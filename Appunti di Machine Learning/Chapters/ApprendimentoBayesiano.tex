\chapter{Apprendimento Bayesano}
Nell'ambito Bayesano si cambia l'approccio avendo la valutazione d'ipotesi in
base alla loro probabilità. Si studia la probabilità rispetto ai dati e rispetto
alle conoscenze pregresse. Non troviamo un'ipotesi che combacia ma che è
probabile.\\
Bisogna studiare come scegliere le ipotesi, usando risultati noti del calcolo
probabilistico e quindi come funziona l'apprendimento. Useremo le nozioni di
probabilità e probabilità condizionata, oltre ovviamente alla \textbf{regola di
  Bayes}. Il ``meglio'' è definito quindi tramite probabilità.\\
Si assume quindi che le quantità d'interesse siano ``governate'' da
distribuzioni di probabilità e che la decisione migliore può essere presa
ragionando su tali distribuzioni e sull'insieme di dati di
training. L'apprendimento Bayesano è importante per due ragioni principali:
\begin{enumerate}
  \item si ha una manipolazione esplicita delle probabilità rispetto ad altri
  approcci pratici di alcuni tipi di problemi di apprendimento (infatti si hanno
  spesso paragoni con gli alberi decisionali e con le reti neurali)
  \item fornisce una prospettiva utile per comprendere metodi di apprendimento
  che non manipolano effettivamente le probabilità
\end{enumerate}
Dal punto di vista delle funzionalità si ha che ogni esempio di training
osservato può aumentare o diminuire, in modo incrementale, la stima di
probabilità relativa alla correttezza di un'ipotesi. Inoltre, come già
anticipato, la conoscenza pregressa può essere combinata con i dati osservati
per determinare la probabilità finale delle varie ipotesi. Si ha inoltre che le
varie ipotesi possono effettuare predizioni probabilistiche e le istanze possono
essere quindi classificate combinando le predizioni delle varie ipotesi, che
sono pesate tramite il peso delle loro probabilità. Con il metodo Bayesano si
ottiene quindi uno ``standard'' per prendere decisioni ottimali rispetto al
quale è possibile misurare ulteriori misure pratiche.\\
Si hanno però alcune difficoltà legate all'apprendimento Bayesano:
\begin{itemize}
  \item si necessita avere la conoscenza di varie probabilità
  \item si hanno costi computazionali non indifferenti
\end{itemize}
Inquadrando nuovamente il fulcro del machine learning ricordiamo che si sta
cercando la ``miglior'' ipotesi $h$, contenuta dello spazio delle ipotesi $H$, a
partire dai dati contenuti in un training set $D$. Nell'apprendimento Bayesano
si ha che la ``miglior'' ipotesi altro non è che la più probabile. Ovviamente
potrei avere più ipotesi.\\
Il punto centrale di questo tipo è dato dal \textbf{definizione di Bayes} che
fornisce un metodo diretto per calcolare la probabilità di tale ipotesi in base:
\begin{itemize}
  \item alla sua probabilità conosciuta a priori
  \item alle probabilità di osservare vari dati data l'ipotesi
  \item ai dati stessi
\end{itemize}
\begin{definizione}[definizione di Bayes]
  Il definizione enuncia che:
  \[P(h|D)=\frac{P(D|h)P(h)}{P(D)}\]
  Avendo:
  \begin{itemize}
    \item $P(h)$ che è la probabilità conosciuta a priori di $h$. Tale
    probabilità riflette qualsiasi conoscenza di base sulla possibilità
    che $h$ sia corretta  
    \item $P(D)$ che è la probabilità conosciuta a priori di $D$, ovvero la
    probabilità che $D$ sia osservato
    \item $P(D|h)$ che è la probabilità di osservare $D$ in presenza
    dell'ipotesi $h$
    \item $P(D|h)$ che è la probabilità a posteriori di $h$. Tale probabilità
    riflette la ``confidenza'' di avere $h$  dopo che $D$ è stato osservato
  \end{itemize}
\end{definizione}
In molti scenari di apprendimento, il learner considera un insieme d'ipotesi
candidate H ed è interessato a trovare l'ipotesi $h\in H$ più probabile in base
ai dati osservati in $D$.
\begin{definizione}
  Definiamo le \textit{ipotesi maximum a posteriori (MAP)} ogni ipotesi
  massimamente probabile:
  \[h_{MAP}=\operatorname*{argmax}_{h\in H}P(h|D)\]
  \[\operatorname*{argmax}_{h\in H}\frac{P(D|h)P(h)}{P(D)}\]
  Ma $P(D)$ può essere cancellato in quanto costante e indipendente da $h$,
  ottenendo:
  \[h_{MAP}=\operatorname*{argmax}_{h\in H}P(D|h)P(h)\]
\end{definizione}
Spesso si assume anche che ogni ipotesi è, a priori, equiprobabile e quindi
possiamo semplificare i conti.
\begin{definizione}
  Dato che $P(D|h)$ viene spesso chiamata \textbf{likehood
    (\textit{probabilità})} di $D$ data $h$ viene definita \textbf{ipotesi
    maximum likehood (ML)} ogni ipotesi che massimizza $P(D|h)$:
  \[h_{ML}=\operatorname*{argmax}_{h\in H}P(D|h)\]
  potendo quindi trascurare $P(h)$ in quanto equivalente $\forall\, h\in H$
\end{definizione}

Si può usare il definizione di Bayes per specificare un algoritmo di apprendimento
molto semplice detto \textbf{algoritmo Brute-Force MAP LEARNING} che si articola
in 2 step:
\begin{enumerate}
  \item $\forall\, h\in H$ calcolo la probabilità a posteriori tramite il definizione
  di Bayes:
  \[P(h|D)=\frac{P(D|h)P(h)}{P(D)}\]
  
  \item restituisco l'ipotesi $h_{MAP}$ con la più alta probabilità a
  posteriori:
  \[h_{MAP}=\operatorname*{argmax}_{h\in H}P(h|D)\]
\end{enumerate}
Però, per specificare il problema d'apprendimento per l'algoritmo, bisogna
obbligatoriamente specificare i valori di:
\begin{itemize}
  \item $P(h)$
  \item $P(D|h)$
\end{itemize}
e quindi dobbiamo obbligatoriamente fare una serie di assunzioni:
\begin{itemize}
  \item il training set deve essere privo di rumore, avendo:
  \[d_i=c(x_i)\]
  \item il target concept deve essere contenuto nello spazio delle ipotesi:
  \[\exists h\in H \mbox{t.c. }h(x)=c(x),\,\,\,\forall\, x\in X\]
  \item devo assumere che le ipotesi siano equiprobabili e quindi:
  \[P(h)=\frac{1}{|H|}\,\,\,\forall\, h\in H\]
  e avendo quindi:
  \[P(D|h)=
    \begin{cases}
      1&\mbox{se } d_i=h(x_i),\,\,\,\forall\, d_i\in D\\
      0&\mbox{altrimenti}
    \end{cases}
  \]
\end{itemize}
Si ha quindi che il problema di apprendimento per l'algoritmo è completamente
definito. Possiamo quindi specificare che, nel primo step, ovvero quando si
cerca $P(h|D)$, posso avere due casi:
\begin{enumerate}
  \item $h$ è inconsistente con $D$, avendo quindi:
  \[P(h|D)=\frac{0\cdot P(h)}{P(D)}=0\]
  \item $h$ è consistente con $D$, avendo quindi:
  \[P(h|D)=\frac{1\cdot \frac{1}{|H|}}{P(D)}=\frac{1\cdot
      \frac{1}{|H|}}{\frac{|VS_{H, D}|}{|H|}}=\frac{1}{|VS_{H, D}|}\] 
\end{enumerate}
Si ha quindi che questa analisi implica che, sotto le condizioni sopra definite,
ogni ipotesi coerente è un'ipotesi MAP, perché per ogni ipotesi coerente si ha
che:
\[P(h|D)=\frac{1}{|VS_{H, D}|}\]
\begin{figure}
  \centering
  \includegraphics[scale = 0.6]{img/map.pdf}
  \caption{Nell'immagine le evoluzioni di probabilità delle
    ipotesi. Nel primo grafico tutte le ipotesi hanno la stessa probabilità, è
    il punto di partenza. Con gli altri due grafici si ha che man mano che i
    dati di addestramento si accumulano, la probabilità a posteriori delle
    ipotesi inconsistenti diventa zero mentre la probabilità totale che si somma
    a 1 è condivisa equamente tra le ipotesi consistenti rimanenti.}
\end{figure}
Si ha che ogni learner consistente ha in output ipotesi MAP se si assume a
priori la distribuzione uniforme delle probabilità su $H$ dati di training
deterministici e privi di rumore.\\
Riprendendo per esempio anche l'algoritmo find-S si ha che:
\begin{itemize}
  \item ha in output ipotesi consistenti e quindi ipotesi MAP sotto la
  distribuzione di probabilità $P(h)$ e $P(D|h)$
  \item $\forall\, P(h)$ che favorisce le ipotesi più specifiche find-S trova
  appunto le ipotesi MAP
\end{itemize}
A riprova che il metodo Bayesano è un modo per caratterizzare il comportamento
degli algoritmi di apprendimento.\\
inoltre, identificando $P(h)$ e $P(D|h)$ base alle quali l'output è l'ipotesi
ottimale, è possibile caratterizzare le ipotesi implicite dell'algoritmo ovvero
il \textbf{bias induttivo} dell'algoritmo (avendo anche che l'inferenza
induttiva è modellata da un sistema di ragionamento probabilistico equivalente
basato sul definizione di Bayes). \\
Si introduce ora il problema di apprendere funzioni target a valori continui
(reti neurali regressione lineare etc$\ldots$). Si ha che n base a determinate
ipotesi, qualsiasi algoritmo di apprendimento che minimizzi l'errore quadratico
tra l'ipotesi di output e i dati di addestramento, produrrà un'ipotesi
ML. Prepariamo quindi il nostro insieme di assunzioni per il problema:
\begin{itemize}
  \item $h:X\to\mathbb{R},\,\,\,\forall\, h\in H$
  \item li esempi sono della forma $\langle x_i, d_i\rangle$
  \item la funzione target è definita come $f:X\to\mathbb{R}$
  \item si hanno $m$ esempi di training dove il valore target di ogni esempio è
  ``sporcato'' dal rumore casuale $e_i$ secondo una distribuzione di probabilità
  normale con media nulla, avendo $d_i=f(x_i)+e_i$
\end{itemize}
Avendo quindi che $h_{ML}=\operatorname*{argmax}_{h\in H}P(D|h)$ e che gli
eventi di training vengono assunti come indipendenti si ha che:
\[h_{ML}=\operatorname*{argmax}_{h\in H}\prod_{i=1}^mP(d_i|h)\]
Dato quindi l'errore e$_i$ distribuito normalmente con media zero e varianza
sconosciuta $\sigma^2$ si ha che anche ogni $d_i$ segue la stessa
distribuzione attorno al target $f(x_i)$. Poiché stiamo scrivendo l'espressione
per $P(D|h)$, assumiamo che $h$ sia la descrizione corretta per $f$, quindi:
\[\mu=f(x_i)=h(x_i)\]
e avendo quindi, per la distribuzione normale:
\[h_{ML}=\operatorname*{argmax}_{h\in H}
  \prod_{i=1}^m\frac{1}{\sqrt{2\pi\sigma^2}}
  e^{-\frac{1}{2\sigma^2}(d_i-h(x_i))^2}\]
È comune massimizzare il logaritmo meno complicato, a causa della monotonia di
questa funzione, avendo:
\[h_{ML}=\operatorname*{argmax}_{h\in
    H}\prod_{i=1}^m\frac{1}{\sqrt{2\pi\sigma^2}}
  -\frac{1}{2\sigma^2}(d_i-h(x_i))^2\]
ma il primo termine è costante e indipendente da $h$ e quindi può essere
cancellato:
\[h_{ML}=\operatorname*{argmax}_{h\in
    H}\prod_{i=1}^m -\frac{1}{2\sigma^2}(d_i-h(x_i))^2\]
Sapendo che massimizzare questo termine negativo equivale a ridurre al minimo il
termine positivo corrispondente:
\[h_{ML}=\operatorname*{argmin}_{h\in
    H}\prod_{i=1}^m \frac{1}{2\sigma^2}(d_i-h(x_i))^2\]
e avendo che anche tutte le costanti sono indipendenti da $h$ e quindi possono
essere rimosse:
\[h_{ML}=\operatorname*{argmin}_{h\in H}\prod_{i=1}^m (d_i-h(x_i))^2\]
trovando che $h_{ML}$ è ciò che minimizza gli errori quadratici. \\
Si specifica la scelta della normale in quanto:
\begin{itemize}
  \item buona approssimazione di molti tipi di rumore nei sistemi fisici 
  \item il definizione del Limite Centrale mostra che la somma di un numero
  sufficientemente grande di variabili casuali indipendenti e identicamente
  distribuite obbedisce a una distribuzione Normale 
\end{itemize}
\textbf{\textit{Si considera solo il rumore sul valore del target e non sugli
    attributi che descrivono le istanze stesse}}.\\
Anche in questo caso si usa il Rasoio di Occam, scegliendo di usare il principio
\textbf{Minimum Description Length (\textit{MDL})}, scegliendo la spiegazione
più breve per i dati osservati.\\
Tramite MDL posso ``giustificare'' la scelta di $h_{MAP}$ in base alla teoria
dell'informazione, infatti:
\[h_{MAP}=\operatorname*{argmax}_{h\in H}P(D|h)P(h)\]
\[=\operatorname*{argmax}_{h\in H}\log_2P(D|h)+\log_2P(h)\]
\[=\operatorname*{argmin}_{h\in H}-\log_2P(D|h)-\log_2P(h)\]
e queste equazioni possono essere interpretate come un'affermazione che sono
preferite ipotesi brevi, assumendo un particolare schema di rappresentazione per
la codifica d'ipotesi e dati.\\
Il principio MDL fornisce un modo per scambiare la complessità delle ipotesi
per il numero di errori commessi dall'ipotesi.\\

\textbf{Su slide altre informazioni su teoria dell'informazione.}
\section{Classificatore Bayesano ottimo}
Ci si chiede quindi qual è la classificazione più probabile della nuova istanza
secondo i dati di training.\\
Vediamo con un esempio che non basta applicare $h_{MAP}$:
\begin{esempio}
  Sia $H=\{h_1, h_2, h_3\}$ con $P(h_1)=0.4$ e $P(h_2)=P(h_3)=0.3$.\\
  Si ha quindi che:
  \[h_{MAP}=h_1\]
  Consideriamo però una nuova istanza $x$ classificata positiva per $h_1$ e
  negativa per le altre due. Si ha quindi che:
  \begin{itemize}
    \item la probabilità che $x$ sia positivo è 0.4
    \item la probabilità che $x$ sia negativo è 0.6
  \end{itemize}
  e quindi la classificazione più probabile non è quella di $h_{MAP}$.
\end{esempio}
Abbiamo che la classificazione più probabile si ottiene combinando le previsioni
di tutte le ipotesi, ponderate in base alle loro probabilità posteriori. Data
$P(v_j|D)$ come la probabilità che la classificazione corretta sia $v_j$:
\[P(v_j|D)=\sum_{h_i\in H}P(v_j|h_i)P(h_i|D)\]
ottenendo che il classificatore Bayesano ottimo è:
\[\operatorname*{argmax}_{v_j\in V}\sum_{h_i\in H}P(v_j|h_i)P(h_i|D)\]
dato $V$ come insieme dei target (?).
\begin{esempio}
  Vediamo un esempio chiarificatore.\\
  Sia $V=\{+,-\}$ e siano:
  \begin{itemize}
    \item $P(h_1, D) =0.4,\,\,\, P(-, h_1) = 0,\,\,\, P(+, h_1) = 1$
    \item $P(h_2, D) =0.3\,\,\, P(-, h_2) = 1,\,\,\, P(+, h2) = 0$
    \item $P(h_3, D) =0.3,\,\,\, P(-, h_3) = 1,\,\,\, P(+, h3) = 0$
  \end{itemize}
  Si hanno:
  \[\sum_{h_i\in H}P(+|h_i)P)(h_i|D)=0.4\]
  \[\sum_{h_i\in H}P(-|h_i)P)(h_i|D)=0.6\]
  \[\operatorname*{argmax}_{v_j\in \{+,-\}}\sum_{h_i\in H}P(v_j|h_i)P(h_i|D)=-\]
\end{esempio}
\section{Classificatore Bayesano naive}
Il Classificatore Bayesano naive si applica alle attività di learning in
cui ogni istanza $x$ è descritta da una giunzione di valori di attributi e in
cui la funzione target $f(x)$ può prendere un valore qualsiasi dall'insieme
finito $V$. Descriviamo gli esempi di training come $\langle a_1, a_2,\ldots
a_n\rangle $. \\
Applicando il metodo Bayesano si ha:
\[v_{MAP}=\operatorname*{argmax}_{v_j\in V}P(v_j|a_1, a_2,\ldots a_n)\]
\[=\operatorname*{argmax}_{v_j\in V}\frac{P(a_1, a_2,\ldots
    a_n|v_j)P(v_j)}{P(a_1, a_2,\ldots a_n)}\]
e, rimuovendo le i fattori indipendenti:
\[v_{MAP}=\operatorname*{argmax}_{v_j\in V}P(a_1, a_2,\ldots a_n|v_j)P(v_j)\]
Avendo che:
\begin{itemize}
  \item $P(v_j)$ può essere stimato tramite la frequenza di $v_j$ in $D$
  \item $P(a_1, a_2,\ldots a_n|v_j)$ non può essere stimato in questo modo am il
  numero di questi termini è pari a $|X|\cdot |V|$
\end{itemize}
Con il classificatore Bayesano naive si hanno diverse semplificazioni. In primis
i valori degli attributi sono condizionatamente indipendenti comprando:
\begin{itemize}
  \item $P(a_1, a_2,\ldots a_n|v_J)=\prod_iP(a_i|v_j)$
  \item il numero dei termini $a_1, a_2,\ldots a_n$ è pari a:
  \[|DA|\cdot |DT|+|DT|\]
  ove:
  \begin{itemize}
    \item $DA$ sta per attributi distinti/unici
    \item $DT$ sta per valori di target distinti/unici
  \end{itemize}
  \item non si ha la ricerca esplicita dentro $H$ ma solo il contro delle
  frequenze 
\end{itemize}
Si ha quindi il classificatore Bayesano naive:
\[v_{NB}=\operatorname*{argmax}_{v_j\in V}P(v_j)\prod_iP(a_i|v_j)\]
\newpage
\begin{esempio}
  Vediamo un esempio chiarificatore.\\
  Sia dato il seguente dataset, con il target \textit{PlayTennis}:
  \begin{figure}[H]
    \centering
\    \includegraphics[scale = 0.7]{img/cbn.jpg}
  \end{figure}
  Si ha la nuova istanza:
  \[\langle Outlook=Sunny, Temperature=Cool, Humidity=High,
      Wind=Strong \rangle\]
  Si ha quindi che:
  \[\prod_iP(a_i|v_j)=(Outlook=sunny|v_j)\cdot
      P(Temperature=cool|v_j)\]\[\cdot P(Humidity=high|v_j)\cdot
      P(Wind=strong|v_j)\]
 
  Avendo la stima delle probabilità:
  \[P(PlayTennis=yes)=\frac{9}{14}=0.64\]
  \[P(PlayTennis=no)=\frac{5}{14}=0.36\]
  In modo analogo calcolo le probabilità condizionali. Per esempio per
  $Wing=Strong$:
  \[P(Wing=Strong|PlayTennis=yes)=\frac{3}{9}=0.33\]
  \[P(Wing=Strong|PlayTennis=no)=\frac{3}{5}=0.60\]
  Posso quindi calcolare $v_{NB}$:
  \[P(yes)\cdot P(Sunny|yes)\cdot P(cool|yes)\cdot P(high|yes)\cdot
    P(cool|yes)=0.0053\]
  \[P(yes)\cdot P(Sunny|no)\cdot P(cool|no)\cdot P(high|no)\cdot
    P(cool|no)=0.0206\]
  Quindi si ha che:
  \[v_{NB}=no\]
  e normalizzando:
  \[\frac{0.0206}{0.0206+0.0053}\]
\end{esempio}
Si è visto come, normalmente, le probabilità sono stimate dalla frazione di
volte in cui si osserva che l'evento si verifica sul numero totale di
opportunità $N$:
\[\frac{n_c}{N}\]
è spesso questo metodo fornisce una buona stima.\\
Si ha però un limite se $n_c$ è molto piccolo, avendo risultati errati con:
\begin{itemize}
  \item sottovalutazione delle probabilità a causa di un bias
  \item se addirittura $n_c$ è nullo esso ``dominerà'' sul classificatore
  Bayesano 
\end{itemize}
Si introduce un nuovo approccio Bayesano sfruttando il cosiddetto
\textbf{m-estimate}, ovvero: 
\[\frac{n_c+m\cdot p}{n+m}\]
dove $p$ è una stima precedente della probabilità che desideriamo determinare,
ed $m$ è una costante chiamata \textit{equivalent sample size} che determina
quanto sia importante il peso di $p$ rispetto ai dati osservati.\\
In assenza d'informazioni aggiuntive $p$ ha distribuzione uniforme quindi si
ha, per $k$ numero di possibili valori di attributo:
\[p=\frac{1}{k}\]
Si nota che per $m$ nullo si ha che l'm-estimate è uguale a:
\[\frac{n_c}{n}\]
e quindi $m$ può essere interpretato come il numero di campioni virtuali
distribuiti su $p$ a cui vengono aggiunti gli $n$ esempi effettivi osservati.
\subsubsection{Esercitazione su Bayes concept learning}
Si ricorda che si cerca una strategia per trovare la miglior ipotesi nello
spazio delle ipotesi. In questo caso si parla di ``migliore'' in termini di più
``probabile''. \\
Quando si fa inferenza bayesana e tutto governato da \textit{incertezza} ma
permette di sfruttare conoscenze a priori.\\
Ricordiamo quindi la formula di Bayes:
\[P(\theta|D)=\frac{P(D|\theta)P(\theta)}{P(D)}\]
con:
\begin{itemize}
  \item $D$ è il dataset
  \item $\theta$ che per noi è l'ipotesi
  \item $P(\theta|D)$ che è la distribuzione a posteriori
  \item $P(\theta)$ è la distribuzione a priori
  \item $P(D|\theta)$ è la verosimiglianza
  \item $P(D)$ è l'evidenza
\end{itemize}
Sapendo che:
\[P(D)=\sum_i(D, \theta_i)=\sum_i P(D|\theta_i)P(\theta_i)\]
$P(D)$ è invariante rispetto al valore dell'ipotesi e quindi possiamo benissimo
rimuoverlo nel calcolo in quanto non influisce sulla probabilità a posteriori:
\[P(\theta_i|D)\varpropto P(D|\theta_i)P(\theta_i)\]
(si ricorda che $\varpropto$ indica \textbf{proposizionale a} quindi che due
cose sono uguali al più di una costante)\\
Ricordiamo inoltre che la massima ipotesi a posteriori è:
\[h_{MAP}=\operatorname*{argmax}_{h\in H}P(h|D)=
  \operatorname*{argmax}_{h\in H}P(D|h)P(h)\]
Se si sa che anche la conoscenza a priori è ininfluente, essendo distribuita
seconda una distribuzione uniforme, e si parla d'ipotesi di massima
verosimiglianza:
\[h_{ML}=\operatorname*{argmax}_{h\in H}P(D|h)\]
L'approccio di Bayes nel machine learning è abbastanza oneroso, dovendo
calcolare tutte le probabilità a posteriori di ogni ipotesi (se $H$ è grande
diventa molto costoso).
\begin{esercizio}
  Si assuma di avere:
  \begin{table}[H]
    \centering
    \begin{tabular}{c||c|c|c|c}
      temp & $h_1$ & $h_2$ & $h_3$ & $h_4$\\
      \hline
      \hline
      hot & no & no & yes & yes\\
      cold & no &yes & no & yes \\
      \hline
      \hline
      $P(D|h)$ & 0 & 0 & 1 &0
    \end{tabular}
  \end{table}
  Si calcoli $h_{MAP}$, assumendo una conoscenza a priori con distribuzione
  uniforme su tutto lo spazio $H$.\\
  Si hanno quindi 4 ipotesi, ciascuna che esprime il valore dei due attributi
  $hot $ e $cold$. L'ultima riga ci dice anche la verosimiglianza di ogni
  ipotesi.\\
  Faccio quindi i conti, sapendo che ogni $P(h_i)=\frac{1}{4}$, avendo
  distribuzione uniforme per tutte le ipotesi.\\
  Calcolo innanzitutto:
  %14
  \[P(D)=\sum_i(D, h_i)=\sum_i
    P(D|h_i)P(h_i)=0\cdot\frac{1}{4}+0\cdot\frac{1}{4}+
    1\cdot\frac{1}{4}+0\cdot\frac{1}{4}=\frac{1}{4}\]
  
  e quindi:
  \[P(h_0|D)=\frac{P(D|h_0)P(h_0)}{P(D)}=
    \frac{0\cdot\frac{1}{4}}{\frac{1}{4}}=0\]
  \[P(h_1|D)=\frac{P(D|h_1)P(h_1)}{P(D)}=
    \frac{0\cdot\frac{1}{4}}{\frac{1}{4}}=0\]
  \[P(h_2|D)=\frac{P(D|h_2)P(h_2)}{P(D)}=
    \frac{1\cdot\frac{1}{4}}{\frac{1}{4}}=1\]
  \[P(h_3|D)=\frac{P(D|h_3)P(h_3)}{P(D)}=
    \frac{0\cdot\frac{1}{4}}{\frac{1}{4}}=0\]
  Si ottiene infine:
  \[h_{MAP}=1\]
\end{esercizio}
\begin{esercizio}
  Si consideri un problema di diagnostica medica con due ipotesi alternative:
  \begin{enumerate}
    \item il paziente ha il cancro (indichiamo la cosa con $cancer$)
    \item il paziente non ha il cancro (indichiamo la cosa con $\neg cancer$)
  \end{enumerate}
  Si hanno inoltre due possibili outcome per i dati dei laboratori:
  \begin{itemize}
    \item positivi, indicati con ``+'' (che indica se si pensa che si abbia il
    cancro)
    \item negativi, indicati con ``-'' (che indica se si pensa che non si abbia
    il cancro)
  \end{itemize}
  Abbiamo quindi le seguenti probabilità:
  \begin{itemize}
    \item $P(cancer)=0.008$
    \item $P(\neg cancer)=0.992$
    \item $P(test=+|cancer)=0.98$
    \item $P(test=-|cancer)=0.02$
    \item $P(test=+|\neg cancer)=0.03$
    \item $P(test=-|\neg cancer)=0.97$
  \end{itemize}
  Si supponga ora che un laboratorio studi un nuovo paziente e che ottenga un
  risultato positivo ``+'' e vediamo se possiamo dire se ha il cancro:
  \[P(cancer|test=+)\varpropto P(test=+|cancer)P(cancer)=0.0078\]
  \[P(\neg cancer|test=+)\varpropto P(test=+|\neg cancer)P(\neg cancer)=0.0298\]
  Quindi:
  \[h_{MAP}=\neg cancer\]
  Posso inoltre determinare la corretta probabilità normalizzando a 1:
  \[P(cancer|test=+)=\frac{0.0078}{0.0078+0.0298}=0.21\]
  \[P(\neg cancer|test=+)=\frac{0.0298}{0.0078+0.0298}=0.79\]
  Si nota quindi come il risultato dipenda molto dalla conoscenza a priori (che
  deve essere disponibile).
\end{esercizio}
\begin{esercizio}
  Si consideri la seguente tabella con un attributo e il target, entrambi
  booleani:
  \begin{table}[H]
    \centering
    \begin{tabular}{c||c}
      Temperature & Play Tennis\\
      \hline
      \hline
      H & yes\\
      H & yes\\
      H & no\\
      C & yes\\
      H & yes\\
      C & no\\
      C & no\\
      C & no\\
      C & yes\\
    \end{tabular}
  \end{table}
  Si hanno quindi vari calcoli:
  \begin{itemize}
    \item $P(H|yes)=0.6$
    \item $P(H|no)=0.25$
    \item $P(C|yes)=0.4$
    \item $P(C|no)=0.75$
    \item $P(yes)=0.56$
    \item $P(no)=0.44$
  \end{itemize}
  Si ha anche:
  \[P(H)=P(H|yes)P(yes)+P(H|no)P(no)=0.6\cdot 0.56+0.25\cdot 0.44 \]
  \[= 0.336+0.11=0.447\]
  \[P(C)=P(C|yes)P(yes)+P(C|no)P(no)=0.4\cdot 0.56+0.75\cdot 0.44 \]
  \[= 0.224+0.33=0.554\]
\end{esercizio}
Ricordiamo che per Naive Bayes si ha, avendo una sequenza $d_1, d_2\ldots d_n$ di
osservazioni: 
\[h_{MAP}=\operatorname*{argmax}_{h\in H}P(h|d_1, d_2\ldots d_n)\]
\[=\operatorname*{argmax}_{h\in H}\frac{P(d_1, d_2\ldots d_n|h)P(h)}{P(D)}\]
\[\varpropto\operatorname*{argmax}_{h\in H}P(d_1, d_2\ldots d_n|h)P(h)\]
Per semplificare, tramite naive Bayes, si può assumere:
\[P(d_1, d_2\ldots d_n|h)=\prod_iP(d_i|h)\]
ovvero indipendenza condizionata (dicendo che un valore $d:i$ è indipendente da
un qualunque $d_j$) e quindi si ha, sostituendo:
\[h_{MAP}=\operatorname*{argmax}_{h\in H}P(h)\prod_iP(d_i|h)\]
\newpage
\begin{esercizio}
  Dati 6 esempi con tre attributi e target $T$ (attributi e target sono
  booleani): 
  \begin{table}[H]
    \centering
    \begin{tabular}{c||c|c|c|c}
      Esempi & $A$ & $B$ & $C$ & $T$\\
      \hline
      \hline
      $x_1$ & 1 & 1 & 1 & 0\\
      $x_2$ & 0 & 1 & 1 & 0\\
      $x_3$ & 1 & 0 & 1 & 1\\
      $x_4$ & 0 & 0 & 0 & 1\\
      $x_5$ & 0 & 1 & 0 & 1\\
      $x_6$ & 1 & 1 & 0 & ?\\
    \end{tabular}
  \end{table}
  Si completi la tabella con la label $T$ per $x_6$.\\
  Dividiamo in:
  \begin{itemize}
    \item $h_0=[T=0]$
    \item $h_1=[T=1]$
  \end{itemize}
  Si ha quindi per la prima ipotesi ($h_0$), sempre pensando a $x_6$:
  \[P(T=0|A=1, B=1, C=0)=\frac{P(A=1, B=1, C=0|T=0)P(T=0)}{P(x_6)}\]
  elimino l'evidenza:
  \[\varpropto P(A=1, B=1, C=0|T=0)P(T=0)\]
  uso l'assunzione di Naive Bayes:
  \[\varpropto P(A=1|T=0)P(B=1|T=0)P(C=0|T=0)P(T=0)
    \varpropto \frac{1}{2}\cdot 1\cdot 0\cdot \frac{2}{5}\]
  Avendo quindi, svolgendo il conto:
  \[P(h_0|x_6)\varpropto  \frac{1}{2}\cdot 1\cdot  0\cdot\frac{2}{5}=0\]
  

  Si passa poi alla seconda ipotesi ($h_1$), sempre pensando a $x_6$:
  \[P(T=1|A=1, B=1, C=0)=\frac{P(A=1, B=1, C=0|T=1)P(T=1)}{P(x_6)}\]
  elimino l'evidenza:
  \[\varpropto P(A=1, B=1, C=0|T=1)P(T=1)\]
  uso l'assunzione di Naive Bayes:
  \[\varpropto P(A=1|T=1)P(B=1|T=1)P(C=0|T=1)P(T=1)
    \varpropto \frac{1}{3}\cdot \frac{1}{3}\cdot \frac{2}{3}\cdot\frac{3}{5}\]
  Avendo quindi, svolgendo il conto:
  \[P(h_1|x_6)\varpropto  \frac{1}{3}\cdot \frac{1}{3}\cdot
    \frac{2}{3}\cdot\frac{3}{5}=\frac{2}{45}\]
  So quindi che $h_{MAP}=h_1$, avendo che la probabilità con $h_1$ è maggiore di
  quella con $h_0$. Si ha quindi, avendo che per $x_6$ scelgo $h_1$ che
  corrisponde ad avere $T=1$:
  \begin{table}[H]
    \centering
    \begin{tabular}{c||c|c|c|c}
      Esempi & $A$ & $B$ & $C$ & $T$\\
      \hline
      \hline
      $x_1$ & 1 & 1 & 1 & 0\\
      $x_2$ & 0 & 1 & 1 & 0\\
      $x_3$ & 1 & 0 & 1 & 1\\
      $x_4$ & 0 & 0 & 0 & 1\\
      $x_5$ & 0 & 1 & 0 & 1\\
      $x_6$ & 1 & 1 & 0 & 1\\
    \end{tabular}
  \end{table}
  
  Si può anche calcolare, per completezza:
  \[P(x_6)=\sum_iP(x_6, h_i)=\sum_{i\in\{0, 1\}}P(A=1, B=1, C=0|T=i)P(T=i)\]
  \[= P(A=1|T=0)P(B=1|T=0)P(C=0|T=0)P(T=0)+\]
  \[P(A=1|T=1)P(B=1|T=1)P(C=0|T=1)P(T=1)\]
  \[=( \frac{1}{2}\cdot 1\cdot  0\cdot\frac{2}{5})+(\frac{1}{3}\cdot
    \frac{1}{3}\cdot \frac{2}{3}\cdot\frac{3}{5})\]
  \[=\frac{1}{3}\cdot \frac{1}{3}\cdot
    \frac{2}{3}\cdot\frac{3}{5}=\frac{2}{45}\]

  Posso quindi calcolare:
  \[P(h_1|x_6)=P(T=1|A=1, B=1, C=0)=\frac{P(A=1, B=1, C=0|T=1)P(T=1)}{p(x_6)}\]
  \[=\frac{P(A=1|T=1)P(B=1|T=1)P(C=0|T=1)P(T=1)}{p(x_6)}\]
  \[=\frac{\frac{1}{3}\cdot \frac{1}{3}\cdot
      \frac{2}{3}\cdot\frac{3}{5}}
    {\frac{1}{3}\cdot \frac{1}{3}\cdot
      \frac{2}{3}\cdot\frac{3}{5}}=1\]
  Ugualmente dovrei calcolare $P(h_0|x_6)$ ma sapendo che i valori sono booleani
  e quindi ho solo $h_1$ e $h_0$, avendo $P(h_1|x_6)=1$ so che
  $P(h_0|x_6)=0$. \\
  Si hanno quindi:
  \begin{itemize}
    \item $h_{MAP}=h_1:=[T=1]$
    \item $P(x_6)=\frac{1}{3}\cdot \frac{1}{3}\cdot
    \frac{2}{3}\cdot\frac{3}{5}=\frac{2}{45}=0.04$
    \item $P(x_6|T=0)P(T=0)= \frac{1}{2}\cdot 1\cdot  0\cdot\frac{2}{5}=0$
    \item $P(x_6|T=0)= \frac{1}{2}\cdot 1\cdot  0=0$
    \item $P(A=1|T=0)=\frac{2}{5}$
    \item $P(T=0)=\frac{2}{5}$
    \item $P(h_0|x_6)=P(T=0|x_6)=0$
    \item $P(x_6|T=1)P(T=1)=\frac{1}{3}\cdot \frac{1}{3}\cdot
    \frac{2}{3}\cdot\frac{3}{5}=\frac{2}{45}=0.04$
    \item $P(x_6|T=1)=\frac{1}{3}\cdot \frac{1}{3}\cdot
    \frac{2}{3}=\frac{2}{27}=0.07$
    \item $P(B=1|T=1)=\frac{1}{3}$
    \item $P(T=1)=\frac{3}{5}$
    \item $P(h_1|x_6)=P(T=1|x_6)=1$
  \end{itemize}
\end{esercizio}