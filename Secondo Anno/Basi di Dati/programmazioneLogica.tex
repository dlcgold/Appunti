\chapter{Programmazione Logica}
L'obiettivo della programmazione logica è quello di costruire uno schema logico in grado di descrivere, in maniera corretta ed
efficiente, tutte le informazioni contenute nell'ER, prodotto nella fase di progettazione concettuale.\newline
La progettazione logica risulta solitamente articolata in 2 fasi:
\begin{itemize}
        \item \emph{ristrutturazione dello schema ER}, fase indipendente dal modello logico scelto e si basa sui criteri di 
              ottimazione dello schema.
        \item \emph{traduzione nel modello logico}, con riferimento al modello scelto, con la possibile inclusione di un
              ulteriore ottimazione, in base alle caratteristiche del modello stesso.
\end{itemize}
Gli input e gli output di ogni fase della progettazione logica si possono notare nella \figurename~\ref{img:fig8.1} e
per ottimizzare lo schema ER si utilizza la normalizzazione, studiata nel prossimo capitolo.

Uno schema ER può essere modificato per ottimizzare alcuni indici di prestazione del progetto, anche se è difficile da poter
analizzare, essendo le prestazioni dipendenti da parametri fisici e dal DBMS usato, ma è possibile usare 2 parametri, generalmente
utile e che regolano le prestazioni dei sistemi software, per effettuarne una stima:
\begin{itemize}
        \item \emph{costo di un operazione}, in cui viene valutato, in termine di occorrenze di entità ed associazioni visitate
              in media per rispondere ad un operazione sul database.
        \item \emph{occupazione di memoria}, in cui viene valutato in termini di spazio in memoria necessario per memorizzare
              i dati descritti dallo schema.
\end{itemize}
Per studiare questi parametri abbiamo bisogno di conoscere il \emph{volume dei dati}, ossia il numero di occorrenze di ogni 
concetto ER e la dimensione di ogni attributo, e le \emph{caratteristiche delle operazioni}, con indicazione del 
tipo, frequenza e dei dati coinvolti.\newline
Supponendo di analizzare la situazione, descritta dallo schema \figurename~\ref{img:fig8.2}, con le seguenti operazioni:
\begin{itemize}
    \item OP1: assegna un impiegato a un progetto
    \item OP2: trova i dati di un impiegato, del dipartimento in cui lavora e dei progetti in cui collabora
    \item OP3: trova i dati di tutti gli impiegati di un certo dipartimento
    \item OP4: per ogni sede, trova i dipertimenti con il cognome del direttore e l'elenco degli impiegati del dipartimento
\end{itemize}
Sebbene un analisi effettuata su un numero ristretto di operazioni può sembrare superflua, si può affermare che un certo
che un numero ristretto di operazioni rappresenta la maggior parte del carico generato nel database.\newline
IL volume dei dati e le caratteristiche possono essere descritte mediante una tabella, come si può notare nella 
\figurename~\ref{fig:img8.3} per il caso in esame.

È possibile per ogni operazione descrivere i dati coinvolti tramite uno schema di operazione, ossia un frammento dello
schema ER indicante la parte interessata dall'operazione come si nota nella \figurename~\ref{img:fig8.4}, ed avendo
a disposizione queste informazioni è possibile stimare il costo di un operazione, attraverso la conta del numero di accessi
alle occorrenze delle entità ed associazioni necessarie per l'esecuzione dell'operazione.\newline
Tutte le occorrenze e gli accessi effettuati da un operazione, possono essere riassunti in una \emph{tavola degli accessi},
come si può notare nella \figurename~\ref{img:fig8.5}.

La fase di ristrutturazione di uno schema ER si può suddividere in 4 passi:
\begin{itemize}
    \item \emph{analisi delle ridondanze}, in cui si decide se eliminare o mantenere eventuali ridondanze presenti nello schema,
          indicante la presenza nello schema di dati derivabili da altri dati.
    \item \emph{rimozione delle generalizzazioni}, in quanto non essendo un costrutto del modello relazionale devono essere
          analizzate e sostituite da altri costrutti.
    \item \emph{partizionamento/accorpamento di entità ed associazioni}, in cui si può decidere di accorpare e partizionare
          delle entità e delle associazioni, al fine di ottimizzare le operazioni su di esse.
    \item \emph{scelta degli identificatori principali}, in cui viene scelto l'identificatore per quella entità con più
          di un identificatore.
\end{itemize}
Nei seguenti paragrafi analizziamo in dettaglio tutte le quattro fasi di ristrutturazione dell'ER, incominciando dall'
analisi delle ridondanze.

\subsection{Analisi delle ridondanze}
Le ridondanze presenti eventualmente in uno schema ER possono essere di diverso tipo ma i casi più frequnti sono i seguenti:
\begin{itemize}
    \item attributi derivabili, occorrenza per occorrenza, da altri attributi della stessa entità( o associazione), 
          come ad esempio nella \figurename~\ref{img:fig8.7}, l'importo netto lo si può ottenere mediante 
          l'importo netto e l'iva.
    \item attributi derivabili da attributi di altre entità(o associazioni), di solito attraverso funzioni aggregative,
          come ad esempio nella \figurename~\ref{img:fig8.7} l'attributo importo totale si può ottenere mediante
          la somma dei prezzi dei prodotti di cui un acquisto è composto.
    \item attributi derivabili da operazioni di conteggio di occorrenze, infatti nella \figurename~\ref{img:fig8.7}
          l'attributo Numero di Abitanti lo si può ottenere attraverso la conta delle occorrenze della associazione
          Residenza a cui tale città partecipa.
    \item associazioni derivabili dalla composizione di altre associazioni, in presenza di cicli, come ad esempio
          nella \figurename~\ref{img:fig8.7} l'associazione Docenza può essere derivata dalle associazioni 
          Frequenza ed Insegnamento, in quanto rappresentano informazioni similari.
\end{itemize}
La presenza di un dato derivato presenta come vantaggio quello di ridurre il numero di accessi necessari per ottenere
il dato derivato, con un aggravio di memoria, oggi giorno quasi trascurabile, e la necessità di effettuare operazioni
aggiuntive per mantenere il dato aggiornato.\newline
La decisione di mantenere o eliminare una ridondanza va presa considerando il costo delle operazioni e l'aggravio di memoria
sia in presenza che in assenza del dato ridondante, usando le tecniche presentate nel primo paragrafo di questo capitolo.

\subsection{Eliminazione delle generalizzazioni}
Dato che i sistemi tradizionali per la gestione dei Database non consentono di rappresentare direttamente una generazione,
risulta spesso necessario trasformarle in un'entità/associazione, con l'utilizzo di 3 possibili alternative:
\begin{itemize}
    \item Accorpamento delle figlie della generalizzazione nel genitore, ossia le entità $E1$ ed $E2$ vengono eliminate
          e le loro proprietà vengono aggiunte all'entità genitore $E_0$, a cui viene aggiunto un attributo tipo, per 
          identificare se si rappresenta l'entità $E_1$ oppure la $E_2$.
    \item Accorpamento del genitore della generalizzazione nelle figlie, ossia l'entità $E_0$ viene eliminata e per la
          proprietà dell'ereditarietà, i suoi attributi, il suo identificatore e le relazioni a cui partecipava, vengono
          aggiunti alle entità figlie $E_1$ e $E_2$.
    \item Sostituzione della generalizzazione con associazioni, in cui la generalizzazione si trasforma in due associazioni
          uno a uno che legano rispettivamente l'entità genitore $E_0$ con le sue entità figlie $E_1$ e $E_2$.\newline
          Nello schema vanno introdotti anche che ogni occorrenza di $E_0$ non può partecipare contemporaneamente a tutte 
          le due associazioni ed inoltre se la generalizzazione è totale ogni occorrenza deve partecipare ad almeno una 
          associazione.
\end{itemize}
La scelta tra le varie alternative può essere fatta in maniera analoga a quanto fatto per i dati derivati, anche se 
è possibile stabilire alcune regole di carattere generale:
\begin{itemize}
    \item l'alternativa (1) è conveniente quando le operazioni non fanno molta distinzione tra le occorrenze e gli attributi
          di $E_0, E_1$ e $E_2$;in questo caso infatti, anche se avremmo uno spreco di memoria per la presenza di valori nulli,
          la scelta ci assicura un numero minore di accessi rispetto alle altre nelle quali le occorrenze e gli attributi
          sono distribuiti tra le varie entità.
    \item l'alternativa (2) è possibile solo quando la generalizzazione è totale, altrimenti le occorrenze di $E_0$,
          che non sono occorrenze delle entità figlie non verrebbero rappresentate ed è conveniente quando ci sono 
          operazioni che si riferiscono a solo un entità figlia, in quanto in questo caso si avrebbe un risparmio 
          di memoria rispetto all'opzione (1), in quanto non si hanno valori nulli ed inoltre rispetto all'opzione (3)
          si avrebbe una riduzione del numero di accessi, dato che non bisogna visitare $E_0$ per accedere ad alcuni
          attributi delle entità figlie.
    \item l'alternativa (3) è conveniente quando la generalizzazione non è totale e ci sono operazioni che si riferiscono
          solo ad occorrenze di $E_1(E_2)$ oppure di $E_0$, e dunque fanno distinzione tra entità figlia ed entità genitore.\newline
          In questo caso abbiamo un risparmio di memoria rispetto ad (1), per la mancanza di valori nulli, ma c'è un 
          incremento di accessi per mantenere la consistenza delle occorrenze rispetto ai vincoli introdotti
\end{itemize}
L'alternativa (3), nonostante può sembrare molto poco usata e poco utile, ha il grande vantaggio di generare entità
con pochi attributi e questo si traduce, a livello pratico, in strutture logiche di piccole dimensioni, per le quali
un accesso fisico permette di recuperare molti dati in una volta sola per cui per alcuni casi critici va effettuata
un'analisi più fine, al fine di tenere conto di altri fattori come la dimensione dei domini degli attributi e la quantità
di dati recuperabili con una sola operazione di lettura.\newline
Le alternative analizzate non sono le uniche possibili, infatti è possibile effettuare ristrutturazioni mediante
la combinazione delle tre trasformazioni presentate e un esempio delle possibili ristrutturazioni effettuate alla 
\figurename~\ref{img:fig8.10} si può notare nella \figurename~\ref{img:fig8.11}.





