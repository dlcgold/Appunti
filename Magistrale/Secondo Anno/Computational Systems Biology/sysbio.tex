\documentclass[a4paper,12pt, oneside]{book}

% \usepackage{fullpage}
\usepackage[italian]{babel}
\usepackage[utf8]{inputenc}
\usepackage{amssymb}
\usepackage{amsthm}
\usepackage{graphics}
\usepackage{amsfonts}
\usepackage{listings}
\usepackage{amsmath}
\usepackage{amstext}
\usepackage{engrec}
\usepackage{rotating}
\usepackage{verbatim}
\usepackage[safe,extra]{tipa}
% \usepackage{showkeys}
\usepackage{multirow}
\usepackage{hyperref}
\usepackage{microtype}
\usepackage{fontspec}
\usepackage{enumerate}
\usepackage{physics}
\usepackage{braket}
\usepackage{marginnote}
\usepackage{pgfplots}
\usepackage{cancel}
\usepackage{polynom}
\usepackage{booktabs}
\usepackage{enumitem}
\usepackage{framed}
\usepackage{pdfpages}
\usepackage{pgfplots}
\usepackage{algorithm}
% \usepackage{algpseudocode}
\usepackage[cache=false]{minted}
\usepackage{mathtools}
\usepackage[noend]{algpseudocode}
\newcommand*{\bfrac}[2]{\genfrac{}{}{0pt}{}{#1}{#2}}

\usepackage{tikz}\usetikzlibrary{er}\tikzset{multi  attribute /.style={attribute
    ,double  distance =1.5pt}}\tikzset{derived  attribute /.style={attribute
    ,dashed}}\tikzset{total /.style={double  distance =1.5pt}}\tikzset{every
  entity /.style={draw=orange , fill=orange!20}}\tikzset{every  attribute
  /.style={draw=MediumPurple1, fill=MediumPurple1!20}}\tikzset{every
  relationship /.style={draw=Chartreuse2,
    fill=Chartreuse2!20}}\newcommand{\key}[1]{\underline{#1}}
\usetikzlibrary{arrows.meta}
\usetikzlibrary{decorations.markings}
\usetikzlibrary{arrows,shapes,backgrounds,petri}
\tikzset{
  place/.style={
    circle,
    thick,
    draw=black,
    minimum size=6mm,
  },
  transition/.style={
    rectangle,
    thick,
    fill=black,
    minimum width=8mm,
    inner ysep=2pt
  },
  transitionv/.style={
    rectangle,
    thick,
    fill=black,
    minimum height=8mm,
    inner xsep=2pt
  }
} 
\usetikzlibrary{automata,positioning,chains,fit,shapes}
\usepackage{fancyhdr}
\pagestyle{fancy}
\fancyhead[LE,RO]{\slshape \rightmark}
\fancyhead[LO,RE]{\slshape \leftmark}
\fancyfoot[C]{\thepage}
\usepackage[usenames,dvipsnames]{pstricks}
\usepackage{epsfig}
\usepackage{pst-grad} % For gradients
\usepackage{pst-plot} % For axes
\usepackage[space]{grffile} % For spaces in paths
\usepackage{etoolbox} % For spaces in paths
\makeatletter % For spaces in paths
\patchcmd\Gread@eps{\@inputcheck#1 }{\@inputcheck"#1"\relax}{}{}
\makeatother
\usepackage{lipsum}
\DeclareSymbolFont{symbolsC}{U}{txsyc}{m}{n}
\DeclareMathSymbol{\strictif}{\mathrel}{symbolsC}{74}
\title{Computational Systems Biology}
\author{UniShare\\\\Davide Cozzi\\\href{https://t.me/dlcgold}{@dlcgold}}
\date{}

\pgfplotsset{compat=1.13}
\begin{document}
\maketitle

\definecolor{shadecolor}{gray}{0.80}
\setlist{leftmargin = 2cm}
\newtheorem{teorema}{Teorema}
\newtheorem{definizione}{Definizione}
\newtheorem{esempio}{Esempio}
\newtheorem{corollario}{Corollario}
\newtheorem{lemma}{Lemma}
\newtheorem{osservazione}{Osservazione}
\newtheorem{nota}{Nota}
\newtheorem{esercizio}{Esercizio}
\algdef{SE}[DOWHILE]{Do}{doWhile}{\algorithmicdo}[1]{\algorithmicwhile\ #1}
\tableofcontents
\renewcommand{\chaptermark}[1]{%
  \markboth{\chaptername
    \ \thechapter.\ #1}{}}
\renewcommand{\sectionmark}[1]{\markright{\thesection.\ #1}}
\newcommand{\floor}[1]{\lfloor #1 \rfloor}
\newcommand{\MYhref}[3][blue]{\href{#2}{\color{#1}{#3}}}%
\chapter{Introduzione}
\textbf{Questi appunti sono presi a lezione. Per quanto sia stata fatta
  una revisione è altamente probabile (praticamente certo) che possano
  contenere errori, sia di stampa che di vero e proprio contenuto. Per
  eventuali proposte di correzione effettuare una pull request. Link: }
\url{https://github.com/dlcgold/Appunti}.\\
\chapter{Introduzione alla Systems Biology}
Per descrivere sistemi biologici complessi si hanno vari tipi di modelli.\\
Kitano (il ``padre'' di quest'ambito), nel 2002, disse che per capire i sistemi
biologici complessi bisogna integrare risultati sperimentali e metodi
computazionali, ottenendo quindi la 
vera e propria \textbf{Systems Biology}. Tramite l'interazione di vari
componenti si ottengono tali sistemi. Disse infatti:
\begin{center}
  \textit{To understand complex biological systems requires the integration of
    experimental and computational research — in other words a systems biology
    approach.} 
\end{center}
Weston, nel 2004, ha aggiunto l'importanza dello studio delle interazioni e
delle regolazioni tra i vari componenti del sistema, studiando le risposte alla
genetica o alle perturbazioni ambientali, al fine di capire nuove proprietà del
sistema. Infatti disse:
\begin{center}
  \textit{Systems biology is the analysis of the relationships among the
    elements in a system in response to genetic or environmental perturbations,
    with the goal of understanding the system or the emergent properties of the
    system} 
\end{center}
Ideker (altro ``padre'' di quest'ambito), già nel 2001, aveva definito la System
Biology come l'integrazione dei 
dati sperimentali con i modelli matematici che descrivono componenti e
interazioni, al fine di simulare il comportamento complessivo ``in silico''. Nel
dettaglio, citandolo:
\begin{center}
  \textit{Systems biology studies biological systems by systematically
    perturbing them (biologically, genetically, or chemically); monitoring the
    gene, protein, and informational pathway responses; integrating these data;
    and ultimately, formulating mathematical models that describe the structure
    of the system and its response to individual perturbations} 
\end{center}
Ai metodi standard della biologia quindi si aggiungono le teorie
informatiche, quelle matematiche, quelle fisiche, quelle chimiche, quelle
ingegneristiche. A partire dal fenomeno biologico quindi si effettuando
esperimenti, ottenendo dei dati sperimentali relativi alle funzioni, alle
strutture e alle interazioni delle varie componenti biologiche. A partire da
questi dati si costruisce un \textbf{modello matematico} che porterà alla
produzione 
di \textit{ipotesi} a partire da esso. Inoltre l'insieme di ipotesi produrrà
nuovi dati che potranno essere anche usati per rifinire il modello
stesso. Inoltre tali ipotesi possono portare a sperimentazioni in \textbf{dry
  lab}, quindi ``in silico'' tramite simulazioni, ma anche in \textbf{wet lab},
quindi in laboratorio qualora possibile. Tali sperimentazioni contribuiranno a
migliorare i dati stessi, producendone anche di nuovi. Si ha quindi un sistema
ciclico di costante miglioramento della ricerca stessa, come visualizzabile in
figura \ref{fig:csb}.\\
\begin{figure}
  \centering
  \includegraphics[scale = 0.8]{img/csb.pdf}
  \caption{Grafico rappresentante il processo ciclico della Systems Biology.} 
  \label{fig:csb}
\end{figure}
Un altro aspetto fondamentale del discorso è capire cosa \textbf{non} sia la
\textit{systems biology}. Citando Wolkenhauer \footnote{O. Wolkenhauer, Why
  Systems Biology is (not) called Systems Biology, BIOforum Europe 4/2007}:
\begin{center}
  \textit{Opening then the book, which I discovered in the London bookstore, I
    read the contents list: “Shotgun Fragment Assembly”, “Gene Finding”, “Local
    Sequence Similarities”, ... What?? ... “Protein Structure Prediction”, “Some 
    Computational Problems Associated with Horizontal Gene Transfer” ... what on
    earth has this to do with systems biology, I asked myself?}\\
  \textit{...}\\
  \textit{Most important to me is
    however that cells and proteins are in  teracting in space and time, that
    is, we are dealing here with (nonlinear) dynamic systems. If you ask me
    then, systems biology is a merger of systems theory with cell biology.}
  \\
  \textit{...}\\
  \textit{Systems biology and bioinformatics are different but complementary.}
\end{center}
Infatti tematiche come l'assemblaggio, l'allineamento etc$\ldots$ non sono
tematiche della \textit{systems biolog}y ma della \textit{bioinformatica},
nonostante spesso vengano confuse e sovrapposte. L'analisi diretta dei dati
biologici non è campo della \textit{systems biology} in quanto si perde uno
degli aspetti fondamentali, ovvero quello del \textbf{tempo}, che comporta lo
studio di \textbf{sistemi dinamici}, che appunto di evolvono nel tempo. In
bioinformatica d'altro canto si ha spesso a che fare con dati provenienti da
pochi timestamp (se non direttamente da uno solo). Inoltre, sempre in
bioinformatica, si studiano solitamente poche componenti biologiche, senza
studiarne l'interazione tra esse.\\
La domanda più importante della \textit{systems biology}, della quale possiamo
vedere uno schema generale delle fasi in figura \ref{fig:csb2}, è quindi:
\begin{center}
  \textit{dato un sistema biologico d'interesse, di cui si vogliono studiare le
    funzioni etc$\ldots$, quale approccio modellistico è più adatto per
    descrivere quel sistema?}
\end{center}
Una volta risposto a questo quesito bisogna ovviamente capire quale sia lo
strumento computazionale di cui si ha bisogno per simulare e analizzare tale
sistema. Bisogna infine capire quali predizioni si possono ottenere da questo
modello, che comunque deve prima essere validato. Tra le cose principali che si
vogliono capire abbiamo, ad esempio, se si può controllare il sistema e se si
può riprodurre il tutto in laboratorio riducendo il numero di tentativi e di
conseguenza anche il costo dell'esperimento in \textit{wet lab}.\\
\begin{figure}
  \centering
  \includegraphics[scale = 0.3]{img/csb2.jpg}
  \caption{Schema generale delle fasi tipiche che compongono la systems
    biology.} 
  \label{fig:csb2}
\end{figure}
Possiamo quindi facilmente intuire che uno degli aspetti fondamentali di questo
ambito è quello di fare le corrette \textit{assunzioni}. Citando ancora
Wolkenhauer \footnote{O. Wolkenhauer, Why Systems Biology is (not) called
  Systems Biology, BIOforum Europe 4/2007}:
\begin{center}
  \textit{The modelling process itself is more important than the model. The
    discussion between the experimentalists and the theoretician, ro decide
    which variables to measure and why, how to formally represent interaction in
    a mathematical form is the basis for succesful interdisciplinary research in
    Systems Biology. In light of the complexity of moleculr systems and the
    available experimental data, Systems Biology is the art of making the right
    assumptions in modelling.}
\end{center}
si nota come il raggiungimento delle assunzioni stesse per ottenere il modello
sia una fase di importanza maggiore rispetto al modello stesso. Il modello
infatti rappresenta la realtà ma non è la realtà stessa e partire da assunzioni
false ed errate porterà ad un modello magari funzionante ``dal punto di vista
sintattico'' ma non `` dal punto di vista semantico'', avendo che esso non potrà
mai essere validato. Nella citazione si parla inoltre di \textit{variabili},
come elemento base dei vari modelli. Tra tali variabili si cercano relazioni,
correlazioni etc$\ldots$\\
Normalmente il punto di partenza sono i \textit{dati omici}.
\begin{shaded}
  Quanto qui riportato è tratto da wikipedia
  \footnote{\url{https://it.wikipedia.org/wiki/-omica}} 
  \begin{definizione}
    In biologia molecolare, ci si riferisce comunemente al neologismo omica (in
    inglese omics) per indicare l'ampio numero di discipline biomolecolari che
    presentano il suffisso ``-omica'', come avviene per la genomica o la
    proteomica. Il suffisso correlato -oma (in inglese -omes) indica invece
    l'oggetto di studio di queste discipline (genoma, proteoma).  
  \end{definizione}
  I più importanti ``-oma'' proposti recentemente all'interno della comunità
  scientifica sono:
  \begin{itemize}
    \item il \textbf{trascrittoma} è l'insieme degli mRNA trascritti nell'intero
    organismo, tessuto, cellula; è studiato dalla trascrittomica
    \item il \textbf{metabolom}a comprende la totalità dei metaboliti presenti
    in un organismo; è studiato dalla metabolomica 
    \item il \textbf{metalloma} comprende la totalità delle specie di metalli e
    metalloidi; è studiato dalla metallomica 
    \item il \textbf{lipidoma} comprende la totalità dei lipidi; è studiato dalla
    lipidomica 
    \item l'\textbf{interattoma} comprende la totalità delle interazioni
    molecolari che hanno luogo in un organismo; un nome che comunemente indica
    la disciplina 
    della interattomica è quello di biologia dei sistemi (systems biology)
    \item lo \textbf{spliceoma} (da non confondersi con lo spliceosoma, il
    complesso di 
    proteine ed acidi nucleici coinvolti nello splicing) comprende la totalità
    delle isoforme proteiche dovute a splicing alternativo; è studiato dalla
    spliceomica
    \item l'\textbf{ORFeoma} comprende la totalità delle sequenze di DNA che
    iniziano con 
    un codone ATG e terminano con un codone di stop (sequenze note come ORF,
    open reading frames). Queste sequenze sono ritenute in grado di codificare
    per una proteina o per una parte
    \item \textbf{textoma}: l'insieme della letteratura scientifica disponibile
    alla consultazione (studiato dalla textomica)
    \item \textbf{kinoma}: l'insieme delle protein chinasi (dall'inglese kinase)
    di una cellula. Esistono pubblicazioni scientifiche che citano il termine
    kinomica 
    \item \textbf{glicosiloma}: correlato alle reazioni di glicosilazione
    (studiato dalla glicosilomica)
    \item \textbf{fisioma}: correlato alla fisiologia (studiato dalla fisiomica)
    \item \textbf{neuroma}: l'insieme delle componenti nervose di un organismo
    (studiato dalla neuromica) 
    \item \textbf{predittoma}: l'insieme delle predizioni di struttura proteica
    \item \textbf{reattoma}: l'insieme dei processi biologici
    \item \textbf{ionoma}: insieme dei nutrienti minerali e degli elementi in
    tracce che si trovano in un organismo 
    \item \textbf{connettoma}: l'insieme di tutti i neuroni e le sinapsi di un
    cervello 
  \end{itemize}
\end{shaded}
Si hanno quindi vari ``livelli'' di studio, al variare dei dati omici, per i
quali variano gli strumenti. Ad esempio:
\begin{itemize}
  \item si ha il \textbf{genoma}, studiato tramite il \textit{sequenziamento},
  la \textit{genotipizzazione} etc$\ldots$
  \item il \textbf{trascrittoma}, ottenuto dopo la \textit{trascrizione},
  studiato tramite \textit{microarrays}, \textit{oligonucleotide chips}
  etc$\ldots$
  \item il \textbf{proteoma}, ottenuto dopo la \textit{traduzione}, studiato
  tramite \textit{proteomica MS-based}, \textit{elettroforesi} etc$\ldots$
  \item il \textbf{metaboloma}, ottenuto tramite le \textit{reazioni}, studiato
  tramite \textit{spettroscopia di massa}, \textit{risonanze magnetiche}
  etc$\ldots$
  \item l'\textbf{interattoma}, ottenuto tramite appunto le varie
  \textit{interazioni}, studiato tramite \textit{screens yeast-to-hybrid}
  etc$\ldots$ 
  \item il \textbf{fenomeno}, ottenuto dopo l'\textit{integrazione} delle varie
  interazioni, studiato tramite \textit{gene inactivations} etc$\ldots$
\end{itemize}
Ognuno di questi ``livelli'' ha una panoramica diversa su quello che sta
accadendo, è accaduto, potrebbe accadere o accadrà ad una certa
cellula. Partendo dalle informazioni dinamiche/cinetiche, ovvero dai dati, e
dalle informazioni strutturali dei vari \textit{pathway} si riesce ad ottenere
la rappresentazione matematica. Ovviamente è impensabile pensare di studiare
tutti i ``livelli'' contemporaneamente ma si può studiare solo una parte del
sistema, studiandone un paio di ``livelli'' o poco più. Inoltre ogni ``livello''
ha associato un suo formalismo matematico, legato alla singola modellazione
matematica. Non sempre tali formalismi sono facilmente integrabili (magari in un
caso ho delle EDO e in un altro dei grafi). Si ha quindi non solo un discorso di
\textit{data integration} ma anche di integrazione dei modelli matematici stessi
e questo non sempre è possibile.\\
Nella realtà, inoltre, prima di scegliere un modello bisogna scegliere
l'\textit{approccio} con cui ottenerlo. Generalmente se ne hanno due in
\textit{systems biology}: 
\begin{enumerate}
  \item l'approccio \textbf{top-down}. In questo caso si parte dalle analisi
  omiche, solitamente con pochissimi timestamp, i cui risultati vengono trattati
  con tecniche bioinformatiche, che riducono anche l'influenza degli errori, per 
  ottenere una \textbf{mappa globale di interazioni}, con le interazioni tra
  migliaia di componenti cellulari, dalla quale si ottiene il
  \textbf{modello predittivo del sistema}. Questo approccio è quindi supportato
  da una grande quantitù di dati basati su \textit{high-throughput} e
  \textit{global profiling}
  \item l'approccio \textbf{bottom-up}. In questo caso si parte dalle
  informazioni, prevalentemente di letteratura, le interazioni tra le componenti
  individuali del sistema, ceracondo magari le concentrazioni o il
  \textit{kinetic-rate}. Tali informazioni potrebbero non essere precise. Da 
  queste si formalizza un modello matematico per 
  avere poi comparazioni tra esperimenti e modelli di simulazione, ottenendo
  alla fine il \textbf{modello predittivo del sistema}. Questo approccio soffre
  quindi la mancanza di dati, specialmente di dati quantitativi. Questo
  approccio è più vicino a quello tipico della biologia, avvicinandosi per
  alcuni aspetti al \textit{pensiero riduzionista} (che mira a studiare piccole
  componenti del sistema).\\
  Tale approccio è sicuramente più complesso, per quanto si possa limitare a
  studiare pathway e non l'intero metaboloma, ma per questo anche più
  informativo. 
\end{enumerate}
Ovviamente tali approcci, per quanto sarebbe fantastico, non possono essere
usati in contemporanea. Detto questo solitamente l'approccio top-down studia i
sistemi su larga scala per poi, a volte, procedere con uno studio bottom-up. In
generale comunque la scelta dipende dalla singola situazione. Non esiste un
meglio o un peggio, anche se i modelli generati dall'approccio top-down hanno
generalmente una minor capacità predittiva anche se studiano sistemi più ampi
rispetto all'approccio bottom-up.\\
Bisogna distinguere quindi quali siano le tecniche tipiche della bioinformatica
(ma anche della statistica)
e quali quelle della \textit{systems biology}. L'uso di tecniche per la ricerca
di similarità, correlazioni, causalità probabilistica, clustering (dove si noti
che non ha un ruolo significativo il \textbf{tempo}) etc$\ldots$
non sono di interesse della \textit{systems biology}, che invece è interessata
allo studio delle causalità in cui il \textit{tempo} è intrinseco e
necessario. Questa necessità di avere il \textit{tempo} comporta una maggior
difficoltà nel recuperare i dati e dell'eseguire la sperimentazione ma comporta,
del resto, un forte ``potere di spiegazione e predizione'' da parte del modello
stesso. \\
Vediamo ora qualche definizione di base.
\begin{definizione}
  Definiamo \textbf{modello} come una descrizione rigorosa e assolutamente non
  ambigua di un sistema. Nel dettaglio tale descrizione è ottenuta tramite un
  adeguato formalismo matematico (l'unico per definizione non ambiguo) e un
  adeguato livello di astrazione (importante per non avere informazioni
  ridondanti o inutili nel modello). 
\end{definizione}
\begin{definizione}
  Definiamo \textbf{proprietà/comportamento emergente} ogni
  caratteristica strutturale (quindi di topologia) o dinamica (quindi in
  evoluzione nel tempo) di un sistema che non può essere capita e/o spiegata
  banalmente tramite l'enumerazione delle componenti ma che deve essere derivata
  unicamente come conseguenza tra le componenti stesse del sistema.
\end{definizione}
\begin{definizione}
  Definiamo \textbf{simulazione} come una tecnica ``computer-based'' per
  determinare una qualsiasi caratteristica emergente e/o predire l'evoluzione
  temporale del sistema. 
\end{definizione}
\begin{definizione}
  Definiamo \textbf{metodo computazionale} come una soluzione automatica, basata
  su uno specifico algoritmo, usata per risolvere problemi difficili (da
  intendersi ``difficili'' anche a livello computazionale) e per
  analizzare sistemi in diverse condizioni.
\end{definizione}
Si noti che, come evidenziato da Fawcett e Higginson\footnote{Tim W. Fawcett
  and Andrew D. Higginson, Heavy use of equations impedes communication among
  biologists, PNAS 2012}, l'uso eccessivo dei formalismi matematici rendono
difficile la comunicazione con i biologi, quindi bisogna muoversi di
conseguenza. I modellatori dovrebbero essere preparati a sviluppare nuovi
strumenti matematici e computazionali, invece di ``forzare'' la descrizione e
l'analisi del sistema con un framework preferito e facilmente applicabile (tipo
usare le EDO per tutto a priori). U biologi sperimentali dovrebbero essere aperti
a progettare nuovi protocolli di laboratorio per identificare tutte le
caratteristiche qualitative e, soprattutto, quantitative che ancora mancano (per
aiutare anche i modellisti). \textbf{La parte più interessante del gioco del
  modellismo non è ciò che il modello permette di capire, ma esattamente ciò che
  non è in grado di spiegare}, infatti, secondo, Box:
\begin{center}
  \textit{essentially, all models are wrong, but some are useful.}
\end{center}
e, secondo Bower e Bolouri:
\begin{center}
  \textit{In fact, all modelers should be prepared to answer the question:
    ``what do you know that you did not know before?'' If the answer is ``that i
    was correct'', it is best to look elsewhere.}
\end{center}
Infatti un modello non solo deve rispondere a quello che già si sa ma deve
predire qualcosa che ancora non si sa (magari anche non funzionando).
\section{PCNA ubiquitylation}
Vediamo brevemente uno studio in cui ha partecipato anche la professoressa
Besozzi dove il non funzionamento del modello ha portato ad una nuova scoperta
scientifica \footnote{
  Flavio Amara, Riccardo Colombo, Paolo Cazzaniga, Dario Pescini, Attila
  Csikász-Nagy, Marco Muzi Falconi, Daniela Besozzi, Paolo Plevani , In vivo and
  in silico analysis of PCNA ubiquitylation in the activation of the 
  Post Replication Repair pathway in S. cerevisiae, BMC 2013 }.\\
In questo studio si cercava di studiare la \textbf{Post Replication Repair
  (\textit{PRR})}, ovvero il principale pathway di tolleranza al danno del DNA
che bypassa le lesioni del DNA durante la \textit{fase S}, che è in citologia
(la branca della biologia che studia la cellula dal punto di vista morfologico e
funzionale) una fase del ciclo cellulare, durante la quale il processo
principale è la sintesi e duplicazione del materiale genetico contenuto nel
DNA. Bombardando il lievito con raggi UV si è quindi studiata la proteina
\textbf{PCNA}, ovvero l'\textit{l'antigene nucleare di proliferazione
  cellulare}. La struttura di tale proteina (di forma a ciambella) è in grado di
assumere una peculiare conformazione la quale le consente di contattare il DNA
(DNA clamp) e di promuovere l'azione della polimerasi durante la replicazione
del DNA \footnote{\url{https://it.wikipedia.org/wiki/PCNA}}. I raggi UV
provocano lesioni che vengono ``trattate'' dalla PCNA. Se ne è quindi
studiata l'\textbf{ubiquitazione}, modificazione post-traduzionale di una
proteina dovuta al legame covalente di uno o più monomeri di ubiquitina. Tale
legame porta, solitamente, alla degradazione della proteina
stessa\footnote{\url{https://it.wikipedia.org/wiki/Ubiquitina}}. La
\textit{mono-ubiquitazione} avviene tramite gli enzimi \textit{Rad6}
\textit{Rad8} mentre la \textit{poli-ubiquitazione} tramite gli enzimi
\textit{Rad5} e \textit{Ubc13-Mms2}. La prima comporta errori di trascrizione,
in quanto si aveva sintesi di DNA tra le lesioni, formando \textit{mutageni},
mentre la seconda è ``error free''.\\
Si conoscevano quindi i principali attori del fenomeno, ovvero la proteina e gli
enzimi. C'erano varie cose che però non si conoscevano:
\begin{itemize}
  \item l'ordine spazio temporale della cascata delle interazioni delle varie
  proteine, non sapendo anche i tempi di attivazione dei vari enzimi
  \item se il numero di lesioni influenzasse il bilanciamento tra le
  \textit{mono-ubiquitazioni} e le \textit{poli-ubiquitazioni}
  \item se esistesse una soglia relativa al danno che regolasse l'interazione
  tra i due sub-pathway
\end{itemize}
Si è quindi proceduto, in \textit{wet lab}, irradiando il lievito in modo
controllato, misurando \textit{mono-ubiquitazioni} e le
\textit{poli-ubiquitazioni} al passare del tempo (da 0 a 300 minuti) a varie
dosi di UV, e contemporaneamente studiando un modello matematico (tramite le
varie reazioni, rappresentate tramite \textit{prodotti} e \textit{reagenti}) per
effettuare le simulazioni. Si è visto, in laboratorio, che 
le varie forme ubiquitilate di PCNA sono assenti a basse dosi di UV
($5\frac{J}{m^2}$ e $10\frac{J}{m^2}$), mentre ad alte dosi di UV
($50\frac{J}{m^2}$ e $75\frac{J}{m^2}$) entrambi i segnali sono ancora presenti
dopo 5 ore nei \textit{western blot}. La simulazione matematica confermava
quanto stesse succedendo a 
bassi dosaggi ma non riusciva ad ottenere i risultati ad alti dosaggi. Dopo vari
tentativi, rifacendo gli esperimenti (variando enzimi e geni) e sistemando il
modello (tramite \textit{parameter sweeping/estimation, analisi di sensitività})
si è sospettato che il modello fosse in realtà ``corretto'' ma non 
completo, mancava qualche ipotesi. Da qui la scoperta: si ha anche un altro
pathway, il \textbf{Nucleotide Excision Repair (\textit{NER})} che ``assiste''
la \textit{PCNA} quando le cellule sono gravemente lesionate. NER è infatti
attivo nella \textit{fase S} e serve alla \textit{PRR} per funzionare
correttamente \textit{in vivo}. Risistemando il modello con \textit{NER} ed
enzimi annessi le simulazioni hanno funzionato. \\
Questa è la prova che quando un modello non funziona si può ottenere anche una
scoperta scientifica, ed è una delle situazioni (coi giusti limiti) più
interessanti di questa branca di ricerca.
\section{I Sistemi Complessi}
In \textit{systems biology} si ha quindi a che fare con sistemi che vengono
definiti \textbf{sistemi complessi}, ovviamente presi nella loro ``sottoclasse''
relativa ai sistemi biologici.
\begin{definizione}
  Si definisce un \textbf{sistema complesso} con un sistema consistente di un
  certo numero di componenti più o meno semplici che, prese nel loro insieme,
  danno vita ad un \textit{comportamento emergente}, grazie alle loro
  \textit{mutue interazioni}. 
\end{definizione}
In questo contesto assumono importanza tre concetti chiave:
\begin{enumerate}
  \item \textbf{comportamento non lineare}, quindi non facilmente prevedibile
  \item \textbf{sistema aperto}, ovvero dove l'interazione con l'ambiente da
  parete del sistema è una delle caratteristiche da studiare e modellare
  \item \textbf{sistema dinamico}, ovvero si ha che il sistema evolve nel tempo 
\end{enumerate}
Uno dei punti cruciali è inoltre capire che quando si procede alla modellazione
di un certo sistema non si deve modellare anche cosa ci si aspetta da quel
modello. Tale informazione infatti deve scaturire dalle simulazioni del modello
stesso in modo completamente autonomo.\\
Come visto si studiano quindi insiemi di componenti. L'insieme complessivo delle
funzionalità del sistema non è determinato però da una specifica funzione di
ogni componente ma dalle loro interazioni. Si hanno quindi altri due concetti
chiave:
\begin{enumerate}
  \item \textbf{topologia/architettura interna}
  \item \textbf{moduli funzionali}
\end{enumerate}
Anche componenti molto semplici possono dare vita a un sistema complesso.\\ 
Vediamo quindi qualche esempio di sistema complesso:
\begin{itemize}
  \item un esempio ``semplice'' è quello di una \textbf{reazione enzimatica con
    feedback negativo}. In questo caso si hanno una serie di \textit{reazioni
    lineari} che dal legame di un \textit{substrato} portano ad un
  \textit{prodotto}. La complessità viene data dal \textit{feedback negativo} in
  quanto la produzione del prodotto stesso porta il substrato a non  legare. Si
  ha quindi la cosiddetta \textbf{autoregolazione} che rende questo un vero e
  proprio \textit{sistema complesso}, avendo che il comportamento emergente del
  sistema è in realtà difficile da prevedere.\\
  Ovviamente si potrebbe anche assumere il caso meno semplice dove si ha una
  serie di \textit{reazioni non lineare}. \\
  Si può quindi arrivare anche a parlare di casi più ``estesi'', come quello ad
  esempio di un \textbf{pathway metabolico}. Ci sarebbe inoltre un caso,
  seguendo questo filo pensiero, ancora più estremo, ovvero quello del
  \textbf{metabolismo di un'intera cellula}, come visualizzabile nella figura
  \ref{fig:meta}, dove si hanno moltissime parti che 
  nel dettaglio sono lineari ma che si autoregolano a vicenda, ottenendo quindi
  un \textit{sistema complesso} davvero impossibile da studiare.
  \begin{figure}
    \centering
    \includegraphics[width = \textwidth]{img/metabolome.jpg}
    \caption{Insieme dei pathway che ``compongono'' il metabolismo di un'intera
      cellula. Tale rappresentazione è stata fatta da Donald E. Nelson,
      dell'università di Leeds e dall'azienda Sigma-Aldrich per la International
      Union of Biochemistry and Molecular Biology del 2000.}
    \label{fig:meta}
  \end{figure}
  \item un altro esempio può essere quello di una \textbf{rete di interazioni
    proteina-proteina (\textit{protein-protein interaction network})} dove si
  hanno:  
  \begin{itemize}
    \item \textbf{nodi} che rappresentano le proteine
    \item \textbf{archi} che rappresentano \textbf{interazioni fisiche} e
    \textbf{interazioni funzionali} tra proteine (???)
  \end{itemize}
  In questo caso la ``complessità'' è data soprattutto dal numero
  incredibilmente alto di proteine (quindi di nodi) e di interazioni tra esse
  (quindi di archi) nel nostro sistema. Un esempio è visualizzabile in figura
  \ref{fig:ppn}. 
  \begin{figure}
    \centering
     \includegraphics[scale = 1]{img/ppn.png}
    \caption{Esempio di rete di interazioni
      proteina-proteina,
      \url{https://www.ebi.ac.uk/training/online/courses/network-analysis-of-protein-interaction-data-an-introduction/protein-protein-interaction-networks/}}
     \label{fig:ppn}
  \end{figure}
  \item un altro esempio è quello delle \textbf{reti di regolazione genica
    (\textit{gene regulatory network})} dove appunto si studiano le relazioni
  che si hanno tra le espressioni e le regolazioni tra geni. In questo caso si
  hanno: 
  \begin{itemize}
    \item \textbf{nodi} che rappresentano i geni
    \item \textbf{archi} che rappresentano le regolazioni tra geni
  \end{itemize}
  Anche in questo caso si possono avere feedback e tali reti sono utili per
  studiare l'\textit{over-espressione di geni}. Inoltre deve essere chiaro che
  la modifica in un certo gene si ripercuote, chi più chi meno, sull'intera
  rete anche se non si ha un singolo gene che ``controlla'' l'intera rete ma
  tutti contribuiscono alla funzionalità dell'intero sistema complesso
  \item aumentando ancora il livello di complessità potremmo pensare allo studio
  di un certo pathway, come ad esempio il \textit{segnale di trasduzione}, in
  una cellula tenendo però conto anche della \textit{componente spaziale},
  tridimensionale, della stessa,  
  nonché le interazioni con l'ambiente. La componente spaziale, che ovviamente
  aggiunge complessità, è una parte rilevante del modello, come il ``movimento''
  al suo interno (prestando sempre attenzione a non aggiungere componenti
  inutili al modello stesso). L'interazione con l'ambiente può portare, ad
  esempio, a \textit{cascate di reazioni intra-cellulari} e a vari ``input'',
  come \textit{ormoni, fattori di sopravvivenza, fattori di
    crescita/anti-crescita, fattori di morte etc$\ldots$} da considerare nel
  modello 
  \item un altro esempio ancor più ``complesso'' può essere quello della
  \textbf{crescita tumorale}, magari ponendo al centro dello studio anche il
  rapporto tra essa e la \textbf{vascolarizzazione}, ovvero la distribuzione di
  vasi sanguigni in un tessuto, in quanto magari si vuole studiare la vicinanza
  tra il tumore e i vasi sanguigni. In questo contesto non solo lo spazio
  tridimensionale è di fondamentale importanza ma bisogna anche modellare
  cellule di vario tipo (normali, cancerogene, legate al sistema immunitario, in
  apoptosi, necrotiche
  etc$\ldots$), che interagiscono in modo diverso tra loro, magari avendo anche
  ``mutazioni'' da normali a cancerogene etc$\ldots$ Si hanno quindi componenti
  eterogenee, dovendo per lo più anche modellare i vasi sanguigni e le
  interazioni con le cellule.
  \item un altro esempio, ``complesso'' almeno quanto il precedente, è lo studio
  della \textbf{formazione di biofilm}, ovvero una aggregazione complessa di
  microrganismi contraddistinta dalla secrezione di una matrice adesiva e
  protettiva. Tale barriera è comunque una struttura permeabile permettendo il
  passaggio dei nutrienti. In un biofilm i microrganismi, tendenzialmente batteri, non solo
  crescono ma, soprattutto quelli più interni e ``protetti'', diventano anche
  più resistenti. Questa è una seria complicazione per la loro eliminazione
  quando fuoriescono dal biofilm. Anche qui quindi bisogna modellare lo spazio
  tridimensionale, l'interazione con l'ambiente, l'interazione tra i vari
  microrganismi (anche se si ha solitamente poca eterogeneità)
  \item cambiando prospettiva un altro esempio di \textit{sistema complesso} è
  quello dello \textbf{sviluppo embrionale e della differenziazione cellulare}
  dove, a partire dall'embrione e da cellule staminali si vanno a formare tutti
  i tipi di cellule che formeranno, ad esempio, i tessuti, gli organi
  etc$\ldots$ dell'uomo. In questo caso solitamente si ha un tipo di
  modellazione diverso, basato su componenti semplici 
  \item un altro esempio è quello dello studio dell'\textbf{ecosistema}. Nel
  dettaglio uno degli aspetti studiati è quello della cosiddetta
  \textbf{dinamica preda-predatore}. Tale dinamica descrive il rapporto tra il
  numero di prede e di predatori e osserva un comportamento oscillatorio (se
  aumentano i predatori diminuiscono le prede fino a che non sono abbastanza per
  i predatori, che calano di numero, portando il numero di prede a crescere
  etc$\ldots$) 
  \item infine un ultimo esempio, molto attuale, di \textit{sistema complesso} è
  quello dello \textbf{studio epidemiologico della diffusione di
    epidemie/pandemie} dove la ``complessità'' è incrementata anche dagli
  aspetti sociali e psicologici delle persone, nonché dalla loro eterogeneità
  anche nel dominio epidemiologico (infetti, gravemente infetti, guariti,
  esposti, suscettibili all'infezione, morti etc$\ldots$)
\end{itemize}
In questi esempi si è spesso parlato più o meno esplicitamente di
\textbf{livelli di complessità}. Per poter avere un'idea di quanti possano
essere bisogna considerare vari punti di vista:
\begin{itemize}
  \item un primo punto di vista è dato dalla \textbf{scala spaziale} dei
  fenomeni che si studiano. Possiamo studiare infatti eventi che accadono nel
  range dei nanometri, o meno, fino a pensare ad eventi in scala umana, in
  metri. Inoltre 
  anche un evento che avviene in nanometri può avere conseguenze visibili in
  metri. Questo tipo di complessità è per lo più un problema matematico dal
  punto di vista della gestione della stessa
  \item un secondo punto di vista è dato dalla \textbf{scala temporale} dei
  fenomeni che si studiano. Anche in questo caso si passa dai nanosecondi, o
  meno, ai miliardi di anni. Un evento quasi istantaneo può avere conseguenze
  evolutive tra milioni di anni. La gestione di questo tipo di complessità è un
  grande problema dal punto di vista computazionale. La complessità aumenta
  all'aumentare della scala temporale
  \item altri livelli di complessità sono dati dai \textit{livelli di funzione
    di un organismo}, avendo, ad esempio, che da \textit{trascrittoma, proteoma}
  e \textit{metaboloma}, in ottica pathway, si passa al \textit{fisioma}, in
  ottica cellule, tessuti, organi e direttamente l'uomo
\end{itemize}
Pensando anche solo alla scala spaziale e quella temporale si ha inoltre che
esse sono in sinergia ma è comunque pressoché impossibile pensare ad un modello
che tenga traccia in modo completo o quasi di entrambe queste scale.
\section{Rappresentazione Grafica}
Ai biologi/biotecnologi etc$\ldots$ piace fare diagrammi e mappe concettuali per
rappresentare graficamente le conoscenze biologiche che hanno su un sistema, ad
esempio componenti molecolari e le loro mutue relazioni, formazione di complessi
molecolari, presenza di feedback di regolazione positivi/negativi
etc$\ldots$. Come si intuisce facilmente diagrammi di questo tipo sono soggetti
ad interpretazioni ambigue e limitano anche l'esplicita rappresentazione della
conoscenza biologica. La matematica è l'unico linguaggio non ambiguo e
fortunatamente esistono anche formalismi, come i \textit{grafi}, le \textit{reti
  di Petri} etc$\ldots$ che non solo sono formalmente rigorosi ma hanno anche
un'interpretazione grafica (tanto amata dalle persone). Ovviamente non sempre si
hanno queste soluzioni intermedie. La modellazione
matematica risolve ogni errata interpretazione e descrive in modo non ambiguo
quello che accade nel sistema e può potenzialmente includere ogni tipo di
ipotesi che può poi essere studiata e testata in \textit{wet lab}. In ogni caso
i diagrammi possono avere utilità nella fase preliminare di discussione tra il
biologo e il modellista: può essere un buon punto di partenza ma non sarà mai
sufficiente per modellare il sistema, che si può ottenere solo con la
formalizzazione matematica di componenti e interazioni. Un esempio è
visualizzabile in figura \ref{fig:dia}\footnote{Besozzi D. (2016) Reaction-Based
  Models of Biochemical Networks. In: Beckmann A., Bienvenu L., Jonoska N. (eds)
  Pursuit of the Universal. CiE 2016. Lecture Notes in Computer Science, vol
  9709. Springer, Cham. https://doi.org/10.1007/978-3-319-40189-8\_3}. 
\begin{figure}
  \centering
  \includegraphics[scale = 0.4]{img/dia.jpg}
  \caption{Esempio (senza entrare nei dettagli biologici che sarebbero ora
    superflui) di un diagramma ambiguo, in figura \textit{A}, tipico 
    dell'approccio biologico. Si hanno poi successive migliorie formali fino ad
    arrivare al modello matematico preciso, in figura \textit{D}, e non ambiguo
    formato da 10 reazioni biochimiche.} 
  \label{fig:dia}
\end{figure}
\section{Tipologie di Modelli}
Sistemi biologici differenti necessitano di approcci modellistici differenti,
ovvero di framework matematici, quindi ad un preciso formalismo, e
conseguentemente computazionali diversi. Inoltre bisogna sempre tenere in
considerazione che ogni metodo computazionale legato ad un preciso modello può
rispondere solo a certe tipologie di domande. Non si ha però una corrispondenza
biunivoca tra ogni approccio modellistico e ogni sistema biologico, infatti
diversi modelli potrebbero prestarsi bene ad un certo sistema biologico (anche
se alcuni modelli sono inapplicabili per certi sistemi biologici o per certe
domande su tali sistemi). La scelta del modello è quindi fortemente legata alle
entità che si vogliono rappresentare e alle risposte che si vogliono ottenere
dal modello. \textbf{Non si ha una strategia universalmente valida per scegliere
il miglior approccio modellistico in base al sistema biologico d'interesse.}\\
Il primo passo è quindi l'interazione tra il biologo/biotecnologo e il
modellista. Il primo deve porsi varie domande tra cui:
\begin{itemize}
  \item cosa si sa e cosa non si sa del sistema biologico in questione?
  \item che tipologie di dato di laboratorio sono disponibili?
  \item che tipologie di dato posso misurare effettivamente i9n laboratorio? 
\end{itemize}
Anche il modellista quindi si deve porre delle domande fondamentali, tra cui:
\begin{itemize}
  \item quale formalismo matematico si presta meglio per questo problema?
  \item che strumenti computazionali sono necessari?
  \item che tipo di predizioni mi aspetto dal modello?
\end{itemize}
Queste questioni sono ``in ciclo'' tra di esse e sono la base degli studi in
\textit{systems biology}, dove farsi domande è una parte fondamentale. In merito
all'ultima domanda del biologo è 
interessante notare che un modello \textbf{deve} essere validato in
laboratorio. Qualora non sia possibile, ad esempio un ``caso limite'' emerso
dallo studio del modello, non si può fare nulla (anche se, qualora
si avessero più modelli completamente distinti che portano allo stesso risultato
si può presupporre che ci sia un fondo di verità). \\
La domanda fondamentale resta però:
\begin{center}
  \textit{qual è la questione scientifica? Perché mi serve un modello?}
\end{center}
La risposta a questa domanda deve essere ``sicura'' prima di intraprendere uno
studio di modellazione.\\
Nel dettaglio, durante il corso, si vedranno i quattro approcci modellistici
tradizionali più usati anche se si tratta di una selezione tra la moltitudine
degli approcci presenti:
\begin{enumerate}
  \item \textbf{modelli basati su interazioni (\textit{interaction-based
      models})}
  \item \textbf{modelli basati su vincoli (\textit{constraint-based
      models})}
  \item \textbf{modelli logici (\textit{logici-based models})}
  \item \textbf{modelli meccanicistici (\textit{mechanism-based models})}
\end{enumerate}
Un generale dato un certo sistema biologico d'interesse, dopo aver risposto alla
domanda fondamentale e avendo quindi ben chiaro il fine di tale modello, la
scelta del modello stesso viene presa considerando secondo quattro aspetti
fondamentali:
\begin{enumerate}
  \item la \textbf{dimensione del sistema}, data in primis dal numero di
  componenti e dal numero delle interazioni tra esse. Si distinguono, secondo
  questo aspetto, due grandi macro-categorie di sistemi:
  \begin{enumerate}
    \item \textbf{sistemi small-scale}, se siamo nel range delle unità o delle
    decine di componenti/interazioni
    \item \textbf{sistemi large-scale}, se siamo nel range delle centinaia o
    migliaia (se non oltre) di componenti/interazioni 
  \end{enumerate}
  Questo è già un ottimo fattore discriminatorio per la scelta del sistema 
  \item il \textbf{livello di dettaglio} necessario a descrivere in modo
  completo le componenti del sistema e le loro interazioni. Si ricorda sempre
  però che formalizzare informazioni inutili e/o ridondanti comporta solo
  un'inutile spreco dal punto di vista computazionale
  \item il \textbf{tipo} e la \textbf{qualità} dei \textbf{dati sperimentali}
  che sono già disponibili o che si è in grado di produrre all'evenienza con
  precisi protocolli al fine di supportare la fase di modellazione. Tali dati
  possono essere ad esempio \textit{dati omici, western blots} etc$\ldots$
  \item il \textbf{carico computazionale} che l'approccio scelto comporta in
  fase di simulazione e analisi dei dati. Un esempio è quello della
  \textit{dinamica molecolare}, che studia come interagiscono tra loro più
  molecole (o anche il comportamento interno di una sola). Tali studi
  normalmente impiegano settimane per simulare anche range temporali molto
  ridotti e necessitano di molte informazioni che non possono essere trascurate
  per ottenere un modello ed una simulazione realistici. Questa scelta è spesso
  un trade-off nella scelta di \textbf{approcci modellistici quantitativi} e
  \textbf{approcci modellistici qualitativi}.\\
  L'uso di \textbf{super computer}, di \textbf{tecniche di calcolo parallelo su
    GPU} etc$\ldots$ sono molto comuni in \textit{systems biology}
\end{enumerate}
Una misura fondamentale è poi la \textbf{capacità predittiva del modello} che,
se bassa, ci porta a preferire \textit{modelli qualitativi}, se alta invece a
\textit{modelli quantitativi}.\\
Vediamo quindi un comodo grafico che classifica le quattro tipologie di modelli
in base a questi quattro aspetti\footnote{Besozzi D. (2016) Reaction-Based
  Models of Biochemical Networks. In: Beckmann A., Bienvenu L., Jonoska N. (eds)
  Pursuit of the Universal. CiE 2016. Lecture Notes in Computer Science, vol
  9709. Springer, Cham. https://doi.org/10.1007/978-3-319-40189-8\_3}:
\begin{figure}[H]
  \centering
  \includegraphics[width = \textwidth]{img/choice.pdf}
\end{figure}
Nel grafico notiamo come i \textit{modelli meccanicistici}, tra quelli
analizzati, siano 
quelli con il più alto potere predittivo, che è un aspetto fondamentale ma anche
con i più alti livelli di 
dettaglio, costi computazionali e sfide nella misurazione dei dati
richiesti. L'ovvia conseguenza è che la dimensione del sistema da studiare deve
essere ridotta, avendo quindi \textit{sistemi small-scale}.\\
Si noti che collocare i \textit{modelli logici} non sia così banale in quanto il
livello di dettaglio e la facilità di misurazione possono essere migliorati
usando ad esempio le \textbf{logiche fuzzy}.\\
Un altro aspetto fondamentale da tenere in considerazione è che il processo di
modellazione richiede molto tempo e si basa su continui rifinimenti del modello
stesso, in un processo circolare, come visualizzabile in figura
\ref{fig:cyc}\footnote{Chou IC, Voit EO. Recent developments in parameter
  estimation and structure identification of biochemical and genomic
  systems. Math
  Biosci. 2009;219(2):57-83. doi:10.1016/j.mbs.2009.03.002}. Quindi partendo dai
dati biologici si abbozza un primo modello che viene poi continuamente sistemato
tramite nuove ipotesi, altri dati, analisi \textit{in silico} etc$\ldots$ fino
all'ottenimento di un modello validato.
\begin{figure}
  \centering
  \includegraphics[scale = 0.8]{img/cycle.jpg}
  \caption{Raffigurazione che mostra i dettagli del processo ciclico di
    modellazione in \textit{systems biology}. Molte, ma non tutte, delle keyword
    presenti verranno approfondite nel corso}
  \label{fig:cyc}
\end{figure}
Vediamo ora una breve introduzione ai quattro approcci elencati al fine di poter
fare un confronto tra essi prima di studiarli e formalizzarli nel dettaglio.\\
Prima di fare ciò diamo una più precisa idea dei criteri con cui si classifica
un modello.
\begin{definizione}
  Si definisce \textbf{modello qualitativo} un modello che specifica le
  interazioni tra le componenti del modello stesso.
\end{definizione}
\begin{definizione}
  Si definisce \textbf{modello quantitativo} un modello che assegna un valore ad
  ogni elemento che descrive e anche alle interazioni tra essi. In questo caso
  si possono avere o non avere cambiamenti nel modello.
\end{definizione}
\begin{definizione}
  Si definisce \textbf{modello deterministico} un modello per il quale
  l'evoluzione attraverso i vari stati può essere predetta a partire dallo stato
  corrente, nel dettaglio anche dallo \textit{stato iniziale}. Il comportamento
  evoluzionistico del modello quindi non varierà tra una simulazione e l'altra.
\end{definizione}
\begin{definizione}
  Si definisce \textbf{modello stocastico} un modello che descrive, a partire da
  uno stato corrente, uno stato futuro attraverso una \textit{distribuzione di
    probabilità}.
\end{definizione}
\begin{definizione}
  Un processo è detto \textbf{reversibile/irreversibile} se si può o meno
  procedere in avanti o indietro tra i vari stati.
\end{definizione}
\begin{definizione}
  Con il termine \textbf{periodicità} si intende che il sistema assume una serie
  di stati nell'intervallo di tempo $[t, t+\Delta t]$ ma anche in:
  \[[t+i \Delta t, t+(i+1)\Delta t],\,\,i=1,2,3,\ldots\]
\end{definizione}
\subsection{Modelli Basati su Interazioni}
Questo tipo di modelli vengono usati per \textit{sistemi large-scale} con
centinaia o migliaia di componenti che interagiscono tra loro in modo fisico o
funzionale. Abbiamo vari esempi di questi modelli, tra cui:
\begin{itemize}
  \item \textbf{reti di interazioni proteina-proteina}
  \item \textbf{reti di regolazione genica}
  \item \textbf{reti metaboliche}
  \item \textbf{reti di malattie}, modelli più complessi, modellati tramite un
  particolare tipo di grafo, che sfruttano
  l'integrazione tra \textit{reti di regolazione genica} e grafi/reti
  rappresentanti le malattie e le relazioni tra esse. Si ottiene quindi un grafo
  che mette in relazione componenti genomiche e malattie
\end{itemize}
In questo caso la scelta del formalismo matematico ricade principalmente sulla
\textbf{teoria dei grafi} e si ha quindi un \textit{modello qualitativo} e
\textit{statico}. Infatti il \textit{tempo} non viene considerato in tali
modelli, che di conseguenza non permettono di ottenere informazioni su
eventuali \textbf{comportamenti emergenti}. Non si possono nemmeno ottenere
informazioni quantitative.\\
Parlando di \textit{modelli basati su interazioni} non si può propriamente
parlare di ``simulazioni'' vere e proprie in quanto in primis manca la
modellazione del \textit{tempo} ma anche di altri fattori come il
\textit{kinetic rate}. Inoltre tali modelli difettano anche di una qualsivoglia
modellazione dello \textit{spazio}. Il fulcro dello studio di tali modelli
quindi solitamente si concerta sulle proprietà ``architetturali'' della
struttura della rete, studiando, ad esempio:
\begin{itemize}
  \item la presenza di \textbf{hub}, ovvero nodi in cui sono entranti/uscenti un
  gran numero di archi rispetto agli altri nodi della rete
  \item misure di centralità
  \item presenza di \textit{motivi (motifs)} nella rete
  \item la robustezza topologica
\end{itemize}
Tutte queste misure permettono anche di caratterizzare, caratterizzando la
topologia stessa, una rete rispetto ad un
altra. Infatti si vedranno vari tipologie di rete, tra cui:
\begin{itemize}
  \item \textbf{random network}
  \item \textbf{scale-free network}, caratterizzate da una forte
  \textit{robustezza}
  \item \textbf{hierarchical network}
\end{itemize}
\subsection{Modelli Logici}
Questi modelli possono essere usati sia per sistemi \textit{small-scale} che per
sistemi \textit{large-scale} e alcuni degli esempi sono:
\begin{itemize}
  \item \textbf{reti di regolazione gene-gene}
  \item \textbf{pathway di trasduzione del segnale} (che si ricorda essere la
  capacità di una cellula di convertire uno stimolo esterno in una particolare
  risposta cellulare) 
  \item \textbf{differenziazione cellulare}
  \item \textbf{pathway per la morte cellulare programmata}
\end{itemize}
Il primo caso è un esempio di \textit{sistema large-scale} mentre gli altri di
\textit{sistemi small-scale}.\\
Dal punto di vista del formalismo matematico si ha anche qui la \textbf{teoria
  dei grafi}, a cui viene aggiunta la \textbf{logica booleana}, con i classici
operatori logici $\neg,\land,\lor$, o anche,
preferibilmente, la \textbf{logica fuzzy}, che verrà approfondita più avanti. L'idea di base è quella che lo stato
delle componenti è regolato da altre componenti del sistema stesso. I nodi
possono assumere o valore booleano 0/1 o, in logica fuzzy, qualsiasi valore tra
0 e 1 (con varie conseguenze nel loro studio).\\
Tali modelli sono in grado di simulare il tempo, rientrando quindi nella
categoria dei \textit{sistemi dinamici} ma sono anch'essi della tipologia dei
\textit{modelli qualitativi}. Tali sistemi si prestano ad essere sia
\textit{deterministici} che \textit{non deterministici}.\\
Lo studio di tali modelli solitamente consiste, tramite le simulazioni e le
analisi, nel determinare:
\begin{itemize}
  \item \textbf{cicli}, ovvero sequenze finite di stati complessivi del sistema
  che si ripetono
  \item \textbf{attrattori}, ovvero degli \textit{stati finali} che sono
  raggiungibili da qualsiasi stato iniziale e una volta raggiunti sio resta in
  tali stati
  \item \textbf{bacini di attrattori}, ovvero percorsi che partono da stati
  intermedi e che conducono a degli \textit{attrattori}
\end{itemize}
Tendenzialmente si arriva sempre ad un \textit{ciclo} o ad un insieme di
\textit{attrattori}.\\
La ``potenza'' della \textit{logica fuzzy} permette, come detto, anche di
modellare il \textit{tempo}, derivando quindi un comportamento dinamico del
sistema, descrivendo, ad esempio, la variazione nel tempo tra i valori degli
stati di ogni componente.\\
Un esempio semplice di quello che si può ottenere con tali modelli è
visualizzabile in figura \ref{fig:log}\footnote{Wynn ML, Consul N, Merajver SD,
  Schnell S. Logic-based models in systems biology: a predictive and
  parameter-free network analysis method. Integr Biol
  (Camb). 2012;4(11):1323-1337. doi:10.1039/c2ib20193c}. 
\begin{figure}
  \centering
  \includegraphics[scale = 0.65]{img/logicnet.jpg}
  \caption{Esempio di modellazione di interazioni tra componenti tramite
    funzioni logiche.}
  \label{fig:log}
\end{figure}
\subsection{Modelli Meccanicistici}
Come già anticipato tali modelli si limitano a descrivere \textit{sistemi
  small-scale}. Questa è la classe di approcci modellistici più complessa ed
eterogenea infatti, in primis, richiede una parametrizzazione completa delle
componenti, con un ampio range di formalismi matematici, tra cui spiccano tra
gli altri i \textbf{metodi numerici} e gli \textbf{algoritmi di simulazione
  stocasitca}. Un problema, già solo 
a questo punto, è che non si hanno spesso i dati per effettuare la
parametrizzazione in quanto i biologi/biotecnologi spesso non sono interessati a
misurarli. \\
Le simulazioni con questi modelli sono usate per studiare l'\textbf{evoluzione
  nel tempo}, quindi la dinamica, del sistema. Si usano metodi
\textit{deterministici, stocastici} e \textit{ibridi}, insieme ad una serie
infinita di altre tecniche computazionali, tra cui l'\textit{analisi di
  sensitività} o il \textit{parameters sweeping}.\\
Si arriva quindi a modelli \textit{quantitativi} e \textit{dinamici}. \\
La scelta tra metodi stocastici, solitamente più dispendiosi, e deterministici
dipende anche dal fatto che \textbf{la vita non è deterministica}. Ad esempio
modellare le interazioni tra, ad esempio tra la proteina \textit{Mdm3} e la
proteina \textit{p53}, avendo che la prima inibisce la seconda, comporta una
funzione molto pulita se studiata in modo deterministico quando in realtà, a
causa di molti fattori, non si ha tale precisione se si va a studiare cosa
accade realmente in natura. Da qui l'uso anche di \textit{modelli stocastici}.\\
La stima dei parametri resta comunque un grandissimo problema e spesso si usano
altre tecniche computazionali/modellistiche in pipeline per inferire gli stessi.
\subsection{Modelli Basati su Vincoli}
Tali modelli sono usati esplicitamente e solo per \textit{sistemi large-scale
  per reti metaboliche}. Il formalismo matematico qui usato si compone di
\textbf{matrici stechiometriche}, \textbf{algebra lineare} e tecniche di
\textbf{ricerca operativa} mentre le
simulazioni e le analisi consistono nello studiare le variazioni nelle
\textbf{distribuzione di flusso}, calcolando i valori di flusso di tutte le
reazioni metaboliche, a seconda di perturbazioni/input prefissati.\\
Non è semplice se si può dire di ottenere dei \textit{sistemi quantitativi} in
quanto si studia il comportamento ad uno \textit{steady state}.\\
Tra gli esempi di uso si hanno:
\begin{itemize}
  \item l'\textbf{ingegnerizzazione metabolica}, ovvero l'ottimizzazione, il
  design e la regolarizzazione di certe strutture/funzioni metaboliche al fine
  di ottenere un certo fenotipo metabolico
  \item studiare \textbf{bersagli di farmaci}, attraverso ad esempio lo studio
  del \textit{rewiring metabolico del cancro}
\end{itemize}
L'idea è quindi quella di:
\begin{itemize}
  \item stabilire dei \textbf{vincoli}
  \item stabilire una \textbf{funzione obiettivo} da
  \textit{massimizzare/minimizzare}
  \item determinare automaticamente la distribuzione dei flussi
\end{itemize}
Si parla di \textbf{Flux Balance Analysis (\textit{FBA})}.
\subsection{Confronto tra i Vari Approcci}
Viste queste prime piccole premesse sui quattro approcci possiamo fare qualche
piccolo confronto. \\
In primis abbiamo capito come lo studio del \textit{tempo} sia assente del tutto
nei \textit{modelli basati su interazioni} e che sia di dubbio uso nel caso dei
\textit{modelli basati su vincoli} a causa dello \textit{steady state}. Quindi
se si dovesse, ad esempio, studiare il cambio di concertazione di una certa
molecola al variare del \textit{tempo} tali approcci sarebbero da scartare a
priori.\\
Si è anche visto come, in realtà, praticamente solo i \textit{modelli
  meccanicistici} ci offrono uno studio quantitativo, al costo di una
complessità sia formale, che di dati, che computazionale molto alta e
riducendosi a studiare solo sistemi piccoli. Ne segue che:
\begin{center}
  \textit{I sistemi meccanicistici sono l'approccio modellistico migliore per
    comprendere e acquisire nuove intuizioni il funzionamento del sistema.} 
\end{center}
Nella realtà però ``avere tutto'' è un'utopia quindi non si ha un vero e proprio
vincitore in questa ``gara tra approcci modellistici'', ben riassunti nella
figura \ref{fig:app}\footnote{Bordbar, A., Monk, J., King, Z. et
  al. Constraint-based models predict metabolic and associated cellular
  functions. Nat Rev Genet 15, 107–120 (2014). https://doi.org/10.1038/nrg3643},
in quanto al variare del problema, dei dati, e di mille altri fattori potrei
aver motivi validi per preferire un approccio ad un altro.\\
\begin{figure}
  \centering
  \includegraphics[width = \textwidth]{img/apptable.jpg}
  \caption{Schema riassuntivo delle caratteristiche dei vari approcci.}
  \label{fig:app}
\end{figure}
Inoltre, in questa breve introduzione, si è scoperto come ci siano moltissime
\textbf{dicotomie} in \textit{systems biology}:
\begin{itemize}
  \item \textit{top-down} e \textit{bottom-up}
  \item \textit{qualitativo} e \textit{quantitativo}
  \item \textit{statico} e \textit{dinamico}
  \item \textit{deterministico} e \textit{stocastico}
  \item \textit{discreto} e \textit{continuo} (sia in ottica di
  rappresentazione del \textit{tempo} che della numerazione delle componenti)
  \item \textit{omogeneo} e \textit{eterogeneo}
  \item \textit{a singolo volume} e \textit{multicompartamentale}
\end{itemize}
Tutte queste dicotomie rappresentano la complessità degli studi in
\textit{systems biology}.\\
Integrare i vari modelli è per lo più utopia. Fare \textit{data integration} è
già di per se uno scoglio complesso ma si aggiunge anche la difficoltà di
integrare vari formalismi matematici. Non si ha il modello ``perfetto'' ma si
può scegliere bene in base al sistema biologico da studiare, magari integrando
anche qualche (molto pochi) approccio modellistico diverso, come visibile in
figura \ref{fig:pap}\footnote{Gonçalves E, Bucher J, Ryll A, et al. Bridging the
  layers: towards integration of signal transduction, regulation and metabolism
  into mathematical models. Molecular Biosystems. 2013 Jul;9(7):1576-1583. DOI:
  10.1039/c3mb25489e. PMID: 23525368. }. Nel diagramma si segnala il paper
centrale di Karr et al.: \textit{A whole-cell computational model predicts
  phenotype from genotype}\footnote{Karr JR, Sanghvi JC, Macklin DN, et al. A
  whole-cell computational model predicts phenotype from
  genotype. Cell. 2012;150(2):389-401. doi:10.1016/j.cell.2012.05.044} cruciale
nello studio di un approccio misto per modellare il sistema un'intera cellula di
un piccolo batterio.
\begin{figure}
  \centering
  \includegraphics[scale = 0.25]{img/papers.jpg}
  \caption{Diagramma che mostra varie soluzioni modellistiche al variare del
    problema biologico}
  \label{fig:pap}
\end{figure}
\chapter{Interaction-Based Modelling}
\end{document}

% LocalWords:  rewiring