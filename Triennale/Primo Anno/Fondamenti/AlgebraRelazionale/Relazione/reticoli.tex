\subsection{Reticoli}
Un \emph{Reticolo} è un poset $(S,R)$ in cui per ogni $x,y \in S$ esistono
il massimo minorante(indicato con $x \sqcap y$) e il minimo maggiorante(indicato con $x \sqcup y$).
Se un poset $(S,R)$ è un reticolo anche il suo poset duale è un reticolo e se due
reticoli sono isomorfi come poset allora i reticoli sono detti \emph{isomorfi}.

\begin{prop}
    Se $(L_1,\leq)$ e $(L_2,\leq)$ sono reticoli, anche $(L_1 \times L_2,\leq)$ lo è,
    con ordine parziale prodotto
\end{prop}

%Proprietà del Reticolo
\begin{defi}
Sia $(L,\leq)$ un reticolo. Presi comunque $a,b,c \in L$ valgono le seguenti proprietà:
\end{defi}
\begin{enumerate}
    \item $a \leq a \cup b$ e $b \leq a \cup b$
    \item Se $a \leq c$ e $b \leq c$, allora $a \cup b \leq c$
    \item $a \cap b \leq a$ e $a \cap b \leq b$
    \item Se $c \leq a$ e $c \leq b$ allora $c \leq a \cap b$
    \item $a \cup a = a$ (Idempotenza)
    \item $a \cap a = a$ (Idempotenza)
    \item $a \cup b = b \cup a$ (Commutativa)
    \item $a \cap b = b \cap a$ (Commutativa)
    \item $a \cup (b \cup c) = (a \cup b) \cup c$ (Associativa)
    \item $a \cap (b \cap b) = (a \cap b) \cap c$ (Associativa)
    \item $a \cup(a \cap b) = a$ (Assorbimento)
    \item $a \cap (a \cup b) = a$ (Assorbimento)
\end{enumerate}

%Reticolo Completo
\begin{defi}
    Se ogni sottoinsieme di un reticolo possiede un minimo maggiorante e un massimo minorante
    allora il reticolo si definisce \emph{completo}.
\end{defi}

%Reticolo limitato
\begin{defi}
    Un reticolo è definito \emph{limitato} se esiste un minimo e un massimo assoluti.
\end{defi}

%Reticolo distribuitivo
\begin{defi}
    Un reticolo è detto \emph{distribuitivo} se valgono le seguenti proprietà:
\end{defi}
\begin{enumerate}
    \item $a \cap (b \cup c) = (a \cap b) \cup (a \cap c)$
    \item $a \cup (b \cap c) = (a \cup b) \cap (a \cup c)$
\end{enumerate}

%Reticolo Complementato
\begin{defi}
    Sia $(L,\leq)$ un reticolo distribuitivo limitato, con massimo $1$ e minimo $0$,
e sia $a \in L$, allora $a'$ è il \emph{complemento},il quale se esiste è unico,
 di $a$ se è rispettata la seguente proprietà: $a \cup a' = 1$ e $a \cap a' = 0$.
\end{defi}

\begin{defi}
Un reticolo $(L,\leq)$ è detto \emph{complementato} se è limitato e ogni suo elemento ha il complemento.
\end{defi}
