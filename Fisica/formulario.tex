\documentclass[a4paper,12pt, oneside]{book}

%\usepackage{fullpage}
\usepackage[italian]{babel}
\usepackage[utf8]{inputenc}
\usepackage{amssymb}
\usepackage{amsthm}
\usepackage{graphics}
\usepackage{amsfonts}
\usepackage{amsmath}
\usepackage{amstext}
\usepackage{engrec}
\usepackage{rotating}
\usepackage{verbatim}
\usepackage[safe,extra]{tipa}
\usepackage{showkeys}
\usepackage{multirow}
\usepackage{hyperref}
\usepackage{microtype}
\usepackage{enumerate}
\usepackage{braket}
\usepackage{marginnote}
\usepackage{pgfplots}
\usepackage{cancel}
\usepackage{polynom}
\usepackage{booktabs}
\usepackage{enumitem}
\usepackage{framed}
\usepackage{pdfpages}
\usepackage{pgfplots}
\usepackage{fancyhdr}
 \usepackage[usenames,dvipsnames]{pstricks}
 \usepackage{epsfig}
 \usepackage{pst-grad} % For gradients
 \usepackage{pst-plot} % For axes
 \usepackage[space]{grffile} % For spaces in paths
 \usepackage{etoolbox} % For spaces in paths
 \makeatletter % For spaces in paths
 \patchcmd\Gread@eps{\@inputcheck#1 }{\@inputcheck"#1"\relax}{}{}
 \makeatother
\pagestyle{fancy}
\fancyhead[LE,RO]{\slshape \rightmark}
\fancyhead[LO,RE]{\slshape \leftmark}
\fancyfoot[C]{\thepage}


\title{Formulario di Fisica}
\author{UniShare\\\\Davide Cozzi\\\href{https://t.me/dlcgold}{@dlcgold}}
\date{}

\pgfplotsset{compat=1.13}
\begin{document}
\maketitle

\definecolor{shadecolor}{gray}{0.80}

\newtheorem{teorema}{Teorema}
\newtheorem{definizione}{Definizione}
\newtheorem{esempio}{Esempio}
\newtheorem{corollario}{Corollario}
\newtheorem{lemma}{Lemma}
\newtheorem{osservazione}{Osservazione}
\newtheorem{nota}{Nota}
\newtheorem{esercizio}{Esercizio}
\tableofcontents

\renewcommand{\chaptermark}[1]{%
\markboth{\chaptername
\ \thechapter.\ #1}{}}
\renewcommand{\sectionmark}[1]{\markright{\thesection.\ #1}}
\chapter{Introduzione}
\subsection{Trigonometria}
\begin{center}
\begin{pspicture}(0,-2.1457813)(5.3,2.1457813)
\psline[linecolor=black, linewidth=0.04](4.4,-1.6694922)(4.4,1.5305078)
\psline[linecolor=black, linewidth=0.04](4.4,1.5305078)(0.8,-1.6694922)(4.4,-1.6694922)
\rput[bl](4.4,-2.069492){$A$}
\rput[bl](4.4,1.9305078){$C$}
\rput[bl](0.0,-1.6694922){$B$}
\rput[bl](2.0,-0.06949219){$a$}
\rput[bl](4.7,-0.4494922){$b$}
\rput[bl](2.4,-2.069492){$c$}
\rput[bl](1.29,-1.5894922){$\beta$}
\rput[bl](3.8,-1.3894922){$\alpha$}
\rput[bl](4,0.7705078){$\gamma$}
\end{pspicture}
$$b= a \sin\beta$$
$$c=a\sin \gamma$$
$$b=a\cos \gamma$$
$$c=a cos\beta$$
$$c=b\tan \gamma$$
$$b=c\tan \beta$$
quindi su un piano inclinato:
$$l=\frac{h}{\sin\theta}$$
\end{center}
inoltre:
$$\sin^2(\theta)+\cos^2(\theta)=1$$
$$\sin(2\theta)=2\sin(\theta)\cos(\theta)$$
$$\cos(2\theta)=\cos^2(\theta)-\sin^2(\theta)$$
\section{vettori}
prodotto scalare tra vettori: $\overline{x}\cdot \overline{y}=||x||\cdot ||y||\cdot \cos\theta$\\
prodototto  vettoriale tra vettori: $\overline{x}\times \overline{y}=||x||\cdot ||y||\cdot \sin\theta$
\section{Costanti}
\begin{itemize}
\item \textbf{accelerazione di gravità:} $g=9,81\, \frac{m}{s^2}$
\item \textbf{costante gravitazionale:} $6,67 \times 10^{-11}\, \frac{m^3}{s^2kg}$
\item \textbf{raggio terra:} $R_L=6,37\times 10^{6}\, m$
\item \textbf{massa terra:} $M_T=5,96\times 10^{24}\, kg$
\item \textbf{massa sole:} $M_S =1,99\times 10^{30}\, kg$
\item \textbf{massa luna:} $M_L=7.36\times 10^{22}\, kg$
\end{itemize}
\chapter{Meccanica}
\section{Cinematica}
\subsubsection{Moto rettilineo}
\begin{itemize}
\item \textbf{velocità media: }$v_m=\frac{\Delta \vec{x}}{\Delta t}=\frac{x_2-x_1}{t_2-t_1}=\frac{\vec{v_2}-\vec{v_1}}{2}$
\item  \textbf{velocità istantanea: }$v(t)=\frac{d\vec{x}(t)}{dt}$
\item \textbf{equazione del moto rettilineo uniforme: }$x(t)=x_0+v(t-t_0)$
\item \textbf{accelerazione media:} $a_m=\frac{\vec{v_2}-\vec{v_1}}{t_2-t_1}=\frac{\Delta v}{\Delta t}$
\item \textbf{velocità moto uniformemente accelerato:} $v(t)=v_0+at$
\item \textbf{equazione del moto rettilineo uniformemente accelerato:} $$x(t)=x_0+v_0t+\frac{a}{2}t^2$$
\item \textbf{velocità finale moto uniformemente accelerato:}
$$v_{fin}^2=v_0^2+2a\Delta x$$
\end{itemize}
\newpage
\subsubsection{Moto verticale}
\begin{itemize}
\item \textbf{punto ad altezza h lasciato cadere:}
$$\vec{x}(t)=h-\frac{1}{2} g t^2$$
$$\vec{v}(t)=-gt$$
$$t_{caduta}=\sqrt{\frac{2 h}{g}}$$
$$\vec{v}_{suolo}=-\sqrt{2 g h}$$
\item \textbf{punto ad altezza h spinto in basso con una certa velocità verso il basso:}
$$\vec{x}(t)=h-\vec{v}_1t-\frac{1}{2} g t^2$$
$$\vec{v}(t)=-\vec{v}_1-gt$$
$$t_{caduta}=-\frac{\vec{v}_1}{g}+\frac{1}{g}\sqrt{\vec{v}_1^2+2gh}$$
$$v_{suolo}=-\sqrt{\vec{v_1}^2+2gh}$$
\item \textbf{punto ad altezza 0 spinto in alto con una certa velocità:}
$$\vec{x}(t)=\vec{v_2}t-\frac{1}{2} g t^2$$
$$\vec{v}(t)=\vec{v_2}-gt$$
con $v=0$ si ha l'altezza massima:
$$t_{x_{max}}=\frac{\vec{v_2}}{g}$$
e quindi:
$$x(t_{max})=\frac{1}{2}\frac{\vec{v_2}^2}{g}$$
$$t_{caduta}=\frac{\vec{v_2}}{g}$$
$$t_{tot}=t_{max}+t_c=\frac{2\vec{v_2}}{g}$$
\end{itemize}
\begin{comment}
\subsubsection{Moto nel Piano}\textbf{da sistemare}
\begin{itemize}
\item \textbf{modulo della velocità in componenti cartesiane}: $$v=|\vec{v}|=\sqrt{v_x^2+v_y^2}$$
\item \textbf{modulo della velocità in componenti cartesiane}: 
$$v=|\vec{v}|=\sqrt{v_r^2+v_q^2}$$
\item \textbf{accelerazione nel piano: }$\vec{a}=\vec{a}_T+\vec{a}_n$
\end{itemize}
\end{comment}
\subsubsection{Moto Circolare}
\begin{itemize}
\item \textbf{arco:} $l_a=\frac{\Delta s}{R}$
\item \textbf{angolo:} $\theta=\frac{l_a}{R}$
\item \textbf{velocità angolare media nel moto uniforme:} $\omega_m=\frac{\Delta\theta}{\Delta t}$
\item \textbf{velocità angolare istantanea nel moto uniforme:} $\omega=\frac{v}{R}$
\item \textbf{accelerazione centripeta (quella tangenziale è nulla) nel moto uniforme:} 
$$a=\frac{v^2}{R}=\omega^2R=\omega v$$
\item \textbf{equazioni del moto uniforme:}
$$s(t)=s_0+vt$$
$$\theta(t)=\theta_0+\omega t$$
\item \textbf{periodo:}
$$T=\frac{2\pi R}{v}=\frac{2\pi R}{\omega R}=\frac{2\pi}{\omega}$$
\item \textbf{accelerazione nel caso di moto non uniforme:}
$$\vec{a}=\vec{a}_T+\vec{a}_N$$
$$\alpha_{media}=\frac{\Delta\omega}{\Delta t}$$
$$\alpha_{istantanea}=\frac{1}{R}a_T$$
$$a_N=\omega^2 R$$
$$a_T=\alpha R$$
\item \textbf{equazioni del moto circolare non uniforme:}
$$\omega(t)=\omega_0+\alpha t$$
$$\theta(t)=\theta_0+\omega_0t+\frac{1}{2}\alpha t^2$$
$$a_N=\omega^2 R=(\omega_0+\alpha t)^2 R$$
$$|\vec{v}|=\omega R$$
\end{itemize}
\subsubsection{Moto Parabolico}
\begin{itemize}
\item \textbf{moto parabolico da terra, con angolo e velocità iniziale:}
$$
\begin{cases}
v_x=v_0cos\theta_0\\
v_y=v_0sin\theta_0-gt
\end{cases}
$$
$$\begin{cases}
x(t)=(v_0cos\theta_0)t\\
y(t)=(v_0sin\theta_0)t-\frac{1}{2}gt^2
\end{cases}$$
$$t=\frac{x}{v_0cos\theta_0}$$
$$y(x)=(tan\theta_0)x-\frac{g}{2v_0^2cos^2\theta_0}x^2\mbox{ (traiettoria)}$$
$$x_G=\frac{v_0^2}{g}sin(2\theta_0)\mbox{ (gittata, y(x)=0)}$$
$$x_G=\frac{v_0^2}{g}\mbox{ (gitatta massima)}$$
$$x_M=\frac{1}{2}\frac{v_0^2}{g}sin(2\theta_0)\mbox{ (altezza massima)}$$
$$y_M=\frac{v_0^2}{2g}sin^2\theta_0 \mbox{ (altezza massima lungo la traiettoria)}$$
$$Y_{M_{max}}=\frac{v_0^2}{2g}\mbox{ (altezza massima, la verticale)}$$
$$t_{volo}=\frac{2v_0}{g}sin\theta_0$$
$$t_{{volo}_{max}}=\frac{2v_0}{g}$$
$$\begin{cases}
v_x(t_G)=v_x(t_0)=v_0cos\theta_0\\
v_y(t_G)=-v_y(t_0)=-v_0sin\theta_0
\end{cases}\mbox{ (velocità finali)}$$
\item \textbf{moto parabolico da altezza h:}
$$\begin{cases}
x(t)=v_0t\\
y(t)=h-\frac{1}{2}gt^2
\end{cases}$$
$$\begin{cases}
v_x(t)=v_0\\
v_y(t)=-gt
\end{cases}$$
$$t_{volo}=\frac{x}{v_0}$$
$$y(x)=h-\frac{g}{2v_0^2}x^2\mbox{ (traiettoria)}$$
$$t_{caduta}=\sqrt{\frac{2h}{g}}$$
$$x(t_c)=x_G=v_0t_c=v_0\sqrt{\frac{2h}{g}}\mbox{ (gittata)}$$
$$\begin{cases}v_x(t_c)=v_0\\
v_y(t_c)=-\sqrt{2gh}\end{cases} \mbox{( velocità finali)}$$
$$v_{caduta}=\sqrt{v_0^2+2gh}$$
\end{itemize}
\section{Dinamica}
\begin{itemize}
\item \textbf{seconda legge della dinamica:} $\vec{F}=m\vec{a}$
\item \textbf{forza elastica:}
$$\vec{F}_e=-k\Delta \vec{x}$$
$$\vec{a}=\frac{-k(x-x_0)}{m}$$
\item \textbf{forza peso:}
$\vec{F}_p=mg$
\item \textbf{forza d'attrito:}
$$\vec{f}_{AD}=-\mu_DN$$
$$\vec{f}_{AS}=-\mu_SN$$
\item \textbf{lunghezza piano inclinato:} $$L=\frac{h}{sin\theta}$$
\end{itemize}
\subsubsection{Lavoro e Energia}
\begin{itemize}
\item \textbf{lavoro:}
$$L=\vec{F}_x\vec{\Delta x}$$
$$L=|\vec{F}|\,|\vec{\Delta x}|cos\theta=\vec{F}\vec{s}$$
\item \textbf{energia cinetica:}
$E_k=\frac{1}{2}mv_f^2-\frac{1}{2}mv_0^2$
\item \textbf{energia potenziale}
$E_P=mgz_B-mgz_A$
\item \textbf{lavoro della forza elastica:}
$E_{Pe}=\frac{1}{2}kx^2$
\item \textbf{lavoro della forza d'attrito:}
$W_{AD}=-\mu_DNl_{AB}$
\item \textbf{conservazione dell'energia meccanica con forze conservative:}
$$E_{KB}+E_{PB}=E_{KA}+E_{PA}$$
\item \textbf{conservazione dell'energia meccanica con forze conservative:}
$$E_{KB}+E_{PB}-E_{KA}+E_{PA}=E_{MB}-E_{MA}=\Delta E_M$$
$$W=W_{cons}+W_{non-cons}$$
$$W_{non-cons}=\Delta E_M$$
\item \textbf{energia meccanica nel caso di presenza di forze d'attrito:}
$$\Delta E_M=-\mu_DNl_{AB}$$
\end{itemize}
\subsubsection{Piano inclinato}
\begin{center}
\begin{pspicture}(0,-2.1787758)(5.26,2.1787758)
\psline[linecolor=black, linewidth=0.04](0.82,1.0412241)(1.62,1.8412242)(2.42,1.0412241)(1.62,0.24122417)(0.82,1.0412241)
\psline[linecolor=black, linewidth=0.04](1.22,0.64122415)(4.02,-2.1587758)(0.02,-2.1587758)(0.02,1.8412242)(1.22,0.64122415)
\psline[linecolor=black, linewidth=0.04, linestyle=dashed, dash=0.17638889cm 0.10583334cm, arrowsize=0.05291667cm 2.0,arrowlength=1.4,arrowinset=0.0]{->}(2.02,0.64122415)(3.22,-0.55877584)
\psline[linecolor=black, linewidth=0.04, linestyle=dashed, dash=0.17638889cm 0.10583334cm, arrowsize=0.05291667cm 2.0,arrowlength=1.4,arrowinset=0.0]{->}(2.02,1.4412242)(2.82,2.241224)
\rput[bl](2.82,-1.7587758){$\theta$}
\rput[bl](2.82,0.24122417){$mg\sin\theta$}
\rput[bl](2.82,1.6412242){$mg\cos\theta$}
\end{pspicture}\\
\textbf{forza normale:} $N=mg\cos\theta$\\
\textbf{lavoro attrito:} $W_{AD}=\mu_Dmg\cos\theta l_{AB}$
\end{center}
\section{Gravitazione}
\begin{itemize}
\item terza legge di Keplero: 
$$T^2=k_Sa^3$$
con 
$$r_1+r_2=2a$$
\item legge di gravitazione universale:
$$F=-G\frac{m_1m_2}{r^2}\vec{u}_r$$
$$G=6,67 \times 10^{−11} \frac{N m^2}{kg^2}$$
$$g=\frac{Fm_T}{r_T^2}$$
$$G=6,67\times10^{-11}\frac{Nm^2}{kg^2}6,67\times10^{-11}\frac{Nm^3}{s^2kg}$$
$$g=G\frac{M_t}{r_T^2}$$
\item campo gravitazionale:
$$\vec{\eta}(\vec{r})=\left(-G\frac{M}{r^2}\vec{u}_r\right)$$
$$\vec{\eta}(P)=\sum \vec{\eta}_i=-g\sum \frac{M_i}{r_i^2}\vec{u}_i$$
\item energia potenziale gravitazionale:
$$E_P=-G\frac{Mm}{r}$$
\item velocità di fuga:
$$\frac{1}{2}mv_f^2-G\frac{Mm}{r}=0$$
$$\downarrow$$
$$v_f=\sqrt{\frac{2GM}{r}}$$
\item velocità orbitale:

$$F=m\cdot \frac{v^2}{r}$$
$$\downarrow$$
$$\frac{Mm}{r^2}=m\cdot \frac{v^2}{r}$$
$$v=\sqrt{\frac{GM}{r}}$$
\item teorema del guscio:
$$\rho=\frac{M_T}{V_T}=\frac{M_T}{\frac{4}{3}\pi r^3}$$
$$F=G\frac{m_Tm}{r_T^2}=G\frac{\rho\frac{4}{3}\pi r^3 m}{r^2}=G\frac{4}{3}\rho m r \mbox{ che con } k=G\frac{4}{3}\rho m \to F=-kr \mbox{ negativo attrazione}$$
\item energia:
$$E_P=G\frac{Mm}{r}$$
$$E_K=\frac{1}{2}mv^2=\frac{1}{2}m\omega^2r^2$$
$$G\frac{Mm}{r^2}=ma=m\frac{v^2}{r}\to\omega^2r^2=G\frac{m}{r}$$
$$E_K=G\frac{Mm}{2r}$$
$$E_M=E_K+E_P=G\frac{Mm}{2r}-G\frac{Mm}{r}=-G\frac{Mm}{2r}$$
\end{itemize}
\end{document}