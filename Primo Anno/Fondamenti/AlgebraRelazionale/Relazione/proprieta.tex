Dubbio sulla definizione con dominio e Codominio diverso
Data una relazione $R$ definita su un dominio $S$ si definiscono le seguenti proprietà:
\begin{itemize}
  \item Riflessiva se e solo se $\forall x \in S$ risulta  $xRx$
  \item Irriflessiva se e solo se $\forall x \in S$ risulta $x \slashed{R} x$
  \item Simmetrica se e solo se $\forall x,y \in S$ risulta $xRy \rightarrow yRx$
  \item Asimmetrica: se e solo se $\forall x,y \in S$ risulta $xRy \rightarrow y \slashed{R} x$
  \item Antisimmetrica: se e solo se $\forall x,y \in S$ si ha $xRy \land yRx$ implica $x = y$
  \item Transitiva: se e solo se $\forall x,y,z \in S$ si ha $xRy \land yRz$ implica che $xRz$
\end{itemize}

%Esempi
Esempio: Tutti gli uomini sono mortali,Socrate è un uomo allora Socrate è un mortale

Costanti:Socrate \newline
Predicati:$Uomo(x),Mortale(x)$ \newline
Funzioni: non presenti \newline
\begin{equation*}
\forall x ((Uomo(x) \rightarrow Mortale(x)) \land Uomo(Socrate) \rightarrow Mortale(Socrate))
\end{equation*}

Esempio: un cugino di Marco non ha cani

Costanti: $Marco$\newline
Predicati:$Cugino(x,y),Avere(x,y),Cane(y)$\newline
Funzioni: non sono presenti
\begin{equation*}
\exists x (Cugino(x,Marco) \land \forall y (Cane(y) \rightarrow \neg Avere(x,y)))
\end{equation*}

Esempio: Ogni treno ha un numero identificativo

Costanti: non presenti \newline
Predicati: $Treno(x)$,$Avere(x)$\newline
Funzioni:$id(x)$
\begin{equation*}
\forall x (Treno(x) \rightarrow Avere(id(x)))
\end{equation*}

Esercizio: Tutti i docenti hanno un età maggiore di 24 anni

Costanti: $24$
Predicati:$Docente(x)$,$>(x,y)$
Funzioni:$eta(x)$

$\forall x (Docente(x) \rightarrow eta(x) > 24)$

Esercizi:Tutti i docenti hanno una chiave d'accesso all'edificio U6

Costanti:$U6$
Predicati:$Docente(x)$,$Avere(y)$,$Chiave(x,y)$
Funzioni:non presente

\begin{equation*}
    \forall x (Docente(x) \land \exists y (Chiave(y,U6) \land Avere(y)))
\end{equation*}

Esercizio:Tutti i canali televisivi con una share maggiore del 10\% sono
          considerati canali principali

Costanti:$10\%$
Predicati:$Canale(x)$,$>(x,y)$,$CanalePrincipale(x)$
Funzioni:$share(x)$
\begin{equation*}
    \forall x (Canale(x) \land share(x) > 10\% \rightarrow CanalePrincipale(x))
\end{equation*}

Esercizio:il fratello di Marco ha copiato il compito ed è stato respinto

Costanti:$Marco$,$compito$
Predicati:$Copiare(x,y)$,$Uomo(x)$,$Bocciato(x)$,$Fratello(x,y)$
Funzioni: non presenti
\begin{equation*}
    \exists x (Uomo(x) \land Fratello(x,Marco) \land Copiare(x,compito) \rightarrow Bocciato(x))
\end{equation*}

Esercizio: Tutte le sere gli studenti ascoltano musica Uzbeka e bevono caffè

Costanti:$musicaUzbeka$,$caffè$
Predicati:$Studenti(x)$,$Ascoltare(x,y)$,$Bere(x,y)$,$Sera(y)$
Funzioni:non presenti
\begin{equation*}
\forall x,y (Studente(x) \land Sera(y) \rightarrow (Ascoltare(x,musicaUzbeka) \land Bere(x,caffè)))
\end{equation*}



Sulle relazioni si possono applicare le usuali operazioni insiemistiche quindi, ad esempio,
date $R_1 \subseteq S \times T$ e $R_2 \subseteq S \times T$ anche $R_1 \cup R_2$ è una relazione su $S \times T$.

%Definizione Relazione Complementare e Relazione Inversa
Data una relazione binaria $R \subseteq S \times T$ definiamo \emph{relazione complementare}
$\bar{R} \subseteq S \times T$ come $x \bar{R} y$ se e solo se $(x,y) \not \in R$.
Per definizione si ha $\bar{\bar{R}} = R$ e $R \cup \bar{R} = S \times T$.

Data una relazione binaria $R \subseteq S \times T$ esiste sempre la \emph{relazione inversa}
$R^-1 = \{(y,x) | (x,y) \in R\} \subseteq T \times S$.
Per definizione $(R ^ {-1}) ^ {-1} = R$
