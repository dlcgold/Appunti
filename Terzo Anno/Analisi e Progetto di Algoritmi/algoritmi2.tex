\documentclass[a4paper,12pt, oneside]{book}

%\usepackage{fullpage}
\usepackage[italian]{babel}
\usepackage[utf8]{inputenc}
\usepackage{amssymb}
\usepackage{amsthm}
\usepackage{graphics}
\usepackage{amsfonts}
\usepackage{listings}
\usepackage{amsmath}
\usepackage{amstext}
\usepackage{engrec}
\usepackage{rotating}
\usepackage{verbatim}
\usepackage[safe,extra]{tipa}
\usepackage{showkeys}
\usepackage{multirow}
\usepackage{hyperref}
\usepackage{microtype}
\usepackage{enumerate}
\usepackage{braket}
\usepackage{marginnote}
\usepackage{pgfplots}
\usepackage{cancel}
\usepackage{polynom}
\usepackage{booktabs}
\usepackage{enumitem}
\usepackage{framed}
\usepackage{pdfpages}
\usepackage{pgfplots}
\usepackage{algorithmic}
\usepackage{algpseudocode}
\usepackage[cache=false]{minted}

\usepackage{tikz}\usetikzlibrary{er}\tikzset{multi  attribute /.style={attribute ,double  distance =1.5pt}}\tikzset{derived  attribute /.style={attribute ,dashed}}\tikzset{total /.style={double  distance =1.5pt}}\tikzset{every  entity /.style={draw=orange , fill=orange!20}}\tikzset{every  attribute /.style={draw=MediumPurple1, fill=MediumPurple1!20}}\tikzset{every  relationship /.style={draw=Chartreuse2, fill=Chartreuse2!20}}\newcommand{\key}[1]{\underline{#1}}

\usepackage{fancyhdr}
\pagestyle{fancy}
\fancyhead[LE,RO]{\slshape \rightmark}
\fancyhead[LO,RE]{\slshape \leftmark}
\fancyfoot[C]{\thepage}



\title{Analisi e Progetto di Algoritmi}
\author{UniShare\\\\Davide Cozzi\\\href{https://t.me/dlcgold}{@dlcgold}\\\\Gabriele De Rosa\\\href{https://t.me/derogab}{@derogab} \\\\Federica Di Lauro\\\href{https://t.me/f_dila}{@f\textunderscore dila}}
\date{}

\pgfplotsset{compat=1.13}
\begin{document}
\maketitle

\definecolor{shadecolor}{gray}{0.80}
\setlist{leftmargin = 2cm}
\newtheorem{teorema}{Teorema}
\newtheorem{definizione}{Definizione}
\newtheorem{esempio}{Esempio}
\newtheorem{corollario}{Corollario}
\newtheorem{lemma}{Lemma}
\newtheorem{osservazione}{Osservazione}
\newtheorem{nota}{Nota}
\newtheorem{esercizio}{Esercizio}
\algdef{SE}[DOWHILE]{Do}{doWhile}{\algorithmicdo}[1]{\algorithmicwhile\ #1}
\tableofcontents
\renewcommand{\chaptermark}[1]{%
	\markboth{\chaptername
		\ \thechapter.\ #1}{}}
\renewcommand{\sectionmark}[1]{\markright{\thesection.\ #1}}
\chapter{Introduzione}
\textbf{Questi appunti sono presi a lezione. Per quanto sia stata fatta una revisione è altamente probabile (praticamente certo) che possano contenere errori, sia di stampa che di vero e proprio contenuto. Per eventuali proposte di correzione effettuare una pull request. Link: } \url{https://github.com/dlcgold/Appunti}.\\
\textbf{Grazie mille e buono studio!}
\\
\begin{comment}

\begin{algorithm}
  \If {$i \gets 1$}
  \State $test$
  \Else
  \State $bho$
  \EndIf
  \While {$test$}
  \State $cose$
  \EndWhile
  \For {$cose...$}
  \State $altre cose$
  \EndFor

  \Function{Increment}{$a$}
  \State $a \gets a+1$
  \State \Return $a$
  \EndFunction
\end{algorithm}

\begin{algorithm}
  \Do
  \State ciao
  \doWhile {$ciao$}
\end{algorithm}
\end{comment}

\end{document}


