\documentclass[a4paper,12pt, oneside]{book}

%\usepackage{fullpage}
\usepackage[italian]{babel}
\usepackage[utf8]{inputenc}
\usepackage{amssymb}
\usepackage{amsthm}
\usepackage{graphics}
\usepackage{amsfonts}
\usepackage{listings}
\usepackage{amsmath}
\usepackage{amstext}
\usepackage{engrec}
\usepackage{rotating}
\usepackage[safe,extra]{tipa}
\usepackage{showkeys}
\usepackage{multirow}
\usepackage{hyperref}
\usepackage{microtype}
\usepackage{enumerate}
\usepackage{braket}
\usepackage{marginnote}
\usepackage{pgfplots}
\usepackage{cancel}
\usepackage{polynom}
\usepackage{booktabs}
\usepackage{enumitem}
\usepackage{framed}
\usepackage{pdfpages}
\usepackage{pgfplots}
\usepackage[cache=false]{minted}
\usepackage{fancyhdr}
\pagestyle{fancy}
\fancyhead[LE,RO]{\slshape \rightmark}
\fancyhead[LO,RE]{\slshape \leftmark}
\fancyfoot[C]{\thepage}



\title{Analisi e Progettazione del Software}
\author{UniShare\\\\Davide Cozzi\\\href{https://t.me/dlcgold}{@dlcgold}\\\\Gabriele De Rosa\\\href{https://t.me/derogab}{@derogab} \\\\Federica Di Lauro\\\href{https://t.me/f_dila}{@f\textunderscore dila}}
\date{}

\pgfplotsset{compat=1.13}
\begin{document}
\maketitle

\definecolor{shadecolor}{gray}{0.80}

\newtheorem{teorema}{Teorema}
\newtheorem{definizione}{Definizione}
\newtheorem{esempio}{Esempio}
\newtheorem{corollario}{Corollario}
\newtheorem{lemma}{Lemma}
\newtheorem{osservazione}{Osservazione}
\newtheorem{nota}{Nota}
\newtheorem{esercizio}{Esercizio}
\tableofcontents
\renewcommand{\chaptermark}[1]{%
\markboth{\chaptername
\ \thechapter.\ #1}{}}
\renewcommand{\sectionmark}[1]{\markright{\thesection.\ #1}}
\chapter{Introduzione}
\textbf{Questi appunti sono presi durante le lezioni in aula. Per quanto sia stata fatta una revisione è altamente probabile (praticamente certo) che possano contenere errori, sia di stampa che di vero e proprio contenuto. Per eventuali proposte di correzione effettuare una pull request. Link: } \url{https://github.com/dlcgold/Appunti}.\\
\textbf{Grazie mille e buono studio!}
\chapter{Lezione 1}
Si studia:
\begin{itemize}
\item introduzione all'ingegneria del software
\item progettazione sistemi orientati ad oggetti
\item modellazione a dominio
\item UML e analisi dei casi d'uso
\item design pattern
\item sviluppo test-driven
\item code smell e refactoring
\end{itemize}
Un software è un programma per computer (ma và?!). Possono essere:
\begin{itemize}
\item generici, per un ampio range di clienti
\item custom, per un singolo cliente
\end{itemize}
spesso si usa una base di software pre-esistente per sviluppare codice.  L'ingegneria del software si occupa degli aspetti riguardanti lo sviluppo del software e sfrutta molto vari software pre-esistenti. Si utilizza un approccio semantico ed organizzato. Tutto questo è veicolato anche dalle risorse tecniche ed economiche. Solitamente si presuppone che un software abbia una durata di alcuni anni, si presuppone che un supporto nel tempo, con un cambiamento nello stesso, e una manutenzione mediante varie releases. Ovviamente si parla di programmi complessi. Si ha la seguente legge famosa:
\begin{center}
\textit{Un software per essere utile deve essere continuamente cambiato}
\end{center}
Si hanno due tipologie di progetti:
\begin{enumerate}
\item progetti di routine, soluzione di problemi e riuso di vecchio codice
\item progetti innovativi, soluzioni nuovi
\end{enumerate}
solitamente l'ingegneria del software si occupa di progetti innovativi con:
\begin{itemize}
\item specifiche del progetto variabili
\item cambiamenti continui
\item è una disciplina nuova
\end{itemize}
Un programmatore normalmente lavora da solo, su un programma completo con specifiche note; un ingegnere del software lavora in gruppo, progetta componenti e l'architettura e identifica requisiti e specifiche. Il costo del software spesso supera quello hardware %aggiungere grafico
\\
l'ingegneria del software riguarda lo sviluppo cost-effective di software.\\
Si hanno le seguenti fasi di sviluppo:
\begin{itemize}
\item \textbf{analisi dei requisiti}, che indica cosa deve fare il sistema
\item \textbf{progettazione}, progetto del sistema d implementare 
\item \textbf{sviluppo}, produzione del sistema software
\item \textbf{convalida}, verifica dei requisiti del cliente
\item \textbf{evoluzione}, evoluzione al cambiare di requisiti del cliente
\end{itemize}
\end{document}