\chapter{Introduzione alla Concorrenza}
\label{Capitolo 1}
La \textbf{concorrenza} è presente in diversi aspetti della quotidianità.
Un primo esempio di \textbf{sistema concorrente} non legato all'informatica è 
la \textit{cellula vivente}: può essere vista come un dispositivo che
trasforma e manipola dati per ottenere un risultato. I vari processi all'interno
di una cellula avvengono in modo concorrente, il che la rende un \textit{sistema
  asincrono}. Un secondo esempio può essere
quello dell'\textit{orchestra musicale}: i vari componenti suonano spesso
simultaneamente rappresentando un \textit{sistema sincrono} (ovvero un sistema
che funziona avendo una sorta di ``cronometro'' condiviso dai vari attori). 
Un esempio informatico invece è un \textit{processore multicore} (anche se in
realtà anche se fosse \textit{monocore} sarebbe comunque un sistema concorrente
per ovvie ragioni). Anche una \textit{rete di calcolatori} è un modello
concorrente, nonché i \textit{modelli sociali umani}.
\subsubsection{Caratteristiche comuni}
I modelli concorrenti hanno alcuni aspetti comuni, tra cui:
\begin{itemize}
  \item competizione per l’accesso alle risorse condivise
  \item cooperazione per un fine comune (che può portare a competizione)
  \item coordinamento di attività diverse
  \item sincronia e asincronia
\end{itemize}
\subsubsection{Studio}
Durante lo studio e la progettazione di sistemi concorrenti si hanno diversi problemi
peculiari che rendono il tutto molto complesso. Un sistema
concorrente mal progettato può avere effetti catastrofici.  

Per poter sviluppare modelli concorrenti si necessita innanzitutto di:
\begin{itemize}
  \item \textbf{linguaggi}, per specificare e rappresentare sistemi concorrenti. 
  \begin{itemize}
    \item \textbf{linguaggi di programmazione} (con l'uso di \textit{thread},
    \textit{mutex}, scambio di messaggi, etc$\ldots$ con i vari
    problemi di \textit{race condition}, uso di variabili condivise
    etc$\ldots$).
    
    \item linguaggi rappresentativi, come ad esempio  
    una \textit{partitura musicale} (nella quale si visualizza bene la natura
    \textit{sincrona}).

    \item \textbf{task graph (\textit{grafo delle
        attività})}, nel quale i nodi sono le attività (o eventi) mentre gli archi
    rappresentano una \textit{relazione d'ordine parziale}, come per esempio una
    \textit{relazione di precedenza} sui nodi.

    \item \textbf{algebre di processi}, simile ad un sistema di equazioni, con simboli
    che rappresentano eventi del sistema concorrente e operatori atti a comporre
    fra loro i vari sottoprocessi del sistema concorrente. Ogni ``equazione''
    descrive un processo che costituisce un elemento di un sistema concorrente.
    
    \item \textbf{modelli}, per modellare sistemi concorrenti in astratto. Un
    esempio è dato dalle \textbf{reti di Petri}, che modellano un sistema
    concorrente partendo dalle nozioni di \textit{stato locale} di uno dei
    componenti del sistema e di \textit{evento locale} che ha un effetto su alcune
    componenti (e non tutte). Si ha quindi rappresentato un \textit{sistema
      dinamico} che si evolve nel tempo e la cui evoluzione è rappresentata tramite
    \textit{relazioni di flusso}).

  \end{itemize}
  
  \item \textbf{logica}, per analizzare e specificare sistemi concorrenti.
  \item \textbf{model-checking}, per validare formule relative a proprietà di
  sistemi concorrenti.
\end{itemize}