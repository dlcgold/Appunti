\documentclass[a4paper,12pt, oneside]{book}

%\usepackage{fullpage}
\usepackage[italian]{babel}
\usepackage[utf8x]{inputenc}
\usepackage{float}
\usepackage{amssymb}
\usepackage{amsthm}
\usepackage{graphics}
\usepackage{amsfonts}
\usepackage{amsmath}
\usepackage{amstext}
\usepackage{engrec}
\usepackage{rotating}
\usepackage[safe,extra]{tipa}
\usepackage{showkeys}
\usepackage{multirow}
\usepackage{hyperref}
\usepackage{microtype}
\usepackage{enumerate}
\usepackage{braket}
\usepackage{marginnote}
\usepackage{pgfplots}
\usepackage{cancel}
\usepackage{polynom}
\usepackage{booktabs}
\usepackage{enumitem}
\usepackage{framed}
\usepackage{pdfpages}
\usepackage{pgfplots}
\usepackage{fancyhdr}
\pagestyle{fancy}
\fancyhead[LE,RO]{\slshape \rightmark}
\fancyhead[LO,RE]{\slshape \leftmark}
\fancyfoot[C]{\thepage}
\graphicspath{ {./images/} }


\title{Metodi algebrici per l'informatica}
\author{UniShare\\\\Davide Cozzi\\\href{https://t.me/dlcgold}{@dlcgold}\\\\Gabriele De Rosa\\\href{https://t.me/derogab}{@derogab} \\\\Federica Di Lauro\\\href{https://t.me/f_dila}{@f\textunderscore dila}}
\date{}

\pgfplotsset{compat=1.13}
\begin{document}
\maketitle

\definecolor{shadecolor}{gray}{0.80}

\newtheorem{teorema}{Teorema}
\newtheorem{definizione}{Definizione}
\newtheorem{principio}{Principio}
\newtheorem{esempio}{Esempio}
\newtheorem{corollario}{Corollario}
\newtheorem{lemma}{Lemma}
\newtheorem{osservazione}{Osservazione}
\newtheorem{nota}{Nota}
\newtheorem{algoritmo}{Algoritmo}
\tableofcontents
\renewcommand{\chaptermark}[1]{%
	\markboth{\chaptername
		\ \thechapter.\ #1}{}}
\renewcommand{\sectionmark}[1]{\markright{\thesection.\ #1}}

\chapter{Introduzione}
\textbf{Questi appunti sono presi a lezione. Per quanto sia stata fatta una revisione è altamente probabile (praticamente certo) che possano contenere errori, sia di stampa che di vero e proprio contenuto. Per eventuali proposte di correzione effettuare una pull request. Link: } \url{https://github.com/dlcgold/Appunti}.\\
\textbf{Grazie mille e buono studio!}

\chapter{Ripasso}
Indichiamo con $\mathbb{Z}$ l'insieme dei numeri interi e con $\mathbb{N}$ l'insieme dei numeri naturali (con la convenzione che $0 \in \mathbb{N}$).\\
Una proprietà fondamentale dell'insieme $\mathbb{Z}$ è il cosiddetto Principio del Buon Ordinamento.
\section{Principio del Buon Ordinamento}
\begin{principio}

	Sia $n_o \in \mathbb{Z}$, $\mathbb{Z}_{n_o} = \{n \in \mathbb{Z} | n \geq n_0\}$ \\\\
	Ogni sottoinsieme non vuoto di $\mathbb{Z}_{n_0}$ ammette minimo. \\\\
	$\forall X \subseteq \mathbb{Z}_{n_0}$ con $X \not = \emptyset \qquad \exists x_0 \in X$ tale che $x_0 \leq x \quad \forall x \in X$\\

\end{principio}
Il principio del buon ordinamento è equivalente al principio di induzione.

\section{Principio di Induzione}
\begin{principio}
	Siano $n_0 \in \mathbb{Z}$ e $P=P(n)$ un enunciato valido per $\forall n \geq n_0$\\
	Se \begin{enumerate}
		\item $P(n_0)$ è vero
		\item \begin{enumerate}[label=\Roman*) ]
			      \item $\forall n > n_0$ $P(n-1)$ vero implica $P(n)$ vero
			      \item $\forall n > n_0$ $P(k)$ vero $\forall n_0 \leq k \leq n$ implica $P(n)$ vero
		      \end{enumerate}
	\end{enumerate}
	Allora $P(n)$ è vero $\forall n \geq n_0$
\end{principio}
\begin{shaded}
	\begin{esempio}
		Somma dei primi $n$ numeri interi:
		\begin{equation}
			\sum_{i=1}^{n} i=\frac{n(n+1)}{2}
		\end{equation}
		\begin{proof}
			Si ha che:
			$$P(1): \sum_{i=1}^{1} i = \frac{1 \cdot 2}{2} = 1$$
			Ipotesi:
			$$P(n-1): \sum_{i=1}^{n-1} i = \frac{(n-1)n}{2}$$
			Tesi:
			$$P(n): \sum_{i=1}^{n} i = \frac{n(n+1)}{2}$$
			Per il \textit{Principio di Induzione}:
			$$ \sum_{i=1}^{n} i = (\sum_{i=1}^{n-1} i) + n = \frac{(n-1)n}{2} + n = \frac{(n-1)n + 2n}{2} = \frac{n^2 + n}{2} = \frac{n(n+1)}{2}$$
			$$ \implies P(n) \mbox{ è vera } \forall n \implies \mbox{la tesi è verificata}$$
		\end{proof}
	\end{esempio}
\end{shaded}
\begin{shaded}
	\begin{nota}
		Nomenclature: \\\\
		$X$ = insieme \\
		$|X|$ = cardinalità dell'insieme $X$ = numero degli elementi di $X$ \\
		$P(X)$ = insieme delle parti di $X$ = $\{ Y | Y \subseteq X \}$
	\end{nota}
\end{shaded}
\begin{shaded}
	\begin{esempio}
		Se un insieme ha cardinalità $n$ allora il suo insieme delle parti ha cardinalità $2^n$.
		\begin{equation}
			|X| = n \implies |P(X)| = 2^n
		\end{equation}
	\end{esempio}
	\begin{proof}
		Si ha, per il \textit{Principio di Induzione} che:\\
		\begin{itemize}
			\item $P(0)$: se un insieme $X$ ha cardinalità 0 (ovvero $X = \emptyset$), allora il suo insieme delle parti $P(X)$ ha cardinalità $2^0 = 1$, infatti $P(X) = \{ \emptyset \}$
			\item $P(n-1)$ vera $\implies P(n)$ vera.\\\\
			      Sia $X$ un insieme di cardinalità $n$\\
			      e sia $x_0 \in X$ (che certamente esiste perchè $n > 0$);
			      $$P(X) = A \cup B$$
			      con
			      $$A = \{Y | Y \subseteq X \cap x_0 \in Y\}$$
			      $$B = \{Z | Z \subseteq X \cap x_0 \not\in Z\}$$\\
			      Noto che $A \cup B = \emptyset$ e che $|P(X)| = |A| + |B|$\\\\
			      Considero $\overline{X} = X \setminus \{ x_0 \}$ l'insieme dei sottoinsiemi che non contengono $x_0$ e ne derivo che la cardinalità $|\overline{X}| = n-1$.\\
			      Risulta che $B = P(\overline{X})$\\\\
			      $\exists f: A \to B$ (biunivoca e) invertibile tale che
			      $$Y \to Y \setminus \{ x_0 \}$$
			      $$Z \cup \{ x_0 \} \to Z$$
			      da cui derivo che $|A| = |B| = 2^{n-1}$\\\\
			      Ottengo quindi che
			      $$|P(X)| = |A| + |B| = 2^{n-1} + 2^{n-1} = 2^n$$
			      Dato che $|\overline{X}| = n-1 \implies |P(\overline{X})| = 2^{n-1}$ è vera,\\
			      allora anche $|X| = n \implies |P(X)| = 2^{n}$ è vera.
		\end{itemize}
	\end{proof}
\end{shaded}
\chapter{Algoritmo della Divisione}
\begin{algoritmo}
	Dati $n,m$ interi con $n > m > 0$, l'usuale algoritmo della divisione permette di determinare due interi $q$ e $r$ (il quoziente e il resto della divisione) tali che $mq$ è il multiplo di di $m$ che più si avvicina a $n$ per difetto e $r = n - mq$ misura lo scarto.
\end{algoritmo}
Possiamo generalizzare con il seguente teorema:
\begin{teorema}
	Siano $n, m \in Z$ con $m \not = 0$. Allora esistono e sono unici due interi $q$ e $r$ tali che:
	\begin{itemize}
		\item $n = mq + r$
		\item $0 \leq r < |m|$
	\end{itemize}
\end{teorema}
\begin{definizione}
	Gli interi q e r del teorema precedente si dicono quoziente e resto della divisione di n per m.
\end{definizione}
\begin{proof}
	\textbf{Esistenza di $q$ e $r$.}
	\begin{enumerate}
		\item Supponiamo $n \geq 0$.\\
		      Fissato arbitrariamente $m$ procediamo per induzione su $n$.
		      \begin{enumerate}
			      \item $n = 0$: le condizioni sono verificate con $q = r = 0$ perchè $0 = m \cdot 0 + 0$.
			      \item $n \geq 0$: \begin{enumerate}
				            \item $n < |m|$: le condizioni sono verificate con $q = 0$ e $r = n$.
				            \item $n \geq |m|$\\
				                  $n > n - |m| \geq 0$\\
				                  Per induzione $\exists$ $q_1$ e $r_1$ tali che
				                  $$n - |m| = mq_1 + r_1$$
				                  con $0 \leq r_1 < |m|$.
				                  Quindi
				                  $$n = |m| + mq_1 - r_1$$
				                  con $0 \leq r_1 < |m|$.\\\\
				                  da cui se \begin{itemize}
					                  \item $m > 0$:
					                        $$n = m + mq_1 + r_1 = m (q_1 +1) + r_1$$
					                        Il teorema è vero con $$q = q_1 +1$$ $$r = r_1$$
					                  \item $m < 0$:
					                        $$n = -m + mq_1 + r_1 = m (q_1 -1) + r_1$$
					                        Il teorema è vero con $$q = q_1 -1$$ $$r = r_1$$
				                  \end{itemize}
			            \end{enumerate}
		      \end{enumerate}
		\item Supponiamo $n < 0$.
		      Allora $-n > 0$ e per il punto \textit{(1)} $\exists$ $q_1, r_1$ tali che
		      $$-n = mq_1 + r_1$$
		      con $0 \leq r_1 < |m|$.
		      Pertanto $$n = -mq_1 - r_1$$
		      Aggiungo e sottraggo $|m|$ ottenendo
		      $$n = -mq_1 -|m| + |m| - r_1$$
		      da cui se \begin{itemize}
			      \item $m > 0$:
			            $$n = -mq_1 - m + (m-r_1)$$
			            Il teorema è vero con $$q = -q_1-1$$ $$r=m-r_1$$
			            \begin{nota}
				            $0 \leq r_1 < m$ quindi $-m \leq -r_1 < 0$ e $0 \leq r = m-r_1 < m$
			            \end{nota}
			      \item $m < 0$:
			            $$n = -mq_1+m-m-r_1 = m(-q_1+1)-m-r_1$$
			            Il teorema è vero con $$q = -q_1+1$$ $$r=-m-r_1$$
		      \end{itemize}
	\end{enumerate}
	\textbf{Unicità di $q$ e $r$.}\\
	Siano $$n = mq+r$$ con $0 \leq r < |m|$ e $$n = mq_1 + r_1$$ con $0 \leq r_1 < |m|$.\\
	Mostriamo che $q=q_1$ e $r=r_1$.\\
	Supponiamo PER ASSURDO che $r \not = r_1$; possiamo assumere $r_1 > r$. Quindi
	$$mq+r=mq_1+r_1$$
	$$m(q-q_1)=r_1-r$$
	Pertanto
	$$|m||q-q_1| = |r_1-r| = r_1-r < |m|$$
	$$|m||q-q_1| < m$$
	$$|q-q_1| < 1$$
	$$|q-q_1| = 0 \implies q=q_1$$
	Dato che
	$$n = mq+r = mq_1+r_1 \implies r=r_1$$ che è ASSURDO poiché abbiamo assunto $r \not = r_1$.
\end{proof}
\begin{osservazione}
	Dati $n, m \in \mathbb{Z}$ con $m \not = 0$ esistono infinite coppie di interi $x$ e $y$ che soddisfano la condizione \textit{(1)} del teorema precedente, cioè $n = mx+y$. Infatti, scelto comunque un intero $x$, basta porre $y=n-mx$. È invece unica la coppia $q, r$ che soddisfa entrambe le condizioni \textit{(1)} e \textit{(2)}.
\end{osservazione}


\chapter{Algoritmo di Euclide}
\section{Divisibilità}
\begin{definizione}
	Siano $a,b \in \mathbb{Z}$.\\
	Se esiste $c \in \mathbb{Z}$ con $a=bc$ diciamo che $b$ divide $a$.
\end{definizione}
\begin{nota}
	$b$ divide $a$ è indicato con $b|a$.
\end{nota}
\begin{osservazione}
	Se $b|a$ (quindi anche $-b|a$) diciamo che $a$ è un multiplo di $b$, ovvero $b$ è un fattore (o divisore) di $a$.\\\\
	Ovviamente $\pm1$ e $\pm a$ sono fattori di ogni intero $a$.\\
	Se $b|a$ e $b \not = \pm 1, \pm a$ diciamo che $b$ è un \textbf{divisore proprio} di $a$.
\end{osservazione}
\begin{osservazione}
	Siano $a,b \in \mathbb{Z}$ con $a \not = 0, b \not = 0$\\
	Se 	$a|b$ e $b|a$ allora $b=\pm a$.
	\begin{proof}
		Poiché\\
		\begin{enumerate}
			\item $a|b \implies \exists c_0 \in \mathbb{Z}$ con $b=a c_0$
			\item $b|a \implies \exists c_1 \in \mathbb{Z}$ con $a=b c_1$
		\end{enumerate}
		Sostituisco la \textit{(1.)} nella \textit{(2.)} e trovo
		$$a = b c_1$$
		$$a = a c_0 c_1$$
		$$a - a c_0 c_1 = 0$$
		$$a(1- c_0 c_1) = 0$$
		Da cui per il \textit{principio di annullamento del prodotto} ottengo $a = 0$ e
		$$c_0 c_1 = 1$$
		Quindi
		$$c_0 = c_1 = 1 \implies b = a$$
		$$c_0 = c_1 = -1 \implies b = -a$$
		Ho dimostrato che
		$$a|b \mbox{ e } b|a \iff b=\pm a$$
	\end{proof}
\end{osservazione}
\begin{shaded}
	\begin{esempio}
		Dimostro che se $c|a$ e $c|b$ allora $c|a+b$.
		\begin{proof}
			$$c|a \implies \exists d_0 \in \mathbb{Z} | a=d_0c$$
			$$c|b \implies \exists d_1 \in \mathbb{Z} | b=d_1c$$
			Derivo che
			$$a+b = d_0c + d_1c = c(d_0+d_1)$$
			Essendo $d_0+d_1 \in \mathbb{Z}$ la tesi è dimostrata.
		\end{proof}
	\end{esempio}
	\begin{esempio}
		Dimostro che se $c|a$ e $c|b$ allora $c|a-b$.
		\begin{proof}
			$$c|a \implies \exists d_0 \in \mathbb{Z} | a=d_0c$$
			$$c|b \implies \exists d_1 \in \mathbb{Z} | b=d_1c$$
			Derivo che
			$$a-b = d_0c - d_1c = c(d_0-d_1)$$
			Essendo $d_0-d_1 \in \mathbb{Z}$ la tesi è dimostrata.
		\end{proof}
	\end{esempio}
	\begin{esempio}
		Dimostro che se $c|a$ e $c|b$ allora $c|ax+by, \forall x,y \in \mathbb{Z}$.
		\begin{proof}
			$$c|a \implies \exists d_0 \in \mathbb{Z} | a=d_0c$$
			$$c|b \implies \exists d_1 \in \mathbb{Z} | b=d_1c$$
			Derivo che
			$$ax+by = d_0cx + d_1cy = c(d_0x+d_1y)$$
			Essendo $d_0x+d_1y \in \mathbb{Z}$ la tesi è dimostrata.
		\end{proof}
	\end{esempio}
	\begin{esempio}
		Dimostro che se $c|a$ allora $c|a+b \implies c|b$
		\begin{proof}
			$$a = k_0c \mbox{ con } k_0 \in \mathbb{Z}$$
			$$a+b = k_1c \mbox{ con } k_1 \in \mathbb{Z}$$
			Sostituendo ottengo che
			$$a + b = k_0c + b = k_1c$$
			da cui $$b = k_1c-k_0c = c(k_1-k_0)$$
			Essendo $k_1-k_0 \in \mathbb{Z}$ la tesi è dimostrata.
		\end{proof}
	\end{esempio}
\end{shaded}
\section{Massimo Comune Divisore}
\begin{definizione}
	Siano $a,b \in \mathbb{Z}$ con $a \not = 0$, $b \not = 0$.\\
	Si dice che $d$ è un \textbf{massimo comune divisore} tra $a$ e $b$ se
	\begin{enumerate}
		\item $d|a$ e $d|b$
		\item se $c \in \mathbb{Z}$ con $c|a$ e $c|b$ allora $c|d$
	\end{enumerate}
\end{definizione}
\subsection{Algoritmo di Euclide}
\begin{teorema}
	Esistenza di un Massimo Comune Divisore\\\\
	Siano $a,b \in \mathbb{Z}$ con $a > 0$, $b > 0$.\\
	Allora esiste un \textit{massimo comune divisore} $d$ tra $a$ e $b$.\\\\
	Inoltre $\exists s, t \in \mathbb{Z}$  tali che
	$$d = as+bt \qquad \mbox{\textbf{Identità di Bezout}}$$
\end{teorema}
\begin{proof}
	Suppongo $a \geq b$ ed eseguo l'\textit{Algoritmo della Divisione}..
	$$a = bq_1+r_1 \mbox{ con } 0 \leq r_1 < b$$
	Poi ricorsivamente
	$$r_1 \not = 0, b = r_1q_2 + r_2 \mbox{ con } 0 \leq r_2 < r_1$$
	$$r_2 \not = 0, r_1 = r_2q_3 + r_3 \mbox{ con } 0 \leq r_3 < r_2$$
	$$\vdots$$
	Fino a quando $r_k = 0$.\\
	\begin{nota}
		La successione dei resti è una successione strettamente decrescente di interi non negativi
		$$b > r_1 > r_2 > r_3 > r_4 > \dots > r_k = 0$$
		Dopo un numero finito di passi troverò resto $r_k = 0$.
	\end{nota}

	Proseguendo, se
	\begin{itemize}
		\item $k = 1$: allora
		      $$a= bq_1$$ ed il massimo comune divisore è $$d = b$$
		\item $k > 1$: allora
		      \begin{enumerate}[label=(\arabic*)]
			      \item $a = bq_1 + r_1$
			      \item $b = r_1q_2 + r_2$
			      \item $r_1 = r_2q_3 + r_3$
			      \item $r_2 = r_3q_4 + r_4$

			            $\vdots$

			      \item [(k-1)] $r_{k-3} = r_{k-2}q_{k-1} + r_{k-1}$
			      \item [(k)] $r_{k-2} = r_{k-1}q_k + r_k$
		      \end{enumerate}
		      Considerando $r_k = 0$ quindi il \textit{massimo comune divisore} è dato dall'ultimo resto non nullo che trovo applicando il procedimento dell'\textit{Algoritmo di Euclide} (delle divisioni successive), ovvero $$d=r_{k-1}$$\\
		      Devo quindi mostrare che $r_{k-1}$ soddisfa entrambe le condizioni per essere un \textit{massimo comune divisore}.
		      \begin{enumerate}
			      \item $d|a$ e $d|b$\\\\
			            Considerando i passi dell'Algoritmo di Euclide dal basso verso l'alto e sostituendo man mano...
			            \begin{enumerate}
				            \item [(k)] $r_{k-2} = r_{k-1}q_k + r_k$
				                  $\implies r_{k-1} | r_{k-2}$
				            \item [(k-1)] $r_{k-3} = r_{k-2}q_{k-1} + r_{k-1}$\\
				                  sostituisco $(k)$ in $(k-1)$\\
				                  $r_{k-3} = (r_{k-1}q_k)q_{k-1} + r_{k-1}$\\
				                  $r_{k-3} = r_{k-1}(q_kq_{k-1}+1)$
				                  $\implies r_{k-1} | r_{k-3}$

				                  $\vdots$

				            \item [(2)] $\dots \implies r_{k-1} | b$
				            \item [(1)] $\dots \implies r_{k-1} | a$
			            \end{enumerate}
			            $\vdots$\\
			            fino ad arrivare a dimostrare la prima condizione con $(2)$ e $(1)$.\\

			      \item se $c \in \mathbb{Z}$ con $c|a$ e $c|b$ allora $c|d$\\\\
			            $\exists c \in \mathbb{Z}$ con $c|a$ e $c|b$ ($c$ \textit{divisore comune tra $a$ e $b$})\\
			            quindi $$a = c \overline{a}$$ $$b = c \overline{b}$$
			            con $\overline{a}, \overline{b} \in \mathbb{Z}$.\\
			            Considerando i passi dell'Algoritmo di Euclide dall'alto verso il basso e sostituendo man mano...
			            \begin{enumerate}[label=(\arabic*)]
				            \item $a = bq_1 + r_1$\\
				                  $r_{1} = a-bq_1$\\
				                  $r_{1} = c\overline{a}-c\overline{b}q_1$\\
				                  $r_{1} = c(\overline{a}-\overline{b}q_1)$\\
				                  Essendo $\overline{a}-\overline{b}q_1 \in \mathbb{Z}$ $\implies c|r_1$\\

				                  Scrivo $r_1 = c\overline{r_1}$
				            \item $b = r_1q_2 + r_2$\\
				                  $r_2 = b-r_1q_2$\\
				                  $r_2 = c\overline{b}-c\overline{r_1}q_2$\\
				                  $r_2 = c(\overline{b}-\overline{r_1}q_2)$\\
				                  Essendo $\overline{b}-\overline{r_1}q_2 \in \mathbb{Z}$ $\implies c|r_2$\\

				                  Scrivo $r_2 = c\overline{r_2}$

				                  $\vdots$

				            \item $\dots \implies c|r_{k-1}$
			            \end{enumerate}
			            $\vdots$\\
			            fino ad arrivare a dimostrare la seconda condizione con $(k)$.\\\\

			            Dimostro l'\textbf{Identità di Bezout}\\
			            Considerando i passi dell'Algoritmo di Euclide dall'alto verso il basso e sostituendo man mano...
			            \begin{enumerate}[label=(\arabic*)]
				            \item $a = bq_1 + r_1$\\
				                  $r_1 = a-bq_1$\\
				                  $r_1 = a \cdot 1 + b(-q_1)$\\
				            \item $b = r_1q_2 + r_2$\\
				                  $r_2 = b-r_1q_2$\\
				                  $r_2 = b-(a-bq_1)q_2$\\
				                  $r_2 = b(1+q_1q_2) + a(-q_2)$\\

				                  $\vdots$

				            \item [(k-1)] $r_{k-1} = as + bt$
			            \end{enumerate}
			            fino a quando, continuando in questo modo, determino $s,t \in \mathbb{Z}$ con $r_{k-1} = as + bt$, ovvero l'\textit{identità di Bezout}.
		      \end{enumerate}
	\end{itemize}
\end{proof}

\begin{shaded}
	\begin{esempio}
		Trovare il massimo comune divisore tra $a=520, b=412$ utilizzando l'algoritmo di Euclide.\\\\
		$520 = 412 \cdot 1 + 108 \longrightarrow q_1=1, r_1 = 108$\\
		$412 = 108 \cdot 3 + 88 \longrightarrow q_2=3, r_2 = 88$\\
		$108 = 88 \cdot 1 + 20 \longrightarrow q_3=1, r_3 = 20$\\
		$88 = 20 \cdot 4 + 20 \longrightarrow q_4=4, r_4 = 8$\\
		$20 = 8 \cdot2 + 4 \longrightarrow q_5=2, r_5 = 4$\\
		$8 = 4 \cdot 2 \longrightarrow q_6=2, r_6 = 0$\\
		Dato che $r_6$ è nullo, $r_5 = (520,412) = 4$ è il \textit{massimo comune divisore}.\\\\
		Trovare anche l'Identità di Bezout:\\
		$r_1 = 108 =520 - 412 = a - b$\\
		$r_2 = 88 = b - 108 \cdot 3 = b-(a-b)3 = 4b-3a$\\
		$r_3 = 20 = 108 - 88 \cdot 1 = (a-b) - (4b-3a) = 4a-5b$\\
		$r_4 = 8 = 88-20 \cdot 4 = (4b-3a) - (4a-5b)4 = 24b-19a$\\
		$r_5 = 4 = 20 - 8 \cdot 2 = (4a-5b) - (24b-19a)2 = 42a-53b$\\
		quindi
		$$s = 42$$
		$$t = -53$$
		e l'identità di Bezout è
		$$4 = 42 \cdot 520 - 53 \cdot 412$$
	\end{esempio}
	\begin{esempio}
		Trovare il massimo comune divisore tra $a=589, b=437$ utilizzando l'algoritmo di Euclide.\\\\
		$589 = 437 \cdot 1 + 152 \longrightarrow q_1=1, r_1=152$\\
		$437 = 152 \cdot 2 + 133 \longrightarrow q_2=2, r_2=133$\\
		$152 = 133 \cdot 1 + 19 \longrightarrow q_3=1, r_3=19$\\
		$133 = 19 \cdot 7 + 0 \longrightarrow q_4 = 7, r_4 = 0$\\
		Dato che $r_4$ è nullo, $r_3 = (589,437) = 19$ è il \textit{massimo comune divisore}.\\\\
		Trovare anche l'Identità di Bezout:\\
		$r_1 = 152 = 589 - 437 = a-b$\\
		$r_2 = 133 = 437 - 152 \cdot 2 = b - 152 \cdot 2 = b-2(a-b) = 3b-2a$\\
		$r_3 = 19 = 152-133 = r_1-r_2 = (a-b)-(3b-2a) = 3a-4b$\\
		quindi
		$$s = 3$$ $$t=-4$$
		e l'identità di Bezout è
		$$19 = 3 \cdot 589 - 4 \cdot 437$$
	\end{esempio}
\end{shaded}
\begin{teorema}
	Se $d$ è un massimo comune divisore tra $a$ e $b$, l'unico altro massimo comune divisore è $-d$.

	\begin{proof}
		È chiaro che se $d$ è \textit{massimo comune divisore} tra $a$ e $b$, anche $-d$ lo è.\\\\
		Supponiamo che $\overline{d}$ è un altro \textit{massimo comune divisore} tra $a$ e $b$.
		\begin{enumerate}
			\item $d|a$ e $d|b$
			\item $\forall c \in \mathbb{Z}$, con $c|a$ e $c|b$ si ha $c|d$
		\end{enumerate}
		\begin{enumerate}[label=\arabic*']
			\item $\overline{d}|a$ e $\overline{d}|b$
			\item $\forall c \in \mathbb{Z}$, con $c|a$ e $c|b$ si ha $c|\overline{d}$
		\end{enumerate}
		Applico la $(2.)$ con $c = \overline{d}$ e trovo $\overline{D}|d$.\\
		Applico la $(2')$ con $c = d$ e trovo $d|\overline{D}$.\\\\
		Quindi $d = \pm \overline{d}$.
	\end{proof}
\end{teorema}
\begin{nota}
	Per convenzione si dice \textbf{massimo comune divisore} tra $a$ e $b$ l'unico massimo comune divisore positivo tra $a$ e $b$ e si indica con \textbf{(a,b)}
\end{nota}
\begin{osservazione}
	Siano $a, b \in \mathbb{Z}$ con $a \not = 0, b \not = 0$.\\
	Si può provare che $(a,b) = (-a,b) = (a,-b) = (-a,-b)$.
\end{osservazione}

\subsection{Numeri Primi}
\begin{definizione}
	Due numeri interi $a,b$ si dicono \textbf{primi} \textit{(o coprimi)} tra loro se $(a,b) = 1$.
\end{definizione}
\begin{osservazione}
	Siano $a,b \in \mathbb{Z}$ e sia $d=(a,b)$.\\
	Quindi $a=d\overline{a}$ e $b=d\overline{b}$ con $\overline{a}, \overline{b} \in \mathbb{Z}$.\\
	Allora $(\overline{a}, \overline{b}) = 1$.\\\\
	\begin{proof}
		Sia $t = (\overline{a}, \overline{b})$.\\
		Da cui $$t|\overline{a} \mbox{ e } t|\overline{b}$$
		$$td|a \mbox{ e } td|b$$
		quindi $td$ è un divisore comune di $a$ e $b$, perciò deve dividere il loro massimo comune divisore $d$
		$$td|d$$
		Concludo che $t = 1$.
	\end{proof}
\end{osservazione}
\begin{osservazione}
	Siano $a,b \in \mathbb{Z}$.
	\begin{shaded}
		\begin{nota}
			Se $a|bc$ non è sempre vero che $a|b$ o $a|c$.\\
			Ad esempio a=4, b=2, c=6
		\end{nota}
	\end{shaded}
	Se $a|bc$ e $(a,b) = 1$ allora $a|c$.

	\begin{proof}
		Da ipotesi ho $a|bc$ allora $$bc = ak$$
		con $k \in \mathbb{Z}$.\\
		Inoltre, sempre da ipotesi, ho $(a,b) = 1$ allora, per l'identità di Bezout,
		$$\exists x,y \in \mathbb{Z} \mbox{ tale che  } 1 = ax+by$$
		Moltiplico per $c$:
		$$c = acx+bcy$$
		ma $bc = ak$, quindi
		$$c = acx+aky$$
		$$c = a(cx+ky)$$
		Essendo $cx+ky \in \mathbb{Z} \implies a|c$
	\end{proof}

\end{osservazione}


\chapter{Numeri in base \textbf{b}}

\begin{teorema}
	Sia $b = \mathbb{Z}$ con $b \geq 2$.\\
	Ogni numero intero può essere scritto in un unico e solo modo nella forma
	$$n = d_{k}b^{k} + d_{k-1}b^{k-1} + \dots + d_{1}b^{1} + d_{0}$$
	con $0 \leq d_i < b \quad \forall  i = 0 \dots k$ e $d_k \not = 0$ per $k>0$.

	\begin{proof}
		Per induzione su $n$.
		\begin{itemize}
			\item [$n = 0$:] $n = 0 = 0 \cdot b^0$ vero
			\item [$n > 0$:] supponiamo il teorema vero per ogni $0 \leq m < n$.\\
			      Dividiamo con resto $n$ per $b$ e troviamo
			      $$n = bq + r \mbox{ con } 0 \leq r < b$$
			      Dato che $q < n$, per l'ipotesi induttiva $q$ può essere riscritto come
			      $$q = c_{k-1}b^{k-1} + c_{k-2}b^{k-2} + \dots + c_{1}b^{1} + c_{0}$$
			      con $0 \leq c_i < b$ per $i = 0 \dots (k-1)$.\\
			      Da cui
			      $$n = bq + r$$
			      $$n = b(c_{k-1}b^{k-1} + c_{k-2}b^{k-2} + \dots + c_{1}b^{1} + c_{0}) +r$$
			      $$n = c_{k-1}b^{k} + c_{k-2}b^{k-1} + \dots + c_{1}b^{2} + c_{0}b + r$$
			      Presi $d_{k} = c_{k-1}$, $d_{k-1} = c_{k-2}$, $\dots$, $d_{1} = c_{0}$, $d_{0} = r$ ottengo
			      $$n = d_{k}b^{k} + d_{k-1}b^{k-1} + \dots + d_{2}b^{2} + d_{1}b + d_0$$
			      con $0 \leq d_i < b$ per $i = 0 \dots k$.\\
			      Quindi il teorema è dimostrato.
		\end{itemize}

		\begin{nota}
			L'unicità di questa espressione segue dall'unicità di $q$ ed $r$.
		\end{nota}

	\end{proof}
\end{teorema}
\begin{definizione}
	Fissato $b \in \mathbb{Z}$, $b \geq 2$. Sia $n \geq 0$
	$$n = d_{k}b^{k} + d_{k-1}b^{k-1} + \dots + d_{2}b^{2} + d_{1}b + d_0$$
	con $0 \leq d_i < b$ per $i = 0 \dots k$.\\
	Gli interi $d_i$ con $i = 0 \dots k$ si dicono le cifre di $n$ in base $b$
	$$n = (d_k d_{k-1} \dots d_1 d_0)_b$$
\end{definizione}

\subsection{Conversione da base \textbf{$b$} a base \textbf{$10$}}
\begin{teorema}
	Sia $n \geq 0$ che in base $b$ è rappresentato dalla sequenza di cifre $(d_k d_{k-1} \dots d_1 d_0)_b$.\\\\
	È conveniente impostare la conversione in base $10$ in questo modo
	$$n = ( \dots ((d_kb+d_{k-1})b+d_{k-2})b+ \dots +d_1)b+d_0$$
	Questo metodo comporta solo $k$ moltiplicazioni per $b$ e $k$ addizioni.
\end{teorema}
\begin{shaded}
	\begin{esempio}
		$$n = (61405)_7$$ $$((((6 \cdot 7 + 1)7 + 4)7+0)7+5) = 14950_{10}$$
	\end{esempio}
\end{shaded}

\subsection{Conversione da base \textbf{$10$} a base \textbf{$b$}}
\begin{teorema}
	Osserviamo che $d_0, d_1, \dots, d_k$ sono i resti delle divisioni
	$$n = bq_0 +d_0 \mbox{ con } 0 \leq d_0 < b$$
	$$q_0 = bq_1 +d_1 \mbox{ con } 0 \leq d_1 < b$$
	$$q_1 = bq_2 +d_2 \mbox{ con } 0 \leq d_2< b$$
	$$ \vdots $$
\end{teorema}
\begin{shaded}
	\begin{esempio}
		$$n = 14950_{10}$$
		$$b=7$$\\
		$$14950 = 7 \cdot 2135 + 5$$
		$$2135 = 7 \cdot 305 + 0$$
		$$305 = 7 \cdot 43 + 4$$
		$$43 = 7 \cdot 6 + 1$$
		$$6 = 7 \cdot 0 + 6$$\\
		$$n = 61405_{7}$$
	\end{esempio}
\end{shaded}
\begin{osservazione}
	Il numero di cifre in base $b$ di un intero non negativo
	$$n = d_{k}b^{k} + d_{k-1}b^{k-1} + \dots + d_{2}b^{2} + d_{1}b + d_0$$ è
	$$k+1 = \lfloor \log_bn \rfloor+1 = \lfloor \frac{\log n}{\log b}+1 \rfloor$$
	siccome
	$$b^k \leq n < b^{k+1}$$
	$$k \leq \log_bn < k+1$$
	$$k=\lfloor \log_bn \rfloor$$
\end{osservazione}

\chapter{Relazioni}
\section{Relazioni su un insieme}
\begin{definizione}
	Sia $A$ un insieme non vuoto.\\
	Una relazione $R$ su $A$ è un sottoinsieme di $A \times A$.
\end{definizione}
\begin{nota}
	Se $R$ è una relazione su $A$, $(a,b) \in R$ si scrive anche $aRb$.
\end{nota}
\section{Proprietà delle relazioni}
\begin{definizione}
	Una relazione $R$ su un insieme $A$ si dice:
	\begin{itemize}
		\item \underline{riflessiva} se $\forall a \in A$, $(a,a) \in R$
		\item \underline{simmetrica} se $\forall a,b \in A$, $(a,b) \in R \implies (b,a) \in R$
		\item \underline{antisimmetrica} se $\forall a,b \in A$, $(a,b) \in R$ e $(b,a) \in R \implies a=b$
		\item \underline{transitiva} se $\forall a,b,c \in A$, $(a,b) \in R$ e $(b,c) \in R \implies (a,c) \in R$
	\end{itemize}
\end{definizione}
\begin{shaded}
	\begin{esempio}
		Dato $A = \{a,b,c,d\}$.\\
		Sia $R = \{(a,a), (b,b), (c,c), (d,d), (a,d), (d,c), (a,c), (c,a), (d,a), (c,d)\}$.\\\\
		$R$ è simmetrica, riflessiva, transitiva.
	\end{esempio}
	\begin{esempio}
		Dato $A = \{1,2,3\}$.\\
		Sia $R = \{(1,1), (2,2), (1,2), (2,1), (2,3)\}$.\\\\
		$R$ è \begin{itemize}
			\item NON è riflessiva
			\item NON è simmetrica
			\item NON è antisimmetrica perchè $(1,2) \in R, (2,1 \in R)$ ma $1 \not = 2$.
			\item NON è transitiva perchè $(1,2) \in R, (2,3) \in R$ ma $(1,3) \not\in R$
		\end{itemize}
	\end{esempio}
	\begin{esempio}
		Sia $A$ un insieme qualsiasi e sia $R$ la \textbf{relazione di uguaglianza} tra elementi di $A$, cioè $$(a,b) \in R \iff a=b$$
		$R$ è riflessiva, simmetrica, antisimmetrica, transitiva.
	\end{esempio}
	\begin{esempio}
		Sia $X$ un insieme qualsiasi e sia $P(X)$ l'\textit{insieme delle parti} di $X$.
		Sia quindi $R$  la relazione di inclusione su $P(X)$, cioè
		$$(Y,Z) \in R \iff Y \subseteq Z $$
		con $Y,Z \in P(X)$.\\\\
		$R$ è \begin{itemize}
			\item riflessiva perchè $$\forall Y \in P(X)$$ $$Y \subseteq Y$$ $$(Y,Y) \in R$$
			\item antisimmetrica perchè $$\forall Y,Z \in P(X)$$ $$(Y,Z) \in R \mbox{ e } (Z,Y) \in R \implies Y \subseteq Z \mbox{ e } Z \subseteq Y \implies Y=Z$$
			\item transitiva perchè $$\forall Y,Z,K \in P(X)$$ $$Y \subseteq Z \mbox{ e } Z \subseteq K \implies Y \subseteq K$$ $$(Y,Z) \in R \mbox{ e } (Z,K) \in R \implies (Y,K) \in R$$
		\end{itemize}
	\end{esempio}

\end{shaded}
\subsection{Relazione di Equivalenza}
\begin{definizione}
	Sia $R$ una relazione su un insieme $A$.\\
	Si dice che $R$ è una \textbf{relazione di equivalenza} se $R$ è
	\begin{itemize}
		\item riflessiva
		\item simmetrica
		\item transitiva
	\end{itemize}
\end{definizione}
\subsection{Relazione d'Ordine}
\begin{definizione}
	Sia $R$ una relazione su un insieme $A$.\\
	Si dice che $R$ è una \textbf{relazione d'ordine} parziale se $R$ è
	\begin{itemize}
		\item riflessiva
		\item antisimmetrica
		\item transitiva
	\end{itemize}
\end{definizione}

\section{Classi di Equivalenza}
\begin{definizione}
	Sia $A$ un insieme non vuoto\\ e sia $R$ una relazione di equivalenza su $A$.\\
	Per $a \in A$, si definisce \textbf{classe di equivalenza} di $a$ l'insieme $$[a]_{R} = \{b \in A | (a,b) \in R\}$$
\end{definizione}
\begin{nota}
	$[a]_{R}$ è un sottoinsieme di $A$.
\end{nota}
\begin{nota}
	$[a]_{R} \not = \emptyset$ perchè $R$ è riflessiva dunque $(a,a) \in R$ e pertanto $a \in [a]_{R}$.
\end{nota}
\begin{nota}
	Data $[a]_{R}$, $a$ si definisce \textbf{rappresentante} della classe di equivalenza.
\end{nota}
\begin{shaded}
	\begin{esempio}
		Sia $A = \{a,b,c,d\}$\\e sia $R=\{(a,a),(b,b),(c,c),(d,d),(a,d),(d,c),(a,c),(c,a),(d,a),(c,d)\}$.\\\\
		$$[a]_{R} = [c]_{R} = [d]_{R}= \{a,c,d\}$$
		$$[b]_{R} = \{b\}$$
	\end{esempio}
	\begin{nota}
		Se $$[a]_{R} = \{a,c,d\}$$ allora $$[a]_{R} = [c]_{R} = [d]_{R}$$
	\end{nota}
\end{shaded}
\begin{teorema}
	Sia $A$ un insieme non vuoto\\
	e sia $R$ una relazione di equivalenza.
	$$\forall a,b \in A \mbox{, } [a]_{R} = [b]_{R} \mbox{ oppure } [a]_{R} \cap [b]_{R} = \emptyset$$
	Due classi  di equivalenza o coincidono o non hanno elementi in comune.
	\begin{proof}
		È necessario dimostrare che se $[a]_{R} \cap [b]_{R} \not = \emptyset \implies [a]_{R} = [b]_{R}$\\\\

		$\exists c \in A$ con $c \in [a]_{R} \cap [b]_{R}$.\\
		Quindi $(c,a) \in R$ e $(c,b) \in R$.\\
		Ma $R$ è simmetrica $\implies (a,c) \in R$ e $(b,c) \in R$.\\
		Ma $R$ è transitiva $\implies (a,b) \in R$.\\\\

		Dimostro che $[a]_{R} = [a]_{R}$.\\
		\begin{itemize}
			\item $[a]_{R} \subseteq [b]_{R}$\\\\
			      Sia $x \in [a]_{R}$ allora $(a,x) \in R$\\
			      Io già conosco che $(b,a) \in R$. Per transitività anche $(b,x) \in R$.\\
			      Quindi $x \in [b]_{R}$\\\\
			      ... dal quale $[a]_{R} \subseteq [b]_{R}$.
			\item $[b]_{R} \subseteq [a]_{R}$\\\\
			      Sia $y \in [b]_{R}$ allora $(b,y) \in R$\\
			      Per riflessività anche $(y,b) \in R$.\\
			      Io già conosco che $(b,a) \in R$. Per transitività anche $(y,a) \in R$.\\
			      Per riflessività anche $(a,y) \in R$.\\
			      Quindi $y \in [a]_{R}$\\\\
			      ... dal quale $[b]_{R} \subseteq [a]_{R}$.
		\end{itemize}
	\end{proof}
\end{teorema}

\section{Insieme Quoziente}
\begin{definizione}
	Sia $A$ un insieme non vuoto\\\\
	e sia $R$ una relazione di equivalenza su $A$.
	L'\textbf{insieme quoziente} $A/R$ è definito come
	$$A/R = \{[a]_{R} \mbox{ }|\mbox{ } a \in R\}$$
\end{definizione}
\begin{shaded}
	\begin{esempio}
		Vedi precedente teorema (7).
		$$A/R = \{[a]_{R},[b]_{R}\}$$
	\end{esempio}
\end{shaded}
\begin{osservazione}
	Le relazioni \textbf{di equivalenza} si indicano anche con il simbolo $\sim$. Pertanto:
	\begin{itemize}
		\item $R$ si indica anche con $\sim$
		\item $(a,b) \in R, aRb$ si indica anche con $a \sim b$
		\item $A/R$ si indica anche con $A/\sim$
	\end{itemize}
\end{osservazione}

\section{Partizioni su un Insieme}
\begin{definizione}
	Sia $A$ un insieme.\\
	Una partizione $\mathcal{F}$ di $A$ è una collezione di sottoinsiemi di $A$ tale che \begin{enumerate}
		\item $\forall X \in \mathcal{F}, X \not = \emptyset$
		\item $\displaystyle\bigcup_{x \in \mathcal{F}} X = A$
		\item $\forall X,Y \in \mathcal{F} \mbox{ o } X=Y \mbox{ oppure } X \cap Y = \emptyset$
	\end{enumerate}
\end{definizione}
\begin{teorema}
	Ogni relazione di equivalenza $R$ su un insieme $A$ determina una partizione di $A$ (non vuoto), i cui elementi sono le classi di equivalenza.\\
	Viceversa, ogni partizione $\mathcal{F}$ di $A$ determina una relazione di equivalenza su $A$, le cui classi sono gli elementi di $\mathcal{F}$.
	\begin{proof}
		Sia $R$ una relazione di equivalenza su $A$.\\
		Ogni $a \in A$ appartiene a una e una sola classe di equivalenza rispetto a $R$. Infatti se $a \in [a]_{R}$ e $b \in [b]_{R}$, allora $[a]_{R} = [b]_{R}$.\\
		Quindi le classi di equivalenza sono gli elementi di una partizione di $A$
		$$\mathcal{F} = \{[a]_{R} \mbox{  } | \mbox{  } a \in A\}$$
		tale che $\displaystyle\bigcup_{a \in A} [a]_{R} = A$.\\\\
		Viceversa, sia $\mathcal{F}'$ una partizione di $A$ ed $R'$ una relazione di equivalenza su $A$ tale che
		$$\forall a,b \in A, (a,b) \in R' \iff \exists X \in \mathcal{F} \mbox{  } | \mbox{ } a,b \in X$$
		ovvero $a$ è in relazione con $b$ secondo $R'$ se e solo se esiste un elemento $X$ della partizione $\mathcal{F}$ che contiene sia $a$ che $b$.\\\\
		È immediato verificare che $R'$ è una relazione di equivalenza su $A$, le cui classi di equivalenza sono gli elementi di $\mathcal{F}$.\\\\
		Infine dimostro che $R'$ è una relazione di equivalenza poiché è è riflessiva, simmetrica, transitiva.
	\end{proof}
\end{teorema}
\begin{nota}
	Gli elementi $A/R$ (insieme quoziente) sono gli elementi della partizione determinata da R su A.\\Passare al quoziente significa identificare tra loro elementi equivalenti in $R$.
\end{nota}

\section{Proiezione Canonica}
\begin{definizione}
	Siano \begin{itemize}
		\item $A$ un insieme (non vuoto)
		\item $R$ una relazione di equivalenza su $A$
		\item $A/R = \{ [a]_{r} | a \in A \}$ l' insieme quoziente
	\end{itemize}
	la \textbf{\textbf{proiezione canonica di $A$ su $A/R$}} è $$\pi : A \longrightarrow A/R$$ $$a \longrightarrow [a]_{R}$$ cioè la funzione che associa ad ogni $a \in A$ la sua classe di equivalenza $[a]_{R}$.
\end{definizione}
\begin{nota}
	La proiezione canonica $\pi$ è una funzione \textit{suriettiva}, ma non iniettiva.
\end{nota}

\chapter{Equazioni Diofantee}
\begin{definizione}
	Una equazione diofantea è una equazione della forma $$ax+by=c$$ con
	\begin{itemize}
		\item $a,b,c \in \mathbb{Z}$
		\item $x,y$ sono incognite
		\item $a \not = 0, b \not = 0$
	\end{itemize}
	Vogliamo determinare, se esistono, delle soluzioni \underline{intere} dell'equazione, cioè coppie $$(x_0, y_0) \in \mathbb{Z} \times \mathbb{Z} $$ tali che $$ax_0+by_0=c$$
\end{definizione}
\begin{shaded}
	\begin{esempio}
		$4x+6y=9$ ha soluzioni?\\
		No. $4x+6y=9$ non ha soluzioni Perchè se esistesse $(x_0,y_0) \in \mathbb{Z} \times \mathbb{Z}$ con $4x_0+6y_0=9$ avrei che $2(2x_0+3y_0)=9$ ovvero $2|9$. Ma non è vero che $2|9$, essendo $9$ un numero dispari.
	\end{esempio}
	\begin{esempio}
		$6x+5y=3$ ha soluzioni?\\
		$6x+5y=3$ ha come soluzione, per esempio, $(3,-3)$ e $(8,-9)$.
	\end{esempio}
\end{shaded}
\begin{teorema}
	Sia $ax+by=c$ una equazione diofantea con $a,b,c \in \mathbb{Z}$ e $a\not=0,b\not=0$.\\
	Condizione necessaria e sufficiente affinché l'equazione abbia soluzioni è che $$(a,b)|c$$

	\begin{proof}
		Supponiamo che l'equazione diofantea $ax+by=c$ ammetta soluzioni. Quindi $$\exists (x_0,y_0) \in \mathbb{Z} \times \mathbb{Z}$$ tale che $$ax_0+by_0=c$$
		Posto $$d=(a,b)$$ so che $$d|a \mbox{ e } d|b$$ quindi $d$ divide ogni combinazione lineare a coefficienti interi di $a$ e $b$, compresa $ax_0+by_0$:
		$$d|ax_0+by_0$$
		Essendo $ax_0+by_0 = c$ otteniamo $$d|c$$ come volevamo.\\\\

		Viceversa sia $$d|c$$
		Quindi $$c=d\overline{c}$$ con $\overline{c} \in \mathbb{Z}$.
		Per l'identità di Bezout $\exists s,t$ tali che $$d=as+bt$$
		Moltiplicando per $\overline{c}$ ottengo
		$$c = d\overline{c} = (as+bt)\overline{c}$$
		$$c = as\overline{c}+bt\overline{c}$$
		$$c = a(s\overline{c})+b(t\overline{c})$$
		Pertanto $$(x_0 = s\overline{c}, y_0 = t\overline{c})$$ è una soluzione dell'equazione diofantea $ax+by=c$.
	\end{proof}
\end{teorema}
\begin{shaded}
	\begin{esempio}
		Determiniamo, se esiste, una soluzione dell'equazione diofantea $74x+22y=10$.\\\\
		\textbf{Calcolo il Massimo Comune Divisore} $(74,22)$\\
		$74 = 22 \cdot 3 + 8$\\
		$22 = 8 \cdot 2 + 6$\\
		$8=6 \cdot 1 + 2$\\
		$6 = 2 \cdot 3$\\
		Quindi $$(74,22)=2$$\\
		Poiché $2|10$ l'equazione ammette soluzioni.\\\\
		\textbf{Ricavo l'Identità di Bezout} $a=74, b=22$\\
		$8 = a- 3b$\\
		$6 = b-2 \cdot 8 = b -2(a-3b) = 7b-2a$\\
		$2 = 8-6=a-3b-(7b-2a)=3a-10b$\\
		Quindi $$(74,22) = 2 = 3a-10b$$
		Dato che $10 = 2 \cdot 5$, moltiplico l'identità di Bezout per 5
		$$10 = 15a -50b$$
		Di conseguenza, una soluzione di $74x+22y=10$ è $(15,-50)$.
	\end{esempio}
\end{shaded}
Come si determinano, se esistono, tutte le soluzioni dell'equazione diofantea $ax+by=c$?
\begin{teorema}
	Data l'equazione diofantea $ax+by=c$ con $a,b,c \in \mathbb{Z}$ e $a \not = 0, b \not = 0$.\\
	Supponiamo che se $d = (a,b)$ allora $d|c$.\\
	Sia $(x_0,y_0) \in \mathbb{Z} \times \mathbb{Z}$ una soluzione di $ax+by=c$.\\
	Allora tutte e sole le soluzioni di $ax+by=c$ sono date dalle coppie $(x_k,y_k)$, al variare di $k \in \mathbb{Z}$, dove
	$$x_k = x_0 + \frac{b}{d}k$$
	$$y_k = y_0 - \frac{a}{d}k$$
	\begin{nota}
		$$\overline{b} = \frac{b}{d} \in \mathbb{Z} \mbox{, } \overline{a} = \frac{a}{d} \in \mathbb{Z}$$
	\end{nota}
	\begin{proof}
		Dobbiamo provare che $ \forall k \in \mathbb{Z}, (x_k,y_k)$ è soluzione dell'equazione diofantea $ax+by=c$.
		Si ha
		$$ax_k+by_k = ax_0 + \frac{ab}{d}k + by_0 - \frac{ab}{d}k = ax_0+by_0$$
		Per ipotesi $(x_0,y_0)$ è soluzione, quindi $ax_0+by_0=c$.\\

		Viceversa, devo mostrare che ogni soluzione dell'equazione diofantea è di tipo $(x_k,y_k)$ per un certo $k \in \mathbb{Z}$.\\
		Sia $\overline{x}, \overline{y} \in \mathbb{Z} \times \mathbb{Z}$ una soluzione di $ax+by=c$. Quindi
		$$a\overline{x}+b\overline{y} = ax_0+by_0$$
		Da cui
		$$a(\overline{x}-x_0) = b(y_0-\overline{y})$$
		Dato $d=(a,b)$ e considerando $a=\overline{a}d, b=\overline{b}d$
		$$\overline{a}d(\overline{x}-x_0) = \overline{b}d(y_0-\overline{y})$$
		Divido per $d$ entrambi i membri
		$$\overline{a}(\overline{x}-x_0) = \overline{b}(y_0-\overline{y})$$
		Noto che
		$$\overline{b} \mbox{ } | \mbox{ } \overline{a}(\overline{x}-x_0)$$
		e sapendo che $(\overline{a}, \overline{b})=1$, allora
		$$\overline{b} \mbox{ } | \mbox{ } \overline{x}-x_0$$
		ovvero $\overline{x}-x_0 =\overline{b}h$, per $h \in \mathbb{Z}$\\
		Sostituendo trovo
		$$\overline{a}(\overline{x}-x_0) = \overline{b}(y_0-\overline{y})$$
		$$\overline{a}\overline{b}h = \overline{b}(y_0-\overline{y})$$
		$$y_0 - \overline{y} = \overline{a}h$$
		In tutto ho trovato
		$$\overline{x} = x_0+\overline{b}h=x_0+\frac{b}{d}h$$
		$$\overline{y} = y_0-\overline{a}h=y_0-\frac{a}{d}h$$
		Ho ricavato $\overline{y}$ direttamente, mentre $\overline{x}$ sostituendo in $\overline{a}(\overline{x}-x_0) = \overline{b}(y_0-\overline{y})$.\\
		Quindi una generica soluzione dell'equazione diofantea è nella forma voluta.
	\end{proof}
\end{teorema}
\begin{shaded}
	\begin{esempio}
		Determinare tutte le soluzioni di $74x+22y=10$.\\\\
		Dall'esempio 19 precedente, conosciamo che $(74,22) = 2$ e che $(15,-50)$ è una soluzione particolare dell'equazione diofantea.\\\\
		Tutte le soluzioni sono date dalle coppie $(x_k,y_k)$ con $k \in \mathbb{Z}$, dove
		$$x_k = x_0 + \frac{b}{d}k = 15+\frac{22}{2}k = 15 + 11k$$
		$$y_k = y_0 - \frac{a}{d}k = -50-\frac{74}{2}k = -50-37k$$
	\end{esempio}
\end{shaded}

\chapter{Stime Temporali}
\section{Somma}
\begin{esempio}
	Suppongo di voler sommare due numeri $n$ e $m$ scritti in base $2$
	$$n = (1111000)_2$$
	$$m = (11110)_2$$
	Aggiungo i $0$ a sinistra di $m$ affinché abbia lo stesso numero $k$ di bit di $n$. Procedo con la somma:
	$$ \begin{tabular}{r}
			${}^1 {}^1 {}^1 {}^0 {}^0 {}^0 {}^0$ \\
			$1111000$                            \\
			$0011110$                            \\
			\hline
			$10010110$
		\end{tabular} $$
\end{esempio}
Generalizziamo l'esempio.\\
Supponiamo di voler sommare $n$ con $k$ bit ed $m$ con $l$ bit; con $l \leq k$.\\\\
Possiamo assumere che $n$ ed $m$ abbiano entrambi $k$ bit, ovvero $l=k$. Se così non fosse, cioè $l<k$, basta aggiungere degli $0$ a sinistra nella scrittura di $m$.\\\\
Scriviamo $n$ sopra $m$ in colonna ed applichiamo la seguente procedura:
\begin{algoritmo}
	Fissiamo una singola colonna.
	\begin{enumerate}
		\item Guardiamo il bit della prima riga e il bit della seconda riga che appartengono alla colonna fissata e guardiamo eventuali riporti sopra il primo bit.
		\item Se entrambi i bit della colonna sono $0$ e non c'è alcun riporto, scriviamo $0$ nella riga del risultato e procediamo oltre, ovvero consideriamo la colonna immediatamente a sinistra di quella fissata.
		\item Se accade una e una sola delle seguenti eventualità
		      \begin{enumerate}
			      \item entrambi i bit della colonna fissata sono $0$ e c'è riporto
			      \item i bit della colonna fissata sono uno $0$, l'altro $1$ e non c'è riporto
		      \end{enumerate}
		      Scriviamo $1$ nella riga del risultato e procediamo oltre, ovvero consideriamo la colonna immediatamente a sinistra di quella fissata.
		\item Se accade una e una sola delle seguenti eventualità
		      \begin{enumerate}
			      \item entrambi i bit considerati sono $1$ e non c'è riporto
			      \item uno dei bit considerati è $0$ e l'altro è $1$ e c'è riporto
		      \end{enumerate}
		      Scriviamo $0$ nella riga del risultato, segniamo $1$ riporto e procediamo oltre, ovvero consideriamo la colonna immediatamente a sinistra di quella fissata.
		\item Se entrambi i bit considerati sono $1$ e c'è riporto scriviamo $1$ nella riga del risultato, segniamo $1$ riporto e procediamo oltre, ovvero consideriamo la colonna immediatamente a sinistra di quella fissata.
	\end{enumerate}
\end{algoritmo}
Eseguire questa procedura una volta si dice una \textbf{operazione bit}.
\begin{nota}
	Il tempo impiegato da un computer per effettuare un calcolo è proporzionale al numero di operazioni bit necessarie. La costante di proporzionalità dipende dal computer usato e non tiene conto del tempo necessario per operazioni di tipo amministrativo.
\end{nota}
Quindi sommare due numeri di $k$ bit significa eseguire $k$ operazioni.
\section{Moltiplicazione}
\begin{esempio}
	Suppongo di voler moltiplicare un numero $n$ di $k$ bit e un numero $m$ di $l$ bit scritti in base $2$ con $l \leq k$.
	$$n = (10011)_2$$
	$$m = (1011)_2$$
	procedo con la moltiplicazione
	$$ \begin{tabular}{r}
			$10011$                              \\
			$1011$                               \\
			\hline
			${}^1 {}^1 {}^1 {}^1 {}^1 {}^0 {}^0$ \\
			$10011$                              \\
			$10011/$                             \\
			$00000//$                            \\
			$10011///$                           \\
			\hline
			$11010001$
		\end{tabular} $$
\end{esempio}
Generalizziamo l'esempio.\\\\
Moltiplicando $n$ per $m$ ottengo $l' \leq l$ righe, una per ogni bit pari a $1$ nella scrittura di $m$.\\ Ciascuna riga corrisponde ad una copia di $n$ traslata a sinistra di una certa distanza.\\\\
Dobbiamo eseguire $l'-1$ somme.\\ Ogni somma parziale ha un numero di bit maggiore di $k$, perciò ciascuna somma comporta solo $k$ operazioni bit non banali (alcuni dei bit vanno solo, di passo in passo, ricopiati).\\\\
Le operazioni bit necessarie per la moltiplicazione sono $$(l'-1)k \leq (l-1)k < lk$$

\section{Notazione O-grande}
\begin{definizione}
	Siano $f,g : \mathbb{N}^{+} \rightarrow \mathbb{R}^{+}$\\
	Si dice che $f \in O(g)$ se esistono due costanti $B>0, C>0$ tali che $$\forall n > B, \mbox{  } f(n) < Cg(n)$$
\end{definizione}
\begin{osservazione}
	Se $f \in O(g)$ e $g \in O(h)$ allora $f \in O(h)$\\
	Quindi se $f \in O(g)$ posso rimpiazzare $g$ con una funzione che cresce più velocemente di $g$. Nella pratica però vogliamo scegliere $g$ in modo che la stima sia la migliore possibile per limitare $f$, preferendo funzioni $g$ che siano semplici da descrivere.
\end{osservazione}
\begin{osservazione}
	Se esiste finito $$ \lim_{n \to \infty} \frac{f(n)}{g(n)} $$ allora $f \in O(g)$.
\end{osservazione}
\begin{osservazione}
	Se $f(n)$ è un polinomio di grado $d$ con coefficiente diretto positivo, cioè se $$f(n) = a_{d} n^{d} + a_{d-1} n^{d-1} + \dots + a_{1} n + a_0$$ con $a_d > 0$, allora $f \in O(n^d)$.
\end{osservazione}
\begin{osservazione}
	Se $f(n)$ è la funzione che restituisce il numero di bit di $n$, per quanto visto in precedenza, si ha $f(n) \in O(\log n)$.\\
	La stessa stima vale per qualunque altra base $b$.
\end{osservazione}
La notazione di \textit{O-grande} può essere estesa a più variabili.
\begin{definizione}
	Siano $f,g : \mathbb{N}^{+} \times \mathbb{N}^{+} \times \dots \times \mathbb{N}^{+} \rightarrow \mathbb{R}^{+}$\\
	Si dice che $f \in O(g)$ se esistono due costanti $B>0, C>0$ tali che se $$n_j > B \mbox{ \space\space} \forall j = 1, \dots, r$$ si ha $$f(n_1,n_2,\dots,n_r) < Cg(n_1,n_2,\dots,n_r)$$
\end{definizione}
\begin{esempio}
	Riguardo i paragrafi di \textit{somma} (7.1) e \textit{moltiplicazioni} (7.2) di numeri interi positivi in base $2$. Abbiamo \begin{itemize}
		\item il tempo necessario a sommare due numeri di $k$ bit
		      $$Tempo((k \mbox{ bit}) + (k \mbox{ bit})) \in O(k)$$
		\item il tempo necessario a moltiplicare $k$ bit per $l$ bit
		      $$Tempo((k \mbox{ bit}) \cdot (l \mbox{ bit})) \in O(kl)$$
	\end{itemize}
	Se vogliamo esprimere il tempo intermini di $n$ ed $m$ anziché delle loro cifre binarie $k$ e $l$ abbiamo
	$$Tempo(n+m) \in O(max\{\log n,\log m\})$$
	$$Tempo(n\cdot m) \in O(\log n \cdot \log m)$$
	\begin{nota}
		Queste stime temporali valgono per una qualunque altra base $b$.
	\end{nota}
	\begin{nota}
		Per la moltiplicazione esistono algoritmi più efficienti di quello descritto.
	\end{nota}
\end{esempio}

\chapter{Congruenze}
\section{Congruenza modulo $n$}
\begin{definizione}
	Sia $n \in \mathbb{Z}$, $n \geq 1$.\\
	Si dice che $a,b \in \mathbb{Z}$ sono \textbf{congrui modulo $n$}, e scriviamo $$a \equiv b \bmod n$$
	se
	$$n | (a-b)$$
	cioè se $\exists k \in \mathbb{Z}$ tale che
	$$a-b=nk$$
\end{definizione}
\begin{osservazione}
	La definizione si può estendere ai casi: \begin{enumerate}
		\item [$n = 0$: ] si ha quindi che
		      $$a \equiv b \bmod 0$$
		      $$0|(a-b)$$
		      cioè se e solo se $$a-b = 0 \cdot k$$ per $k \in \mathbb{Z}$
		      ovvero solamente quando $$a=b$$\\
		      La congruenza modulo $0$ coincide con la relazione di uguaglianza in $\mathbb{Z}$.
		\item [$n < 0$: ] si ha quindi che
		      $$a \equiv b \bmod n$$
		      $$n | (a-b)$$
		      cioè se e solo se $$a-b=nk$$ per $k \in \mathbb{Z}$.
		      Ma allora è anche vero che $$a-b = (-n)(-k)$$
		      da cui $$a \equiv b \bmod -n$$
	\end{enumerate}
\end{osservazione}
\begin{teorema}
	Per ogni intero $n \geq 1$ la relazione di \textit{congruenza modulo $n$} definisce una \textbf{relazione di equivalenza} su $\mathbb{Z}$.

	\begin{proof}
		La congruenza modulo $n$ definisce su $\mathbb{Z}$ la relazione $R$ così definita
		$$\forall a,b \in \mathbb{Z}, \mbox{\space} aRb \iff a \equiv b \bmod n$$
		che gode della proprietà
		\begin{itemize}
			\item Riflessiva
			      $$\forall a \in \mathbb{Z}, \mbox{\space} a \equiv a \bmod n$$
			      Infatti $a-a=0=0 \cdot n$.
			\item Simmetrica
			      $$\forall a,b \in \mathbb{Z}, \mbox{\space se } a \equiv b \bmod n \mbox{ allora } b \equiv a \bmod n$$
			      Infatti
			      $$a \equiv b \bmod n$$ $$a-b=nk$$
			      implica
			      $$b-a=-nk = n(-k)$$
			      per un $k \in \mathbb{Z}$.
			\item Transitiva
			      $$\forall a,b,c \in \mathbb{Z} \mbox{\space se } a \equiv b \bmod n \mbox{ e } b \equiv c \bmod n \implies a \equiv c \bmod n$$
			      Infatti da \begin{enumerate}[label=\Roman*) ]
				      \item $a \equiv b \bmod n$\\
				            $a-b = nk, \mbox{\space} k \in \mathbb{Z}$
				      \item $b \equiv c \bmod n$\\
				            $b-c = nt, \mbox{\space} t \in \mathbb{Z}$
			      \end{enumerate}
			      Dalla (II) ricavo che
			      $$b = c+nt$$
			      Sostituendo alla (I) ottengo
			      $$a-(c+nt) = nk$$
			      $$\vdots$$
			      $$a-c = n(s+t)$$
			      Essendo $s+t \in \mathbb{Z}$ ho dimostrato che $a \equiv c \bmod n$.
		\end{itemize}
		Perciò, essendo $R$ una relazione riflessiva, simmetrica e transitiva allora $R$, per definizione, è una \textbf{relazione di equivalenza}.
	\end{proof}
\end{teorema}
\begin{shaded}
	\begin{esempio}
		Sia $n=2$.\\
		Allora $a \equiv b \bmod 2$ se, per definizione, $2|(a-b)$.\\\\
		Ad esempio, se $a=5$: $$5 \equiv b \bmod 2$$
		Noto che $b$ deve essere dispari affinché sia congruo a $5 \bmod 2$\\\\
		Invece, se $a=6$: $$6 \equiv b \bmod 2$$
		Noto che $b$ deve essere pari affinché sia congruo a $6 \bmod 2$
	\end{esempio}
	\begin{esempio}
		Sia $n = 3$.\\
		Allora esempi di $a,b$ congrui $\bmod 3$ sono
		$$9 \equiv 6 \bmod 3$$
		$$9 \equiv 9 \bmod 3$$
	\end{esempio}
	\begin{esempio}
		Sia $n = 5$.\\
		$$a = 0 \bmod 5 \longrightarrow a = 5, 10, 15, 20, \dots$$
		$$a = 1 \bmod 5 \longrightarrow a = 6, 11, 16, 21, \dots$$
		$$a = 2 \bmod 5 \longrightarrow a = 7, 12, 17, 22, \dots$$
		$$a = 3 \bmod 5 \longrightarrow a = 8, 13, 18, 23, \dots$$
		$$a = 4 \bmod 5 \longrightarrow a = 9, 14, 19, 24, \dots$$
	\end{esempio}
\end{shaded}
\begin{definizione}
	Le classi di equivalenza della congruenza modulo $n$ si dicono \textbf{classi di resto} modulo $n$.

	\begin{proof}
		Per $a \in \mathbb{Z}$, la classe di equivalenza di $a$ su $R$\\
		$$[a]_{R} = \{ b \in \mathbb{Z}, (a,b) \in R \}$$
		nel caso in cui $R$ sia la congruenza modulo $n$, la \textbf{classe di resto} di $a$ su $n$ è
		$$[a]_n = \{ b \in \mathbb{Z}, a \equiv b \bmod n \}$$
		ovvero
		$$[a]_n = \{ b \in \mathbb{Z}, n|a-b \}$$
		da cui $n|a-b \longrightarrow a-b=nk \longrightarrow b = a+n(-k)$ allora
		$$[a]_n = \{ b \in \mathbb{Z}, a+nk \mbox{\space} | \mbox{\space} k \in \mathbb{Z}\}$$
	\end{proof}
\end{definizione}
\begin{nota}
	Rispetto all'esempio 26 precedente, noto che non può esistere un numero che non sia congruente a nessuno tra $0 \bmod 5, 1 \bmod 5, 2 \bmod 5, 3 \bmod 1, 4 \bmod 5$!
\end{nota}
\begin{osservazione}
	Ogni intero è congruo modulo $n$ solamente ad uno degli interi $0,1,\dots,n-1$.

	\begin{proof}
		Sia $a \in \mathbb{Z}$.\\
		La divisione con resto fornisce
		$$a=nq+r$$ con $0 \leq r < n$. Dal quale trovo
		$$a-r=qr$$ ovvero proprio $$a \equiv r \bmod n$$
		cioè $$[a]_n = [r]_n$$\\
		Questo dimostra che ogni $a \in \mathbb{Z}$ è congruo modulo $n$ a uno degli interi $0,1,\dots,n-1$, ovvero tutti e i soli possibili resti.\\

		Viceversa i possibili resti non possono essere congrui modulo $n$ tra loro.\\Se $i,j \in \mathbb{Z}$, con $$0 \leq i < n$$ $$0 \leq j < n$$ assumendo $i \geq j$ ho che $$0 \leq i-j \leq n-1$$
		e quindi $$i − j = kn$$ se e solo se $k = 0$, cioè $$i = j$$
	\end{proof}
\end{osservazione}
\begin{definizione}
	L'\textbf{insieme quoziente} di $\mathbb{Z}$ rispetto alla relazione di congruenza modulo $n$ si indica con $\mathbb{Z}_n$ e rappresenta l'insieme delle classi dei resti modulo $n$:
	$$\mathbb{Z}_n = \{ [0]_n,[1]_n,\dots,[n-1]_n \}$$
\end{definizione}

\begin{shaded}
	\begin{esempio}
		Sia $n=5$.\\
		$$\mathbb{Z}_5 = \{ [0]_5,[1]_5,[2]_5,[3]_5,[4]_5  \}$$
		dove
		$$[0]_5 = \{ 0+5k \mbox{\space}|\mbox{\space} k \in \mathbb{Z} \}$$
		$$[1]_5 = \{ 1+5k \mbox{\space}|\mbox{\space} k \in \mathbb{Z} \}$$
		$$[2]_5 = \{ 2+5k \mbox{\space}|\mbox{\space} k \in \mathbb{Z} \}$$
		$$[3]_5 = \{ 3+5k \mbox{\space}|\mbox{\space} k \in \mathbb{Z} \}$$
		$$[4]_5 = \{ 4+5k \mbox{\space}|\mbox{\space} k \in \mathbb{Z} \}$$
	\end{esempio}
\end{shaded}
\begin{nota}
	$\mathbb{Z}_n$ è una partizione di $\mathbb{Z}$.
\end{nota}
\begin{shaded}
	\begin{esempio}
		Sia $n=2$.\\
		$$\mathbb{Z}_2 = \{ [0]_2,[1]_2  \}$$
		Noto che \begin{itemize}
			\item $[0]_2$ è la classe di equivalenza dei numeri pari
			\item $[1]_2$ è la classe di equivalenza dei numeri dispari
		\end{itemize}
	\end{esempio}
\end{shaded}
\begin{osservazione}
	Casi particolari:
	\begin{enumerate}
		\item [$n = 0$: ] la congruenza modulo $0$ è l'uguaglianza.\\\\
		      Sia $a \in \mathbb{Z}$, allora $[a]_0 = \{a\}$. Quindi le classi di equivalenza sono tante quanti gli elementi di $\mathbb{Z}$, ovvero infinite.
		\item [$n = 1$: ] la congruenza modulo $1$ è sempre verificata.\\\\
		      Dati $a,b \in \mathbb{Z}, \mbox{\space} 1|(a-b)$ sempre.\\
		      Sia $a \in \mathbb{Z}$, allora $[a]_1 = \mathbb{Z}$. Quindi ho una sola classe di equivalenza.
	\end{enumerate}
\end{osservazione}

\section{Congruenze lineari}
\begin{definizione}
	Una \textbf{congruenza lineare} è una congruenza della forma $$ax \equiv b \bmod n$$ dove \begin{itemize}
		\item $a,b \in \mathbb{Z}$
		\item $n \geq 1 \in \mathbb{Z}$
		\item $x$ è incognita
	\end{itemize}
	Si dice soluzione di $ax \equiv b \bmod n$ ogni $c \in \mathbb{Z}$ che soddisfa $$ac \equiv b \bmod n$$
\end{definizione}
\begin{shaded}
	\begin{esempio}
		La congruenza lineare $$2x \equiv 3 \bmod 7$$ ha soluzione $c=5$ perché $$2 \cdot 5 = 10 \equiv 3 \bmod 7$$
		In generale ogni $$c_k = 5+7k \mbox{\space} \in \mathbb{Z}$$ con $k \in \mathbb{Z}$ è soluzione.
	\end{esempio}
	\begin{esempio}
		La congruenza lineare $$2x \equiv 3 \bmod 4$$ non ha soluzioni.\\
		Se esistesse $c \in \mathbb{Z}$ con $$2c-3=4k$$ con $k \in \mathbb{Z}$, avremmo $$3 = 2c-4k$$ ovvero $2|3$ che è assurdo.
	\end{esempio}
\end{shaded}
\begin{teorema}
	Data la congruenza lineare $$ax \equiv b \bmod n$$
	Sia $d = (a,n)$ con $a = \overline{a}d$, $n = \overline{n}d$.
	\begin{enumerate}
		\item La congruenza lineare $ax \equiv b \bmod n$ ammette soluzioni se e solo se $d|b$.
		\item Se $c$ è una soluzione di $ax \equiv b \bmod n$ allora tutte e sole le soluzioni di $ax \equiv b \bmod n$ sono interi della forma $$c + k\overline{n}$$ al variare di $k \in \mathbb{Z}$, dove $\overline{n} = \frac{n}{d}$.\\\\
		      In particolare $ax \equiv b \bmod n$ ha esattamente $d$ soluzioni non congrue fra loro, modulo $n$.
	\end{enumerate}

	\begin{proof}
		La congruenza lineare
		$$ax \equiv b \bmod n$$
		ammette soluzione se e solo se $\exists c \in \mathbb{Z}$:
		$$ac \equiv b \bmod n$$
		quindi se e solo se $\exists c, k_0 \in \mathbb{Z}$ tale che
		$$ac = b +k_0 n$$
		da cui
		$$ac + n(-k_0) = b$$\\
		In tutto  la congruenza lineare $ax \equiv b \bmod n$ ammette soluzioni se e solo se l'equazione diofantea
		$$ax+ny=b$$
		ammette soluzioni.\\
		Ma, dalla teoria delle equazioni diofantee \textit{(vedi Teorema 9.)}, $ax \equiv b \bmod n$ ammette soluzioni se e solo se
		$$(a,n)|b$$ cioè $$d|b$$
		Il punto 1. è così dimostrato.\\\\
		Inoltre se $(c, -k_0)$ è soluzione di $ax+ny=b$ allora tutte e sole le soluzioni di $ax+ny=b$ sono le coppie $(x_k,y_k)$ con
		$$x_k=c+\frac{n}{d}k = c+\overline{n}k$$
		$$y_k=-k_0-\frac{a}{d}k = -k_0-\overline{a}k$$
		Quindi tutte e sole le soluzioni di $ax+b \bmod n$ sono gli interi $c+k\overline{n}$, $k \in \mathbb{Z}$.\\\\
		Infine, devo provare che prendendo $0 \leq k \leq d-1$, ottengo $d$ soluzioni nella forma $c+k\overline{n}$
		$$c, c+\overline{n}, c+2\overline{n}, \dots, c+(d-1)\overline{n}$$
		fra loro non congrue modulo $n$.\\
		Per assurdo, prendo $i \not = j$ con $0 \leq i,j \leq d-1$
		e ipotizzo che siano congrue modulo $n$
		$$c+i\overline{n} \equiv c+j\overline{n} \bmod n$$
		Ottengo che
		$$n | (c+i\overline{n} - (c+ j\overline{n}))$$
		$$n | (i-j)\overline{n}$$
		cioè
		$$(i-j)\overline{n} = n \cdot s \mbox{,\space\space }s \in \mathbb{Z}$$
		da cui, dato che $n = d\overline{n}$,
		$$(i-j)\overline{n} = d \cdot \overline{n} \cdot s$$
		$$(i-j) = d \cdot s$$
		Risulta che $(i-j)$ è multiplo di $d$, ma questo non può essere poiché $0 < i-j \leq d-1$ e l'unico multiplo di $d$ minore di $d-1$ è $0$: assurdo (poiché ho assunto $i \not = j$)!\\\\
		Quindi le soluzioni $c+k\overline{n}$ con $0 \leq k \in \mathbb{Z} \leq d-1$ non sono congrue modulo $n$ tra loro.\\\\
		Invece,se $c+k\overline{n}$ con $k \in \mathbb{Z} > d-1$, la divisione con resto porge
		$$k=dq+r$$ con $0 \leq r \leq d-1$. Da cui
		$$c+k\overline{n}$$
		$$c+(dq+r)\overline{n}$$
		$$c+dq\overline{n}+r\overline{n}$$
		ma $n = d\overline{n}$
		$$c+qn+r\overline{n}$$
		Dato che $0 \leq r \leq d-1$, si conclude che $c+k\overline{n}$ è congrua modulo $n$ ad una delle soluzioni sopra elencate $$c+k\overline{n} \equiv c+r\overline{n} \bmod n$$
		Anche il punto 2. è così dimostrato.

	\end{proof}
\end{teorema}
\begin{shaded}
	\begin{esempio}
		Trovo, se esistono, le soluzioni della congruenza lineare
		$$35x \equiv 23 \bmod 16$$
		Riduco i coefficienti. Poiché $$35 \equiv 3 \bmod 16 \mbox{\space \space e \space \space} 23 \equiv 7 \bmod 16$$
		allora la congruenza lineare di partenza equivale a
		$$3x \equiv 7 \bmod 16$$
		Calcolo il massimo comune divisore:
		$$16 = 3 \cdot 5 + 1$$
		$$3 = 1 \cdot 3 + 0$$
		$\implies d = (16,3) = 1$\\\\
		L'equazione diofantea associata è $3x+16y=7$\\
		$\longrightarrow d|b = 1|7$ quindi la congruenza lineare ha soluzioni.\\\\
		Ricavo l'identità di Bezout:
		$$1 = 16 \cdot 1 + 3 \cdot (-5)$$
		Moltiplicando per $7$ ottengo
		$$7 = 16 \cdot 7 + 3 \cdot (-35)$$
		$$3 \cdot (-35) = 7 - 16 \cdot 7$$
		Dunque una soluzione di $$3x \equiv 7 \bmod 16$$ è
		$$x_0 = -35$$
		mentre tutte le soluzioni sono nella forma
		$$x_k = -35 + 16k$$ con $k \in \mathbb{Z}$.
	\end{esempio}
	\begin{esempio}
		Trovo, se esistono, le soluzioni della congruenza lineare
		$$15x \equiv 6 \bmod 18$$
		L'equazione diofantea associata è $15x+18y=6$.
		Calcolo il massimo comune divisore:
		$$18 = 15 \cdot 1 + 3$$
		$$15 = 3 \cdot 5$$
		$\implies d = (18,15) = 3$.\\\\
		$\longrightarrow d|b = 3|6$ quindi la congruenza lineare ha soluzioni.\\\\
		Ricavo l'identità di Bezout:
		$$3 = 18 \cdot 1 + 15 \cdot (-1)$$
		Moltiplicando per $2$:
		$$6 = 18 \cdot 2 + 15 \cdot (-2)$$
		Una soluzione alla congruenza lineare $15x \equiv 6 \bmod 18$ è
		$$x_0 = -2$$
		Tutte le soluzioni sono della forma
		$$x_k = -2 + \frac{18}{3}k = -2 +6k$$ con $k \in \mathbb{Z}$.\\\\
		Solo $3$ tra queste soluzioni sono non congrue tra loro modulo 18, ovvero quelle con $k = 0,1,2$:
		$$x_0 = -2+6 \cdot 0 \qquad x_1 = -2+6 \cdot 1 \qquad x_2 = -2+6 \cdot 2$$
		$$x_0 = -2 \qquad\qquad x_1 = 4 \qquad\qquad x_2 = 10$$
	\end{esempio}
\end{shaded}

\section{Teorema Cinese del Resto}
Il \textit{Teorema Cinese del Resto} è utile per risolvere sistemi di congruenza.
\begin{teorema}[Teorema Cinese del Resto]
	Siano $$n_1, n_2, \dots , n_r \in \mathbb{Z}^+$$
	a due a due coprimi (cioè $(n_i, n_j)=1$ per $i \not = j$).
	E siano $$b_1, b_2, \dots , b_r \in \mathbb{Z}$$
	Il sistema
	$$\begin{cases}
			x \equiv b_1 \bmod n_1   \\
			x \equiv b_2 \bmod n_2 & \\
			\vdots                   \\
			x \equiv b_r \bmod n_r & \\
		\end{cases}$$
	è risolubile.\\\\
	Inoltre se $c$ e $c'$ sono due soluzioni del sistema, allora $$c \equiv c' \bmod N$$
	dove $$N = n_1 \cdot n_2 \cdot \dots \cdot n_r = \prod_{i = 1}^{r} n_i$$

	\begin{proof}
		Definiamo $$N_i = \frac{N}{n_i} = \prod_{j \not = i} n_j \mbox{\space} \forall i=1,\dots,n$$
		Poiché $(n_i, n_j) = 1$ per $i \not = j$ si ha che $(N_i, n_i) = 1$.\\\\
		La congruenza lineare $$N_i y \equiv 1 \bmod n_i$$ per $i = 1, \dots, r$, ammette soluzioni.\\\\
		Pongo $$c = \sum_{i = 1}^{r} N_iy_ib_i = N_1y_1b_1 + \dots + N_ry_rb_r$$
		allora $c$ è una soluzione del sistema di congruenze, cioè
		$$\forall j = 1 \dots r, \mbox{\space \space} c \equiv b_j \bmod n_j$$
		Infatti, fissato $j \not = i$: $$c \equiv N_jy_jb_j \bmod n_j$$
		ma $N_jy_j \equiv 1 \bmod n_j$ quindi $$c \equiv b_j \bmod n_j$$
		Ho dimostrato che $c$ è soluzione del sistema.\\\\
		Sia $c'$ un'altra soluzione del sistema, allora
		$$\forall j = 1 \dots r, \mbox{\space \space} c' \equiv b_j \bmod n_j$$
		ma so già che $c \equiv b_j \bmod n_j$ quindi
		$$\forall j = 1 \dots r, \mbox{\space \space} c \equiv c' \bmod n_j$$
		ovvero $\forall j = 1 \dots r$
		$$n_j | c-c'$$
		$$c-c' = k n_j, \mbox{\space \space } k \in \mathbb{Z}$$
		Per $j=1$
		$$c-c' = k_0 n_1, \mbox{\space \space } k_0 \in \mathbb{Z}$$
		ma per $j=2$
		$$c-c' = k_1 n_2, \mbox{\space \space } k_1 \in \mathbb{Z}$$
		Dato che $n_1$ ed $n_2$ sono coprimi tra loro, allora
		$$c-c' = k_2 n_1 n_2, \mbox{\space \space } k_2 \in \mathbb{Z}$$
		ma per $n_3$
		$$c-c' = k_3 n_3, \mbox{\space \space } k_3 \in \mathbb{Z}$$
		ed essendo $n_3$ coprimo con tutti gli altri
		$$c-c' = k_4 n_1 n_2 n_3, \mbox{\space \space } k_4 \in \mathbb{Z}$$
		$$\vdots$$
		Proseguendo in questo modo ottengo che
		$$c-c' = k n_1 n_2 n_3 \dots n_r, \mbox{\space \space } k \in \mathbb{Z}$$
		$$c-c' = k N \dots n_r, \mbox{\space \space } k \in \mathbb{Z}$$
		Cioè
		$$N | c - c'$$
		ovvero
		$$c \equiv c' \bmod N$$
		\begin{shaded}
			\begin{definizione}[Numero Primo]
				Un intero $p > 1$ si dice numero primo se $\forall a,b \in \mathbb{Z}$ se $p|ab$ allora $p|a$ o $p|b$.
			\end{definizione}
		\end{shaded}
		Chiamiamo $d = (N_i, n_i)$
		$$p|d \implies p|n_i \mbox{ e } p|N_i$$
		$$p|d \implies p|n_i \mbox{ e } p|\prod_{j \not = i} n_j$$
		$p$ è primo, quindi se $p|N_i$ allora divide uno qualsiasi dei suoi fattori: $p|n_{j_0}$
		$$p|d \implies p|n_i \mbox{ e } p|n_{j_0} \mbox{, } j_0 \not = i$$
		Ma $d$ deve essere uguale a $1$ poiché i moduli sono, per ipotesi, a due a due coprimi. Invece ho trovato un fattore di $n_i$ e di $n_{j_0}$ che è assurdo!
	\end{proof}
\end{teorema}
\begin{shaded}
	\begin{esempio}
		Risolvere il sistema
		$$\begin{cases}
				x \equiv 2 \bmod 3   \\
				x \equiv 3 \bmod 5 & \\
				x \equiv 2 \bmod 7 & \\
			\end{cases}$$
		Calcolo $$N = n_1n_2n_3 = 3 \cdot 5 \cdot 7 = 105$$ e poi
		$$N_1=\frac{N}{n_1}=n_2n_3=5\cdot7=35$$
		$$N_2=\frac{N}{n_2}=n_1n_3=3\cdot7=21$$
		$$N_3=\frac{N}{n_3}=n_1n_2=3\cdot5=15$$
		Risolvo le seguenti congruenze lineari
		$$N_1y \equiv 1 \bmod n_1 \rightarrow 35y \equiv 1 \bmod 3 \rightarrow 2y_1 \equiv 1 \bmod 3 \longrightarrow y_1=2$$
		$$N_2y \equiv 1 \bmod n_2 \rightarrow 21y \equiv 1 \bmod 5 \rightarrow y_2  \equiv 1 \bmod 5 \longrightarrow y_2=1$$
		$$N_3y \equiv 1 \bmod n_3 \rightarrow 15y \equiv 1 \bmod 7 \rightarrow y_3  \equiv 1 \bmod 7 \longrightarrow y_3=1$$
		Una soluzione del sistema è
		$$c = \sum_{i=1}^{3} N_iy_ib_i = N_1y_1b_1 + N_2y_2b_2 + N_3y_3b_3 = 35 \cdot 2 \cdot 2 + 21 \cdot 1 \cdot 3 + 15 \cdot1 \cdot2 = 233$$
		Ogni altra soluzione $c'$ è congrua a $233 \bmod 105$.\\
		Tutte e sole le soluzioni in $\mathbb{Z}$ sono
		$$233+105k, \mbox{\space\space} k \in \mathbb{Z}$$
		La minima soluzione positiva è $23 = 233-2\cdot105$
	\end{esempio}
\end{shaded}

\chapter{Strutture algebriche}
\section{Struttura algebrica}
\subsection{Operazione Binaria}
\begin{definizione}
	Sia $A$ un insieme non vuoto.\\
	Una \textbf{operazione \textit{binaria}} su $A$ è una funzione
	$$* : A \times A \longrightarrow A$$
	$$\mbox{\space\space\space\space\space\space\space} (a,b) \longrightarrow a * b$$
	In altre parole, è una regola per associare ad ogni coppia ordinata $(a,b)$ di elementi di $A$, uno e un solo elemento di $A$.
\end{definizione}
\subsection{Proprietà di una operazione \textbf{binaria}}
Una funzione
$$* : A \times A \longrightarrow A$$
si dice \begin{itemize}
	\item \underline{associativa}, se $$\forall a,b,c \in A \qquad (a*b)*c = a*(b*c)$$
	\item \underline{commutativa}, se $$\forall a,b \in A \qquad (a*b) = (b*a)$$
	\item dotata di \underline{elemento neutro}, se $$\exists e \in A: \quad \forall a \in A \qquad a*e=a=e*a$$
\end{itemize}
\subsection{Definizione di Struttura Algebrica}
\begin{definizione}
	Una struttura algebrica è un insieme non vuoto $A$ con una o più operazioni (binarie) su $A$.
\end{definizione}

\section{Gruppi}
\subsection{Definizione di Gruppo}
\begin{definizione}
	Una struttura algebrica $(G, *)$
	dove \begin{itemize}
		\item $G$ è un insieme non vuoto
		\item $*$ è un'operazione binaria su $G$
	\end{itemize}
	si dice \textbf{gruppo} se:
	\begin{enumerate}
		\item l'operazione $*$ è associativa, cioè $$\forall g,h,k \in G, \qquad (g*h)*k=g*(h*k)$$
		\item esiste un elemento neutro in $G$ rispetto all'operazione $*$, cioè $$\exists e \in G \quad | \quad \forall g \in G \qquad g*e=e=e*g$$
		\item ogni elemento di $G$ ha un inverso rispetto all'operazione $*$, cioè $$\forall g \in G \quad \exists g^{-1} \in G: \qquad g*g^{-1}=e=g^{-1}*g$$
	\end{enumerate}
\end{definizione}

\subsubsection{Gruppo abeliano}
\begin{definizione}
	Se $*$ è commutativo, il gruppo si dice \textbf{abeliano} o \underline{commutativo}.
\end{definizione}

\begin{shaded}
	\subsection{Esempi di Gruppo}
	\begin{esempio}
		$(\mathbb{Z}, +)$ è un gruppo.\\
		$$+: \quad \mathbb{Z} \times \mathbb{Z} \rightarrow \mathbb{Z}$$
		$$ \qquad \qquad (a,b) \rightarrow a+b$$
		In particolare è un gruppo abeliano con elemento neutro $0$ ed $-a$ inverso di $a$ rispetto a $+$.
	\end{esempio}
	\begin{esempio}
		$(\mathbb{Z}, \cdot)$ \underline{non} è un gruppo.\\
		Dato che \underline{non} tutti gli elementi di $\mathbb{Z}$ hanno inverso in $\mathbb{Z}$.
	\end{esempio}
	\begin{esempio}
		$(\mathbb{R}, \cdot)$ \underline{non} è un gruppo.\\
		$$\cdot: \quad \mathbb{R} \times \mathbb{R} \rightarrow \mathbb{R}$$
		$$ \qquad \qquad (a,b) \rightarrow a\cdot b$$
		Dato che $0$ non ha inverso in $\mathbb{R}$.
	\end{esempio}
	\begin{esempio}
		$(\mathbb{R}^{*} = \mathbb{R}-\{0\}, \cdot)$ è un gruppo.\\
		$$\cdot: \quad \mathbb{R} \times \mathbb{R} \rightarrow \mathbb{R}$$
		$$ \qquad \qquad (a,b) \rightarrow a\cdot b$$
		In particolare è un gruppo abeliano con elemento neutro $1$.
	\end{esempio}
	\begin{esempio}
		$(\mathbb{Q}^{*}, \cdot)$, come nel precedente, è un gruppo abeliano con elemento neutro $1$.
	\end{esempio}
	\begin{esempio}
		$(Mat(4 \times 4, \mathbb{Z}), \times)$ \underline{non} è un gruppo.
		$$\times: \quad Mat(4 \times 4, \mathbb{Z}) \times Mat(4 \times 4, \mathbb{Z}) \rightarrow Mat(4 \times 4, \mathbb{Z})$$
		$$ \qquad \qquad (A,B) \rightarrow A \times B$$
		Dato che $\times$ è associativa, esiste l'elemento neutro (matrice identità), ma \underline{non} ogni elemento ammette inverso.
	\end{esempio}
	\begin{esempio}
		Sia $$GL(n, \mathbb{Z}) = \{ A \in Mat(n, \mathbb{Z}) \quad | \quad det(A) \not = 0 \}$$ l'insieme delle matrici $n \times n$ a coefficienti interi con determinante diverso da $0$, \underline{non} è un gruppo
		$$GL(n, \mathbb{Z}) \times GL(n, \mathbb{Z}) \rightarrow GL(n, \mathbb{Z})$$
		dato che è associativa, ma l'inverso $\not\in GL(n, \mathbb{Z})$
	\end{esempio}
	\begin{esempio}
		Sia $$GL(n, \mathbb{R}) = \{ A \in Mat(n, \mathbb{R}) \quad | \quad det(A) \not = 0 \}$$ l'insieme delle matrici $n \times n$ a coefficienti reali con determinante diverso da $0$, è un gruppo (non abeliano) rispetto al prodotto tra matrici.
		$$GL(n, \mathbb{R}) \times GL(n, \mathbb{R}) \rightarrow GL(n, \mathbb{R})$$
		dato che è associativa e l'inverso $\in GL(n, \mathbb{R})$
		\begin{nota}
			Il gruppo $GL(n, \mathbb{R})$ si dice \textbf{gruppo generale lineare}.
		\end{nota}
	\end{esempio}
	\begin{esempio}
		Sia $\mathbb{R}^2 = \{ (x,y) \quad | \quad x,y \in \mathbb{R} \}$ uno spazio vettoriale.\\\\
		Somma
		$$+: \quad \mathbb{R}^2 \times \mathbb{R}^2 \rightarrow \mathbb{R}^2$$
		$$ \qquad \qquad (x_1,y_1), (x_2,y_2) \rightarrow (x_1+x_2, y_1+y_2)$$
		Prodotto Scalare
		$$\cdot: \quad \mathbb{R}^2 \times \mathbb{R}^2 \rightarrow \mathbb{R}^2$$
		$$ \qquad \qquad \alpha, (x_1,y_1) \rightarrow (\alpha x_1, \alpha y_1)$$
		Ha le seguenti proprietà:
		\begin{enumerate}
			\item esistenza del \underline{vettore nullo} $$\exists 0_v \in \mathbb{R}^2 \quad | \quad \forall v \in V \qquad v+0_v = v = 0_v+v$$
			\item \underline{commutatività} $$\forall v_1,v_2 \in V \qquad v_1+v_2=v_2+v_1$$
			\item \underline{associatività} $$\forall v_1,v_2,v_3 \in V \qquad (v_1+v_2)+v_3=v_1+(v_2+v_3)$$
			\item esistenza dell'\underline{elemento inverso} $$\forall v \in V \quad \exists (-v) \in V \qquad v+(-v) = 0_v$$
		\end{enumerate}
		Perciò $\mathbb{R}^2$ è un gruppo.
	\end{esempio}
	\begin{esempio}
		$(\mathbb{Z}_n, +)$ è un gruppo abeliano con elemento neutro $[0]_n$ e inverso di $[a]_n$ la classe $[n-a]_n$
	\end{esempio}
\end{shaded}

\section{Somma e Prodotto in $\mathbb{Z}_n$}
Definiamo le operazioni di somma e prodotto (di classi di resto) in $\mathbb{Z}_n$ come segue.
\begin{definizione}
	Dati $[a]_n, [b]_n \in \mathbb{Z}_n$.\\
	Somma $$[a]_n + [b]_n = [a+b]_n$$
	Prodotto $$[a]_n \cdot [b]_n = [a \cdot b]_n$$
\end{definizione}
\begin{shaded}
	\begin{esempio}
		In $\mathbb{Z}_5$\\
		$$[1]_5 + [3]_5 = [1+3]_5 = [4]_5$$
		$$[2]_5 \cdot [3]_5 = [2 \cdot 3]_5 = [6]_5$$
		\begin{nota}
			$$[1]_5 + [3]_5 = [4]_5$$ ma $[1]_5 = [6]_5$, quindi $$[6]_5 + [3]_5 = [9]_5$$ che è corretto dato che $[9]_5 = [4]_5$.\\\\
			Attenzione! Devo verificare che la definizione sia \underline{\textbf{ben posta}}, cioè che non dipenda dal rappresentante scelto per le classi di resto.
		\end{nota}
	\end{esempio}
\end{shaded}
\begin{teorema}
	Fissato $n \in \mathbb{Z}$ con $n \geq 1$.\\
	Siano $a,b,c,d \in \mathbb{Z}$, con $$[a]_n = [b]_n$$ $$[c]_n = [d]_n$$
	Allora
	$$[a]_n + [c]_n = [b]_n + [d]_n$$
	$$[a]_n \cdot [c]_n = [b]_n \cdot [d]_n$$

	\begin{nota}
		Le seguenti sono affermazioni equivalenti:
		$$a \equiv b \bmod n \qquad n | a-b \qquad [a]_n = [b]_n$$
	\end{nota}

	\begin{proof}
		Siano\\\\
		\begin{minipage}{0.45\textwidth}
			$$[a]_n = [b]_n$$
			$$a \equiv b \bmod n$$
			$$n|a-b$$
			$$a = b+nk, \quad k\in \mathbb{Z}$$
		\end{minipage}%
		\hfill
		\begin{minipage}{0.45\textwidth}
			$$[c]_n = [d]_n$$
			$$c \equiv d \bmod n$$
			$$n|c-d$$
			$$c = d+nh, \quad h\in \mathbb{Z}$$
		\end{minipage}%

		% ---------------
		Devo dimostrare che $[a]_n + [c]_n = [b]_n + [d]_n$.\\
		Quindi
		$$[a]_n + [c]_n = [a+c]_n$$
		ma $a = b+nk$ e $c=d+nh$
		$$[a]_n + [c]_n = [b+nk+d+nh]_n$$
		$$[a]_n + [c]_n = [b+d+n(k+h)]_n$$
		ma $[b+d+n(k+h)]_n = [b+d]_n$
		$$[a]_n + [c]_n = [b+d]_n$$
		$$[a]_n + [c]_n = [b]_n + [d]_n$$\\

		Devo dimostrare che $[a]_n \cdot [c]_n = [b]_n \cdot [d]_n$.\\
		Quindi
		$$[a]_n \cdot [c]_n = [ac]_n$$
		ma $a = b+nk$ e $c=d+nh$
		$$[a]_n \cdot [c]_n = [(b+nk)(d+nh)]_n$$
		$$[a]_n \cdot [c]_n = [bd+nkd+nhb+n^2kh]_n$$
		$$[a]_n \cdot [c]_n = [bd+n(kd+hb+nkh)]_n$$
		ma $[bd+n(kd+hb+nkh)]_n = [bd]_n$
		$$[a]_n \cdot [c]_n = [bd]_n$$
		$$[a]_n \cdot [c]_n = [b]_n \cdot [d]_n$$
	\end{proof}
\end{teorema}

\subsection{Proprietà di somma e prodotto in $\mathbb{Z}_n$}
\subsubsection{Proprietà della somma in $\mathbb{Z}_n$}
\begin{itemize}
	\item \underline{associativa}
	      $$\forall [a]_n, [b]_n, [c]_n \in \mathbb{Z}_n \qquad ([a]_n+[b]_n)+[c]_n = [a]_n+([b]_n+[c]_n)$$
	      \begin{proof}
		      $$([a]_n+[b]_n)+[c]_n = [a]_n+([b]_n+[c]_n)$$
		      $$[(a+b)+c]_n = [a+(b+c)]_n$$
		      È dimostrato per le proprietà della somma in $\mathbb{Z}$.
	      \end{proof}
	\item \underline{commutativa}
	      $$\forall [a]_n, [b]_n \in \mathbb{Z}_n \qquad [a]_n+[b]_n = [b]_n+[a]_n$$
	      \begin{proof}
		      $$[a]_n+[b]_n = [b]_n+[a]_n$$
		      $$[a+b]_n = [b+a]_n$$
		      È dimostrato per le proprietà della somma in $\mathbb{Z}$.
	      \end{proof}
	\item esistenza dell'\underline{elemento neutro}
	      $$\forall [a]_n \in \mathbb{Z} \qquad\quad \exists [b]_n \in \mathbb{Z} \quad | \quad [a]_n+[b]_n=[a]_n=[b]_n+[a]_n$$
	      \begin{nota}
		      $[b]_n = [0]_n$
	      \end{nota}
	\item esistenza dell'\underline{elemento inverso}
	      $$\forall [a]_n \in \mathbb{Z} \qquad\quad \exists [b]_n \in \mathbb{Z} \quad | \quad [a]_n+[b]_n=[0]_n=[b]_n+[a]_n$$
	      \begin{nota}
		      $[b]_n = [n-a]_n$
	      \end{nota}
\end{itemize}
\begin{nota}
	$\mathbb{Z}_n$ è un gruppo abeliano!
\end{nota}

\subsubsection{Proprietà del prodotto in $\mathbb{Z}_n$}
\begin{itemize}
	\item \underline{associativa}
	      $$\forall [a]_n, [b]_n, [c]_n \in \mathbb{Z}_n \qquad ([a]_n \cdot [b]_n) \cdot [c]_n = [a]_n \cdot ([b]_n \cdot [c]_n)$$
	      \begin{proof}
		      $$([a]_n \cdot [b]_n) \cdot [c]_n = [a]_n \cdot ([b]_n \cdot [c]_n)$$
		      $$[(a \cdot b) \cdot c]_n = [a \cdot (b \cdot c)]_n$$
		      È dimostrato per le proprietà del prodotto in $\mathbb{Z}$.
	      \end{proof}
	\item \underline{commutativa}
	      $$\forall [a]_n, [b]_n \in \mathbb{Z}_n \qquad [a]_n \cdot [b]_n = [b]_n \cdot [a]_n$$
	      \begin{proof}
		      $$[a]_n \cdot [b]_n = [b]_n \cdot [a]_n$$
		      $$[ab]_n = [ba]_n$$
		      È dimostrato per le proprietà del prodotto in $\mathbb{Z}$.
	      \end{proof}
	\item esistenza dell'\underline{elemento neutro}
	      $$\forall [a]_n \in \mathbb{Z} \qquad\quad \exists [b]_n \in \mathbb{Z} \quad | \quad [a]_n \cdot [b]_n=[a]_n=[b]_n \cdot [a]_n$$
	      \begin{nota}
		      $[b]_n = [1]_n$
	      \end{nota}
\end{itemize}

\subsubsection{Proprietà distributive in $\mathbb{Z}_n$}
\begin{itemize}
	\item $\forall [a]_n, [b]_n, [c]_n \in \mathbb{Z}_n \qquad [a]_n \cdot ([b]_n + [c]_n) = [a]_n \cdot [b]_n + [a]_n \cdot [c]_n$
	\item $\forall [a]_n, [b]_n, [c]_n \in \mathbb{Z}_n \qquad ([a]_n + [b]_n) \cdot [c]_n = [a]_n \cdot [c]_n + [b]_n \cdot [c]_n$
\end{itemize}

\section{Invertibili in $\mathbb{Z}_n$}
Data $[a]_n \in \mathbb{Z}_n$, esiste $[b]_n \in \mathbb{Z}_n$ con $[a]_n[b]_n=[1]_n$?
\begin{shaded}
	\begin{esempio}
		Sia $n=7$.\\
		\begin{minipage}{0.45\textwidth}
			$$[a]_7 = [3]_7$$
		\end{minipage}%
		\hfill
		\begin{minipage}{0.45\textwidth}
			$$[b]_7 = [5]_7$$
		\end{minipage}%

		$$[a]_7[b]_7=[3]_7[5]_7=[15]_7=[1]_7$$
	\end{esempio}
	\begin{esempio}
		Sia $n=6$ e sia $[a]_6 = [2]_6$.\\
		$$\mathbb{Z}_6= \{ [0]_6,[1]_6,[2]_6,[3]_6,[4]_6,[5]_6 \}$$
		Testo tutti gli elementi:

		\begin{minipage}{0.45\textwidth}
			$$[2]_6[0]_6=[0]_7$$
			$$[2]_6[1]_6=[2]_7$$
			$$[2]_6[2]_6=[4]_7$$\\
		\end{minipage}%
		\hfill
		\begin{minipage}{0.45\textwidth}
			$$[2]_6[3]_6=[0]_7$$
			$$[2]_6[4]_6=[2]_7$$
			$$[2]_6[5]_6=[4]_7$$\\
		\end{minipage} $\space$\\
		Concludo che $[2]_6$ non è invertibile in $\mathbb{Z}_6$.
	\end{esempio}
\end{shaded}
\begin{definizione}[Invertibilità]
	Un elemento $[a]_n \in \mathbb{Z}_n$ si dice \textbf{invertibile} (rispetto al prodotto) se esiste $[b]_n \in \mathbb{Z}_n$ tale che $$[a]_n[b]_n=[1]_n=[b]_n[a]_n$$
\end{definizione}
\begin{osservazione}
	Sia $[a]_n = [0]_n$.\\
	Cerchiamo $[b]_n \in \mathbb{Z}_n$ con $$[0]_n[b]_n=[1]_n$$
	Ma $[0]_n[b]_n=[0]_n$ $$[0]_n=[1]_n$$
	Ovvero $$n|1-0$$ $$n|1$$
	che è valida solo per $n=1$.\\\\
	Concludo quindi che se $n \geq 2$ allora $[0]_n$ \underline{non} è invertibile!
\end{osservazione}
Esiste un criterio per stabilire se una classe di $\mathbb{Z}_n$ è invertibile:
\begin{teorema}
	Fissati $a, n \in \mathbb{Z}$ con $n>1$.\\
	La classe $[a]_n \in \mathbb{Z}_n$ è \textbf{invertibile} se e solo se $$(a,n)=1$$

	\begin{proof}
		Suppongo che $[a]_n \in \mathbb{Z}_n$ sia invertibile.\\
		Quindi $\exists [b]_n \in \mathbb{Z}_n$ con $$[a]_n[b]_n=[1]_n$$
		Quindi
		$$[ab]_n = [1]_n$$
		$$ab \equiv 1 \bmod n$$
		$$n|ab-1$$
		$$ab=1+nk \qquad k \in \mathbb{Z}$$
		$$ab+n(-k) =1$$
		Posto $d = (a,n)$ allora $$d|a \qquad d|n$$ da cui $$d|ab \qquad d|n(-k)$$
		Di conseguenza $$d|ab+n(-k)$$ $$d|1$$
		Segue che $d=1$.

		Viceversa se $(a,n)=1$ per l'identità di Bezout $$\exists s,1 \in \mathbb{Z} \qquad 1=as+nt$$
		Ma allora $$as=1-nt$$ $$as \equiv 1 \bmod n$$ $$[as]_n= 1$$ $$[a]_n [s]_n = 1$$
	\end{proof}
\end{teorema}
\begin{osservazione}
	Se $[a]_n$ è invertibile allora il suo inverso è unico e si indica con $[a]_n^{-1}$
\end{osservazione}
\begin{shaded}
	\begin{esempio}
		In $\mathbb{Z}_{51}$, $[13]_{51}$ è invertibile, dato che $(13,51)=1$.
	\end{esempio}
	\begin{esempio}
		Gli elementi invertibili in $$\mathbb{Z}_8 = \{ [0]_8, [1]_8, [2]_8, [3]_8, [4]_8, [5]_8, [6]_8, [7]_8 \}$$ sono
		$$[1]_8 \qquad [3]_8 \qquad [5]_8 \qquad [7]_8$$
		I rispettivi inversi sono
		$$[1]_8 \qquad [3]_8 \qquad [5]_8 \qquad [7]_8$$
	\end{esempio}
	\begin{esempio}
		Gli elementi invertibili in $$\mathbb{Z}_7 = \{ [0]_7, [1]_7, [2]_7, [3]_7, [4]_7, [5]_7, [6]_7 \}$$ sono
		$$[1]_7 \qquad [2]_7 \qquad [3]_7 \qquad [4]_7 \qquad [5]_7 \qquad [6]_7$$
		I rispettivi inversi sono
		$$[1]_7 \qquad [4]_7 \qquad [5]_7 \qquad [2]_7 \qquad [3]_7 \qquad [6]_7$$

		\begin{nota}
			Sia $p \in \mathbb{Z}$ un numero primo.
			$$\mathbb{Z}_p = \{ [0]_p, [1]_p, \dots, [p-1]_p \}$$
			Gli invertibili in $\mathbb{Z}_p$ sono tutte le classi tranne $[0]_p$, ovvero
			$$\mathbb{Z}_p^{*} = \mathbb{Z}_p - \{ [0]_p \} = \{ [1]_p, \dots, [p-1]_p \}$$
		\end{nota}
	\end{esempio}
\end{shaded}
\begin{nota}
	Gli insiemi delle classi
	$$\mathbb{Z}_n = \{ [0]_n, [1]_n, \dots, [n-1]_n \}$$
	scritti in \underline{rappresentazione standard} possono essere egualmente scritti anche con la seguente \underline{\textbf{rappresentazione bilanciata}}
	$$\mathbb{Z}_n = \{ \Big[-\frac{n}{2}\Big]_n, \dots, [-1]_n, [0]_n, [1]_n, \dots, \Big[\frac{n}{2}\Big]_n \}$$
	\begin{shaded}
		\begin{esempio}
			L'insieme delle classi
			$$\mathbb{Z}_7 = \{ [0]_7, [1]_7, [2]_7, [3]_7, [4]_7, [5]_7, [6]_7 \}$$
			può essere egualmente rappresentato in modo bilanciato nel modo seguente
			$$\mathbb{Z}_7 = \{ [-3]_7, [-2]_7, [-1]_7, [0]_7, [1]_7, [2]_7, [3]_7 \}$$
		\end{esempio}
	\end{shaded}
\end{nota}

\chapter{Funzione di Eulero}
\section{Definizione della funzione di Eulero}
\begin{definizione}
	La \textbf{funzione di Eulero} $$\varphi: \mathbb{N}^* \rightarrow \mathbb{N}^*$$ è definita da\\
	$$\varphi(1)=1$$
	$$\varphi(n)=|\{ k \in \mathbb{Z} : 1 \leq k \leq n-1 \mbox{ e } (k,n)=1  \}|,\mbox{ per } n \geq 2$$
\end{definizione}
\begin{esempio}
	Calcolo $\varphi(8)$:\\
	Dato che $$\{ k \in \mathbb{Z} : 1 \leq k \leq 7 \mbox{ e } (k,8)=1 \} = \{ 1,3,5,7 \}$$
	trovo che
	$$\varphi(8)=|\{ k \in \mathbb{Z} : 1 \leq k \leq 7 \mbox{ e } (k,8)=1  \}| = 4$$
\end{esempio}

\section{Proprietà della funzione di Eulero}
Proprietà della funzione di Eulero:
\begin{enumerate}
	\item Se $p$ è un numero primo,
	      $$\varphi(p)=|\{ k \in \mathbb{Z} : 1 \leq k \leq p-1 \mbox{ e } (k,p)=1  \}| = p-1$$
	      \begin{proof}
		      Immediata dalla definizione di numero primo.
	      \end{proof}
	\item Se $p$ è un numero primo ed $m \geq 1$ numero naturale,
	      $$\varphi(p^m) = p^{m-1}(p-1)$$
	      \begin{proof}
		      Dalla definizione
		      $$\varphi(p^m)=|\{ k \in \mathbb{Z} : 1 \leq k \leq p^m-1 \mbox{ e } (k,p^m)=1 \}|$$\\
		      Riscrivo
		      $$\{ k \in \mathbb{Z} : 1 \leq k \leq p^m-1 \mbox{ e } (k,p^m)=1 \} = *$$
		      come differenza di
		      $$* = \{ 1,2,\dots, p^m \} - \{ k \in \mathbb{Z} : 1 \leq k \leq p^m \mbox{ e } (k,p^m) \not= 1 \}$$\\
		      So che $$|\{ 1,2,\dots, p^m \}| = p^m \mbox{ (elementi)}$$
		      e che $$|\{ k \in \mathbb{Z} : 1 \leq k \leq p^m \mbox{ e } (k,p^m) \not= 1 \}| = p^{m-1} \mbox{ (elementi)}$$\\
		      Quindi ho dimostrato che
		      $$\varphi(p^m)=|\{ k \in \mathbb{Z} : 1 \leq k \leq p-1 \mbox{ e } (k,p)=1  \}| = p^m - p^{m-1} = p^m(p-1)$$
	      \end{proof}
	\item $\varphi$ è moltiplicativa, cioè
	      $$\forall a, b \in \mathbb{N}^* \mbox{ con } (a,b)=1 \qquad \varphi(ab)=\varphi(a)\varphi(b)$$
	      \begin{proof}
		      Dalle definizione..
		      $$\varphi(a) =  |\{ r \in \mathbb{Z} | 1 \leq r \leq a-1  \mbox{ e } (r,a) =1 \}|$$
		      $$\varphi(b) =  |\{ s \in \mathbb{Z} | 1 \leq r \leq b-1  \mbox{ e } (s,b) =1 \}|$$
		      $$\varphi(ab) = |\{ c \in \mathbb{Z} | 1 \leq r \leq ab-1 \mbox{ e } (c,ab)=1 \}|$$
		      Siano $r,s \in \mathbb{Z}$ con\\
		      \begin{minipage}{0.45\textwidth}
			      $$1 \leq r \leq a-1$$
			      $$(a,r) =1$$
		      \end{minipage}%
		      \hfill
		      \begin{minipage}{0.45\textwidth}
			      $$1 \leq s \leq b-1$$
			      $$(s,b) =1$$
		      \end{minipage}%

		      Per il teorema Cinese del resto, il sistema di congruenze
		      $$\begin{cases}
				      x \equiv r \bmod a   \\
				      y \equiv s \bmod b & \\
			      \end{cases}$$
		      ammette soluzioni, tra le quali una e una sola soluzione $c$ compresa tra $1$ e $ab-1$.\\\\
		      Affermo che $(c,ab)=1$.\\
		      Perché se così non fosse, esisterebbe un numero $p$ primo tale che
		      $$p|(c,ab)$$
		      $$p|c \qquad \mbox{ e } \qquad p|ab$$
		      $$p|c \qquad \mbox{ e } \qquad p|a \mbox{ o } p|b$$\\
		      Suppongo che $p|a$ (e $p|c$).
		      Allora $$c \equiv r \bmod a$$
		      $$c = r+ah, \qquad h \in \mathbb{Z}$$
		      da cui $$p|r$$ $$p|c-ah$$ ma è assurdo che $p$ divida sia $r$ che $a$ dal fatto che so che $r$ e $a$ sono primi, $(r,a)=1$.\\
		      Concludo che $(c,ab)=1$.\\
		      Poiché ogni coppia di interi $r$ e $s$ dà luogo a un intero $c$ con $1 \leq t \leq ab-1$ e $(c,ab)=1$ abbiamo che $\varphi(a)\varphi(b) \leq \varphi(ab).$
		      \\\\
		      Viceversa, sia $c \in \mathbb{Z}$ con $1 \leq c \leq ab-1$ e $(c,ab)=1$.\\
		      Divido $c$ per $a$ e trovo
		      $$c = aq+r \qquad \mbox{ con } 0 \leq r < a$$\\
		      Non può essere $r=0$ perché altrimenti avremmo $c = aq \rightarrow a|c$, da cui $a|ab$ contro il fatto che $(c,ab)=1$.\\\\
		      Quindi
		      $$c = aq+r \qquad \mbox{ con } 1 \leq r < a$$\\
		      Devo mostrare che $r$ e $a$ sono coprimi. Affermiamo che $(r,a)=1$.\\
		      Posto $d = (r,a)$, si ha che $d|a$ e $d|r$. Da cui

		      \begin{minipage}{0.45\textwidth}
			      $$d|aq+r$$
			      $$\downarrow$$
			      $$d|c$$
			      $$\qquad \searrow$$
		      \end{minipage}%
		      \hfill
		      \begin{minipage}{0.45\textwidth}
			      $$d|a$$
			      $$\downarrow$$
			      $$d|ab$$
			      $$\swarrow \qquad$$
		      \end{minipage}%

		      $$d|(c, (a,b))$$
		      ma $(c, (a,b)) = 1$.\\\\
		      Concludo che $$\varphi(ab)=\varphi(a)\varphi(b) \qquad \mbox{ quando } (a,b)=1$$

	      \end{proof}

\end{enumerate}

Le proprietà della funzione di Eulero permettono di calcolarla facilmente.
Sia $n \geq 2$. Scrivo la sua fattorizzazione $$n = p_1^{e_1} p_2^{e_2} \dots p_r^{e_r}$$ con \begin{itemize}
	\item $p_i$ primo per $i=1\dots r$
	\item $p_i \not = p_j$ per $i \not = j$
	\item $e_i \geq 1$ per $i = 1 \dots r$
\end{itemize}
Osservo che $(p_1^{e_1}, (p_2^{e_2}, \dots, p_r^{e_r})) = 1$; posso utilizzare la proprietà 3 con
$$a=p_1^{e_1} \qquad b=p_2^{e_2} \dots p_r^{e_r}$$ quindi
$$\varphi(n) = \varphi(p_1^{e_1})\varphi(p_2^{e_2} \dots p_r^{e_r})$$\\
Nuovamente osservo che $(p_2^{e_2}, p_3^{e_3}, \dots, p_r^{e_r}) = 1$; posso utilizzare la proprietà 3 con
$$a=p_2^{e_2} \qquad b=p_3^{e_3} \dots p_r^{e_r}$$ quindi
$$\varphi(n) = \varphi(p_1^{e_1})\varphi(p_2^{e_2})\varphi(p_3^{e_3} \dots p_r^{e_r})$$
$$\vdots$$
Procedo in questo modo fino a trovare che $$\varphi(n) = \varphi(p_1^{e_1})\varphi(p_2^{e_2})\varphi(p_3^{e_3}) \dots \varphi(p_r^{e_r})$$

\begin{esempio}
	Sia $n=12 = 2^2 \cdot 3$.\\
	$$\varphi(12) = \varphi(2^2)\varphi(3)$$
	Dato che \begin{itemize}
		\item $\varphi(3)=2$ per la proprietà 1
		\item $\varphi(2^2)=2^1(2-1) = 2$ per la proprietà 2
	\end{itemize}
	Allora $$\varphi(12) = \varphi(2^2)\varphi(3)=2^1(2-1) \cdot 2 = 4$$
\end{esempio}

\begin{osservazione}
	$\varphi$ è iniettiva?
	\begin{shaded}
		\begin{definizione}[Funzione Iniettiva]
			Una funzione $f: \mathbb{N}^* \rightarrow \mathbb{N}$ è iniettiva se
			$$f(x_1)=f(x_2) \implies x_1=x_2$$
			oppure
			$$x_1 \not = x_2 \implies f(x_1) \not = f(x_2)$$
		\end{definizione}
	\end{shaded}
	Noto che
	$$\varphi(8) = |\{ [1]_8, [3]_8, [5]_8, [7]_8 \}| = 4$$
	$$\varphi(12) = |\{ [1]_8, [5]_8, [7]_8, [11]_8 \}| = 4$$\\
	$\implies \varphi$ \underline{non} è \underline{iniettiva}.
\end{osservazione}

\begin{osservazione}
	Siano invertibili in $\mathbb{Z}_{n>1}: [a]_n$ con $(a,n)=1$.\\
	Il numero di \underline{invertibili} in $\mathbb{Z}_n$ è $\varphi(n)$.
\end{osservazione}



\chapter{Teoremi di Fermat ed Eulero}
\section{Teorema di Fermat}
\subsection{Ultimo Teorema di Fermat}
\begin{teorema}[Ultimo Teorema di Fermat]
	Sia $n > 2, n \in \mathbb{N}$. Allora $$x^n + y^n = z^n$$ non ha soluzioni banali.
\end{teorema}
\subsection{Piccolo Teorema di Fermat}
\begin{teorema}[Piccolo Teorema di Fermat]
	Siano \begin{itemize}
		\item $p$ un numero primo
		\item $a \in \mathbb{Z}$
	\end{itemize}
	Allora $$a^p \equiv a \bmod p$$\\
	Inoltre se $p \not| a$ allora $$a^{p-1} \equiv 1 \bmod p$$

	\begin{proof}
		Supponiamo che $p \not | a$.\\
		Considero le classi di resto
		$$[0]_p, [a]_p, [2a]_p, \dots, [(p-1)a]_p$$
		Affermo che sono tra loro tutte distinte
		$$[ra]_p=[sa]_p \iff r=s$$ con $0 \leq r_1s \leq p-1$.\\
		Infatti
		$$[ra]_p=[sa]_p$$
		$$ra \equiv sa \bmod p$$
		$$p|(r-s)a$$
		$$p|r-s \qquad \mbox{ con } 0 \leq |r-s| \leq p-1$$
		ma l'unica possibilità è $$r-s=0$$ cioè $r=s$.\\\\
		Abbiamo quindi che l'insieme
		$$\{ [0]_p, [a]_p, [2a]_p, \dots, [(p-1)a]_p \}$$ coincide con $$\{ [0]_p, [1]_p, [2]_p, \dots, [(p-1)]_p \}$$
		dato che entrambi hanno $p$ classi di resto \textit{modulo p}.\\\\
		Eliminando la classe $[0]_p$ che compare in entrambi, l'insieme
		$$\{ [a]_p, [2a]_p, \dots, [(p-1)a]_p \}$$ coincide con $$\{ [1]_p, [2]_p, \dots, [(p-1)]_p \}$$\\
		Calcolo il prodotto degli elementi in entrambi gli insiemi
		$$ [a]_p \cdot [2a]_p \cdot \dots \cdot [(p-1)a]_p  = [(p-1)!]_p$$
		$$ [1]_p \cdot [2]_p \cdot \dots \cdot [(p-1)]_p = [(p-1)! a^{p-1}]_p$$\\
		Dato che gli insiemi coincidono, i prodotti dei loro elementi coincidono $$[(p-1)!]_p = [(p-1)! a^{p-1}]_p$$
		da cui
		$$(p-1)! \equiv (p-1)! a^{p-1} \bmod p$$
		$$p|(p-1)! (a^{p-1}-1)$$
		ma naturalmente $p \not | (p-1)!$ quindi
		$$p|(a^{p-1}-1)$$
		ovvero
		$$a^{p-1} \equiv 1 \bmod p$$
		$\longrightarrow [a]^{p-1}_p = [a^{p-1}]_p = [1]_p$.\\
		Abbiamo dimostrato che se $p \not | a$ allora $a^{p-1} \equiv 1 \bmod p$.\\\\
		Dimostriamo che se $a \in \mathbb{Z}$ allora $$a^p \equiv a \bmod p$$
		Infatti se \begin{itemize}
			\item $p|a$ ($a$ è multiplo di $p$) allora

			      \begin{minipage}{0.45\textwidth}
				      $$a \equiv 0 \bmod p$$
				      $$a^p \equiv 0 \bmod p$$
			      \end{minipage}%
			      \hfill
			      \begin{minipage}{0.45\textwidth}
				      $$\implies a^p \equiv a \bmod p$$
			      \end{minipage}%

			\item $p \not| a$ ($a$ non è multiplo di $p$) allora, per quanto già detto,
			      $$a^{p-1} \equiv 1 \bmod p$$
			      e, per la definizione di congruenza, $$a \equiv a \bmod p \qquad \mbox{ riflessività }$$
			      Utilizzando la proprietà seguente
			      \begin{shaded}
				      \begin{nota} $\forall a,b,c,d \in \mathbb{Z}$.
					      Se $$a \equiv b \bmod n \qquad c \equiv d \bmod n$$
					      Allora $$a+c \equiv b+d \bmod n \qquad ac \equiv bd \bmod n$$
				      \end{nota}
			      \end{shaded}
			      posso concludere che
			      $$a^p \equiv a \bmod p$$
		\end{itemize}

	\end{proof}
\end{teorema}

\section{Teorema di Eulero}
Una generalizzazione del Teorema di Fermat è dovuta a Eulero.
\subsection{Formula del Binomio di Newton}
\begin{definizione}[Formula del binomio di Newton]
	$$(a+b)^n = \sum_{k=0}^{n} \binom{n}{k} a^{k} b^{n-k}$$
	dove $\binom{n}{k} = \frac{n!}{k!(n-k)!}$, numero di sottoinsiemi di cardinalità $k$ in un insieme di cardinalità $n$.
\end{definizione}
\begin{nota}Noto che
	$$\binom{n}{k} = \binom{n-1}{k-1} + \binom{n-1}{k}$$
\end{nota}
\begin{nota}
	Noto che \\

	\begin{minipage}{0.45\textwidth}
		$$\binom{n}{k} = \binom{n}{n-k}$$
	\end{minipage}%
	\hfill
	\begin{minipage}{0.45\textwidth}
		$$\binom{n}{0} = \binom{n}{n} = 1$$
	\end{minipage}%

\end{nota}
\begin{nota}
	Se $p$ è un numero primo, i binomiali $$\binom{p}{k} = \frac{p!}{k!(p-k)!}$$ sono multipli di $p$.
	\begin{proof}
		Il numeratore $p!$ è multiplo di $p$.\\
		Il denominatore $k!(p-k)!$ non è multiplo di $p$ se $1 \leq k \leq p-1$.\\
		Perciò il multiplo di $p$ a numeratore non viene eliminato dal denominatore
		$\implies \binom{p}{k}$ è multiplo di $p$ quando $1 \leq k \leq p-1$.
	\end{proof}
\end{nota}

\subsection{Teorema di Eulero}
\begin{teorema}[Teorema di Eulero]
	Siano \begin{itemize}
		\item $n \geq 1, n \in \mathbb{Z}$
		\item $a \in \mathbb{Z}$
	\end{itemize}
	con $(a,n)=1$.\\
	Allora $$a^{\varphi(n)} \equiv 1 \bmod n$$

	\begin{osservazione}
		Il teorema di Eulero per $n=p$ primo diventa
		$$a^{p-1} \equiv 1 \bmod p$$
		con $(a,p)=1 \quad \rightarrow \quad p \not | a$
	\end{osservazione}

	\begin{proof}
		Divisa in due casi:
		\begin{enumerate}
			\item $n$ potenza di un numero primo: $$n=p^\alpha$$ con $\alpha \geq 1 \in \mathbb{Z}, p$ numero primo.\\\\
			      Per induzione su $\alpha$: \begin{enumerate}[label=\Roman*) ]
				      \item $\alpha = 1$: $n=p \rightarrow $ vero perché è il teorema di Fermat
				      \item $\alpha \geq 2$: assumiamo il teorema vero per $\alpha-1$ e lo proviamo per $\alpha$.\\
				            Suppongo vero
				            $$a^{\varphi(p^{\alpha-1})} \equiv 1 \bmod p^{\alpha-1}$$
				            con $(a, p^{\alpha-1}) = 1$.\\\\
				            Devo dimostrare che
				            $$a^{\varphi(p^{\alpha})} \equiv 1 \bmod p^{\alpha}$$
				            con $(a, p^{\alpha}) = 1$.\\\\
				            Considero $a \in \mathbb{Z}$ con $(a,p^\alpha) = 1$.\\
				            Allora logicamente $(a,p^{\alpha-1}) = 1$.\\\\
				            Per ipotesi induttiva
				            $$a^{\varphi(p^{\alpha-1})} \equiv 1 \bmod p^{\alpha-1}$$
				            quindi
				            $$p^{\alpha-1} | a^{\varphi(p^{\alpha-1})}-1$$
				            $$a^{\varphi(p^{\alpha-1})} = 1 + b \cdot p^{\alpha-1} \qquad \mbox{ con } b \in \mathbb{Z}$$
				            ma $\varphi(p^{\alpha-1}) = p^{\alpha-2}(p-1)$
				            $$a^{p^{\alpha-2} (p-1)} = 1+bp^{\alpha-1}$$\\
				            Elevo alla $p$ e trovo
				            $$\Big( a^{p^{\alpha-2} (p-1)} \Big)^p = \Big( 1+bp^{\alpha-1} \Big)^p$$
				            $$a^{p^{\alpha-1} (p-1)} = \Big( 1+bp^{\alpha-1} \Big)^p$$\\
				            Applico il binomio di Newton $\Big[ (a+b)^n = \sum_{k=0}^{n} \binom{n}{k} a^{k} b^{n-k} \Big]$ e diventa
				            $$a^{p^{\alpha-1} (p-1)} = \sum_{k=0}^{p} \binom{p}{k} (1^k b p^{\alpha-1})^k$$
				            ma $\binom{p}{k}$ è multiplo di $p$ (vedi Nota 33.)
				            $$a^{p^{\alpha-1} (p-1)} = 1 + \sum_{k=1}^{p-1} \binom{p}{k} (b p^{\alpha-1})^k + (bp^{\alpha-1})^p$$\\
				            So che \begin{itemize}
					            \item $(bp^{\alpha-1})^p$ è multiplo di $p^\alpha$
					            \item $(b p^{\alpha-1})^k$ è multiplo di $p^\alpha$
				            \end{itemize}
				            Quindi
				            $$a^{p^{\alpha-1} (p-1)} = 1+p^\alpha h, \qquad h \in \mathbb{Z}$$
			      \end{enumerate}
			      Ho dimostrato il caso 1: $$a^{\varphi(p^\alpha)} \equiv 1 \bmod p^a$$
			\item $n$ qualsiasi.\\\\
			      Scrivo $n$ in fattori primi
			      $$n = p_1^{\alpha_1} p_2^{\alpha_2} \dots p_r^{\alpha_r}$$
			      con $p_i \not= p_j$ per $i \not= j$ per definizione.\\\\
			      Sia $a \in \mathbb{Z}$ con $(a,n)=1$.
			      Allora $$(a, p_i^{\alpha_i}) \qquad i=1 \dots r$$\\
			      Dato che $p_i^{\alpha_i}$ è potenza di un numero primo, applico il punto 1 e ottengo
			      $$a^{\varphi(p_i^{\alpha_i})} \equiv 1 \bmod p_i^{a_i}$$
			      Conosco che
			      $$\varphi(n) = \varphi(p_1^{\alpha_1}) \varphi(p_2^{\alpha_2}) \dots \varphi(p_r^{\alpha_r})$$
			      è multiplo di $\varphi(p_i^{\alpha_i})$.\\\\
			      Quindi
			      $$a^{\varphi(n)} \equiv 1 \bmod p_i^{\alpha_i} \qquad \mbox{ per } i = 1 \dots r$$
			      Allora
			      $$p_i^{\alpha_i} | a^{\varphi(n)}-1 \qquad \forall i$$
			      \begin{nota}
				      Se $a|c$ e $b|c$ con $(a,b)=1$, allora $ab|c$.
			      \end{nota}
			      Dunque
			      $$n | a^{\varphi(n)} -1$$
			      $$p_1^{\alpha_1} p_2^{\alpha_2} \dots p_r^{\alpha_r} | a^{\varphi(n)} -1$$
		\end{enumerate}
	\end{proof}
\end{teorema}


\chapter{Potenze modulo $n$}
% da rifare meglio!
\section{Metodo dei quadrati ripetuti}
Algoritmo efficiente per calcolare $$a^n \bmod m$$\\
Scriviamo l'esponente $n$ in base $2$
ottenendo $$n=(d_{k-1} d_{k-2} \dots d_1 d_0)$$
cioè $$n = \sum_{i=0}^{k-1} = d_i 2^i$$\\
Costruiamo la seguente tabella
$$(n)_2 \qquad \qquad \qquad \qquad c_0 = 1$$
$$d_{k-1} \qquad c_1 \equiv c_{0}^{2} \cdot a^{d_{k-1}} \bmod m$$
$$d_{k-2} \qquad c_2 \equiv c_{1}^{2} \cdot a^{d_{k-2}} \bmod m$$
$$\vdots$$
$$d_{1} \qquad c_{k-1} \equiv c_{k-2}^{2} \cdot a^{d_{1}} \bmod m$$
$$d_{0} \qquad c_k \equiv c_{k-1}^{2} \cdot a^{d_{0}} \bmod m$$\\
Risulta $a^n \bmod m = c_k$

\begin{shaded}
	\begin{esempio}
		Calcoliamo con il metodo dei quadrati ripetuti
		$$3^{90} \bmod 91$$
		Scriviamo $90$ in base $2$:
		$$(90)_{10} = (1011010)_2$$
		Quindi
		$$(n)_2 \qquad \qquad \qquad \qquad c_0 = 1$$
		$$1 \longrightarrow \qquad c_1 \equiv c_{0}^{2} \cdot 3^1 = 3 \bmod 91$$
		$$0 \longrightarrow \qquad c_2 \equiv c_{1}^{2} \cdot 3^0 = 9 \bmod 91$$
		$$1 \longrightarrow \qquad c_3 \equiv c_{2}^{2} \cdot 3^1 = 9^2 \cdot 3 \equiv 61 \equiv -30 \bmod 91$$
		$$1 \longrightarrow \qquad c_4 \equiv c_{3}^{2} \cdot 3^1 = (-30)^2 \cdot 3 \equiv -30 \bmod 91$$
		$$0 \longrightarrow \qquad c_5 \equiv c_{4}^{2} \cdot 3^0 = (-30)^2 \equiv -10 \bmod 91$$
		$$1 \longrightarrow \qquad c_6 \equiv c_{5}^{2} \cdot 3^1 = (-10)^2 \cdot 3 \equiv 27 \bmod 91$$
		$$0 \longrightarrow \qquad c_7 \equiv c_{6}^{2} \cdot 3^0 = 27^2 \equiv 1 \bmod 91$$\\
		Risulta $$3^{90} \equiv 1 \bmod 91$$
	\end{esempio}
\end{shaded}


\chapter{Crittografia}
\section{Sistemi Crittografici}
Un sistema crittografico si può rappresentare come $$ \mathcal{P} \overset{\text{$f$}}{\longrightarrow} \mathcal{C} \overset{\text{$f^{-1}$}}{\longrightarrow} \mathcal{P} $$
dove \begin{itemize}
	\item $\mathcal{P} = $ insieme dei messaggi in chiaro, per esempio l’insieme delle lettere dell'alfabeto, tradotti in forma numerica. Possono darsi i casi: una lettera alla volta, blocchi di lettere in una volta (coppie, terne, k-ple di lettere). Il modo in cui si associano le lettere ai numeri può non essere segreto.
	\item $\mathcal{C} = $ insieme dei messaggi cifrati.
	\item $f= $ funzione di cifratura.
	\item $f^{-1} = $ funzione di decifratura = inversa di $f$.
\end{itemize}

\section{Mappe lineari affini}
Le \textit{mappe lineari affini} sono dei sistemi crittografici a \textbf{chiave simmetrica}.
\begin{esempio}
	Esempio di Mappa lineare affine:\\
	\begin{itemize}
		\item $\mathcal{P} = \mathbb{Z}_N$, ad esempio con $N = 26$ (lettere dell'alfabeto inglese)
		\item $\mathcal{C} = \mathbb{Z}_N$
		\item $f: \mathcal{P} \rightarrow \mathcal{C}$\\
		      $f(p) = p+b \qquad b \in \mathbb{Z}_N$ fissato
		\item $f^{-1}: \mathcal{C} \rightarrow \mathcal{P}$\\
		      $f^{-1}(c) = c-b$
	\end{itemize}
	\begin{nota}
		Se $b=3$, il sistema crittografico è conosciuto con il nome di \textbf{cifrario di Cesare}.
	\end{nota}
\end{esempio}
\begin{esempio}
	Esempio di Mappa lineare affine:\\
	\begin{itemize}
		\item $\mathcal{P} = \mathbb{Z}_N$, ad esempio con $N = 26$ (lettere dell'alfabeto inglese)
		\item $\mathcal{C} = \mathbb{Z}_N$
		\item $f: \mathcal{P} \rightarrow \mathcal{C} \qquad f: \mathbb{Z}_N \rightarrow \mathbb{Z}_N$\\
		      $f(p) = ap+b \qquad a,b \in \mathbb{Z}_N$ fissati
		      \begin{nota}
			      $a$ deve essere invertibile $\implies a \not = 0 \implies \exists a^{-1}$.
		      \end{nota}
		\item $f^{-1}: \mathcal{C} \rightarrow \mathcal{P}$\\
		      $f^{-1}(c) = a^{-1}(b-c)$
	\end{itemize}
\end{esempio}
\begin{shaded}
	\begin{esempio}[Esempio Numerico]
		\begin{itemize} $N = 26$, $a=3$, $b=3$.
			\item $\mathcal{P} = \mathbb{Z}_{26}$
			\item $\mathcal{C} = \mathbb{Z}_{26}$
			\item $f: \mathbb{Z}_{26} \rightarrow \mathbb{Z}_{26} \qquad f(p) = ap+b \qquad f(p) = 3p+3$
			\item $f^{-1}: \mathbb{Z}_{26} \rightarrow \mathbb{Z}_{26} \qquad f^{-1} = a^{-1} (c-b)$\\\\
			      Devo ricavare la funzione inversa $f^{-1} = a^{-1} (c-b)$.\\
			      Calcolo $(3, 26)$:
			      $$26 = 3 \cdot 8 + 2$$
			      $$3 = 2 \cdot 1 + 1$$
			      $$2 = 1 \cdot 2 + 0$$
			      $\rightarrow (3, 26) = 1 \implies 3$ è invertibile.

			      Identità di Bezout:
			      $$2 = 26 - 8 \cdot 3 = b - 8a$$
			      $$1 = 3  - 1 \cdot 2 = a - 1 \cdot (b-8a) = 9a-b$$

			      $$1 = 9 \cdot 3 - 1 \cdot 26$$

			      Trovo che $a^{-1} = 9$
			      $$9 \cdot 3 = 27 = 1$$

			      Quindi $f^{-1}(c) = 9 (c-3) = 9c -9 \cdot 3 = 9c -27$\\\\
		\end{itemize}

		Voglio crittare e decrittare $p_1=3$, $p_2=9$, $p_3=1$ e $p_4=15$:
		\begin{enumerate}
			\item $p_1 = 3$:\\
			      $$f(p_1) = f(3) = 3 \cdot 3 + 3 = 12 = c_1$$
			      $$f^{-1}(c_1) = f(12) = 9 \cdot 12 - 27 = 108 - 27 = 81 = 3 = p_1$$
			\item $p_2 = 9$:\\
			      $$f(p_2) = f(9) = 3 \cdot 9 + 3 = 30 = 4 = c_2$$
			      $$f^{-1}(c_2) = f(4) = 9 \cdot 4 - 27 = 36 - 27 = 9 = p_2$$
			\item $p_3 = 1$:\\
			      $$f(p_3) = f(1) = 3 \cdot 1 + 3 = 6 = c_3$$
			      $$f^{-1}(c_3) = f(6) = 9 \cdot 6 - 27 = 54 - 27 = 27 = 1 = p_3$$
			\item $p_4 = 15$:\\
			      $$f(p_4) = f(15) = 3 \cdot 15 + 3 = 48 = 22 = c_4$$
			      $$f^{-1}(c_4) = f(22) = 9 \cdot 22 - 27 = 198 - 27 = 171 = 15 = p_4$$
		\end{enumerate}

		\begin{nota}
			Si sotto-intendono le classi di resto!\\
			Si scrive $30 = 4$ solo perché $[30]_{26} = [4]_{26}$.
		\end{nota}

	\end{esempio}
\end{shaded}

\newpage
\section{RSA}
L'\textit{RSA} è un sistema crittografico a \textbf{chiave asimmetrica}.\\

\begin{minipage}{0.45\textwidth}

	$$A\mbox{lice}$$\linebreak[2]

	Sceglie due numeri primi $p$ e $q$ distinti e dispari.\\\\
	Calcola $N = p \cdot q$.\\\\
	Calcola $\varphi(N) = (p-1)(q-1)$.\\\\
	Sceglie $r \in \mathbb{Z}$ con ($r, \varphi(N))=1$.\\\\
	Calcola, con l'algoritmo di Euclide, $s,t \in \mathbb{Z}$ con $$1=rs+\varphi(N)t$$\\
	Pubblica la coppia $(r, N)$.\\\\\\\\\\\\\\\\\\\\\\

	Riceve il messaggio $a$ crittato da $B\mbox{ob}$.\\\\
	Calcola $$b = a^s \bmod N$$ e ritrova il messaggio originale $b$.


\end{minipage}%
\hfill
\begin{minipage}{0.45\textwidth}
	\begin{tabular}{|p{\textwidth}}

		$$B\mbox{ob}$$\linebreak[2]                                                             \\\\\\\\\\\\\\\\\\\\\\\\\\\\\\\\\\

		Vuole mandare ad $A\mbox{lice}$ il messaggio $b$, dove $b \in \mathbb{Z}$, $0 < b < N$. \\\\
		Calcola $$a = b^r \bmod N $$                                                            \\
		Spedisce $a$ ad $A\mbox{lice}$.                                                         \\\\\\\\\\\\\\\\\\
	\end{tabular}
\end{minipage}%

\newpage
\begin{proof}
	Perchè $A\mbox{lice}$ calcolando $a^s \bmod N$ ritrova $b$?\\
	Il motivo è il teorema di Eulero.
	\begin{enumerate}
		\item supponiamo che $(b, N) = 1$\\\\
		      Bob critta $b$ calcolando $b^r \bmod N = a$\\
		      Alice decritta $a$ calcolando $b = a^s \bmod N$\\\\

		      Alice sa che $$1 = rs + \varphi(N)t$$\\
		      Quindi $$b = b^1 \bmod N = b^{rs + \varphi(N)t} \bmod N = b^{rs} b^{\varphi(N)t} \bmod N$$\\
		      So che $b$ ed $N$ sono coprimi, per il teorema di Eulero $$b^{\varphi(N)} \equiv 1 \bmod N$$
		      Deriva che $$b^{\varphi(N)t} \equiv 1 \bmod N$$
		      Allora $b^{rs} b^{\varphi(N)t} \bmod N= b^{rs} \bmod N$:
		      $$b = b^1 \bmod N = b^{rs + \varphi(N)t} \bmod N= b^{rs} b^{\varphi(N)t} \bmod N=$$ $$= b^{rs} \bmod N = (b^r)^s \bmod N = a^s \bmod N$$

		\item supponiamo che $(b, N) \not = 1$\\\\
		      So che $N = p \cdot q$. Quindi, data la supposizione,
		      $$\mbox{o } (b,p) \not=1 \qquad \mbox{o } (b,q) \not=1 $$\\
		      Supponiamo $(b,p) \not=1$. Allora $p|b$,
		      $$b = k \cdot p \qquad \mbox{per un certo } k \in \mathbb{Z} < q$$
		      Le condizioni suddette non sono vere entrambe, perciò $(b,q)=1$, ovvero $q \not| b$.
		      Applico il teorema di Eulero a $b$ e $q$ (di Fermat poiché sono coprimi) e
		      $$b^{\varphi(q)} \equiv 1 \bmod q$$
		      $$b^{q-1} \equiv 1 \bmod q$$
		      A maggior ragione si ha
		      $$b^{\varphi(N)} \equiv 1 \bmod q$$
		      $$b^{(p-1)(q-1)} \equiv 1 \bmod q$$
		      da cui
		      $$b^{-t\varphi(q)} \equiv 1 \bmod q$$
		      Quindi trovo che
		      $$b^{-t \cdot \varphi(N)} = 1 + q \cdot n \qquad n \in \mathbb{Z}$$
		      Moltiplico questa ultima uguaglianza per $b$
		      $$b^{1-t \cdot \varphi(N)} = b + b \cdot q \cdot n \qquad n \in \mathbb{Z}$$
		      Dalla solita identità di Bezout $1 = rs + \varphi(N)t$ ho che $$1-\varphi(N)t = rs$$ quindi
		      $$b^{1-t \varphi(N)} = b + b q n \qquad n \in \mathbb{Z}$$
		      $$b^{rs} = b + bqn$$
		      $$b^{rs} = b + kpqn \qquad \mbox{ perchè } b=kp$$
		      $$b^{rs} = b + nkN$$ che è congruo modulo N...
		      $$b^{rs} \equiv b \bmod N$$
	\end{enumerate}
\end{proof}

\newpage

Supponiamo che una terza persona, $C\mbox{arl}$, intercetti il messaggio $a$ crittato che $B\mbox{ob}$ ha mandato ad $A\mbox{lice}$.\\

\begin{minipage}{0.20\textwidth}

	$$A\mbox{lice}$$\\\\\\\\\\\\\\\\\\\\\\\\\\\\\\\\\\\\\\\\\\\\

\end{minipage}%
\hfill
\begin{minipage}{0.50\textwidth}
	\begin{tabular}{|p{\textwidth}}

		$$C\mbox{arl}$$

		Intercetta il messaggio $a$ crittato che $B$ob ha spedito ad $A$lice.                            \\\\
		Conosce la coppia $(N, r)$ scelta da $A$lice poiché è pubblica.                                  \\\\
		Per tentare di decrittare il messaggio, $C$arl deve calcolare, come fa $A$lice,
		$$b = a^s \bmod N$$
		Ma $C$arl non conosce $s$.                                                                       \\
		$C$arl dovrebbe calcolare $s$ attraverso l'algoritmo delle divisioni successive da $\varphi(N)$. \\
		$C$arl dovrebbe calcolare $\varphi(N) = (p-1)(q-1)$                                              \\
		$C$arl conosce $N=pq$, ma se $p$ e $q$ sono numeri primi abbastanza grandi, da questa informazioni non può ricostruire $p$ e $q$.

		$C$arl non riesce a decrittare il messaggio.
	\end{tabular}
\end{minipage}%
\hfill
\begin{minipage}{0.20\textwidth}
	\begin{tabular}{|p{\textwidth}}

		$$B\mbox{ob}$$ \\\\\\\\\\\\\\\\\\\\\\\\\\\\\\\\\\\\\\\\\\\\\\
	\end{tabular}
\end{minipage}%

\newpage

\begin{osservazione}
	L'RSA si basa sul fatto che fattorizzare numeri primi impiega un tempo computazionale enorme se i numeri scelti sono abbastanza grandi.
\end{osservazione}

% inserire esempio di RSA

\newpage

\subsection{RSA per la \textbf{firma digitale}}

\begin{minipage}{0.45\textwidth}

	$$A\mbox{lice}$$\linebreak[2]

	Svolge tutti i calcoli del normale RSA ricavando così
	\begin{itemize}
		\item chiave pubblica $(N_a, r_a)$
		\item chiave pubblica $s_a$
	\end{itemize}

\end{minipage}%
\hfill
\begin{minipage}{0.45\textwidth}
	\begin{tabular}{|p{\textwidth}}

		$$B\mbox{ob}$$\linebreak[2]

		Svolge tutti i calcoli del normale RSA ricavando così
		\begin{itemize}
			\item chiave pubblica $(N_b, r_b)$
			\item chiave pubblica $s_b$
		\end{itemize}
	\end{tabular}
\end{minipage}\\\\\\%

1° caso: $N_a < N_b$\\\\
\begin{minipage}{0.45\textwidth}
	$A\mbox{lice}$ calcola
	$$F_a = F^{s_a} \bmod N_a$$
	e poi
	$$F_{a,b} = F_{a}^{r_{b}} \bmod N_{b}$$\\
	$A\mbox{lice}$ spedisce $F_{a,b}$ a $B\mbox{ob}$
\end{minipage}%
\hfill
\begin{minipage}{0.45\textwidth}
	\begin{tabular}{|p{\textwidth}}
		$B\mbox{ob}$ calcola
		$$F_{a} = F_{a,b}^{s_{b}} \bmod N_{b}$$
		e poi
		$$F = F_a^{r_a} \bmod N_a$$ \\
		E ricava la firma $F$ di $A\mbox{lice}$
	\end{tabular}
\end{minipage}\\\\\\%

2° caso: $N_b < N_a$\\\\
\begin{minipage}{0.45\textwidth}
	$A\mbox{lice}$ calcola
	$$F_b = F^{r_b} \bmod N_b$$
	e poi
	$$F_{a,b} = F_{b}^{s_{a}} \bmod N_{a}$$\\
	$A\mbox{lice}$ spedisce $F_{a,b}$ a $B\mbox{ob}$
\end{minipage}%
\hfill
\begin{minipage}{0.45\textwidth}
	\begin{tabular}{|p{\textwidth}}
		$B\mbox{ob}$ calcola
		$$F_{b} = F_{a,b}^{r_{a}} \bmod N_{a}$$
		e poi
		$$F = F_b^{s_b} \bmod N_b$$ \\
		E ricava la firma $F$ di $A\mbox{lice}$
	\end{tabular}
\end{minipage}\\%

\newpage

\chapter{Numeri Primi}

\begin{definizione}[Numero Primo]
	Un intero $p \in \mathbb{Z}, p>1$ si dice \textbf{primo} se
	$$p|ab \implies p|a \quad \mbox{o} \quad p|b \qquad\qquad a,b \in \mathbb{Z}$$
\end{definizione}

\begin{definizione}[Numero Irriducibile]
	Un intero $p \in \mathbb{Z}, p>1$ si dice \textbf{irriducibile} se
	$$a|p \implies a=\pm 1 \quad \mbox{o} \quad a=\pm p \qquad\qquad a \in \mathbb{Z}$$
\end{definizione}

\begin{teorema}
	Sia $p \in \mathbb{Z}$ con $p > 1$.\\
	Allora $p$ è \textbf{primo se e solo se} $p$ è \textbf{irriducibile}.

	\begin{proof}Nei due versi:
		\begin{itemize}
			\item $p$ primo $\rightarrow p$ irriducibile\\\\
			      Sia $a \in \mathbb{Z}$ con $$a|p$$ quindi $$p=ab \qquad \mbox{per un certo } b \in \mathbb{Z}$$
			      Ma $$p|p$$
			      $$p|ab$$ quindi
			      $$p|a \quad \mbox{o} \quad p|b$$
			      \begin{itemize}
				      \item $p|a$: dato che $p|a$ e $a|p \implies a = \pm p$
				      \item $p|b$: quindi
				            $$b=pc \qquad \mbox{per un certo } c \in \mathbb{Z}$$
				            Da cui derivo $$p=ab=apc$$
				            Essendo $p=p$ posso dedurre che
				            $$ac=1$$
				            $$a=\pm 1$$
			      \end{itemize}
			\item $p$ irriducibile $\rightarrow p$ primo\\\\
			      Supponiamo che $p|ab$ con $a,b \in \mathbb{Z}$, dunque
			      $$ab=pq \qquad \mbox{per un certo } q \in \mathbb{Z}$$
			      Sia $d=(a,p)$.
			      Deriva $d|p$.\\
			      Poiché $p$ è irriducibile
			      $$\mbox{o }\quad d=1 \quad\mbox{ o }\quad d=p$$
			      Nel caso \begin{itemize}
				      \item $d=p$ allora $p|a$
				      \item $d=1$ allora $\exists s,t \in \mathbb{Z}$
				            tale che $$1=as+pt$$
				            Moltiplicando per $b$ trovo
				            $$b = abs + pbt$$
				            da cui $$p|b$$
			      \end{itemize}
		\end{itemize}
	\end{proof}
\end{teorema}

\begin{lemma}
	Sia $p$ un numero primo.\\
	Se $p$ divide un prodotto di $m \geq 2$ numeri interi, allora $p$ divide almeno uno dei fattori.

	\begin{proof}
		Per induzione su $n$.
		Per \begin{itemize}
			\item $m = 2$: l'enunciato è vero(segue dalla definizione di numero primo).
			\item $m > 2$:\\
			      assumo $m>2$ e il risultato vero per $m-1$.
			      Supponiamo che $$p|a_1a_2 \dots a_m$$
			      Da cui $$p|(a_1a_2 \dots a_{m-1}) a_m$$
			      Allora $$\mbox{o} \quad p|a_1a_2 \dots a_{m-1} \quad \mbox{o} \quad p|a_m$$
			      Se \begin{itemize}
				      \item $p|a_m$ ho dimostrato la tesi.
				      \item $p|a_1a_2 \dots a_{m-1}$ devo procedere per induzione: $p|a_i \quad 1 \leq i \leq m-1$
			      \end{itemize}
		\end{itemize}
	\end{proof}
\end{lemma}

\subsection{Teorema della fattorizzazione unica}
Il \textit{teorema della fattorizzazione unica} è chiamato anche \textbf{teorema fondamentale dell'Aritmetica}.
\begin{teorema}[Teorema della fattorizzazione unica]
	Ogni numero intero $n \geq 2$ si può scrivere come prodotto di numeri primi (non necessariamente distinti).
	Tale fattorizzazione è essenzialmente unica, cioè se
	$$n = p_1 p_2 \dots p_s = q_1 q_2 \dots q_t$$
	allora $s=t$ e (a meno di cambiare l'ordine dei fattori) $p_i = q_i \quad \forall 1 \leq i \leq s$.

	\begin{proof}
		\underline{Esistenza della fattorizzazione}\\
		Per induzione su $n$.
		\begin{itemize}
			\item $n=2$ vero perchè $2$ è primo.
			\item $n>2$, allora \begin{itemize}
				      \item se $n=p$ numero primo, vero
				      \item se $n$ non è un numero primo, allora
				            $$n=ab \qquad \mbox{con } 1 < a,b < n$$
				            Per ipotesi induttiva
				            $$a=p_1p_2 \dots p_s \qquad b = q_1q_2 \dots q_t$$
				            con \begin{itemize}
					            \item $p_i \mbox{ primo } \quad 1 \leq i \leq s$
					            \item $q_j \mbox{ primo } \quad 1 \leq j \leq t$
				            \end{itemize}
				            Quindi
				            $$n = ab = p_1 p_2 \dots p_s q_1 q_2 \dots q_t$$
			      \end{itemize}
		\end{itemize}
		\underline{Unicità della fattorizzazione}\\
		Supponiamo che
		$$n = p_1p_2 \dots p_s \qquad p_i \mbox{ primo } \forall i$$
		$$n = q_1q_2 \dots q_t \qquad q_j \mbox{ primo } \forall j$$
		Quindi
		$$p_1p_2 \dots p_s = q_1q_2 \dots q_t$$
		Poiché
		$$p_1 | p_1 p_2 \dots p_s$$
		segue che
		$$p_1 | q_1 q_2 \dots q_t$$
		e quindi
		$$p_1 | q_j \qquad \mbox{ per almeno un } j \mbox{ con } 1 \leq j \leq t$$
		A meno di riordinare i fattori $q_1q_2 \dots q_t$, suppongo che
		$$p_1 | q_1$$ e pertanto $$p_1 = q_1$$
		Segue che
		$$p_1 p_2 \dots p_s = p_1 q_2 \dots q_t$$
		$$p_2 \dots p_s = q_2 \dots q_t$$
		$$\vdots$$
		Procedo nuovamente per induzione ricorsivamente ottenendo che
		$$s=t \qquad \mbox{ e } \qquad p_i = q_i \quad \mbox{ per } i = 1 \dots s$$

	\end{proof}
\end{teorema}

\subsection{Teorema di Euclide}
\begin{teorema}[Teorema di Euclide]
	Esistono infiniti numeri primi.

	\begin{proof}
		\underline{Per assurdo}, supponiamo che i numeri primi siano finiti e siano
		$$p_1, p_2, p_3, \dots, p_n$$
		Consideriamo il numero intero
		$$M = p_1p_2p_3 \dots p_n +1$$
		Ho che $M \geq 2$ e $M \in \mathbb{Z}$.
		Per il teorema fondamentale dell'Aritmetica, $M$ si scompone in prodotto di fattori primi, ovvero $\exists p$ con $p|M$.
		Ma i numeri primi sono tutti e soli $p_1, p_2, p_3, \dots, p_n$, quindi $p$ deve essere uno di questi.
		Perciò
		$$p=p_i \qquad \mbox{ per un certo } 1 \leq i \leq n$$
		Quindi
		$$p_i | M$$
		$$p_i | (p_1p_2 \dots p_n+1)$$
		Nella divisione di $M$ per $p_i$ il quoziente è $p_1p_2 \dots p_{i-1} p_{i+1} \dots p_n$ e il resto è $1$.\\
		Assurdo!
	\end{proof}
\end{teorema}

\begin{definizione}
	$\pi(n)$ conta i numeri primi da $1$ ad $n$.
	$$\pi(n) = | \{ p \quad | \quad p \mbox{ primo e } p \leq n \} |$$
\end{definizione}

\begin{nota}
	Noto che \begin{itemize}
		\item se $\pi(n) = \pi(n-1) \implies n$ \underline{non} è primo.
		\item se $\pi(n) = \pi(n-1)+1 \implies n$ è primo.
	\end{itemize}
\end{nota}

\subsection{Teorema di Euclide}
\begin{teorema}[Teorema dei numeri primi]
	La densità media dei numeri primi tra $1$ e $n$ è asintoticamente uguale a
	$$\frac{1}{\ln n}$$
	ovvero\\
	$$\lim\limits_{n \to +\infty} \frac{\pi(n)}{\frac{n}{\ln n}} = 1$$
\end{teorema}

\section{Test di Primalità}
Si tratta di un test in grado di dirci quando un numero intero positivo è primo.\\\\
Per determinare un numero primo di data grandezza scegliamo random un numero $n$ intero della grandezza voluta;
\begin{itemize}
	\item se $n$ è pari, considero $n+1$.
	\item se $n$ è dispari applico il test a $n$, $n+2$, $n+4$, \dots
\end{itemize}
fin quando non trovo un numero primo, che sarà il più piccolo numero primo $\geq n$.\\

\begin{osservazione}
	Una conseguenza del teorema dei numeri primi è che dopo circa $\ln n$ trovo un numero primo.
\end{osservazione}

\begin{definizione}[Test deterministico]
	Considero $3 \dots \lfloor \sqrt{n} \rfloor$
	e verifico se dividono o no $n$.
	Se $n$ non è primo, allora
	$$n = ab \qquad 1 < a,b < n $$
\end{definizione}
\begin{nota}
	Se $n = ab$ con $a>\sqrt{n}$ e $b>\sqrt{n}$ allora $ab>n$, assurdo.
\end{nota}

\begin{definizione}[Test probabilistico]
	Test che risponde con certezza quando $n$ \underline{non} è un numero primo. \\
	Invece mostra che $n$ è primo \underline{non} con certezza, ma solo con una certa probabilità.
\end{definizione}

\subsection{Pseudoprimi di Fermat}

\begin{definizione}
	Sia \begin{itemize}
		\item $n>1$ un intero dispari
		\item $b \in \mathbb{Z}$ con $(b,n)=1$
	\end{itemize}
	Se $b^{n-1} \equiv 1 \bmod n$
	Allora $n$ è \textbf{pseudoprimo} (di Fermat) rispetto alla base $b$.
\end{definizione}

\begin{osservazione}
	La definizione di pseudoprimo è giustificata dal \underline{Piccolo teorema di Fermat}.\\
	Infatti se $p$ è primo e $b \in \mathbb{Z}$ con $(b,p)=1$ (cioè con $p \not | b$), il Piccolo teorema di Fermat assicura che
	$$b^{p-1} \equiv 1 \bmod p$$
\end{osservazione}

\begin{osservazione}
	Se $p$ è primo, allora $p$ è pseudoprimo rispetto ad ogni $b \in \mathbb{Z}$ con $(b,p)=1$.
\end{osservazione}

\begin{osservazione}
	Ogni intero dispari $n>1$ è pseudoprimo rispetto alle basi (banali) $b=\pm1$ (questo perché è $n-1$ è pari).
\end{osservazione}

\begin{nota}
	Dato $n>1$ intero dispari, $b \in \mathbb{Z}$ con $(b,n)=1$
	\begin{itemize}
		\item se $b^{n-1} \not\equiv 1 \bmod n$, allora $n$ \underline{non} è primo.
		\item se $b^{n-1} \equiv 1 \bmod n$, allora $n$ è primo.
	\end{itemize}
\end{nota}

\begin{shaded}
	\begin{esempio}
		Sia $n = 91$.\\\\
		$n$ è pseudoprimo rispetto alla base $b=3$.\\
		$n$ \underline{non} è pseudoprimo rispetto alla base $b=2$.\\\\
		Verifico che
		$$3^{90} \equiv 1 \mod 91$$
		$90_{10} = (1011010)_{2}$, quindi\\\\
		$1 \rightarrow c_1 = 1^2 \cdot 3^1 = 3 \bmod 91$\\
		$0 \rightarrow c_2 = 3^2 \cdot 3^0 = 9 \bmod 91$\\
		$1 \rightarrow c_3 = 9^2 \cdot 3^1 = 81 \cdot 3 = -30 \bmod 91$\\
		$1 \rightarrow c_4 = (-30)^2 \cdot 3^1 = 900 \cdot 3 = -30 \bmod 91$\\
		$0 \rightarrow c_5 = (-30)^2 \cdot 3^0 = 900 = -10 \bmod 91$\\
		$1 \rightarrow c_6 = (-10)^2 \cdot 3^1 = 300 = 27 \bmod 91$\\
		$1 \rightarrow c_7 = (27)^2 \cdot 3^0 = 1 \bmod 91$\\\\
		Verifico che
		$$2^{90} \not\equiv 1 \mod 91$$
		Quindi\\\\
		$1 \rightarrow c_1 = 1^2 \cdot 2^1 = 2 \bmod 91$\\
		$0 \rightarrow c_2 = 2^2 \cdot 2^0 = 4 \bmod 91$\\
		$1 \rightarrow c_3 = 4^2 \cdot 2^1 = 16 \cdot 2 = 32 \bmod 91$\\
		$1 \rightarrow c_4 = 32^2 \cdot 2^1 = 1024 \cdot 2 = 46 \bmod 91$\\
		$0 \rightarrow c_5 = 46^2 \cdot 2^0 = 2114 = 23 \bmod 91$\\
		$1 \rightarrow c_6 = 23^2 \cdot 2^1 = 529 \cdot 2 = 1058 = 57 = -34 \bmod 91$\\
		$1 \rightarrow c_7 = (-34)^2 \cdot 2^0 = 1156 = 246 = 64 \bmod 91 \not\equiv 1 \bmod 91$
	\end{esempio}
\end{shaded}

\subsubsection{Proprietà degli Pseudoprimi di Fermat}

\begin{osservazione}
	Sia
	$$b^{n-1} = 1 \bmod n \qquad (b,n)=1 \quad 0<b<n$$
	$\implies$ ho $\varphi(n)$ possibili basi.
\end{osservazione}

\begin{teorema}
	Per ogni numero intero $b>1$ esistono infiniti numeri composti che sono pseudoprimi rispetto alla base $b$.

	\begin{proof}
		Sia $p$ un numero primo dispari con $p \not| b$ e $p \not|  b^2-1$. Osserviamo che esistono infiniti numeri primi con queste proprietà.
		Sia\\
		$$n=\frac{ b^{2p} -1 }{b^2 -1} = \frac{ (b^{p})^2 -1 }{b^2 -1} = \frac{b^p - 1}{b-1} \cdot \frac{b^p + 1}{b+1}$$
		Ora
		$$\frac{b^p-1}{b-1} = \underbrace{b^{p-1} + b^{p-2} + \dots + b+1}_{\in \mathbb{Z}} > 1$$
		e
		$$\frac{b^p+1}{b+1} = b^{p-1} - b^{p-2} + b^{p-3} - b^{p-4} + \dots + b^2 -b+1$$
		$$ = \underbrace{b^{p-2}(b-1) + \dots + b(b-1) +1}_{\in \mathbb{Z}} > 1$$
		quindi $n$ è un numero composto.\\
		Inoltre
		$$n=\frac{ b^{2p} -1 }{b^2 -1} = \frac{ (b^{2})^p -1 }{b^2 -1} = (b^2)^{p-1}  + (b^2)^{p-2} + \dots + b^2 +1$$
		da cui
		$$n-1 = (b^2)^{p-1}  + (b^2)^{p-2} + \dots + b^2$$
		Segue che $n-1$ è somma di $p-1$ termini, con $p-1$ pari, che sono tutti pari se $b$ è pari oppure tutti dispari se $b$ è dispari.
		In tutto $n-1$ è pari cioè $2|n-1$ (e $n$ è dispari).\\\\
		Poi
		$$(n-1)(b^2-1) = n(b^2-1) - (b^2-1) = b^{2p} -1-b^2+1=b^{2p} -b^2 = b^2(b^{2p-2} -1)$$
		Per il teorema di Fermat $b^{p-1} \equiv 1 \bmod p$ e pertanto
		$$b^{2p-2} = (b^{p-1})^2 \equiv 1^2 \equiv 1 \bmod p$$
		cioè $$p|b^{2p-2}-1$$
		Quindi $p|(n-1)(b^2 -1)$ e $p \not | b^2 -1$ per ipotesi.\\
		Segue che $p|n-1$.\\
		Abbiamo allora $n-1 = 2pk$, $k \in \mathbb{Z}$ (notare che $p$ è dispari).\\
		Mostriamo che $n$ è pseudoprimo rispetto alla base $b$.\\
		Innanzitutto
		$$n=\underbrace{(b^2)^{p-1} + (b^2)^{p-2}+\dots+b^2}_{\mbox{multiplo di }b}+1$$
		dunque $(b,n)=1$.\\
		Poi $n(b^2-1) = b^{2p}-1$ cioè $n|b^{2p}-1$ ovvero $b^{2p} \equiv 1 \bmod n$.\\
		Allora
		$$b^{n-1} = b^{2pk} = (b^{2p})^k \equiv 1^k = 1 \bmod n$$
		La tesi segue dal fatto che abbiamo infinite scelte per $p$ numero primo dispari con $p \not | b$ e $p \not | b^2 -1$.


	\end{proof}
\end{teorema}

\begin{teorema}
	Sia $n>1$ un intero composto dispari.
	Se $n$ \underline{non} è pseudoprimo rispetto ad almeno una base $\overline{b}$, allora $n$ \underline{non} è pseudoprimo per almeno la metà delle basi possibili.

	\begin{proof}
		\begin{nota}
			Considero sempre le basi in $\mod n$.
		\end{nota}
		\begin{enumerate}
			\item Se $n$ è pseudoprimo rispetto alle basi $a$ e $b$, allora $n$ è pseudoprimo rispetto alle basi $ab$ e $ab^{-1}$ (dove $b^{-1}$ è l'inverso di $b \mod n$).\\\\
			      Infatti
			      $$(a,b)^{-1} = a^{n-1} b^{n-1} \equiv 1 \cdot 1 = 1 \bmod n$$
			      $$(a,b^{-1}) = a^{n-1} (b^{-1})^{n-1} = a^{n-1} (b^{n-1}) \equiv 1 \cdot (1)^{-1} = 1 \cdot 1 = 1 \bmod n$$

			\item Sia $\{ b_1, b_2, \dots , b_s \}$ l'insieme di tutte le basi rispetto alle quali $n$ è pseudoprimo.
			      $$\{ b \qquad 0<b<n \quad (b,n)=1 \} \qquad\qquad \varphi(n)$$
			      Considero l'insieme $$\{ \overline{b}b_1, \overline{b}b_2, \dots, \overline{b}b_s \}$$
			      Affermo che $(\overline{b}b_i, n) =1$ per $i=1 \dots s$. Infatti
			      $$(\overline{b}b, n)=1 \iff (\overline{b},n) =1 \mbox{ e } (b_i,n)=1 \quad \forall i$$
			      Affermo che $n$ \underline{non} è pseudoprimo rispetto alla base $\overline{b}b_i \quad \forall i$\\
			      perchè se $n$ fosse pseudoprimo rispetto a $\overline{b}b_i$, per l'osservazione 1, allora $n$ sarebbe pseudoprimo anche rispetto alla base
			      $$(\overline{b}b_i) b^{-1} = \overline{b} \qquad \mbox{ASSURDO}$$
			      (dato che $(\overline{b}b_i = a)$).

			\item Affermo che
			      $$\overline{b}b_i = \overline{b}b_j \implies b_i = b_j$$
			      Infatti
			      $$\overline{b}b_i = \overline{b}b_j$$
			      $$\overline{b}^{-1}(\overline{b}b_i) = \overline{b}^{-1}(\overline{b}b_j)$$
			      $$b_i = b_j$$
			      quindi
			      $$i=j$$
			      Concludendo ho trovato che le basi rispetto alle quali $n$ è pseudoprimo sono $s$, allora ne esistono (almeno) $s$ rispetto alle quali $n$ è pseudoprimo.
		\end{enumerate}
	\end{proof}
\end{teorema}

\subsection{Test di Primalità}
Sia $n>1$ un intero dispari.
\begin{enumerate}
	\item Scegliamo random un intero $b$ con $0 < b < n$
	\item Calcoliamo, con l'algoritmo di Euclide, $d=(b,n)$
	      \begin{itemize}
		      \item se $d>1$ allora $n$ \underline{non} è primo.
		      \item se $d=1$ allora $b$ è una base, calcoliamo $b^{n-1} \bmod n$
	      \end{itemize}
	\item \begin{itemize}
		      \item se $b^{n-1} \not\equiv 1 \bmod n$ allora $n$ \underline{non} è primo.
		      \item se $b^{n-1} \equiv 1 \bmod n$ allora $n$ è pseudoprimo rispetto alla base $b$ e \underline{forse} $n$ è primo.
	      \end{itemize}
\end{enumerate}
Scegliamo quindi un altro valore per $b$ come al punto \textit{1} e ripeto la procedura.\\

Supponiamo di aver applicato la procedura $k$ volte con gli interi $b_1, b_2, \dots b_k$ e supponiamo che $n$ sia pseudoprimo rispetto alle basi $b_1, b_2, \dots b_k$ (cioè $b_i^{n-1} \equiv 1 \bmod n$ per $i \dots k$).\\

Qual è la probabilità che $n$ sia composto (e che ci "ha fregato" $k$ volte)?\\
Se $n$ è composto e $b_1^{n-1} \equiv 1 \bmod n$ vuol dire che $b_1$ è una base rispetto alla quale $n$ è pseudoprimo.
Per il teorema precedente, tali basi sono al più la metà di quelle possibili, ovvero la probabilità che
$$b_1^{n-1} \equiv 1 \bmod n \qquad \mbox{ e } \qquad n \mbox{ è composto}$$ è $\leq \frac{1}{2}$.\\
Considerando quindi ognuna delle $k$ scelte di $b$ come un evento indipendente, le probabilità che $n$ è composto ma supera il test $k$ volte è $\leq \frac{1}{2} k$.

\subsection{Numeri di Carmichael}
Esistono dei numeri interi composti che sono pseudoprimi rispetto ad ogni base possibile.
\begin{definizione}[Numeri di Carmichael]
	Sia $n > 1$ un intero dispari composto.
	Si dice che $n$ è un numero intero di Carmichael se
	$$b^{n-1} \equiv 1 \bmod n$$
	per ogni $b \in \mathbb{Z}$ con $(b,n)=1$.
\end{definizione}
\begin{nota}
	I numeri di Carmichael minori di $1000$ sono: $561, 1105, 1729, 2465, 2821, 6601, 8911$.
\end{nota}

\subsubsection[Caratterizzazione]{Caratterizzazione dei numeri di Carmichael}
Un numero composto $n>1$ è di Carmichael se e solo se
\begin{itemize}
	\item $n$ è libero da quadrati \textit{(= la fattorizzazione contiene solamente esponenti uguali a 1)}
	\item $p-1|n-1$ per ogni divisore primo $p$ di $n$.
\end{itemize}

\begin{proof}
	Scrivo $$n = p_{1}^{a_1} p_{2}^{a_2} \dots p_{r}^{a_r}$$
	dove $p_1 \dots p_r$ sono numeri primi distinti.\\\\
	Per definizione $n$ è un numero di Carmichael se e solo se
	$$n\mbox{ è dispari } \qquad \mbox{ e } \qquad b^{n-1} \equiv 1 \bmod n \quad \forall b $$
	con $0<b<n$ e $(b,n)=1$.\\\\
	Pongo
	$$P = m.c.m( \varphi(p_{1}^{a_1}), \varphi(p_{2}^{a_2}), \dots, \varphi(p_{r}^{a_r}) )$$
	$$P = m.c.m( p_{1}^{a_{1}-1}(p_{1}-1), p_{2}^{a_{2}-1}(p_{2}-1), \dots, p_{r}^{a_{r}-1}(p_{r}-1) )$$\\
	Sia poi $b$ con $0<b<n$ e $(b,n)=1$
	\begin{itemize}
		\item $(b, p_{i}^{a_i})$ per $i = 1 \dots r$
		\item per il teorema di Eulero $$b^{\varphi(p_{i}^{a_i})} \equiv 1 \bmod p_{i}^{a_i} \qquad i=1 \dots r$$
		\item a maggior ragione $$b^{l} \equiv 1 \bmod p_{i}^{a_i} \qquad i=1 \dots r$$
		\item $$b^{l} \equiv 1 \bmod p_{i}^{a_i}$$ perchè\\\\

		      \begin{minipage}{0.45\textwidth}

			      $$p_{1}^{a_1} | b^{l-1}$$
			      $$p_{2}^{a_2} | b^{l-1}$$
			      $$\dots$$
			      $$p_{r}^{a_r} | b^{l-1}$$

		      \end{minipage}%
		      \hfill
		      \begin{minipage}{0.45\textwidth}
			      \begin{tabular}{|p{\textwidth}}

				      $$\implies (\prod p_{i}^{a_i}) | b^{l-1} = n| b^{l-1}$$ \\
			      \end{tabular}
		      \end{minipage}\\\\\\%

		\item $$b^t \equiv 1 \bmod n \iff l|t$$
	\end{itemize}
	In particolare abbiamo che
	$$b^{n-1} \equiv 1 \bmod n \iff l|n-1$$
	$$n \mbox{ è un numero di Carmichael } \iff l|n-1$$
	con $l=m.c.m( p_{1}^{a_{1}-1}(p_{1}-1), p_{2}^{a_{2}-1}(p_{2}-1), \dots, p_{r}^{a_{r}-1}(p_{r}-1) )$.
	$$n \mbox{ è un numero di Carmichael } \iff p_{r}^{a_{r}-1}(p_{r}-1)|n-1 \quad \mbox{ per } i \dots r$$\\
	Ora $p_i|n$ pertanto $p_i \not | n-1 \rightarrow \begin{cases}
			a_1 = 1 & \quad \forall i \\
			p_i -1  & \quad \forall i
		\end{cases}$

\end{proof}

\begin{corollario}
	Un numero di Carmichael è prodotto di almeno 3 numeri primi distinti.
	\begin{proof}
		Sia $n$ un numero di Carmichael con $n=p \cdot q$ ($p$ e $q$ primi, $p<q$).\\
		Allora
		$$n-1 = pq-1 = (p-1)(q-1) + (p-1) + (q-1)$$
		Per la caratterizzazione dei numeri di Carmichael\\ sappiamo che $p-1|n-1$ e $q-1|n-1$.\\
		Ottengo che
		$$p-1|n-1=(p-1)(q-1) + (p-1) +(q-1) \implies p-1|q-1$$
		Analogamente
		$$q-1|n-1 \implies q-1|p-1$$
		Ma allora
		$$p-1=q-1 \implies p=q$$
		che è ASSURDO!
	\end{proof}
\end{corollario}

\begin{shaded}
	\begin{esempio}
		Dato $n=561$, verificare se è un numero di Carmichael.

		$$561=3 \cdot 11 \cdot 17$$

		\begin{enumerate}
			\item $n= 3 \cdot 11 \cdot 17$ è libero da quadrati.\\
			\item devo controllare che $p-1|n-1 \quad \forall p $ divisore primo di $n$:\\\\
			      \begin{minipage}{0.25\textwidth}

				      $$3-1 | 561-1$$
				      $$2|560$$
				      $$\checkmark$$

			      \end{minipage}%
			      \hfill
			      \begin{minipage}{0.25\textwidth}
				      \begin{tabular}{|p{\textwidth}}

					      $$11-1 | 561-1$$
					      $$10|560$$
					      $$\checkmark$$
				      \end{tabular}
			      \end{minipage}%
			      \hfill
			      \begin{minipage}{0.25\textwidth}
				      \begin{tabular}{|p{\textwidth}}

					      $$17-1 | 561-1$$
					      $$16|560$$
					      $$\checkmark$$
				      \end{tabular}
			      \end{minipage}\\%

			      $\implies n = 561$ è un numero di Carmichael.

		\end{enumerate}

	\end{esempio}
\end{shaded}

% MEMO: Magari inserire gli esercizi random svolti durante questa lezione

\chapter{Anelli e Campi}
\section{Anelli}
\subsection{Anello}
\begin{definizione}[Anello]
	Un anello è una struttura algebrica $(A,+, \cdot)$ tale che
	\begin{enumerate}
		\item \textbf{$(A,+)$ è un gruppo abeliano}
		\item \textbf{$\cdot$ è associativo}, cioè $\forall a,b,c \in A \qquad (ab)c=a(bc)$
		\item valgono le \textbf{leggi distributive}, cioè $\forall a,b,c \in A$
		      \begin{itemize}
			      \item $a(b+c)=ab+ac$
			      \item $(a+b)c=ac+bc$
		      \end{itemize}
		\item \textbf{$\exists 1_A \in A$} tale che $\forall a \in A \qquad 1_A \cdot a = a = a \cdot 1_A$
	\end{enumerate}
\end{definizione}

\begin{shaded}
	\begin{esempio}
		Esempi pratici:
		\begin{enumerate}
			\item $(\mathbb{Z}, +, \cdot)$ + un anello \underline{commutativo}
			\item $Mat(n \times n, \mathbb{Z})$ rispetto alla somma e al prodotto tra matrici è un anello \underline{non commutativo}
			\item $\mathbb{Z}_n$ rispetto alla somma e al prodotto di classi di resto è un anello \underline{commutativo}
		\end{enumerate}
	\end{esempio}
\end{shaded}

\subsubsection{Anello Commutativo}
\begin{definizione}
	Un anello $A$ si dice \underline{commutativo} se $$\forall a,b \in A \qquad ab=ba$$
\end{definizione}
\begin{shaded}
	\begin{esempio}
		$\mathbb{R}$, $\mathbb{Q}$, $\mathbb{C}$ sono anelli commutativi
	\end{esempio}
\end{shaded}

\section{Campi}
\subsection{Campo}

\begin{definizione}[Campo]
	Un campo $k$ è un anello commutativo in cui ogni elemento (tranne $O_k$) ammette inverso.\\\\
	Ovvero un campo $k$ è un anello in cui
	\begin{enumerate}
		\item $\forall a,b \in k \qquad ab=ba$
		\item $\forall a \in k \quad \mbox{con} \quad a \not = O_k \qquad \exists a^{-1}  \in k \quad \mbox{tale che} \quad a \cdot a^{-1} = 1_k = a^{-1} \cdot a$
	\end{enumerate}
\end{definizione}

\begin{shaded}
	\begin{esempio}
		\begin{itemize}
			Esempi pratici:
			\item $\mathbb{Q}$ è un campo.
			\item $\mathbb{R}$ è un campo.
			\item $\mathbb{C}$ è un campo.
			\item $\mathbb{Z}$ \underline{non} è un campo.
			\item $\mathbb{Z}_{p}$ con $p$ primo è un campo.
		\end{itemize}
	\end{esempio}
\end{shaded}

\chapter{Polinomi su un campo}
Sia $K$ un campo, indichiamo con $K[X]$ l'anello dei polinomi a coefficienti in $K$, nell'indeterminata $x$.\\
Ovvero $K[x]$ è l'insieme di tutti i polinomi
$$p(x) = a_{n}x^{n} + a_{n-1}x^{n-1} + \dots + a_{1}x + a_{0}$$
con \begin{itemize}
	\item $n \in \mathbb{Z}$
	\item $a_i \in K \quad \forall i = 0 \dots n$
\end{itemize}

\section{Operazioni in $K[x]$}
Dati
$$p(x) = a_{n}x^{n} + a_{n-1}x^{n-1} + \dots + a_{1}x + a_{0} = \sum_{k=0}^{n} a_i x_i$$
$$q(x) = b_{m}x^{m} + b_{m-1}x^{m-1} + \dots + b_{1}x + b_{0} = \sum_{k=0}^{m} a_j x_j$$
definiamo...
\subsection{Somma in $K[x]$}
$$p(x)+q(x) = \sum_{k=0}^{max(n,m)} (a_k+b_k)x^k$$

$$= a_0 + b_0 + (a_1+b_1)x + \dots + (a_m+b_m)x^m + (a_{m+1}+b_{m+1})x^{m+1} + \dots + (a_n+b_n)x^n$$

\begin{nota}
	L'elemento neutro della somma è: $0_{K[x]} = 0_K = 0$
\end{nota}

\subsection{Prodotto in $K[x]$}
$$p(x)q(x) = \sum_{k=0}^{n+m} c_kx^k$$
con $c_k = \displaystyle\sum_{k=i+j} a_ib_j$

\begin{nota}
	L'elemento neutro del prodotto è: $1_{K[x]} = 1_K = 1$
\end{nota}

\subsection{Osservazioni su $K[x]$}

\begin{nota}
	Quindi $K[x]$ è un anello commutativo con $0_{K[x]}$ e $1_{K[x]}$ che coincidono con $0_K$ e $1_K$.
\end{nota}

\begin{nota}
	Per l'anello $K[x]$ si può sviluppare una teoria parallela a quella sviluppata per $\mathbb{Z}$.
\end{nota}

\section{Coefficiente Direttore}
\begin{definizione}
	Dato $p(x) \in K[x]$ polinomio non nullo con
	$$p(x) = a_{n}x^{n} + a_{n-1}x^{n-1} + \dots + a_{1}x + a_{0}$$
	il coefficiente $a_n \not = 0$ si dice \textbf{coefficiente direttore} di $p(x)$.
\end{definizione}

\begin{nota}
	Se $a_n = 1$ allora $p(x)$ è \textbf{monico}.
\end{nota}

\section{Grado di un polinomio}
\begin{definizione}
	Dato $p(x) \in K[x]$ polinomio non nullo con
	$$p(x) = a_{n}x^{n} + a_{n-1}x^{n-1} + \dots + a_{1}x + a_{0}$$
	e
	$$a_n \not = 0$$
	l'intero non negativo $n$ si dice \textbf{grado di $p(x)$}\\
	e lo si indica con $\partial p(x) = n$.
\end{definizione}
\begin{nota}
	Per convenzione, il polinomio nullo ha grado $\partial p(x) = -1$
\end{nota}

\section{Algoritmo della divisione}
\begin{teorema}[Algoritmo della divisione]
	Siano $$a(x), b(x) \in K[x] \quad \mbox{con} \quad b(x) \not = 0$$
	Esistono e sono unici due polinomi $q(x), r(x) \in K[x]$ tali che
	\begin{enumerate}
		\item $a(x) = b(x)q(x)+r(x)$
		\item $\partial r(x) < \partial b(x)$
	\end{enumerate}

	\begin{proof} Dimostro esistenza e unicità:\\

		\underline{Esistenza di $q(x)$ e $r(x)$}\\\\
		Per induzione su $n = \partial a(x)$
		\begin{itemize}
			\item[$n = -1$:] $a(x)=0$ e il teorema è vero con $q(x) = 0 = r(x)$
			\item[$n \geq 0$:] allora poniamo $m = \partial b(x)$;
			      \begin{itemize}
				      \item se $n < m$ il teorema è vero con $q(x)=0$ e $r(x)=a(x)$
				      \item se $n \geq m$ allora scriviamo
				            $$a(x) = a_{n}x^{n} + a_{n-1}x^{n-1} + \dots + a_{1}x + a_{0}$$
				            $$b(x) = b_{m}x^{m} + b_{m-1}x^{m-1} + \dots + a_{1}x + a_{0}$$
				            con $b(x) \not = 0$ (quindi $b_n \not = 0, \exists b_m^{-1} \in K)$.\\\\
				            Considero il polinomio
				            $$a'(x) = a_n(x) - a_nb_m^{-1} b(x) x^{n-m}$$
				            Risulta
				            $$a'(x) = a_nx^n + a_{n-1}x^{n-1}+ \dots + a_1x+a_0 - a_nb_m^{-1}x^{n-m}(b_mx^m+\dots+b_1x+b_0)$$
				            dunque $\partial a'(x) \leq n-1$.\\\\
				            Per induzione esistono due polinomi $q'(x), r'(x) \in K[x]$ tali che
				            $$a'(x) = b(x)q'(x) + r'(x)$$
				            con $\partial r'(x) < \partial b(x)$.\\\\
				            Poiché $a(x) = a'(x) + a_nb_m^{-1}x^{n-m}b(x)$ abbiamo
				            $$a(x) = a'(x) + a_nb_m^{-1}x^{n-m}b(x)$$
				            $$= q'(x)b(x)+r'(x) + a_nb_m^{-1}x^{n-m}b(x)$$
				            $$= (q'(x) + a_nb_m^{-1}x^{n-m})b(x) +r'(x)$$
				            Posto quindi
				            $$q(x) = q'(x) + a_nb_m^{-1}x^{n-m} $$
				            $$r(x) = r'(x)$$
				            sono verificate le condizioni 1 e 2.\\
			      \end{itemize}
		\end{itemize}

		\underline{Unicità di $q(x)$ e $r(x)$}\\\\
		Supponiamo che
		$$a(x) = b(x)q(x) + r(x), \qquad \partial r(x) < \partial b(x)$$
		$$a(x) = b(x)q_1(x) + r_1(x), \qquad \partial r_1(x) < \partial b(x)$$
		Quindi deve essere
		$$b(x) ( q(x) - q_1(x) ) = r_1(x) - r(x)$$
		Se fosse $q(x) \not = q_1(x)$ sarebbe
		$$\partial ( b(x)  ( q(x) - q_1(x) )) \geq \partial b(x)$$
		e, d'altra parte, $\partial ( r_1(x) -r(x) ) < \partial b(x)$, assurdo.\\\\
		Ne segue che $q(x) = q_1(x)$ e quindi $r(x) = r_1(x)$.
	\end{proof}
\end{teorema}

\begin{definizione}[Quoziente e Resto]
	I polinomi $q(x)$ e $r(x)$ si dicono rispettivamente \textbf{quoziente e resto} della divisione di $a(x)$ per $b(x)$.
\end{definizione}

\begin{shaded}
	\begin{esempio}
		Divido $a(x)=x^3-2x^2+x-1$ per $b(x)=2x^2-5$:
		$$\polylongdiv[style=B,div=:,vars=x]{x^3-2x^2+x-1}{2x^2-5}$$
		Ottengo $$q(x)=\frac{1}{2}x-1$$ $$r(x)=\frac{7}{2}x-6$$
	\end{esempio}

	% TODO: Aggiungo esempio divisione in modulo

\end{shaded}

\subsection{Divisibilità}

\begin{definizione}[Divisibilità]
	Se $r(x) = 0$ si dice che $b(x)$ divide $a(x)$, ovvero che $a(x)$ è divisibile per $b(x)$, e si scrive $$b(x) | a(x)$$\\
	\begin{nota}
		$$b(x) | a(x) \iff \exists c(x) \in K[x]: \quad a(x)=b(x)c(x)$$
	\end{nota}
\end{definizione}

\section{Massimo Comune Divisore}

\begin{definizione}[Massimo Comune Divisore]
	Sia
	\begin{itemize}
		\item $K[x]$ l'anello dei polinomi a coefficienti in $K$
		\item $a(x), b(x) \in K[x]$ due polinomi non nulli
	\end{itemize}
	Si dice massimo comune divisore tra $a(x)$ e $b(x)$, ogni polinomio $d(x) \in K[x]$ tale che
	\begin{enumerate}
		\item $d(x) | a(x)$ e $d(x) | b(x)$
		\item se $c(x) \in K[x]$ con $c(x) | a(x)$ e $c(x) | b(x)$ allora $c(x) | d(x)$
	\end{enumerate}
\end{definizione}

\subsection{Esistenza di un Massimo Comune Divisore}
\begin{teorema}
	Per ogni $a(x), b(x) \in K[x]$ con $a(x) \not = 0, a(x) \not = 0$, esiste un massimo comune divisore $d(x)$ fra $a(x)$ e $b(x)$.\\\\
	Esistono inoltre i polinomi $s(x), t(x) \in K[x]$ tali che sia
	$$d(x) = a(x) s(x) + b(x) t(x)$$

	\begin{proof}
		Analoga a quella in $\mathbb{Z}$.\\
		Applico l'algoritmo delle divisioni successive:\\
		\begin{enumerate}
			\item[(1)] $a(x) = b(x) q_1(x) + r_1(x) \qquad \qquad \qquad \partial r_1(x) < \partial b(x)$
			\item[(2)] $b(x) = r_1(x) q_2(x) + r_2(x) \qquad \qquad \qquad \partial r_2(x) < \partial r_1(x)$
			\item[(3)] $r_1(x) = r_2(x) q_3(x) + r_3(x) \qquad \qquad \qquad \partial r_3(x) < \partial r_2(x)$
			\item[$\vdots$] $\vdots$
			\item[(k-1)] $r_{k-3}(x) = r_{k-2}(x) q_{k-1}(x) + r_{k-1}(x) \qquad \partial r_{k-1}(x) < \partial r_{k-2}(x)$
			\item[(k)] $r_{k-2}(x) = r_{k-1}(x) q_k(x)$
		\end{enumerate}
		L'ultimo resto non nullo è un massimo comune divisore tra $a(x)$ e $b(x)$.
		\begin{nota}
			Per determinare $s(x)$ e $t(x)$ si procede come in $\mathbb{Z}$.
		\end{nota}
	\end{proof}
\end{teorema}

% TODO: Aggiungo altro esempio divisione in modulo

Il Massimo Comune Divisore tra polinomi è unico a meno di una costante moltiplicativa non nulla.

\begin{teorema}
	Sia $d(x)$ un massimo comune divisore tra $a(x)$ e $b(x)$. Allora $d'(x)$ è un massimo comune divisore tra $a(x)$ e $b(x)$ se e solo se $$d'(x)=k d(x)$$ con $k \in K^{*}$.

	\begin{proof}
		Da dimostrare. % TODO: da dimostrare #18
	\end{proof}
\end{teorema}

\begin{osservazione}
	Dato quanto detto, esiste uno e un solo polinomio \textbf{monico} $d(x)$ che sia massimo comune divisore tra $a(x)$ e $b(x)$.
	Tale polinomio è indicato con il simbolo $$(a(x), b(x))$$ ed è chiamato \textbf{massimo comune divisore tra $a(x)$ e $b(x)$}.
	In particolare, se il grado del massimo comune divisore è zero, allora tale massimo comune divisore è $1$. In questo caso $a(x)$ e $b(x)$ si dicono \textbf{coprimi}.
\end{osservazione}

\begin{shaded}
	\begin{esempio}
		%$$\polylongdiv[style=B,div=:,vars=x]{x^5-x^2+x+1}{3x^2+2x+2}$$
	\end{esempio}
	\begin{figure}[H]
		\includegraphics[width=\linewidth,scale=1]{polydiv1}
	\end{figure}
\end{shaded}

\begin{shaded}
	\begin{esempio}
	\end{esempio}
	\begin{figure}[H]
		\includegraphics[width=\linewidth,scale=1]{polydiv2}
		\includegraphics[width=\linewidth,scale=1]{polydiv3}
	\end{figure}
\end{shaded}

\begin{definizione}
	Sia $a(x) \in K[x]$ un polinomio di grado $n > 0$. Si dice che $a(x)$ è un \textbf{polinomio primo} in $K[x]$ se ogni volta che $a(x) | b(x)c(x)$, con $b(x),c(x) \in K[x]$, si ha $a(x) | b(x)$ oppure $a(x) | c(x)$.
\end{definizione}

\begin{osservazione}
	Se un polinomio primo $a(x)$ divide il prodotto $n \geq 2$ polinomi, segue dalla definizione (per induzione su $n$) che $a(x)$ divida almeno uno dei fattori.
\end{osservazione}

\begin{definizione}
	Sia $a(x) \in K[x]$ un polinomio di grado $n>0$. Si dice che $a(x)$ è un polinomio irriducibile (in $K[x]$) se $a(x)$ è divisibile solo per i polinomi di grado $0$ e per i polinomi della forma $h \cdot a(x)$ con $h \in K^{*}$. In caso contrario, si dice che a(x) riducibile.

	Detto diversamente: il polinomio $a(x)$ è irriducibile se e solo se è fattorizzabile soltanto come
	$$a(x) = h^{-1}(ha(x)) \qquad \mbox{ con } h \in K^{*}$$
\end{definizione}

\begin{teorema}
	Un polinomio $a(x) \in K[x]$ è irriducibile se e solo se è primo.

	\begin{proof}
		Analoga a quella vista in $\mathbb{Z}$.
	\end{proof}
\end{teorema}

\begin{osservazione}
	La nozione di irriducibilità di un polinomio $a(x) \in K[x]$ dipende dal campo $K$ cui appartengono i coefficienti del polinomio.
	Se $K$ è un sottocampo di un campo $F$, si può riguardare $a(x)$ come polinomio in $F[x]$. Può accadere che $a(x)$ sia irriducibile in $K[x]$ ma riducibile in $F[x]$.
\end{osservazione}

\begin{shaded}
	\begin{esempio}
		Il polinomio $a(x) = x^2 -2$ è irriducibile in $\mathbb{Q}[x]$, ma è riducibile in $\mathbb{R}[x]$ perchè
		$$x^2 -2 = (x-\sqrt{2}) (x+\sqrt{2}) \quad \mbox{in} \quad \mathbb{R}[x]$$
	\end{esempio}
	\begin{esempio}
		Il polinomio $a(x) = x^2 +1$ è irriducibile in $\mathbb{Q}[x]$ e in $\mathbb{R}[x]$, ma è riducibile in $\mathbb{C}[x]$ perchè
		$$x^2 +1 = (x-i) (x+i) \quad \mbox{in} \quad \mathbb{C}[x]$$
	\end{esempio}
\end{shaded}

\begin{teorema}[Teorema della fattorizzazione unica]
	Ogni polinomio $a(x) \in K[x]$ di grado $n>0$ può essere scritto come prodotto di $s \geq 1$ polinomi irriducibili (non necessariamente distinti).

	Tale fattorizzazione è essenzialmente unica, nel senso che se
	$$a(x) = p_1(x) \dots p_s(x) = q_1(x) \dots q_t(x)$$ dove i polinomi
	$$p_i(x), q_j(x) \qquad (1 \leq i \leq s)$$ sono irriducibili, si possono ordinare i fattori in modo che $$s=t$$ e
	$$p_1(x) = h_1q_1(x), \dots , p_s(x) = h_sq_s(x)$$
	con $h_i \in K^{*} \qquad (q \leq i \leq s)$

	\begin{proof}
		Da dimostrare. % TODO: da dimostrare #18
	\end{proof}
\end{teorema}

\begin{corollario}
	Ogni polinomio $a(x) \in K[x]$ di grado $n>0$ si può scrivere come
	$$a(x) = ka_1(x) \dots a_s(x)$$
	dove $k \in K^{*}$ è il coefficiente direttore di $a(x)$ e i polinomi $a_1(x), \dots, a_s(x)$ sono monici e irriducibili. Tale scrittura è unica a meno dell'ordine.
\end{corollario}


% to be continued...			
\chapter{Radici di un Polinomio}
\chapter{Costruzione di Campi}
\chapter{Permutazioni}
\chapter{Teoria dei Codici}
\chapter{Codici Lineari}



\end{document}