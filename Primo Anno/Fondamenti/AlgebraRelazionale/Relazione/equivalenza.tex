\section{Relazioni di Equivalenza}
Si definisce $R$ una \emph{relazione di equivalenza} se e solo se la relazione binaria
$R$ è riflessiva, simmetrica e transitiva.

%Inserire Esempi

Data una relazione di equivalenza $R$ definita su un insieme $S$, si definisce
\emph{classe di equivalenza} di un elemento $x \in S$ come $[x] = \{y | (x,y) \in R \}$

\begin{thm}
Se $R$ è una relazione di equivalenza su $S$, allora le classi di Equivalenza
generate da $R$ partizionano $S$
\end{thm}
%Fare la Dimostrazione

Data una relazione di equivalenza in S, la partizione che essa determina si dice
\emph{insieme quoziente} di $S$ rispetto alla relazione di equivalenza e si indica con $S/$

