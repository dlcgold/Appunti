\documentclass[a4paper,12pt, oneside]{book}

% \usepackage{fullpage}
\usepackage[italian]{babel}
\usepackage[utf8]{inputenc}
\usepackage{amssymb}
\usepackage{amsthm}
\usepackage{graphics}
\usepackage{amsfonts}
\usepackage{listings}
\usepackage{amsmath}
\usepackage{amstext}
\usepackage{engrec}
\usepackage{rotating}
\usepackage{verbatim}
\usepackage[safe,extra]{tipa}
% \usepackage{showkeys}
\usepackage{multirow}
\usepackage{hyperref}
\usepackage{microtype}
\usepackage{fontspec}
\usepackage{enumerate}
\usepackage{physics}
\usepackage{braket}
\usepackage{marginnote}
\usepackage{pgfplots}
\usepackage{cancel}
\usepackage{polynom}
\usepackage{booktabs}
\usepackage{enumitem}
\usepackage{framed}
\usepackage{pdfpages}
\usepackage{pgfplots}
\usepackage{algorithm}
% \usepackage{algpseudocode}
\usepackage[cache=false]{minted}
\usepackage{mathtools}
\usepackage[noend]{algpseudocode}
\newcommand*{\bfrac}[2]{\genfrac{}{}{0pt}{}{#1}{#2}}

\usepackage{tikz}\usetikzlibrary{er}\tikzset{multi  attribute /.style={attribute
    ,double  distance =1.5pt}}\tikzset{derived  attribute /.style={attribute
    ,dashed}}\tikzset{total /.style={double  distance =1.5pt}}\tikzset{every
  entity /.style={draw=orange , fill=orange!20}}\tikzset{every  attribute
  /.style={draw=MediumPurple1, fill=MediumPurple1!20}}\tikzset{every
  relationship /.style={draw=Chartreuse2,
    fill=Chartreuse2!20}}\newcommand{\key}[1]{\underline{#1}}
\usetikzlibrary{arrows.meta}
\usetikzlibrary{decorations.markings}
\usetikzlibrary{arrows,shapes,backgrounds,petri}
\tikzset{
  place/.style={
    circle,
    thick,
    draw=black,
    minimum size=6mm,
  },
  transition/.style={
    rectangle,
    thick,
    fill=black,
    minimum width=8mm,
    inner ysep=2pt
  },
  transitionv/.style={
    rectangle,
    thick,
    fill=black,
    minimum height=8mm,
    inner xsep=2pt
  }
} 
\usetikzlibrary{automata,positioning,chains,fit,shapes}
\usepackage{fancyhdr}
\pagestyle{fancy}
\fancyhead[LE,RO]{\slshape \rightmark}
\fancyhead[LO,RE]{\slshape \leftmark}
\fancyfoot[C]{\thepage}
\usepackage[usenames,dvipsnames]{pstricks}
\usepackage{epsfig}
\usepackage{pst-grad} % For gradients
\usepackage{pst-plot} % For axes
\usepackage[space]{grffile} % For spaces in paths
\usepackage{etoolbox} % For spaces in paths
\makeatletter % For spaces in paths
\patchcmd\Gread@eps{\@inputcheck#1 }{\@inputcheck"#1"\relax}{}{}
\makeatother
\usepackage{lipsum}
\DeclareSymbolFont{symbolsC}{U}{txsyc}{m}{n}
\DeclareMathSymbol{\strictif}{\mathrel}{symbolsC}{74}
\title{Data and Computational Biology, Old Version}
\author{UniShare\\\\Davide Cozzi\\\href{https://t.me/dlcgold}{@dlcgold}}
\date{}

\pgfplotsset{compat=1.13}
\begin{document}
\maketitle

\definecolor{shadecolor}{gray}{0.80}
\setlist{leftmargin = 2cm}
\newtheorem{teorema}{Teorema}
\newtheorem{definizione}{Definizione}
\newtheorem{esempio}{Esempio}
\newtheorem{corollario}{Corollario}
\newtheorem{lemma}{Lemma}
\newtheorem{osservazione}{Osservazione}
\newtheorem{nota}{Nota}
\newtheorem{esercizio}{Esercizio}
\algdef{SE}[DOWHILE]{Do}{doWhile}{\algorithmicdo}[1]{\algorithmicwhile\ #1}
\tableofcontents
\renewcommand{\chaptermark}[1]{%
  \markboth{\chaptername
    \ \thechapter.\ #1}{}}
\renewcommand{\sectionmark}[1]{\markright{\thesection.\ #1}}
\newcommand{\floor}[1]{\lfloor #1 \rfloor}
\newcommand{\MYhref}[3][blue]{\href{#2}{\color{#1}{#3}}}%
\chapter{Introduzione}
\textbf{Questi appunti sono presi a lezione. Per quanto sia stata fatta
  una revisione è altamente probabile (praticamente certo) che possano
  contenere errori, sia di stampa che di vero e proprio contenuto. Per
  eventuali proposte di correzione effettuare una pull request. Link: }
\url{https://github.com/dlcgold/Appunti}.
\chapter{Introduzione alla Biologia Computazionale}
La \textbf{biologia} nasce come una disciplina altamente \textbf{descrittiva}
mentre altre discipline, come, ad esempio, informatica, matematica o fisica,
sono discipline \textbf{generaliste}.\\
I biologi propongono \textbf{modelli}, come ad esempio i \textit{pathway}, che
sono il diretto risultato di osservazioni sperimentali e interpretazione dei
risultati.
\begin{definizione}
  Un \textbf{pathway (\emph{percorso}) biologico} è una serie di interazioni
  tra molecole in una cellula che porta a un determinato prodotto o un
  cambiamento in una cellula. Tale \emph{percorso} può innescare
  l'assemblaggio di nuove molecole, come un grasso o una proteina. I
  \emph{percorsi} possono anche attivare e disattivare i geni o stimolare una
  cellula a muoversi. I pathway più comuni sono coinvolte nel metabolismo, nella
  regolazione dell'espressione genica e nella trasmissione dei segnali e
  svolgono un ruolo chiave negli studi avanzati di genomica.\\
  Tra le principali categorie si hanno:
  \begin{itemize}
    \item Metabolic pathway
    \item Genetic pathway
    \item Signal transduction pathway
  \end{itemize}
\end{definizione}
Un altro aspetto chiave negli ultimi 25 anni è stato quello della
mole di dati prodotti, tramite, ad esempio, \textbf{Next Generation Sequencing
  (\textit{NGS})}, con la produzione di \textit{DNAseq} e \textit{RNAseq}, o
alla cosiddetta \textbf{single-cell analysis}. Tutte queste tecnologie
\textit{high-throughput} usate in biologia computazionale e in bioinformatica
richiedono una forte conoscenza algoritmica, matematica e statistica per la
gestione di questa enorme quantità di dati (essendo anche nell'ambito
\textbf{big data}) in ambito biomedico. Ovviamente le conoscenze, i tempi (ma
anche gli spazi), gli strumenti da usare e sviluppare etc$\ldots$ variano al
variare del tipo di studio.\\ 
Un altro aspetto non trascurabile è la scala di misura di ciò che viene
studiato, ad esempio:
\begin{itemize}
  \item \textit{organismi}, ad esempio per gli organismi multicellulari si passa
  da $10\mu m$ a $50/85m$ 
  \item \textit{tessuti}, ad esempio per i tessuti umani siamo in un range
  maggiore di $10^4 \mu m^3$
  \item \textit{cellule}, ad esempio per quelle umane si va da $30\mu m^3$ a
  $10^6 \mu m^3$ con:
  \begin{itemize}
    \item membrane
    \item nuclei
    \item ribosomi
    \item mitocondri e cloroplasti
    \item altri organelli e strutture intracellulari
    \item proteine
    \item materiale genomico (DNA e RNA e affini strutture: ad esempio istoni) 
    \item $\ldots$
  \end{itemize}
\end{itemize}
Parlando di tipi di organismi distinguiamo in primis:
\begin{itemize}
  \item \textbf{eucarioti}. In questo caso si hanno cellule più complesse, con
  numerosi organelli e soprattutto il \textbf{nucleo}, dove sono contenute le
  informazioni
  \item \textbf{procarioti}, come i \textit{batteri}. In questo caso si hanno
  cellule piccole e semplici. Non hanno un nucleo ma solo una regione, detta
  \textbf{nucleoide}, dove sono contenute le informazioni
\end{itemize}
In aggiunta si hanno anche i \textbf{virus}.\\
\textit{Per ulteriori informazioni sui tipi di organismi guardare online}.\\
Parlando di DNA si ha che ogni cellula umana contiene circa 2 metri di DNA e un
organismo umano contiene moltissime cellule rendendo lo studio del DNA davvero
complesso (anche dal punto di vista computazionale si hanno file di genomi
davvero molto pesanti, di centinaia di $MB$).\\
\textbf{Riprendere da appunti di Bioinformatica il passaggio da DNA a RNA e da
  RNA a Proteine}.\\
Ad essere interessanti non sono solo le dimensioni di ciò che viene studiato ma
anche i vari \textbf{tempi}. Vediamo una piccola tabella d'esempio:
\begin{table}[H]
  \small
  \centering
  \begin{tabular}{c|c|c}
    \textbf{Proprietà} & \textbf{E. coli} & \textbf{Uomo}\\
    \hline
    \hline
    diffusione di proteine in una cellula & $0.1 s$ & $\sim 100 s$\\
    \hline
    trascrizione di un gene & $\sim 1m$ ($80\frac{bp}{s}$) & $\sim 100 s$\\
    \hline
    generazione di una cellula & da $30 m$ a ore & da $20h$ a statico\\
    \hline
    transizione di stato proteico & da $1\mu s$ a $100\mu s$
                                          & da $1\mu s$ a $100\mu s$\\
    \hline
    rate di mutazione & $\sim \frac{10^{-9}}{\frac{bp}{generazione}}$
                                      & $\sim \frac{10^{-8}}{\frac{bp}{anno}}$\\
  \end{tabular}
\end{table}
Qualche nota:
\begin{itemize}
  \item i tempi di trascrizione di un gene umano includono i tempi di
  preprocessamento dell'\textit{mRNA}
  \item per la generazione di una cellula di E. Coli si hanno 30 minuti in
  presenza di nutrienti
  \item 
\end{itemize}
Si studiano quindi i vari \textbf{modelli} per la biologia computazionale che
possono essere di varie tipologie:
\begin{itemize}
  \item \textbf{continui}, tramite equazioni differenziali ordinarie
  \item \textbf{discreti}
  \item \textbf{stocastici}
\end{itemize}
Si studiano, in ottica analisi di cancro, anche \textbf{grafi mutazionali} e
\textbf{evoluzioni clonali} (tramite Single-cell analysis).\\
Un aspetto fondamentale è costituito dall'RNA, che trasposta le informazioni dal
DNA (contenuto nel nucleo) al citoplasma della cellula, dove funge da
intermediario per il processo di sintesi delle proteine.
\begin{teorema}[Dogma principale di Francis Crick]
  Si ha quindi il dogma principale della biologia molecolare:
  \begin{center}
    \textbf{il flusso d'informazione è unidirezionale}
  \end{center}
  ovvero, in termini più estesi:
  \begin{center}
    \emph{una volta che le ``informazioni'' sono passate nelle proteine, non
      possono uscirne nuovamente. Il trasferimento di informazioni da acido
      nucleico ad acido nucleico, o da acido nucleico a proteina, può essere
      possibile, ma il trasferimento da proteina a proteina, o da proteina ad
      acido nucleico è impossibile. Per ``informazione'' si intende qui la
      precisa determinazione della sequenza, sia delle basi nell'acido nucleico
      che dei residui amminoacidici nella proteina.} 
  \end{center}
\end{teorema}
Geni, proteine e cellule sono il \textit{linguaggio macchina} della vita.\\
Veniamo quindi alla distinzione delle due branche di
studio. \textbf{Bioinformatica} e \textbf{Biologia (del Sistema) Computazionale}
sono due aspetti sovrapposti del modo in cui usiamo l'approccio computazionale
alla Biologia e alla Medicina, manipolando oggetti simili ma con enfasi diversa
e diverse scale spazio-temporali. In entrambe si usano ontologie, formalismi
descrittive ma anche, lato più pratico, database. Nel dettaglio:
\begin{itemize}
  \item la \textbf{Bioinformatica} si occupa in primis dell'\textbf{analisi di
    sequenze} ovvero, tra le altre cose, di studio del genoma, RNA folding,
  folding di proteine e studio dei database necessari a questi studi. Si usano
  algoritmi di pattern matching e altri metodi di analisi delle stringhe
  \item la \textbf{Biologia (del Sistema) Computazionale} studia, tra le altre
  cose:
  \begin{itemize}
    \item modelli e inferenze sulle proprietà dei sistemi, studiando simulazioni
    e nuove proprietà
    \item ricostruzione di reti metaboliche e regolatorie e di modelli di
    progressione 
  \end{itemize}
  Si usano, ad esempio, metodi di machine learning per l'analisi dei dati
  prodotti e si simulano modelli biologici in modo sia deterministico che
  stocastico (tramite ad esempio Gillespie e Monte Carlo) e si fa analisi di
  raggiungibilità 
\end{itemize}
D'altro canto, tecniche come la \textbf{Polymerase chain reaction
  (\textit{PCR})} ed altre sono appannaggio di biologi e biotecnologi.
L'interesse per un biologo computazionale e per un bioinformatico è quello di
aiutare altri ricercatori a svolgere le proprie attività. Ad esempio i biologi
traggono vantaggio in ottica di acquisire conoscenze di base o anche al ricevere
strumenti atti al progettare e pianificare esperimenti. Gli esperimenti
biologici sono costosi sia dal punto di vista dei materiali che di persone e
tempo. \\
In biologia computazionale si è quindi interessati a comprendere, anche in
termini computazionali, l'interazione complessiva di:
\begin{itemize}
  \item processi intracellulari (regolatori e metabolici)
  \item cellule singole
  \item popolazioni cellulari 
\end{itemize}
Un altro compito dei biologi computazionali è quello di capire cosa
succede quando si ha la possibilità di perturbare un sistema e vedere quali sono
gli effetti della perturbazione, in particolare vedere cosa succede usando un
dato farmaco piuttosto che un altro per intervenire su una certa patologia,
parlando, in questo caso, del cosiddetto \textbf{momento traslazionale} della
\textbf{medicina traslazionale}. Con ``momento'' ci si riferisce al
trasferimento di conoscenze delle attività di pura ricerca alle \textbf{attività
  di produzione}, ovvero all'\textit{attività clinica}, con il passaggio alla
``vita vera''.
\chapter{Esempio del Repressilator}
Introduciamo un esempio che rientra nell'ambito della \textit{synthetic
  biology}, di M. B. Elowitz e S. Leibler\footnote{M. B. Elowitz, 
  S. Leibler, A synthetic oscillatory network of transcriptional regulators,
  Nature 403(20), January 2000},  che sarà rivisto sotto diversi aspetti durante
il corso. Questo è un esempio di un sistema biologico ``ingegnerizzato'', uno
dei primi esempi di sistema biologico, di \textbf{biologia sintetica}.
\section{Il Modello Biologico}
In questo sistema si hanno tre geni, che per praticità chiamiamo \textit{gene
  A}, \textit{gene B} e \textit{gene C}, ognuno dei quali, dopo essere
trascritti e tradotti producono il rispettivo \textit{mRNA} e poi, nel
citoplasma, tali \textit{mRNA} vengono usati per sintetizzare le tre rispettive
\textit{proteine}. \\
Quello che succede è che la trascrizione dei 3 geni può partire solo se non c'è
proteina attaccata ad una sezione, detta \textit{promotrice del processo di
  trascrizione}. Tale proteina è detta anche \textit{promotore} o
\textit{inibitore}. Diciamo quindi che: 
\begin{itemize}
  \item per il \textit{gene A} non deve esserci la \textit{proteina C} attaccata
  per avere la trascrizione del gene stesso
  \item per il \textit{gene B} non deve esserci la \textit{proteina A} attaccata
  per avere la trascrizione del gene stesso
  \item per il \textit{gene C} non deve esserci la \textit{proteina B} attaccata
  per avere la trascrizione del gene stesso
\end{itemize}
È quindi un processo ciclico.
\begin{figure}
  \centering
  \includegraphics[scale = 1]{img/repr.pdf}
  \caption{Schema di base del Repressilator, con le frecce bidimensionali che
    rappresentano l'azione di inibizione delle proteine.}
  \label{fig:repr}
\end{figure}
Nel dettaglio del Repressilator le proteine (prodotte dai rispettivi geni che si
indicano con la prima lettera minuscola) sono, in ordine (\textit{A, B, C}):
\begin{itemize}
  \item $TetR$ prodotta dal gene $tetR$
  \item $\Lambda cI$ prodotta dal gene $\lambda cI$
  \item $LacI$ prodotta dal gene $lacI$
\end{itemize}
Il punto fondamentale, come visibile in figura \ref{fig:repr}, è capire che se
sto producendo una grande quantità di una 
certa proteina allora sicuramente non avrò produzione di quella di cui
tale proteina inibisce la trascrizione del gene e così via. Nel nostro caso se
si produce tanta \textit{proteina A} non avremo produzione di \textit{proteina
  B} e di conseguenza avremo produzione della \textit{proteina C}, ma nel
momento in cui questa terza viene prodotta cala la produzione della
\textit{proteina A} comportando la produzione della \textit{proteina B}
etc$\ldots$. Ho, in pratica, un sistema oscillatorio, con 3 proteine che si
reprimono l'una con l'altra.\\
La rappresentazione ``su carta'' di questo comportamento è abbastanza semplice,
come vedremo, modellandola tramite un insieme di equazioni differenziali. Il
problema è passare dalla teoria alla pratica. Questo sistema ``ingegnerizzato'',
di equazioni differenziali, è in grado di confermare quanto visualizzabile poi
tramite esperimenti. \\ 
Vediamo quindi come viene effettivamente costruito il sistema sperimentale
usando delle colonie di E. Coli, sfruttando la loro biologia. Nei batteri il DNA
non è, come detto, racchiuso nel nucleo ma ``circola'' in una regione, detta
\textit{nucleoide}, abbastanza accessibile all'interno del citoplasma. Nei
batteri il DNA circola in forme dette \textbf{plasmidi} quindi potenzialmente si
può sintetizzare un particolare plasmide e inserirlo in un batterio, il quale lo
userà per sintetizzare proteine. Prima è stato comunque pensato il modello
matematico e poi stato effettivamente costruito l'esperimento (al contrario
dell'ordine con cui si stanno ora spiegando quindi). \\
I due ricercatori hanno costruito due plasmidi (di cui per ora non approfondiamo
i dettagli):
\begin{itemize}
  \item un plasmide che codifica il\textit{ Repressilator}, ovvero che contiene
  i 3 geni che codificano le 3 proteine. Prima di ogni gene si ha attaccata una
  \textit{zona di induzione} 
  \item un plasmide che codifica un \textit{Reporter}, che codifica una
  particolare proteina, detta \textbf{green fluorescent protein
    (\textit{Gfp})}. La \textit{Gfp} è una proteina usata spesso in quanto fa si
  che un certo sistema diventi fluorescente, di colore verde, una volta che
  viene illuminato con una certa luce (un laser ad una determinata
  frequenza). Questo plasmide fa si che, quando $TetR$ è presente in abbondanza
  la trascrizione del gene \textit{gfp} viene bloccata e quindi diminuisce la
  quantità di \textit{Gfp}. Quindi, come $TetR$ oscilla per il sistema di
  \textit{mutua repressione}, si vedrà al microscopio un'oscillazione della
  fluorescenza della colonia di batteri. 
\end{itemize}
Si ha un ulteriore ``trucco''. Se si lascia una colonia di E. Coli senza alcun
controllo si avrebbe che ogni batterio inizierebbe il ciclo per conto suo, in
modo non sincrono, impedendo una corretta visualizzazione della
fluorescenza. Questo trucco è quello di inibire la produzione di $LacI$,
interferendo con la sua espressione, usando
un'ulteriore induttore, detto \textit{IPTG} ($isopropyl
\,\beta\mbox{-}D\mbox{-}1\mbox{-}thiogalactopyranoside$), e ottenendo così la
sincronia delle  
cellule dopo questo impulso iniziale di \textit{IPTG} (che poi decade
velocemente lasciando tutti gli E. Coli nello stesso stato iniziale).
\section{Il Modello Matematico}
Facciamo quindi un passo indietro e vediamo il modello matematico del
Repressilator.\\
Per prevedere il comportamento complessivo del sistema ingegnerizzato, si è
quindi scritto un modello matematico che rappresenta la variazione 
dell'RNA e delle proteine espresse.\\
Per farlo indichiamo (\textbf{questo indice va sistemato}):
\begin{itemize}
  \item $\alpha$, proteine/cellula dal promotore non represso
  \item $\alpha_0$, proteine/cellula dal promotore represso
  \item $\beta$, rapporto proteina/velocità di decadimento dell'\textit{mRNA}
  \item $n$, \textit{coefficiente di cooperatività di Hill} (nel caso del
  Repressilator si ha $n=2$)
  \item $m_i$, i-esimo \textit{mRNA}
  \item $p_i$, i-esima proteina che funge da repressore
\end{itemize}
L'intero sistema viene modellato con \textit{coppie di equazioni
  differenziali}. Si hanno quindi: 
\begin{itemize}
  \item un'equazione che ci rappresenta la velocità di variazione dell'i-esimo
  mRNA:
  \[\frac{\dd{m_i}}{\dd{t}}=-m_i+\frac{\alpha}{1+p_j^n}+\alpha_0\]
  Tale velocità dipende dalla quantità che già si ha di mRNA, dalla presenza
  della proteina che lo reprime (essendo sotto nella frazione al crescere il
  termine tende a zero, mentre al diminuire tende a 1)
  \item un'equazione che ci rappresenta la velocità di variazione dell'i-esima
  proteina che funge da repressore:
  \[\frac{\dd{p_i}}{\dd{t}}=\beta(m_i-p_i)\]
  Tale velocità dipende da quanto mRNA si ha a disposizione meno la quantità di
  proteina che si ha a disposizione in quel dato momento. Maggiore è la
  quantità di mRNA e maggiore è la produzione fino a che la proteina stessa
  non supera un certo livello di quantità, avendo che ``satura''
\end{itemize}
\textbf{Nelle formule forse indici delle proteine sbagliati}.\\
In ordine si hanno, per i geni:
\begin{table}[H]
  \centering
  \begin{tabular}{c|c|c|c}
    Indice & 1 & 2 & 3\\
    \hline
    \hline
    $i$ & $lacI$ & $ tetR$ & $\lambda cI$\\
    \hline
    $j$ & $\lambda cI$ & $lacI$ & $ tetR$ 
  \end{tabular}
\end{table}
Con ``velocità di variazione'' si intende in pratica un tasso di cambio di
concentrazione delle due \textit{specie molecolari}, ovvero un'entità che
osserviamo nel modello (in questo caso mRNA o proteina). \\
Le concentrazioni si esprimono con l'unità di misura $K_M$ e il tempo in
$\tau_{mRNA}$, ovvero la velocità di trascrizione.
Integrando numericamente le due equazioni differenziali otteniamo un
comportamento periodico.\\
L'esperimento è stato fatto poi osservando come tutto questo diventa osservabile
in una colonia di E. Coli, opportunamente trattata, usando delle foto (dove si è
osservato anche un drift verso l'alto nel grafico oscillatorio a causa del fatto
che la colonia si espande).\\
La conoscenza di tipo matematico deve però essere trasferita in un esperimento
reale che funzioni (e i ricercatori devono essere in grado di manipolare
entrambi gli aspetti, si quello della modellazione matematica che quello più
biologico e chimico). In questo caso per ottenere oscillazioni stabili servono
determinati prerequisiti:
\begin{itemize}
  \item usare inibitori artificiali piccoli, con la cosiddetta \textit{low
    leakiness} 
  \item la velocità di decadimento di proteine e mRNA doveva essere simile, per
  ottenere l'oscillazione. Questo si ottiene attaccando \textit{ssrA} ad ogni
  repressore 
  \item servono curve di repressione piuttosto ``ripide''. Per questo si è
  usato un promotore con multipli \textit{binding sites} (arrivando alla scelta
  di quelle date proteine), usando repressori
  cooperativi (questo è rappresentato con il parametro $n$) 
  \item usare un \textit{Reporter} non stabile, attaccando una variante di
  \textit{ssrA} a \textit{Gfp}
\end{itemize}
\begin{figure}[H]
  \centering
  \includegraphics[scale = 0.55]{img/reprprot.pdf}
  \includegraphics[scale = 0.55]{img/reprmrna.pdf}
  \includegraphics[scale = 0.55]{img/reprmix.pdf}
  \caption{Grafici relativi al modello del Repressilator ottenuti tramite
    Mathematica. In primis, a sinistra la quantità di 
    repressore/proteina rispetto al tempo, a destra quella di mRNA (nel primo
    caso per le 3 proteine e nel secondo per i 3 mRNA). I grafici
    cambiano drasticamente quando l'insieme dei valori dei parametri viene
    modificato. In basso le quantità di mRNA (nel caso di $tetR$) rispetto al
    repressore/proteina (in questo caso ovviamente $TetR$) associata rispetto al
    tempo. Si nota che c'è un piccolo delay nel grafico, che rappresenta il
    tempo di traduzione. Le scale dei tre grafici sono indicative. I parametri
    sono specificati nel notebook di Mathematica presente nella pagina Moodle.}   
\end{figure}
\chapter{Modellazione di Sistemi Biologici}
Cerchiamo ora di capire come classificare i problemi, come analizzarli e
comprenderli (anche tramite machine learning) e avere coscienza delle risorse
online disponibili per la tematica.\\
Buona parte della ricerca in biologia computazionale ha come obiettivo quello di
ottenere il passaggio dai risultati di laboratorio alle applicazioni cliniche
(ed è qualcosa di molto complesso). Per quanto ci sia interesse verso tutte le
patologie la più interessante e più studiata (soprattutto in questo corso) è il
\textbf{cancro}. Un esempio di un sistema particolare dove i tumori si
sviluppano è quello delle cosiddette \textbf{cripte coloniche (\textit{colonic
    crypts})}, avendo che questo sistema è relativamente semplice da studiare
dal punto di vista computazionale.\\
Le \textit{cripte coloniche} si trovano nell'intestino e sono delle sorta di
``pozzetti'', morfologicamente divisibili in varie aree.
Alla base delle cripte ci sono delle \textbf{cellule staminali epiteliali}, che
sono quelle che poi danno luogo ai tessuti dell'epitelio.\\
\textit{Dal punto di vista matematico tutti gli essere viventi sono di topologia
isomorfa a dei tubi.}\\
Tornando al discorso delle cellule staminali si ha che essere si suddividono e,
man mano che si suddividono tendono a spingere verso l'alto le cellule che si
trovano ``al di sopra'' di loro. Man mano che tali cellule vengono spinte
anch'esse tendono a dividersi spingendo le altre cellule verso il \textit{lumen
  della cripta}. In questo processo di suddivisione queste cellule si
differenziano e le cellule staminali danno luogo ad una progenie che possiamo,
dal punto di vista in primis computazionale, rappresentare come un
\textit{albero}. Si hanno le cellule di tipo diverso, più o meno
differenziate che continuano a salire verso la superficie dell'epitelio e poi
tendono a salire su quelli che sono detti i \textit{villi intestinali}. Questo è
un interessante processo che può essere simulato, tra i vari modi, in modo tale
che si simuli cosa accade quando le varie differenziazioni non funzionano
perché, ad esempio, si ha una cellula che ha acquisito una mutazione, mutazioni
che danno luogo ad una crescita non corretta, ad una \textit{displasia}, che è
la fase iniziale da cui poi si sviluppano i \textit{tumori del colon}. Si vuole
quindi fare queste simulazioni e farle in modo il più fedele possibile.\\
Per capire se una cellula si sta comportando in modo corretto o meno dobbiamo
misurarne il comportamento. In primis vogliamo misurare due cose, tra le tante:
\begin{enumerate}
  \item \textbf{gene expression}
  \item \textbf{gene alterations}, ovvero le varie mutazioni del genome, le
  cosiddette le \textit{copy number variations} etc$\ldots$
\end{enumerate}
La tecnologia a disposizione per queste tematiche si è molto evoluta ma tra le
tante tecnologie si segnalano:
\begin{itemize}
  \item \textit{microarrays} per l'espressione genica, usati però molti anni fa
  essendo una delle prime tecnologie per misurare, in modo indiretto ma
  parallelo, l'espressione dei geni
  \item \textit{Next Generation Sequencing (NGS)} per praticamente qualsiasi
  cosa, anche per l'espressione genica, in modo diretto tramite particolari
  esperimenti (\textbf{nella rec non ho capito il nome di tali
    esperimenti}). NGS ha avuto molta fama da circa il 2006 in poi, con il
  monopolio poi di Illumina, anche se di recente si hanno nuove tecnologie che
  stanno rivoluzionando il settore (che producono read più lunghe)
\end{itemize}
\subsection{Microarrays}
% 16.23 lezione 2
\end{document}

% LocalWords:  clock traslazionale pathway Repressilator nucleoide plasmide
% LocalWords:  cooperatività read
