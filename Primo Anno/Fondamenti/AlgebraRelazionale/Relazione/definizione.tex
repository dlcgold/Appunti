\chapter{Relazioni}
Si definisce \textit{relazione n-aria} un sottoinsieme del prodotto cartesiano
rappresentato da tutte le coppie che rispettano la relazione voluta tra gli $n$ insiemi,
che può essere definita in maniera estensionale e/o intensionale. \newline
Si definisce \textit{arietà} di una relazione il numero e il tipo degli argomente
di una relazione.

%Definizione Dominio e Codominio di una relazione
\textbf{Dominio}:insieme degli elementi $x$ tali che $(x,y) \in R$ per qualsiasi $y$.\newline
\textbf{Codominio}:insieme degli elementi $y$ tali che $(x,y) \in R$ per qualsiasi $x$.

\begin{align*}
A \times B & = \{(1,1),(1,4),(1,5),(2,1),(2,4),(2,5),(3,1),(3,4),(3,5)\} \\
R \subseteq A \times B  & = \{(1,1),(1,4),(1,5),(2,4),(2,5),(3,4),(3,5)\}\\
R \subseteq B \times A & = \{(1,1),(1,2),(1,3)\} \\
\end{align*}
