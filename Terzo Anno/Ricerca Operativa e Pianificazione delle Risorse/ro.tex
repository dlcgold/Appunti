\documentclass[a4paper,12pt, oneside]{book}

%\usepackage{fullpage}
\usepackage[italian]{babel}
\usepackage[utf8]{inputenc}
\usepackage{amssymb}
\usepackage{amsthm}
\usepackage{graphics}
\usepackage{amsfonts}
\usepackage{listings}
\usepackage{amsmath}
\usepackage{amstext}
\usepackage{engrec}
\usepackage{rotating}
\usepackage[safe,extra]{tipa}
\usepackage{showkeys}
\usepackage{multirow}
\usepackage{hyperref}
\usepackage{microtype}
\usepackage{enumerate}
\usepackage{braket}
\usepackage{marginnote}
\usepackage{pgfplots}
\usepackage{cancel}
\usepackage{polynom}
\usepackage{booktabs}
\usepackage{enumitem}
\usepackage{framed}
\usepackage{pdfpages}
\usepackage{pgfplots}
\usepackage[cache=false]{minted}

\usepackage{tikz}\usetikzlibrary{er}\tikzset{multi  attribute /.style={attribute ,double  distance =1.5pt}}\tikzset{derived  attribute /.style={attribute ,dashed}}\tikzset{total /.style={double  distance =1.5pt}}\tikzset{every  entity /.style={draw=orange , fill=orange!20}}\tikzset{every  attribute /.style={draw=MediumPurple1, fill=MediumPurple1!20}}\tikzset{every  relationship /.style={draw=Chartreuse2, fill=Chartreuse2!20}}\newcommand{\key}[1]{\underline{#1}}

\usepackage{fancyhdr}
\pagestyle{fancy}
\fancyhead[LE,RO]{\slshape \rightmark}
\fancyhead[LO,RE]{\slshape \leftmark}
\fancyfoot[C]{\thepage}



\title{Ricerca Operativa e Pianificazione delle Risorse}
\author{UniShare\\\\Davide Cozzi\\\href{https://t.me/dlcgold}{@dlcgold}\\\\Gabriele De Rosa\\\href{https://t.me/derogab}{@derogab} \\\\Federica Di Lauro\\\href{https://t.me/f_dila}{@f\textunderscore dila}}
\date{}

\pgfplotsset{compat=1.13}
\begin{document}
\maketitle

\definecolor{shadecolor}{gray}{0.80}
\setlist{leftmargin = 2cm}
\newtheorem{teorema}{Teorema}
\newtheorem{definizione}{Definizione}
\newtheorem{esempio}{Esempio}
\newtheorem{corollario}{Corollario}
\newtheorem{lemma}{Lemma}
\newtheorem{osservazione}{Osservazione}
\newtheorem{nota}{Nota}
\newtheorem{esercizio}{Esercizio}
\tableofcontents
\renewcommand{\chaptermark}[1]{%
	\markboth{\chaptername
		\ \thechapter.\ #1}{}}
\renewcommand{\sectionmark}[1]{\markright{\thesection.\ #1}}
\chapter{Introduzione}
\textbf{Questi appunti sono presi a lezione. Per quanto sia stata
  fatta una revisione è altamente probabile (praticamente certo)
  che possano contenere errori, sia di stampa che di vero e proprio
  contenuto. Per eventuali proposte di correzione effettuare una
  pull request. Link: } \url{https://github.com/dlcgold/Appunti}.\\
\textbf{Grazie mille e buono studio!}
\chapter{Introduzione alla Ricerca Operativa}
La\textbf{ Ricerca Operativa} è essenziale nel \textit{problem
  solving} e nell'ambito del \textit{decision making}.
Sostanzialmente quindi si studia l'ottimizzazione, massimizzando le
performances, l'accuratezza dei costi etc$\ldots$ per raggiungere un
obiettivo. \\ \textit{Sulle slides ci sono vari esempi introduttivi di
  vita reale}\\
Un altro problema studiato dalla riceca operativa sono le previsioni,
mediante algoritmi predittivi che studiano i \textit{pesi} delle
osservazioni (cosa utile nel \textbf{Machine Learning} in quanto sono
un uso di base delle \textbf{Reti Neurali}, \textit{vari esempi
  introduttivi sulle slides}).\\
\textbf{La ricerca operativa si occupa di formalizzare un problema in
  un modello matematico e calcolare una soluzione ottimo o
  approssimata}. Essa costituisce un approccio scientifico alla
risoluzione di problemi complessi da ricondurre alla matematica
applicata. È utile in ambiti economici, logistici, di progettazione di
servizi e di sistemi di trasporto e, ovviamente, nelle tecnologie.
\textit{È la branca della matematica più applicata}.\\
Il \textit{primo passo} consiste nel costruire un modello traducendo il
problema reale in linguaggio anturale in un linguaggio matematico, che
non è ambiguo. Il \textit{secondo passo} consiste nella costruzione delle
soluzioni del modello tramite algoritmi e programmi di calcolo. Il
\textit{terzo passo}, ovvero l'ultimo, è l'interpretazione e la
valutazione delle soluzioni del modello rispetto a quelle del problema
reale.\\
La ricerca operativa ha origini nel 1800 in un ambiente puramente
matematico. È stata resa ``\textit{algoritmica}'' con la Macchina di
Turing. \textbf{La ricerca operativa usa anche tecniche numeriche e
  non solo analitiche}.\\
Negli ultimi hanno si sono sviluppati, mediante il concetto di
\textbf{gradiente}, nuovi algoritmi per il \textbf{deep network}.\\
\section{Modelli nella R.O.}
\begin{definizione}
Data una funzione $f:\mathbb{R} \to mathbb{R}$ e $X\subseteq
\mathbb{R}^n$ un \textbf{problema di ottimizzazione} può esssere
formulato come:
\[opt\,\,f(x)\,\,s.t. \,\,x\in X\]
dove con $opt={\min, \max}$ indendiamo che opt può essere o min o max,
portando ad un problema di minimizzazione con $\min\,f(x)$ o di
massimizzazione $\max\,f(x)$. \\
$f(x)$ è detta \textbf{funzione obiettivo} e vale che:
\[max[f(x):\,x\in X] = -min[-f(x):\,x\in X]\]
Inoltre $x\subseteq\mathbb{R}^n$ è \textbf{l'insieme delle soluzioni
  ottenibili} o anche \textbf{regione ammissibile}.\\
Infine $x\in X$ rappresenta il \textbf{vettore delle variabili
  decisionali} e si tratta di variabili numeriche i cui valori
rappresentano la soluzione del problema. \\
\textbf{Si capisce che essendo in $\mathbb{R}$ si hanno infinite
  soluzioni}.\\
% aggiungere fine definizione

Se $X=\mathbb{R}^n$ si ha un'\textbf{ottimizzazione non vincolata}, altrimenti,
$x\subset \mathbb{R}$ si ha un'\textbf{ottimizzazione vincolata}, dove la
ricerca dei punti di ottimo della funzione obiettivo è fatta su un
sottoinsieme proprio dello spazio di definizione tenendo però conto
dei vincoli. Se ho una funzione obiettivo lineare non si può avere
un'ottimizzazione non vincolata (non saprei cercare massimi e minimi
senza vincoli).\\
Abbiamo poi l'\textbf{ottimizzazione intera o a numeri interi}
se $x\in \mathbb{Z}^n$ e si possono avere ottimizzazioni miste se si
hanno interi e reali. Si ha anche l'\textbf{ottimizzazione binaria}
quando si hanno due vie decisionali. \\
\textbf{Se non specificato si intende $X\subseteq\mathbb{R}}$.
\end{definizione}
\begin{definizione}
Quando l'insieme $X$ delle soluzioni ammissibili di un problema di
ottimizzazione è espresso in un sistema di equazioni o disequazioni si
parla di \textbf{problema di programmazione matematica (PM)}.\\
Come vincolo si ha un'espressione $g_i(x)\{\leq, =, \geq\} 0$
($g_i\geq 0$ etc $\ldots$) e con
$g_i:X\to \mathbb{R}$ che è una generica funzione che lega due
variabili. \\
\textbf{Si possono avere più vincoli ma si ha sempre l'uguale in ogni
  vincolo per permettere il funzionamento degli algoritmi}.\\
La regione ammissibile è $X\subseteq\mathbb{R}^n$ che è l'intersezione
di tutti i vincoli del problema
\[X=\{x\in\mathbb{R}^n|\, g_i(x)\{\leq, 0, \geq\}\]
%termina definizione
\end{definizione}
\begin{esempio}
  abbiamo la funzione obiettivo
  \[\min_{x,y}(x^2+y^2)\]
  con i 3 vincoli:
  \[x+y \leq 3\]
  \[x\geq 0\]
  \[y\geq 0\]
  la regione ammissibile è:
  \[\{x\in\mathbb{R}^2|\,x+y \leq 3,\,\,x\geq 0,\,\,y\geq 0\}\]
  % inserire grafico
\end{esempio}
Si possono avere problemi con regione non ammissibile, ovvero con
$X=\emptyset$, che implica che il problema è mal posto oppure bisogna
abbassare qualche vincolo. Si può avere un problema illimitato con:
\[\forall c \in \mathbb{R}\exists x_c\in X:f(x_c)\leq c\,\, se\,\,
  opt = min\]
\[\forall c \in \mathbb{R}\exists x_c\in X:f(x_c)\geq c\,\, se\,\,
  opt = max\]
Infine si può avere una sola soluzione ottima o più (anche infinite)
soluzioni ottime
% finire frase
\begin{esempio}
  abbiamo la funzione obiettivo
  \[\min_{x,y}(x^2+y^2)\]
  con i 3 vincoli:
  \[x+y \leq -1\]
  \[x\geq 0\]
  \[y\geq 0\]
  Non ha soluzione (è matematicamente impossibile) e il problema
  non è ammissibile
\end{esempio}
\begin{esempio}
  abbiamo la funzione obiettivo
  \[\max_{x,y}(x^2+y^2)\]
  con i 2 vincoli:
  \[x\geq 0\]
  \[y\geq 0\]
  Ha come soluzione infinito
\end{esempio}
\begin{esempio}
  abbiamo la funzione obiettivo
  \[\max_{x,y,z}(z)\]
  con i 4 vincoli:
  \[x+y+z = 2\]
  \[0\leq x \leq 1\]
  \[0\leq y \leq 1\]
  \[0\leq z \leq 1\]
  Ha infinite soluzioni (tutte le soluzioni con $z=1$ e $x+y=1$, in quanto cerco
  il max di z e come ultio vincolo ho che al massimo è 1)
  % aggiungere grafico
\end{esempio}

\end{document}