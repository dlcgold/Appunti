\documentclass[a4paper,12pt, oneside]{book}

% \usepackage{fullpage}
\usepackage[italian]{babel}
\usepackage[utf8]{inputenc}
\usepackage{amssymb}
\usepackage{amsthm}
\usepackage{graphics}
\usepackage{amsfonts}
\usepackage{listings}
\usepackage{amsmath}
\usepackage{amstext}
\usepackage{engrec}
\usepackage{rotating}
\usepackage{verbatim}
\usepackage[safe,extra]{tipa}
% \usepackage{showkeys}
\usepackage{multirow}
\usepackage{hyperref}
\usepackage{microtype}
\usepackage{fontspec}
\usepackage{enumerate}
\usepackage{physics}
\usepackage{braket}
\usepackage{marginnote}
\usepackage{pgfplots}
\usepackage{cancel}
\usepackage{polynom}
\usepackage{booktabs}
\usepackage{enumitem}
\usepackage{framed}
\usepackage{pdfpages}
\usepackage{pgfplots}
\usepackage{algorithm}
% \usepackage{algpseudocode}
\usepackage[cache=false]{minted}
\usepackage{mathtools}
\usepackage[noend]{algpseudocode}
\newcommand*{\bfrac}[2]{\genfrac{}{}{0pt}{}{#1}{#2}}

\usepackage{tikz}\usetikzlibrary{er}\tikzset{multi  attribute /.style={attribute
    ,double  distance =1.5pt}}\tikzset{derived  attribute /.style={attribute
    ,dashed}}\tikzset{total /.style={double  distance =1.5pt}}\tikzset{every
  entity /.style={draw=orange , fill=orange!20}}\tikzset{every  attribute
  /.style={draw=MediumPurple1, fill=MediumPurple1!20}}\tikzset{every
  relationship /.style={draw=Chartreuse2,
    fill=Chartreuse2!20}}\newcommand{\key}[1]{\underline{#1}}
\usetikzlibrary{arrows.meta}
\usetikzlibrary{decorations.markings}
\usetikzlibrary{arrows,shapes,backgrounds,petri}
\tikzset{
  place/.style={
    circle,
    thick,
    draw=black,
    minimum size=6mm,
  },
  transition/.style={
    rectangle,
    thick,
    fill=black,
    minimum width=8mm,
    inner ysep=2pt
  },
  transitionv/.style={
    rectangle,
    thick,
    fill=black,
    minimum height=8mm,
    inner xsep=2pt
  }
} 
\usetikzlibrary{automata,positioning,chains,fit,shapes}
\usepackage{fancyhdr}
\pagestyle{fancy}
\fancyhead[LE,RO]{\slshape \rightmark}
\fancyhead[LO,RE]{\slshape \leftmark}
\fancyfoot[C]{\thepage}
\usepackage[usenames,dvipsnames]{pstricks}
\usepackage{epsfig}
\usepackage{pst-grad} % For gradients
\usepackage{pst-plot} % For axes
\usepackage[space]{grffile} % For spaces in paths
\usepackage{etoolbox} % For spaces in paths
\makeatletter % For spaces in paths
\patchcmd\Gread@eps{\@inputcheck#1 }{\@inputcheck"#1"\relax}{}{}
\makeatother
\usepackage{lipsum}
\DeclareSymbolFont{symbolsC}{U}{txsyc}{m}{n}
\DeclareMathSymbol{\strictif}{\mathrel}{symbolsC}{74}
\title{Data and Computational Biology}
\author{UniShare\\\\Davide Cozzi\\\href{https://t.me/dlcgold}{@dlcgold}}
\date{}

\pgfplotsset{compat=1.13}
\begin{document}
\maketitle

\definecolor{shadecolor}{gray}{0.80}
\setlist{leftmargin = 2cm}
\newtheorem{teorema}{Teorema}
\newtheorem{definizione}{Definizione}
\newtheorem{esempio}{Esempio}
\newtheorem{corollario}{Corollario}
\newtheorem{lemma}{Lemma}
\newtheorem{osservazione}{Osservazione}
\newtheorem{nota}{Nota}
\newtheorem{esercizio}{Esercizio}
\algdef{SE}[DOWHILE]{Do}{doWhile}{\algorithmicdo}[1]{\algorithmicwhile\ #1}
\tableofcontents
\renewcommand{\chaptermark}[1]{%
  \markboth{\chaptername
    \ \thechapter.\ #1}{}}
\renewcommand{\sectionmark}[1]{\markright{\thesection.\ #1}}
\newcommand{\floor}[1]{\lfloor #1 \rfloor}
\newcommand{\MYhref}[3][blue]{\href{#2}{\color{#1}{#3}}}%
\chapter{Introduzione}
\textbf{Questi appunti sono presi a lezione. Per quanto sia stata fatta
  una revisione è altamente probabile (praticamente certo) che possano
  contenere errori, sia di stampa che di vero e proprio contenuto. Per
  eventuali proposte di correzione effettuare una pull request. Link: }
\url{https://github.com/dlcgold/Appunti}.
\chapter{Introduzione alla Biologia Computazionale}
La \textbf{biologia} nasce come una disciplina altamente \textbf{descrittiva}
mentre altre discipline, come, ad esempio, informatica, matematica o fisica,
sono discipline \textbf{generaliste}.\\
I biologi propongono \textbf{modelli}, come ad esempio i \textit{pathway}, che
sono il diretto risultato di osservazioni sperimentali e interpretazione dei
risultati.
\begin{definizione}
  Un \textbf{pathway (\emph{percorso}) biologico} è una serie di interazioni
  tra molecole in una cellula che porta a un determinato prodotto o un
  cambiamento in una cellula. Tale \emph{percorso} può innescare
  l'assemblaggio di nuove molecole, come un grasso o una proteina. I
  \emph{percorsi} possono anche attivare e disattivare i geni o stimolare una
  cellula a muoversi. I pathway più comuni sono coinvolte nel metabolismo, nella
  regolazione dell'espressione genica e nella trasmissione dei segnali e
  svolgono un ruolo chiave negli studi avanzati di genomica.\\
  Tra le principali categorie si hanno:
  \begin{itemize}
    \item Metabolic pathway
    \item Genetic pathway
    \item Signal transduction pathway
  \end{itemize}
\end{definizione}
Un altro aspetto chiave negli ultimi 25 anni è stato quello della
mole di dati prodotti, tramite, ad esempio, \textbf{Next Generation Sequencing
  (\textit{NGS})}, con la produzione di \textit{DNAseq} e \textit{RNAseq}, o
alla cosiddetta \textbf{single-cell analysis}. Tutte queste tecnologie
\textit{high-throughput} usate in biologia computazionale e in bioinformatica
richiedono una forte conoscenza algoritmica, matematica e statistica per la
gestione di questa enorme quantità di dati (essendo anche nell'ambito
\textbf{big data}) in ambito biomedico. Ovviamente le conoscenze, i tempi (ma
anche gli spazi), gli strumenti da usare e sviluppare etc$\ldots$ variano al
variare del tipo di studio.\\ 
Un altro aspetto non trascurabile è la scala di misura di ciò che viene
studiato, ad esempio:
\begin{itemize}
  \item \textit{organismi}, ad esempio per gli organismi multicellulari si passa
  da $10\mu m$ a $50/85m$ 
  \item \textit{tessuti}, ad esempio per i tessuti umani siamo in un range
  maggiore di $10^4 \mu m^3$
  \item \textit{cellule}, ad esempio per quelle umane si va da $30\mu m^3$ a
  $10^6 \mu m^3$ con:
  \begin{itemize}
    \item membrane
    \item nuclei
    \item ribosomi
    \item mitocondri e cloroplasti
    \item altri organelli e strutture intracellulari
    \item proteine
    \item materiale genomico (DNA e RNA e affini strutture: ad esempio istoni) 
    \item $\ldots$
  \end{itemize}
\end{itemize}
Parlando di tipi di organismi distinguiamo in primis:
\begin{itemize}
  \item \textbf{eucarioti}. In questo caso si hanno cellule più complesse, con
  numerosi organelli e soprattutto il \textbf{nucleo}, dove sono contenute le
  informazioni
  \item \textbf{procarioti}, come i \textit{batteri}. In questo caso si hanno
  cellule piccole e semplici. Non hanno un nucleo ma solo una regione, detta
  \textbf{nucleoide}, dove sono contenute le informazioni
\end{itemize}
In aggiunta si hanno anche i \textbf{virus}.\\
\textit{Per ulteriori informazioni sui tipi di organismi guardare online}.\\
Parlando di DNA si ha che ogni cellula umana contiene circa 2 metri di DNA e un
organismo umano contiene moltissime cellule rendendo lo studio del DNA davvero
complesso (anche dal punto di vista computazionale si hanno file di genomi
davvero molto pesanti, di centinaia di $MB$).\\
\textbf{Riprendere da appunti di Bioinformatica il passaggio da DNA a RNA e da
  RNA a Proteine}.\\
Ad essere interessanti non sono solo le dimensioni di ciò che viene studiato ma
anche i vari \textbf{tempi}. Vediamo una piccola tabella d'esempio:
\begin{table}[H]
  \small
  \centering
  \begin{tabular}{c|c|c}
    \textbf{Proprietà} & \textbf{E. coli} & \textbf{Uomo}\\
    \hline
    \hline
    diffusione di proteine in una cellula & $0.1 s$ & $\sim 100 s$\\
    \hline
    trascrizione di un gene & $\sim 1m$ ($80\frac{bp}{s}$) & $\sim 100 s$\\
    \hline
    generazione di una cellula & da $30 m$ a ore & da $20h$ a statico\\
    \hline
    transizione di stato proteico & da $1\mu s$ a $100\mu s$
                                          & da $1\mu s$ a $100\mu s$\\
    \hline
    rate di mutazione & $\sim \frac{10^{-9}}{\frac{bp}{generazione}}$
                                      & $\sim \frac{10^{-8}}{\frac{bp}{anno}}$\\
  \end{tabular}
\end{table}
Qualche nota:
\begin{itemize}
  \item i tempi di trascrizione di un gene umano includono i tempi di
  preprocessamento dell'\textit{mRNA}
  \item per la generazione di una cellula di E. Coli si hanno 30 minuti in
  presenza di nutrienti
  \item 
\end{itemize}
Si studiano quindi i vari \textbf{modelli} per la biologia computazionale che
possono essere di varie tipologie:
\begin{itemize}
  \item \textbf{continui}, tramite equazioni differenziali ordinarie
  \item \textbf{discreti}
  \item \textbf{stocastici}
\end{itemize}
Si studiano, in ottica analisi di cancro, anche \textbf{grafi mutazionali} e
\textbf{evoluzioni clonali} (tramite Single-cell analysis).\\
Un aspetto fondamentale è costituito dall'RNA, che trasposta le informazioni dal
DNA (contenuto nel nucleo) al citoplasma della cellula, dove funge da
intermediario per il processo di sintesi delle proteine.
\begin{teorema}[Dogma principale di Francis Crick]
  Si ha quindi il dogma principale della biologia molecolare:
  \begin{center}
    \textbf{il flusso d'informazione è unidirezionale}
  \end{center}
  ovvero, in termini più estesi:
  \begin{center}
    \emph{una volta che le ``informazioni'' sono passate nelle proteine, non
      possono uscirne nuovamente. Il trasferimento di informazioni da acido
      nucleico ad acido nucleico, o da acido nucleico a proteina, può essere
      possibile, ma il trasferimento da proteina a proteina, o da proteina ad
      acido nucleico è impossibile. Per ``informazione'' si intende qui la
      precisa determinazione della sequenza, sia delle basi nell'acido nucleico
      che dei residui amminoacidici nella proteina.} 
  \end{center}
\end{teorema}
Geni, proteine e cellule sono il \textit{linguaggio macchina} della vita.\\
Veniamo quindi alla distinzione delle due branche di
studio. \textbf{Bioinformatica} e \textbf{Biologia (del Sistema) Computazionale}
sono due aspetti sovrapposti del modo in cui usiamo l'approccio computazionale
alla Biologia e alla Medicina, manipolando oggetti simili ma con enfasi diversa
e diverse scale spazio-temporali. In entrambe si usano ontologie, formalismi
descrittive ma anche, lato più pratico, database. Nel dettaglio:
\begin{itemize}
  \item la \textbf{Bioinformatica} si occupa in primis dell'\textbf{analisi di
    sequenze} ovvero, tra le altre cose, di studio del genoma, RNA folding,
  folding di proteine e studio dei database necessari a questi studi. Si usano
  algoritmi di pattern matching e altri metodi di analisi delle stringhe
  \item la \textbf{Biologia (del Sistema) Computazionale} studia, tra le altre
  cose:
  \begin{itemize}
    \item modelli e inferenze sulle proprietà dei sistemi, studiando simulazioni
    e nuove proprietà
    \item ricostruzione di reti metaboliche e regolatorie e di modelli di
    progressione 
  \end{itemize}
  Si usano, ad esempio, metodi di machine learning per l'analisi dei dati
  prodotti e si simulano modelli biologici in modo sia deterministico che
  stocastico (tramite ad esempio Gillespie e Monte Carlo) e si fa analisi di
  raggiungibilità 
\end{itemize}
D'altro canto, tecniche come la \textbf{Polymerase chain reaction
  (\textit{PCR})} ed altre sono appannaggio di biologi e biotecnologi.
L'interesse per un biologo computazionale e per un bioinformatico è quello di
aiutare altri ricercatori a svolgere le proprie attività. Ad esempio i biologi
traggono vantaggio in ottica di acquisire conoscenze di base o anche al ricevere
strumenti atti al progettare e pianificare esperimenti. Gli esperimenti
biologici sono costosi sia dal punto di vista dei materiali che di persone e
tempo. \\
In biologia computazionale si è quindi interessati a comprendere, anche in
termini computazionali, l'interazione complessiva di:
\begin{itemize}
  \item processi intracellulari (regolatori e metabolici)
  \item cellule singole
  \item popolazioni cellulari 
\end{itemize}
Un altro compito dei biologi computazionali è quello di capire cosa
succede quando si ha la possibilità di perturbare un sistema e vedere quali sono
gli effetti della perturbazione, in particolare vedere cosa succede usando un
dato farmaco piuttosto che un altro per intervenire su una certa patologia,
parlando, in questo caso, del cosiddetto \textbf{momento traslazionale} della
\textbf{medicina traslazionale}. Con ``momento'' ci si riferisce al
trasferimento di conoscenze delle attività di pura ricerca alle \textbf{attività
  di produzione}, ovvero all'\textit{attività clinica}, con il passaggio alla
``vita vera''.\\
% 2.57

\end{document}

% LocalWords:  clock  traslazionale pathway
