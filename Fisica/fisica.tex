\documentclass[a4paper,12pt, oneside]{book}

%\usepackage{fullpage}
\usepackage[italian]{babel}
\usepackage[utf8]{inputenc}
\usepackage{amssymb}
\usepackage{amsthm}
\usepackage{graphics}
\usepackage{amsfonts}
\usepackage{amsmath}
\usepackage{amstext}
\usepackage{engrec}
\usepackage{rotating}
\usepackage{verbatim}
\usepackage[safe,extra]{tipa}
\usepackage{showkeys}
\usepackage{multirow}
\usepackage{hyperref}
\usepackage{microtype}
\usepackage{enumerate}
\usepackage{braket}
\usepackage{marginnote}
\usepackage{pgfplots}
\usepackage{cancel}
\usepackage{polynom}
\usepackage{booktabs}
\usepackage{enumitem}
\usepackage{framed}
\usepackage{pdfpages}
 \usepackage[usenames,dvipsnames]{pstricks}
 \usepackage{epsfig}
 \usepackage{pst-grad} % For gradients
 \usepackage{pst-plot} % For axes
 \usepackage[space]{grffile} % For spaces in paths
 \usepackage{etoolbox} % For spaces in paths
 \makeatletter % For spaces in paths
 \patchcmd\Gread@eps{\@inputcheck#1 }{\@inputcheck"#1"\relax}{}{}
 \makeatother
\usepackage{pgfplots}
\usepackage{fancyhdr}
\pagestyle{fancy}
\fancyhead[LE,RO]{\slshape \rightmark}
\fancyhead[LO,RE]{\slshape \leftmark}
\fancyfoot[C]{\thepage}


\title{Fisica}
\author{UniShare\\\\Davide Cozzi\\\href{https://t.me/dlcgold}{@dlcgold}}
\date{}

\pgfplotsset{compat=1.13}
\begin{document}
\maketitle

\definecolor{shadecolor}{gray}{0.80}

\newtheorem{teorema}{Teorema}
\newtheorem{definizione}{Definizione}
\newtheorem{esempio}{Esempio}
\newtheorem{corollario}{Corollario}
\newtheorem{lemma}{Lemma}
\newtheorem{osservazione}{Osservazione}
\newtheorem{nota}{Nota}
\newtheorem{esercizio}{Esercizio}
\tableofcontents

\renewcommand{\chaptermark}[1]{%
\markboth{\chaptername
\ \thechapter.\ #1}{}}
\renewcommand{\sectionmark}[1]{\markright{\thesection.\ #1}}

\chapter{Introduzione}
\textbf{Questi appunti sono presi a lezione. Per quanto sia stata fatta una revisione è altamente probabile (praticamente certo) che possano contenere errori, sia di stampa che di vero e proprio contenuto. Per eventuali proposte di correzione effettuare una pull request. Link: } \url{https://github.com/dlcgold/Appunti}.\\
\textbf{Grazie mille e buono studio!}

\chapter{Meccanica}
Si comincia con la Meccanica, la branca della fisica classica che studia il moto dei corpi, esprimendolo con leggi quantitative. Si ha la seguente divisione:
\begin{itemize}
\item \textbf{Cinematica}, dove si studia il moto e le sue caratteristiche indipendentemente dalle cause
\item \textbf{Dinamica}, dove si studia l'influenza delle forze nel moto 
\end{itemize}
Si utilizzano i cosiddetti \textit{punti materiali} per semplificare lo studio dei fenomeni. Un punto materiale infatti non ha estensione ma è dotato di una massa. In pratica ha dimensioni trascurabili rispetto allo spazio nel quale si muove.\\
Un altro strumento essenziale per lo studio dei fenomeni è il \textit{sistema di riferimento} mediante gli assi ortogonali:
\begin{center}
\includegraphics[scale=0.7]{img/ref.png}
\end{center}
\newpage
e si hanno le seguenti formule:
$$R=\sqrt{X^2+Y^2}$$
$$sin \vartheta=\frac{Y}{R}$$
$$cos \vartheta=\frac{X}{R}$$
$$tan \vartheta = \frac{Y}{X}$$
$$\vartheta= arctan \frac{Y}{X}$$
e per gli angoli si usano i \textit{radianti} in quanto adimensionali. L'angolo in radianti infatti è:
$$\vartheta_{rad}=\frac{Lunghezza\_arco}{raggio}$$
dove le due unità di misura esprimenti una lunghezza vengono "semplificate".\\
Si ricordano inoltre le basi del calcolo vettoriale. Tra due vettori posso fare somme e sottrazioni 
La somma non è altro che la diagonale maggiore del parallelogramma che si forma tra i due vettori. Inoltre se $\vec{A}=(a_x,a_y)$ e $\vec{B}=(b_x,b_y)$ si ha:
$$\vec{C}=\vec{A}+\vec{B}=(a_x+b_x, a_y+b_y)$$
la sottrazione è la diagonale minore e:
$$\vec{C}=\vec{A}-\vec{B}=(a_x-b_x, a_y-b_y)$$
\begin{center}
\includegraphics[scale=0.5]{img/vec.png}
\quad
\includegraphics[scale=0.5]{img/vec2.png}
\end{center}
\section{Cinematica}
Innanzitutto qualche definizione:
\begin{itemize}
\item \textbf{Moto:} posizione in funzione del tempo in un dato sistema di riferimento
\item \textbf{Traiettoria:} luogo dei punti attraversati dal punto materiale in movimento
\item \textbf{Velocità:} variazione della posizione
\item \textbf{Accelerazione:} variazione della velocità
\item \textbf{Quiete:} assenza di movimento in un certo sistema di riferimento
\end{itemize}
Come grandezze fondamentali del movimento si hanno quindi \textit{posizione}, \textit{velocità} e \textit{accelerazione}, tutte e tre funzioni del tempo.
\subsection{Moto Rettilineo}
Rappresentando su un piano cartesiano avente la posizione come ordinata e il tempo come ascisse e rappresentando vri momenti del moto si ottiene una curva. Questa curva rappresenta la legge oraria.\\
Si ha la traiettoria più semplice, una retta. Il moto del punto quindi è esprimibile come funzione solo di $$\vec{x}(t)$$, che sarà la nostra equazione del moto.\\
Si passa quindi da un sistema di riferimento a 3 assi:
\begin{center}
\includegraphics[scale=0.5]{img/rett.png}
\end{center}
ad uno a un asse:
\begin{center}
\includegraphics[scale=0.3]{img/ret2.png}
\end{center}
La scelta dell'origine della coordinata spaziale ($x=0$) e di quella temporale ($t=0$) sono arbitrari.\\
Si definisce la \textbf{distanza} come una quantità scalare la lunghezza del tratto percorso da un punto per cambiare posizione.
\subsubsection{Velocità}
Per ottenere la velocità di un punto materiale ne misuro la posizione in due diversi istanti di tempo. Si ha:
\begin{itemize}
\item \textbf{Spostamento:} $\Delta \vec{x}= x(t_2)-x(t_1)=x_2-x_1$ è un vettore che descrive la differenza di posizione tra due punti. Viene misurato in \textit{Metri (m)} secondo il Sistema Internazionale (SI). Il metro è definito come la distanza percorsa dalla luce in $\frac{1}{299792458}s$ 
\item \textbf{Intervallo di Tempo:} $\Delta t=t_2-t_1$ che viene misurato in \textit{Secondi (s)} secondo il Sistema Internazionale (SI). Il secondo è definito come la durata di $9192631770$ periodi della radiazione corrispondente alla transizione tra 2 livelli iperfini dello stato fondamentale dell'atomo di Cesio-133
\end{itemize}
Possiamo quindi definire la \textbf{Velocità Media:}
$$v_m=\frac{\Delta \vec{x}}{\Delta t}=\frac{x_2-x_1}{t_2-t_1}=\frac{\vec{v_2}-\vec{v_1}}{2}$$
Questa grandezza però non fornisce nessuna indicazione sulle caratteristiche effettive del moto. Provo a spezzare il moto in più intervalli temporali al fine di studiarne ogni variazione. Si ottiene quindi la \textbf{Velocità Istantanea}:
$$v=\lim_{\Delta t \to 0}\frac{\Delta \vec{x}}{\Delta t}=\frac{d \vec{x}(t)}{d t}$$
La velocità istantanea rappresenta la rapidità di variazione temporale della posizione nell'istante \textit{t} considerata. Il segno della velocità indica la direzione del moto sull'asse delle ascisse. La velocità è a sua volta funzione del tempo:
$$v(t)=\frac{d\vec{x}(t)}{dt}$$
che è ben rappresentata dai seguenti grafici:
\begin{center}
\includegraphics[scale=0.36]{img/ist.png}
\end{center}
Se \textit{v} è costante si parla di \textit{Moto Rettilineo Uniforme}. \\
Si ha quindi:
$$\Delta x = v\Delta t\to x-x_0=v(t-t_0)\to x=x_0+v(t-t_0)$$
che vale anche per $v$ non costante ma per intervalli di tempo approssimati 0, infatti tra brevi istanti di tempo si può approssimare la velocità istantanea $v(t)=\frac{dx}{dt}$ come una velocità costante. Disegniamo ora un grafico velocità tempo con la curva rappresentante la legge oraria, indicando velocità e tempo in due momenti del moto:
\begin{center}
\includegraphics[scale=0.5]{img/gra.png}
\end{center}
calcolare l'area sottesa alla curva implica calcolare la differenza di posizione. Approssimo la curva ad una retta e procedo col banale calcolo del trapezio sottostante:
$$A=(t_1-t_0)(\frac{\vec{v_1}-\vec{v_0}}{2})+(t_1-t_0
)\vec{v_0}=(\frac{\vec{v_1}-\vec{v_0}}{2})\Delta t+\vec{v_0}\Delta t$$
$$A=\frac{\Delta t}{2}(\vec{v_1}-\vec{v_0}+2\vec{v_0})=\frac{\Delta t}{2}(\vec{v_0}+\vec{v_1})=\Delta t v_{med}$$
Nota quindi l'equazione del moto $$\vec{x}(t)$$ possiamo ricavare $v(t)$ derivando, infatti la posizione si ottiene, partendo dal grafico sopra, riducendo al massimo gli intervalli di tempo e calcolando la somma delle aree dei vari rettangolini .\\
Si può procedere anche al calcolo di $$\vec{x}(t)$$ avendo nota $\vec{v}(t)$. Sappiamo che lo spostamento totale è: $\Delta \vec{x}=\sum_{i=1}^n \Delta \vec{x}_i=\sum_{i=1}^n v_{m_i} \Delta t$ e che, per intervalli infinitesimi $dx=\vec{v}(t) dt$. Si ha quindi:
$$\Delta x=\underbrace{\int_{x_0}^x dx}_{\vec{x}(t)-x_0}=\int_{t_0}^t \vec{v}(t) dt$$
$$\downarrow$$
$$\vec{x}(t)=x_0+\int_{t_0}^t \vec{v}(t) dt$$
che è l'equazione del moto rettilineo per una velocità qualunque.\\
Possiamo ora anche riscrivere la forma completa della velocità media, essendo $x-x_0=\int_{t_0}^t \vec{v}(t) dt$ si ha:
$$v_m=\frac{1}{t-t_0}\int_{t_0}^t \vec{v}(t) dt$$
Possiamo analizzare ora il \textit{moto rettilineo uniforme} con \textit{v} costante. Essendo \textit{v} costante, e non più dipendente dal tempo, può essere portata fuori dall'integrale:
$$\vec{x}(t)=x_0+v\int_{t_0}^t dt=x_0+v(t-t_0)$$
che è l'equazione generale del moto rettilineo uniforme dove lo spostamento varia linearmente col tempo.\\
La velocità di esprime in metri al secondo ($\frac{m}{s}$ o \textit{m/s}) o in kilometri all'ora $\frac{km}{h}$ o \textit{km/h}). Per passare da \textit{km/h} a \textit{m/s} divido la grandezza in \textit{km/h} per 3,6, per passare da \textit{m/s} a \textit{km/h} moltiplico la grandezza in \textit{m/s} per 3,6. 
\subsubsection{Accelerazione}
Si ha che in due istanti di tempo diversi abbiamo due diverse velocità: $\vec{v}(t_1)=\vec{v_1}$ e $\vec{v}(t_2)=\vec{v_2}$. Si definisce l'\textbf{Accelerazione Media:}
$$a_m=\frac{\vec{v_2}-\vec{v_1}}{t_2-t_1}=\frac{\Delta v}{\Delta t}$$
Procediamo come per la velocità, con un grafico accelerazione-tempo e la legge del moto, calcolando l'area sottostante ottengo la differenza di posizione. Si ha una situazione più semplice ancora perché avendo $a$ costante (e quindi $ \overline{a}(t)=a_{med}=\frac{\Delta v}{\Delta t}$ e quindi $v_1=v_0+a(t_1-t_0)$) essa può essere rappresentata come una retta l'area sottostante, che questa volta è letteralmente un trapezio senza approssimazioni, è lo spostamento.
\begin{center}
\includegraphics[scale=0.5]{img/gra2.png}
\end{center}
ovvero:
$$A=x-x_0=t_1-v_0+\frac{t_1(v_1-v_0)}{2}=t_1\frac{v_1+v_0}{2}$$ 
e quindi 
$$v_1=v_0+at_1$$
unendo con $v_1=v_0+a(t_1-t_0)$ si ottiene:
$$x-x_0=\frac{t_1}{2}(v_0+at_1+v_0)=\frac{t_1}{2}(2v_0+at_1)=v_0t_1+\frac{a}{2}t_1^2$$
$$\downarrow$$
$$x=x_0+v_0t_1+\frac{a}{2}t_1^2$$
Ora, come per la velocità, analizziamo intervalli di tempo infinitesimi ricordando che anche l'accelerazione è una funzione del tempo:
$$a(t)=\frac{dv}{dt}=\frac{d}{dt}\left(\frac{dx}{dt}\right)=\frac{d^2x}{dt^2}$$
ovvero la derivata seconda della posizione rispetto al tempo e si ha che:
\begin{itemize}
\item $a=0$ implica un moto rettilineo uniforme (si deriva una costante, \textit{v}, e si ottiene 0)
\item $a>0$ implica una velocità crescente
\item $a<0$ implica una velocità decrescente
\end{itemize}
\begin{shaded}
Proviamo ora a risalire a $\vec{v}(t)$ conoscendo \textit{a(t)}. Sappiamo che $a=\frac{dv}{dt}\rightarrow dv=a(t) dt$. Risolviamo quindi l'equazione differenziale :
$$\int_{\vec{v_0}}^v dv = \int_{t_0}^t a(t) dt\rightarrow \vec{v}(t)=\vec{v_0}+\int_{t_0}^t a(t) dt$$
che è l'equazione generale per la velocità, dove, nel caso di $a\neq 0$, ovvero di accelerazione costante, si ha:
$$\vec{v}(t)=\vec{v_0}+a\int_{t_0}^t dt=\vec{v_0}+a (t-t_0)$$
dove si nota come la velocità sia una funzione lineare del tempo se $t_0=0$, ottenendo $\vec{v}(t)=\vec{v_0}+a t$.
\end{shaded}
Cerchiamo ora l'equazione del moto in caso di\textit{ moto rettilineo uniformemente accelerato}.
si ha che: 
$$\vec{x}(t)=x_0+\int_{t_0}^t \vec{v}(t) dt= x_0+\int_{t_0}^t [\vec{v_0}+a (t-t_0)] dt$$
$$\downarrow$$
$$\vec{x}(t)=x_0+\int_{t_0}^t \vec{v_0} dt +\int_{t_0}^t a (t-t_0) dt$$
\begin{center}
\textit{porto fuori le due costanti, }$\vec{v_0}\,\, e\,\, a$
$$\downarrow$$
$$\vec{x}(t)=x_0+\vec{v_0} \int_{t_0}^t dt +a \int_{t_0}^t (t-t_0) dt$$
$$\downarrow$$
$$\vec{x}(t)=x_0+\vec{v_0} (t-t_0)+\frac{1}{2} a  (t-t_0)^2$$
dove, se si ha $t_0=0$, si ottiene:
$$\vec{x}(t)=x_0+\vec{v_0} (t-t_0)+\frac{1}{2} a  t^2$$
\end{center}
Si ha che $\overline{x}(t)$ con accelerazione costante è una parabola.\\
Ricapitolando si ha.
\begin{itemize}
\item $v=v_0+at$
\item $x=x_0+vt+\frac{1}{2}at^2$
\end{itemize}
Possiamo usare le due formule combinandole. Per esempio dalla prima prendo $$t=\frac{v-v_0}{a}$$ e lo metto nella seconda formula:
$$x=x_0+v\frac{v-v_0}{a}+\frac{1}{2}a \left(\frac{v-v_0}{a}\right)^2=x_0+\frac{v_0}{a}(v-v_0)+\frac{1}{2a}(v-v_0)^2$$
$$=x_0+\frac{1}{a}(v_0v-v_0^2+\frac{1}{2a}(v^2+v_0^2-2vv_0)=x_0+\frac{1}{2a}(2v_0v-2v_0^2+v^2+v_0^2-2v_0v)$$
$$x=x_0+\frac{v^2-v_0^2}{2a}\to v^2-v_0^2=2a(x-x_0)$$
Si nota come sia il termine $a t$ nel caso di $\vec{v}(t)$ che il termine $\frac{1}{2} a  t^2$ nel caso di $a(t)$ non dipendono dalle condizioni iniziali.\\
Si definisce anche la velocità finale come:
$$v_{fin}^2=v_0^2+2a\Delta x$$
L'accelerazione si esprime in metri al secondo quadrato ($\frac{m}{s^2}$, $m/s^2$ o $ms^-2$)

\subsection{Moto Verticale}
Sperimentale si scopre come un qualunque corpo lasciato libero di cadere nei pressi della superficie terrestre si muove verso il basso con un'accelerazione costante $g\simeq 9.81\, ms^{-2}$ (si trascurano attrito dell'aria e si trattano piccole altitudini). Il valore di \textit{g} non è costante in ogni parte del mondo ma può variare fino a circa il $0.6\%$.\\
Impostiamo un sistema di riferimento con l'asse \textit{x} crescente verso l'alto e quindi con $a=-g=-9.81 ms^{-2}$. Si avrà un corpo in caduta libera da un'altezza \textit{h}:
\begin{center}
\includegraphics[scale=0.3]{img/vert.png}
\end{center}
Si hanno le seguenti condizioni iniziali:
\begin{itemize}
\item $t=t_0=0$
\item $x_0=h$
\item $\vec{v_0}=0$
\end{itemize}
Con queste premesse otteniamo:
\begin{itemize}
\item \textbf{Equazione del moto:}
$$\vec{x}(t)=x_0+\vec{v_0} t+\frac{1}{2} a  t^2$$
$$\downarrow$$
$$\vec{x}(t)=h-\frac{1}{2} g t^2$$
\item \textbf{Equazione della velocità:}
$$\vec{v}(t)=\vec{v_0}+a t$$
$$\downarrow$$
$$\vec{v}(t)=-g t$$
\end{itemize}
Posso quindi ottenere il tempo di caduta, ponendo $x=0$ nell'equazione del moto:
$$h-\frac{1}{2} g t^2=0\rightarrow t_c=\sqrt{\frac{2 h}{g}}$$
e posso ottenere la velocità al suolo:
$$v_c=v(t_c)=-g t_c=-g \sqrt{\frac{2 h}{g}}=-\sqrt{2 g h}$$
Imponiamo ora una velocità iniziale $-\vec{v_1}$, quindi verso il basso:
\begin{center}
\includegraphics[scale=0.4]{img/vert2.png}
\end{center} 
Si hanno le seguenti condizioni iniziali:
\begin{itemize}
\item $t=t_0=0$
\item $x_0=h$
\item $\vec{v_0}=-\vec{v_1}$
\end{itemize}
\begin{itemize}
\item \textbf{Equazione del moto:}
$$\vec{x}(t)=x_0+\vec{v_0} t+\frac{1}{2} a  t^2$$
$$\downarrow$$
$$\vec{x}(t)=h-\vec{v_1}t-\frac{1}{2} g t^2$$
\item \textbf{Equazione della velocità:}
$$\vec{v}(t)=\vec{v_0}+a t$$
$$\downarrow$$
$$\vec{v}(t)=-\vec{v_1}-gt$$
\end{itemize}
Posso quindi ottenere il tempo di caduta, ponendo $x=0$ nell'equazione del moto:
$$h-\vec{v_1}t-\frac{1}{2} g t^2=0-\rightarrow \frac{1}{2} g t^2 +\vec{v_1}t-h=0$$
$$\downarrow$$
$$t_c=\frac{-\vec{v_1}\pm \sqrt{\vec{v_1}^2+2gh}}{g}$$
\begin{center}
\textit{ma $t<0$ non è una soluzione fisica, quindi tengo solo la soluzione col +}
\end{center}
$$t_c=-\frac{\vec{v_1}}{g}+\frac{1}{g}\sqrt{\vec{v_1}^2+2gh}$$
e posso ottenere la velocità al suolo:
$$v_c=-\vec{v_1}-gt_c=-\vec{v_1}-g\left[-\frac{\vec{v_1}}{g}+\frac{1}{g}\sqrt{\vec{v_1}^2+2gh}\right]=-\sqrt{\vec{v_1}^2+2gh}$$
Con una velocità iniziale verso il basso avremo un tempo di caduta inferiore e una velocità al suolo maggiore rispetto alla partenza da fermo.\\
Analizziamo ora il moto verticale di un punto materiale lanciato dal basso verso l'alto con velocità $\vec{v_2}$:
\begin{center}
\includegraphics[scale=0.4]{img/vert3.png}
\end{center}
Si hanno le seguenti condizioni iniziali:
\begin{itemize}
\item $t=t_0=0$
\item $x_0=0$
\item $\vec{v_0}=\vec{v_2}$
\end{itemize}
\begin{itemize}
\item \textbf{Equazione del moto:}
$$\vec{x}(t)=x_0+\vec{v_0} t+\frac{1}{2} a  t^2$$
$$\downarrow$$
$$\vec{x}(t)=\vec{v_2}t-\frac{1}{2} g t^2$$
\item \textbf{Equazione della velocità:}
$$\vec{v}(t)=\vec{v_0}+a t$$
$$\downarrow$$
$$\vec{v}(t)=\vec{v_2}-gt$$
\end{itemize}
Inizialmente si ha $v>0$, finché il punto sale verso l'alto, fino a fermarsi. Con $v=0$ si ha l'altezza massima. Si ha quindi:
$$v=\vec{v_2}-gt=0\rightarrow t_{max}=\frac{\vec{v_2}}{g}$$
e quindi:
$$x_{max}=x(t_{max})=\vec{v_2}\frac{\vec{v_2}}{g}-\frac{1}{2}g\frac{\vec{v_2}^2}{g^2}=\frac{1}{2}\frac{\vec{v_2}^2}{g}$$
raddoppiando la velocità iniziale avrò quindi un'altezza 4 volte superiore. Da questp momento in poi di avrà la caduta libera da $h=x-max$ con $\vec{v_0}=0$:
$$t_c=\sqrt{\frac{2h}{g}}=\sqrt{\frac{2x_{max}}{g}}=\sqrt{\frac{2}{g}\left(\frac{1}{2}\frac{\vec{v_2}^2}{g}\right)}=\frac{\vec{v_2}}{g}$$
e quindi si avrà:
$$t_{tot}=t_{max}+t_c=\frac{2\vec{v_2}}{g}$$
\begin{comment}
\subsection{Moto Armonico}
Si ha la seguente equazione del moto per un \textit{moto armonico semplice} lungo un asse rettilineo:
$$\vec{x}(t)=Asin(\omega t+\varphi)$$
con:
\begin{itemize}
\item \textit{A} ampiezza, espressa in $m$ e costante
\item $\omega$ pulsazione, espressa in $s^{-1}$ e costante
\item $\varphi$ fase iniziale
\item $\omega t+\varphi$ è detta fase
\end{itemize}
Si ha inoltre:
\begin{itemize}
\item $sin(\omega t+\varphi)$ che è la traiettoria e si ha che $-1\geq sin(\omega t+\varphi)\leq 1$
\item $x_0=x(0)=asin\varphi$ è la posizione iniziale generica
\end{itemize}
$\vec{x}(t)=Asin(\omega t+\varphi)$ è una funzione periodica con periodo $T=2\pi$ (la posizione si ripete dopo ogni periodo \textit{T}). Consideriamo $T=t_2-t_1=2\pi$. Si ha quindi che:
$$x(t_2)=x(t_1)$$
$$\downarrow$$
$$Asin(\omega t_2+\varphi)=Asin(\omega t_1+\varphi)$$
$$\downarrow$$
$$(\omega t_2+\varphi)-(\omega t_1+\varphi)=2\pi$$
$$\downarrow$$
$$\omega(t_2-t_1)=2\pi$$
$$\downarrow$$
$$T=\frac{2\pi}{\omega}$$
Si ha quindi che:
\begin{itemize}
\item $\omega=\frac{2\pi}{T}$, la pulsazione è inversamente proporzionale al periodo
\item \textbf{Frequenza:} $\nu =\frac{1}{T}$ quindi $\omega=2\pi\ni$
\end{itemize}
Pulsazione e frequenza sono indipendenti dall'ampiezza.\\
Posso ora trovare velocità ed accelerazione dall'equazione del \\moto $\vec{x}(t)=Asin(\omega t+\varphi)$:
\begin{itemize}
\item \textbf{velocità:} $\vec{v}(t)=\frac{d\vec{x}(t)}{dt}=A\omega\, cos(\omega t+\varphi)$
\item \textbf{accelerazione:} $\vec{v}(t)=\frac{d\vec{v}(t)}{dt}=\frac{d^2\vec{x}(t)}{dt^2}=-A\omega^2 \,sin(\omega t+\varphi)=-\omega^2 \vec{x}(t)$
\end{itemize}
Ampiezza di $\vec{v}(t)$ e \textit{a(t)} dipendono dalla pulsazione. Le tre curve \textit{$\vec{x}(t)$}, $\vec{v}(t)$ e \textit{a(t)} hanno lo stesso andamento ma sono sfasate tra loro. Ricordando che $sin(\alpha+\frac{\pi}{2}=cos \alpha$ e che $sin(\alpha+\pi)=-sin\alpha$ notiamo che $\vec{v}(t)$ è sfasata di $\frac{\pi}{2}$ rispetto a \textit{$\vec{x}(t)$} (\textit{quadratura di fase}) e che \textit{a(t)} è sfasata di $\pi$ rispetto a \textit{$\vec{x}(t)$} (\textit{opposizione di fase}) 
\begin{center}
\includegraphics[scale=0.5]{img/arm.png}
\end{center}
\end{comment}
\subsection{Moto nel Piano}
Si passa ora al moto in 2 dimensioni quindi con una traiettoria curva (e non più una retta).\\
Si introducono le coordinate cartesiane (${x}(t)$ e $y(t)$) e quelle polari ($r(t)$ e $\vartheta(t)$). Si hanno le seguenti formule per il passaggio da coordinate cartesiane a polari:
$$r=\sqrt{x^2+y^2}$$
$$tan\,\vartheta=\frac{y}{x}$$
e le seguenti per il passaggio da coordinate polari a cartesiane:
$$x=r\,cos\,\vartheta$$
$$y=r\,sin\,\vartheta$$
\begin{center}
\includegraphics[scale=0.4]{img/pia.png}
\end{center}
Il moto di $P$ è descritto attraverso l'evoluzione del vettore posizione:
$$\vec{r}(t)\equiv (\vec{x}(t),y(t))$$
Si introducono inoltre i versori degli assi $\vec{u}_x,\,\vec{u}_y$, ricordando che $|\vec{u}_x|=|\vec{u}_y|=1$ e che i versori restano fissi nel tempo. Si ottiene quindi:
$$\vec{r}(t)=\vec{x}(t)\vec{u}_x+y(t)\vec{u}_y$$
\begin{center}
\includegraphics[scale=0.4]{img/pia2.png}
\end{center}
Suppongo ora la traiettoria fissata e nota a priori. Fissata un'origine $O$, una posizione $s(t)$ e la velocità $v=\frac{ds}{dt}$ si ha che il moto è completamente determinato. Si ha una generalizzazione del moto rettilineo su una traiettoria curva.\\
Prendiamo ora in considerazione il seguente caso:
\begin{center}
\includegraphics[scale=0.4]{img/pia3.png}
\end{center}
si ha il vettore spostamento:
$$\Delta\vec{r}(t)=\vec{r}(t+\Delta t)-\vec{r}(t)$$
e il vettore velocità media:
$$\vec{v}_m\equiv\frac{\Delta\vec{r}}{\Delta t}$$
e il vettore velocità istantanea:
$$\vec{v}(t)=\lim_{\Delta t\to 0}\frac{\Delta\vec{r}}{\Delta t}=\lim_{\Delta t\to 0}\frac{\vec{r}(t+\Delta t)-\vec{r}(t)}{\Delta t}=\frac{d\vec{r}}{dt}$$
al limite $\Delta t\to 0$ lo spostamento infinitesimo si dispone sulla tangente alla traiettoria nel punto $P$: 
$$d\vec{r}=ds\vec{u}_T$$
con $|\vec{u}_T|=1$ versore della tangente che indica una direzione variabile nel tempo. Per il vettore velocità avremo:
$$\vec{v}=\frac{d\vec{r}}{dt}=\frac{ds}{dt}\vec{u}_T=v\vec{u}_T$$
con $v$ indicate il modulo della velocità e $\vec{u}_T$ la direzione.
\newpage
 Quanto appena descritto è visualizzabile nelle seguenti immagini:
\begin{center}
\includegraphics[scale=0.4]{img/pia4.png}
\end{center}
Analizziamo ora meglio la velocità nelle componenti cartesiane. Essendo $\vec{v}_m=\frac{\Delta\vec{r}}{\Delta t}$ e $\vec{r}(t)=\vec{u}_x+y(t)\vec{u}_y$ si ottiene:
$$\vec{v}=\frac{dx}{dt}\vec{u}_x+\frac{dy}{dt}\vec{u}_y=v_x\vec{u}_x+v_y\vec{u}_y$$
con il modulo della velocità:
$$v=|\vec{v}|=\sqrt{v_x^2+v_y^2}$$
Ecco un'immagine di quanto detto:
\begin{center}
\includegraphics[scale=0.6]{img/pia5.png}
\end{center}
Passiamo alle componenti polari. Essendo $\vec{v}_m=\frac{\Delta\vec{r}}{\Delta t}$ e $\vec{r}(t)=r(t)\vec{u}_r(t)$ (col versore $\vec{u}_r$ mostrato in figura) si ottiene:
$$\vec{v}=\frac{d}{dt}(r\vec{u}_r)=\frac{dr}{dt}\vec{u}_r+r\frac{d\vec{u}_r}{dt}=\frac{dr}{dt} \vec{u}_r+r\frac{d\vartheta}{dr}\vec{u}_\vartheta$$
in quanto solitamente la derivata di un versore è: 
$$\frac{d\vec{u}}{dt}=\frac{\vec{u}(t+dt)-\vec{u}(t)}{dt}=\frac{d\vartheta}{dt}\vec{u}_\bot$$
Ecco un'immagine di quanto detto:
\begin{center}
\includegraphics[scale=0.5]{img/pia6.png}
\end{center}
Possiamo approfondire ancora lo studio della velocità in componenti polari infatti:
$$\vec{v}=\underbrace{\frac{dr}{dt} \vec{u}_r}_{\vec{v_r}}+\underbrace{r\frac{d\vartheta}{dr}\vec{u}_\vartheta}_{\vec{v}_\vartheta}$$
con:
\begin{itemize}
\item $\vec{v_r}$ è la \textbf{velocità radiale} e $|\vec{v_r}|=\frac{dr}{dt}$  è la variazione di $r$
\item $\vec{v_\vartheta}$ è la \textbf{velocità traversa} e $|\vec{v_\vartheta}|=r\frac{d\vartheta}{dt}$ è la variazione della direzione
\end{itemize}
quindi:
$$\vec{v}=\vec{v_r}+\vec{v_\vartheta}$$
e quindi:
$$|\vec{v}|=\sqrt{v_r^2+v_\vartheta^2}=\sqrt{\left(\frac{dr}{dt}\right)^2+r^2\left(\frac{d\vartheta}{dt}\right)^2}$$
\newpage
Ecco un'immagine che spiega quanto detto:
\begin{center}
\includegraphics[scale=0.5]{img/pia7.png}
\end{center}
Passiamo ora all'accelerazione nel piano. Essa è, come sappiamo, la variazione della velocità $\vec{v}=v\vec{u}_T$ ma, se nel moto rettilineo è solo la variazione del modulo, nel moto del piano si ha anche la variazione della direzione. Iniziamo sapendo che $\vec{a}=\frac{d\vec{v}}{dt}$. Quindi:
$$\vec{a}=\frac{d}{dt}(v\vec{u}_t)=\frac{dv}{dt}\vec{u}_T+v\frac{d\vec{u}_t}{dt}$$
ricordando la derivata di un versore si ottiene:
$$\vec{a}=\frac{dv}{dt}\vec{u}_T+v\frac{d\phi}{dt}\vec{u}_N$$
con:
\begin{itemize}
\item $\vec{u}_N$ versore perpendicolare al versore tangente
\item $\frac{dv}{dt}\vec{u}_T$ variazione del modulo velocità, detta $\vec{a}_T$ \textbf{accelerazione tangenziale}
\item $v\frac{d\phi}{dt}\vec{u}_N$ variazione della direzione, detta $\vec{a}_T$ \textbf{accelerazione normale o centripeta}
\end{itemize}
quindi:
$$\vec{a}=\vec{a}_T+\vec{a}_N$$
procedendo con l'analisi della traiettoria si nota come essa possa essere approssimata da una circonferenza con un certo raggio $R$ che può essere usato come raggio di curvatura. Si ha quindi $ds=R\,d\phi$ e quindi:
$$\frac{d\,\phi}{dt}=\frac{1}{R}\frac{ds}{dt}=\frac{1}{R}v$$
Posso quindi sostituire $\frac{d\,\phi}{dt}$ nella formula precedentemente trovata dell'accelerazione ottenendo:
$$\vec{a}=\frac{dv}{dt}\vec{u}_T+\frac{v^2}{R}\vec{u}_N$$
quindi $\vec{a}_N$ può anche essere indicata con $\vec{a}_N=\frac{v^2}{R}\vec{u}_N$. Da questi ultimi due risultati si intuiscono due cose:
\begin{enumerate}
\item se $r\to\infty$ si ha $\vec{a}_N=0$ e quindi un moto rettilineo
\item se $\frac{dv}{dt}=0$ si ha $\vec{a}_T=0$ e quindi un moto curvilineo uniforme con solo il cambiamento della direzione
\end{enumerate}
Si ha infine il modulo dell'accelerazione:
$$a=|\vec{a}|=\sqrt{\left(\frac{dv}{dt}\right)^2+\frac{v^4}{R^2}}$$
Ecco un'immagine di quanto detto:
\begin{center}
\includegraphics[scale=0.5]{img/pia8.png}
\end{center}
Proiettiamo ora l'accelerazione sugli assi del sistema cartesiano:
$$\vec{a}=\frac{dv_x}{dt}\vec{u}_x+\frac{dv_y}{dt}\vec{u}_y=a_x\vec{u}_x+a_y\vec{u}_y$$
analizziamo anche quanto succede sul piano:
\begin{center}
\includegraphics[scale=0.6]{img/pia9.png}
\end{center}
Possiamo ora scrivere le componenti cartesiane in funzione di quella tangenziale e di quella centripeta:
\begin{itemize}
\item per l'ascisse:
$$a_x=(a_T)_x+(a_N)_x=a_tcos\,\alpha+a_ncos\left(\frac{\pi}{2}-\alpha\right)=\frac{dv}{dt}cos\,\alpha+\frac{v^2}{R}sin\,\alpha$$
\item per l'ordinata:
$$a_y=\frac{dv}{dt}sin\,\alpha-\frac{v^2}{R}cos\,\alpha$$
\end{itemize}
\subsection{Moto Circolare}
Si tratta di un caso particolare di moto curvilineo nel piano. In generale si ha il modulo della velocità non uniforme. Si hanno quindi:
\begin{itemize}
\item \textit{coordinate polari}:
\begin{itemize}
\item angolo $\theta(t)$
\item raggio $r(t)=R=costante$
\end{itemize}
\item \textit{coordinate curvilinee}:
\begin{itemize}
\item posizione misurata lungo la traiettoria $s(t)=R\theta (t)$
\end{itemize}
\item \textit{coordinate cartesiane:}
\begin{itemize}
\item $\vec{x}(t)=R\,cos\,\theta(t)$
\item $y(t)=R\,sin\,\theta(t)$
\end{itemize}
\end{itemize}
ovvero:
\begin{center}
\includegraphics[scale=0.52]{img/cir.png}
\end{center}
Iniziamo ad analizzare il moto circolare. Considero il punto $P$ in due istanti $t$ e $t+\Delta t$. Quindi avrò $\theta (t)=\theta_1$ e $\theta(t+\Delta t)=\theta_2$. Nel complesso si ha $\Delta \theta= \theta_2-\theta_1$:
\begin{center}
\includegraphics[scale=0.52]{img/cir2.png}
\end{center}
Si definisce innanzitutto la velocità angolare media:
$$\omega_m=\frac{\Delta\theta}{dt}$$
mentre per la velocità angolare istantanea si ha:
$$\omega=\lim_{\Delta t\to 0}\frac{\Delta\theta}{dt}=\frac{d\theta}{dt}$$
Si indica ora la velocità angolare in funzione di $v$ e $R$, ricordando che $ds=Rd\theta$:
$$\omega=\frac{d\theta}{dt}=\frac{1}{R}\frac{ds}{dt}=\frac{v}{R}$$
quindi la velocità angolare è proporzionale al modulo della velocità ed inversamente proporzionale al raggio. Inoltre $v=\omega R$. Partiamo da qui per approfondire la velocità nel moto circolare. Sappiamo che in generale nel moto curvilineo si ha: $\vec{v}=\frac{dr}{dt}\vec{u_r}+r\frac{d\theta}{dt}\vec{u_\theta}$. Si ha che $\frac{dr}{dt}=0$ quindi:
$$\vec{v}=R\frac{d\theta}{dr}\vec{u}_\theta=R\omega\vec{u}_\theta$$
\newpage
in quanto $R$ è costante e quindi si ha come modulo della velocità:
$$\vec{v}(t)=|\vec{v}(t)|=\omega(t)R$$
graficamente si ha:
\begin{center}
\includegraphics[scale=0.9]{img/cir3.png}
\end{center}
Si ha inoltre che se si parla di moto circolare uniforme si ha che $v=\omega R$ è costante in quanto $\omega$ è costante.\\
Passiamo all'accelerazione nel moto circolare uniforme. Si ha solo l'accelerazione centripeta in quanto $\frac{dv}{dt}=0$
$$\vec{a}=\frac{v^2}{R}\vec{U}_N$$
con $v^2$ costante e si ha il modulo dell'accelerazione pari a:
$$a=|\vec{a}|=\frac{v^2}{R}=\frac{(\omega R)^2}{R}=\omega^2 R=\omega \, v$$
Quindi per il moto lungo la traiettoria si ha:
\begin{itemize}
\item $s(t)=s_0+vt$
\item $\theta(t)=\theta_0+\omega t$
\end{itemize}
e nel moto circolare uniforme si può notare un moto periodico con periodo:
$$T=\frac{2\pi R}{v}=\frac{2\pi R}{\omega R}=\frac{2\pi}{\omega}$$
vediamo ora l'accelerazione in caso di moto non uniforme, quindi con $\vec{a}=\vec{a}_T+\vec{a}_N$ e con $\vec{v}(t)=\omega(t)R$. Definisco un'accelerazione angolare media:
$$\alpha_m=\frac{\omega_2-\omega_1}{\Delta t}=\frac{\Delta\omega}{\Delta t}$$
e l'accelerazione angolare istantanea, si ricorda che $\omega=\frac{v}{r}$:
$$\alpha=\lim_{\Delta t\to 0}\frac{\Delta\omega}{\Delta t}=\frac{d\omega}{dt}=\frac{1}{R}\frac{dv}{dt}=\frac{1}{R}a_T$$
\newpage
visivamente:
\begin{center}
\includegraphics[scale=0.4]{img/cir4.png}
\end{center}
Possiamo quindi riscrivere accelerazione normale e tangenziale in funzione di quantità angolari:
\begin{itemize}
\item $a_N=\frac{v^2}{R}\xrightarrow{v=\omega R} a_N=\omega^2 R$
\item $a_T=\frac{dv}{dt}\xrightarrow{v=\omega R} a_T=\frac{d\omega}{dt}R=\alpha R$
\end{itemize}
quindi:
$$\vec{a}=\alpha R\vec{u}_T+\omega^2 R\vec{u}_N= R(\alpha \vec{u}_T+\omega^2 \vec{u}_N)$$
quindi infine:
\begin{itemize}
\item $\omega(t)=\omega_0+\int_{t_0}^t \alpha dt=\omega_0+\alpha \int_{t_0}^t dt=\omega_0+\alpha t$ in quanto $t_0=0$
\item $\theta(t)=\theta_0+\int_{t_0}^t \omega (t)dt=\theta_0+\int_{t_0}^t (\omega_0+\alpha t)dt=\theta_0+\omega_0t+\frac{1}{2}\alpha t^2$, dove si nota l'analogia con l'accelerazione nel moto rettilineo
\item $a_N=\omega^2 R=(\omega_0+\alpha t)^2 R$
\end{itemize}
Diamo nuovamente un'occhiata alla velocità angolare $\omega=\frac{d\theta}{dt}=\frac{v}{R}$. Si ha che è una quantità scalare. Studiamo ora la notazione vettoriale di $\vec{\omega}$. Questo vettore ha direzione ortogonale alla circonferenza e , visto dalla punta di $\vec{\omega}$ il moto appare antiorario. SI ha quindi che $\vec{\omega}\times \vec{r}=\vec{v}$ e:
$$|\vec{v}|=\omega Rsin\,\frac{\pi}{2}=\omega R$$
anche se il vettore $\vec{\omega}$si può applicare a qualunque punto dell'asse $z$, ottenendo:
$$|\vec{v}|=\omega Rsin\,\phi=\omega R$$
\newpage
graficamente si ha a sinistra il caso con applicazione sull'origine normale e a destra il caso applicazione in un punto a scelta:
\begin{center}
\includegraphics[scale=0.5]{img/cir5.png}
\quad
\includegraphics[scale=0.5]{img/cir6.png}
\end{center}
\subsection{Moto Parabolico}
Consideriamo il moto di un punto materiale lanciato da terra (con angolo $\theta_0$) con una certa velocità iniziale $v_0$:
\begin{center}
\includegraphics[scale=0.5]{img/par.png}
\end{center}
posto $t_0=0$ e $\vec{r}(t_0)=0$
otteniamo per la velocità:
$$\vec{v}(t)=\vec{v}(t_0)+\int_{t_0}^t\vec{a}(t)\,dt=\vec{v_0}-g\vec{u}_y\int_{t_0}^t dt=\vec{v}_0-gt\vec{u}_y$$
$$\downarrow$$
$$
\begin{cases}
v_x=v_0cos\theta_0\\
v_y=v_0sin\theta_0-gt
\end{cases}
$$
\newpage
e per la posizione:
$$\vec{r}(t)=\vec{r}(t_0)+\int_{t_0}^t\vec{v}(t)\,dt$$
$$\downarrow$$
$$
\begin{cases}
x=x(t)=\int_{t_0}^t v_x=(v_0cos\theta_0)t\\
y=y(t)=\int_{t_0}^t v_y=(v_0sin\theta_0)t-\frac{1}{2}gt^2
\end{cases}
$$
notiamo che la componente $x$ rappresenta un moto rettilineo uniforme mentre quella $y$ un moto uniformemente accelerato. Passiamo ora al calcolo della traiettoria $y(x)$. Dall'equazione di $x(t)$ otteniamo:
$$t=\frac{x}{v_0cos\theta_0}$$
quindi:
$$y(x)=(v_0sin\theta_0)\frac{x}{v_0cos\theta_0}-\frac{1}{2}g\frac{x^2}{v_0^2cos\theta^2_0}$$
$$\downarrow$$
$$y(x)=(tan\theta_0)x-\frac{g}{2v_0^2cos^2\theta_0}x^2$$
ovvero l'equazione di una parabola con asse verticale, infatti:
\begin{center}
\includegraphics[scale=0.5]{img/par2.png}
\end{center}
\newpage
Studiamo ora la gittata, che si ha con $y(x)=0$:
\begin{center}
\includegraphics[scale=0.4]{img/par3.png}
\end{center}
Quindi, essendo $y(x)=0$, si ha:
$$tan\theta_0=\frac{g}{2v_0^2cos^2\theta_0}$$
quindi:
$$x_G=\frac{2v_0^2}{g}cos^2\theta_0 tan\theta_0=\frac{2v_0^2}{g}cos\theta_0 sin\theta_0$$
$$\downarrow$$
$$x_G=\frac{v_0^2}{g}sin(2\theta_0)$$
e, fissata la velocità iniziale si ha la gittata massima con $sin(2\theta_0)=1$ (quindi con $\theta_0=\frac{\pi}{4}=45^{\circ}$). Si ha quindi:
$$x_{G_{max}}=\frac{v_0^2}{g}$$
la parabola è simmetrica rispetto all'asse e a $x_M$ si ha l'altezza massima $y_M=y(x_M)$. Si ha:
$$x_M=\frac{1}{2}\frac{v_0^2}{g}sin(2\theta_0)$$
e, mettendo $x_M$ nella formula della traiettoria, si ottiene:
$$y_M=\frac{v_0^2}{2g}sin^2\theta_0$$
che è l'altezza massima lungo la traiettoria.
Ne segue che l'altezza massima si ha sulla verticale (con $\theta_0=\frac{\pi}{2}$) e quindi si ha l'altezza massima pari a:
$$Y_{M_{max}}=\frac{v_0^2}{2g}$$
Possiamo ora calcolare il tempo di volo, ovvero il tempo impiegato a raggiungere la gittata $x_G$:
$$t_G=\frac{x_G}{v_x}=\frac{v_0^2}{g}sin(2\theta_0)\frac{1}{v_0cos\theta_0}$$
$$\downarrow$$
$$t_G=\frac{2v_0}{g}sin\theta_0$$
che con $sin\theta_0=1$, ovvero con $\theta_0=\frac{\pi}{2}$ (sulla verticale), raggiunge il suo massimo:
$$t_{G_{max}}=\frac{2v_0}{g}$$
si hanno inoltre le seguenti velocità finali:
$$\begin{cases}
v_x(t_G)=v_x(t_0)=v_0cos\theta_0\\
v_y(t_G)=-v_y(t_0)=-v_0sin\theta_0
\end{cases}$$
Vediamo ora un punto materiale lanciato orizzontalmente da altezza $h$.
\begin{center}
\includegraphics[scale=0.6]{img/ori.png}
\end{center}
è un moto parabolico con condizioni iniziali diverse:
\begin{itemize}
\item $x(0)=0$
\item $y(0)=h$
\item $v_x(0)=v_0$
\item $v_y(0)=0$
\end{itemize}
con la seguente equazione  del moto:
$$\begin{cases}
x(t)=v_0t\\
y(t)=h-\frac{1}{2}gt^2
\end{cases}$$
che implicano:
$$\begin{cases}
v_x(t)=v_0\\
v_y(t)=-gt
\end{cases}$$
si hanno quindi:
\begin{itemize}
\item tempo di volo: $t=\frac{x}{v_0}$
\item traiettoria: $y(x)=h-\frac{g}{2v_0^2}x^2$
\item tempo di caduta: $y(t)=0\to h-\frac{1}{2}gt^2=0\to t_c=\sqrt{\frac{2h}{g}}$
\item gittata: $x(t_c)=x_G=v_0t_c=v_0\sqrt{\frac{2h}{g}}$
\item velocità finale: $v_x(t_c)=v_0$ \textit{e} $v_y(t_c)=-\sqrt{2gh}\to v_c=\sqrt{v_0^2+2gh}$
\end{itemize}
\subsection{Esercizi}
prodotto scalare:
$$\vec{a} \cdot \vec{b}=ab\,cos\theta$$
prodotto vettoriale:
$$\vec{a} \times \vec{b}=\vec{c}\to |\vec{c}|=ab\,sin\theta$$
\begin{esercizio}
si ha una strada rettilinea di 5.2km percorsa in auto a 43km/h, dopo si ha un altro percorso di 1.2km (fatto in 27m). Trovo velocità media nei due tratti:\\
%aggiungo grafico
$$v_{med}=\frac{\Delta x}{\Delta t}$$
$$\Delta x = \Delta x_1+\Delta x_2= 5.2+1.2=6.4km$$
$$\Delta t = \Delta t_1+\Delta t_2= \frac{\Delta x_1}{v}+\Delta t_2=\frac{5.2}{43}+0.45=0.12+0.45=0.57h$$
$$v_{med}=\frac{\Delta x}{\Delta t}=\frac{6.4}{0.57}=11km/h$$
ora torna alla macchina, tornando indietro di 1.2km in 35m, quindi la nuova velocità media totale sarà:
$$v_{med2}=\frac{5.2}{0.57+0.58}=4.5km/h$$
\textit{sistemare parte finale}
\end{esercizio}
\begin{esercizio}
negli anni '60 c'è stato il record di velocità al suolo con 631km/h in 3.72s. L'accelerazione media è maggiore di quella di gravità?
$$a_{med}=\frac{\Delta v}{\Delta t}=\frac{631}{3.6}\frac{1}{3.72}=47,2m/s^2$$
che è maggiore di 9.81
\end{esercizio}
\begin{esercizio}
la metro va da A a B. Quando parte accelera con 1.2m$/s^2$, per metà tratta accelera così positivamente e poi frena con lo stesso modulo. Tra A e B ci sino 1100m. Calcolo tempo e velocità massima:
$$v_{max}=v\frac{\Delta x}{2}=\sqrt{2a\Delta x\frac{1}{2}}=\sqrt{a\Delta x}=\sqrt{1.2*1100}=36,3m/s$$
$$v=v_0+at\to t=\frac{v_{fin}}{a}=\frac{36.3}{1.2}=30.3s$$
$$\Delta t_{tot}=30.3\cdot 2= 60.6s$$
in quanto accelera e frena con lo stesso modulo
\end{esercizio}
\begin{esercizio}
un tizio urla in un pozzo e l'eco gli torna dopo 2s. Quanto è profondo il pozzo ($v_{suono}=344m/s$)
$$2\Delta x= v\Delta t\to t=\frac{344\cdot 2}{2}=344m$$
\end{esercizio}
\begin{esercizio}
Un aereo per staccarsi dalla pista deve avere una velocità finale di 360km/h. La pista è lunga 1,8km. Si ha accelerazione costante. Qual è l'accelerazione minima?
$$v_{fin}^2=v_0^2+2a\Delta x\to a=\frac{360}{3.6}\frac{1}{2\cdot 1.8\times 10^3}=2.7m/s^2$$
\end{esercizio}
\begin{esercizio}
Un tizio fa cadere una chiave inglese da un grattacielo. Dov'è la chiave dopo 1.5s?
$$\Delta x= -\frac{1}{2}gt^2=-\frac{9.81\cdot 1.2^2}{2}=-11m$$
negativo nel mio sistema di riferimento 
\end{esercizio}
\begin{esercizio}
lancio una palla in alto con $v_0=12m/s$, quanto ci impiega ad arrivare alla massima altezza?
$$v_f=v_0-gt\to t=\frac{v_0}{g}=\frac{12}{9.81}=1.2s$$
quanta strada fa?
$$x=-\frac{v_0^2}{2g}=-\frac{12^2}{2\cdot (-9.81)}=7,3m$$

quanto impiega la palla per arrivare a 5m sopra il punto di lancio?
$$x=v_0t-\frac{1}{2}gt^2\to \frac{1}{2}9,81t^2+12t+5=0$$
$$\downarrow$$
$$t=\frac{12\pm \sqrt{12^2-4\frac{9,81}{2}5}}{2,45}=\begin{cases}
0,53s\\
1,9s
\end{cases}$$
sappiamo che per arrivare all'altezza massima impiega 1,2s, quindi impiegherà 0,53s per arrivare a 5m e 1,9s per salire all'apice e scendere nuovamente a 5m
\end{esercizio}
\begin{esercizio}
Un aereo getta aiuti umanitari ad una quota di 1200m volando a 430km/h. Calcolo a quale angolo devono essere gettati gli aiuti?
$$
\begin{cases}
x=v_{0_x}t\\
y=v_{0_y}t-\frac{1}{2}gt^2=-\frac{1}{2}gt^2
\end{cases}
$$ 
$$t=\sqrt{\frac{2y}{g}}=\sqrt{\frac{2400}{9,81}}=15,6s$$
$$\downarrow$$
$$x=\frac{430}{3,6}\cdot 15,6=1863m$$
$$\downarrow$$
$$tan(\frac{\pi}{2}-\theta)=\frac{\Delta x}{\Delta y}=\frac{1869}{1200}$$
$$\downarrow$$
$$\frac{\pi}{2}-\theta=arctan\left(\frac{1869}{1200}\right)=57^{\circ}$$
\end{esercizio}
\begin{esercizio}
\textit{una pallina viene scagliata contro un muro a 25m/s. Dopo l'impatto torna indietro a -22m/s. Calcolo l'accelerazione media sapendo che l'impatto dura 3,5ms}
$$a_{med}=\frac{\Delta v}{\Delta t}=\frac{25-(-22)}{3,5\times 10^{-3}}=\frac{47}{3,5\times 10^{-3}}$$
\end{esercizio}
\begin{esercizio}
\textit{una pallina viene lanciata su uno scalino. Sapendo che viene lanciata con un angolo di 60 gradi a 42m/s e che atterra dopo 5,53s calcolo l'altezza del gradino}
$$y(5,53)=v_{0_y}t-\frac{1}{2}gt^2=42sin\left(\frac{\pi}{3}\right()-\frac{1}{2}\cdot 9,8\cdot 5,53^2=51,8m$$
calcolo la velocità d'impatto:\\
calcolo prima la velocità sulle ordinate:
$$v_y(5,53)=v_{0_y}-gt=v_0sin\left(\frac{pi}{3}\right)-9,81\cdot 5,53=-17m/ s$$
e infine:
$$v=\sqrt{v_{o_y}^2
v_{0_x}^2}=\sqrt{21^2+(-17)^2}=27m/ s$$
\end{esercizio}
\begin{esercizio}
\textit{ho una pallina che si muove lungo un cerchio di raggio 0,1m. Si ha la velocità iniziale pari a 0,05m/s e dopo 1s si trova a 0,08m. Calcolo accelerazione tangenziale e centripeta a 1s}:\\
parto dall'accelerazione tangenziale
$$x(t)=v_0t+\frac{1}{2}a_Tt^2$$
$$\downarrow$$
$$8\times 10^{-2}=0,05\times 1\frac{1}{2}a_T(1)^2$$
$$\downarrow$$
$$a_T=6\times 10^{-2}m/s^2$$
passo all'accelerazione centripeta:
$$a_N(1)=\frac{v(1)^2}{R}=\frac{(v_0+a_Tt^2)^2}{R}=0,121m/s^2$$
????????????????????
\end{esercizio}
\begin{esercizio}
\textit{Un oggetto posto a 1,2m di altezza viene spinto in avanti, cadendo a 1,5m di distanza. Calcolo il tempo di volo}\\
mi basta il tempo su y:
$$\Delta y= v_{0_y}t+\frac{1}{2}gt^2$$
$$\downarrow v_{0_y}=0$$
$$t=\sqrt{\frac{2\Delta y}{g}}=\sqrt{\frac{2\cdot 1,2}{9,81}}=0,49s$$
trovo ora la velocità su x:
$$v_{0_x}=\frac{\Delta x}{\Delta t}=\frac{1,5}{0,49}=3m/s$$
\end{esercizio}
\newpage
\section{Dinamica}
La dinamica studia le cause del moto. Si studia un punto materiale con una certa massa, detta anche \textit{massa inerziale}, (che è una proprietà intrinseca dei corpi, è una quantità scalare, mentre il peso è una quantità scalare rivolta verso il basso e dipendente dall'accelerazione di gravità, e si esprime in \textit{kg} secondo il \textit{SI}) in un certo ambiente che condiziona quel punto materiale. Si hanno le tre leggi di Newton:
\begin{itemize}
\item \textbf{Prima legge:} \textit{in assenza di forze esterne (ovvero la somma delle forze vettoriali applicate è 0) su un corpo si ha che lo stesso non cambia velocità (il suo moto non cambia)}; si ha quindi un \textbf{sistema inerziale}. Nel caso di presenza di forze nel sistema di ha un \textbf{sistema non inerziale}, dove agiscono forze \textbf{non apparenti}
\item \textbf{Seconda legge:} \textit{si ha una relazione tra forza (espressa in Newton, N, è una quantità vettoriale) e accelerazione. Si ha inoltre la presenza della massa in questa relazione. Si scopre sperimentalmente che vale la seguente relazione, in quanto massa e accelerazioni sono inversamente proporzionali:}
$$\vec{F}=m\vec{a}\to \vec{a}=\frac{\vec{F}}{m}$$
$$\downarrow$$
$$\frac{d^2\vec{x}(t)}{dt^2}=\frac{\vec{F}(t)}{m}$$
ovviamente posso anche scomporre:
$$\vec{F}\equiv
\begin{cases}
\vec{F}_z=m\vec{a}_x\\
\vec{F}_z=m\vec{a}_y\\
\vec{F}_z=m\vec{a}_z\\
\end{cases}$$
\item \textbf{Terza legge:} \textit{è il \textbf{principio di azione-reazione}}
\end{itemize}
%aggungi immagine vettori forze
%approfindisci forze apparenti
Un Newton è: 
$$N=kg\frac{m}{s^2}$$
\newpage
\subsection{Forza elastica}
Si definisce la legge di Hooke:
$$\vec{F}=-k\Delta \vec{x}$$
dove $k$ è la costante elastica. La forza elastica si oppone all'allungamento della molla. Quindi si ha:
$$\vec{F}(x)=-k(x-x_0)=m\vec{a}$$
$$\downarrow$$
$$\vec{a}=\frac{-k(x-x_0)}{m}$$
Il moto poi dell'estremità della molla viene espresso da un moto armonico.
\subsection{Lavoro e Energia}
\subsubsection{Lavoro}
La \textit{conservazione dell'energia} è un principio base della fisica. Ci sono delle combinazioni matematiche di cinematica e forze che costruiscono l'energia, che non è frutto di un'indagine sperimentale. L'energia va individuata i vari aspetti del sistema considerato. Un sistema può essere formato da più punti materiali e l'energia può studiare i sistemi senza doverne scomporre le parti. Se un sistema non scambia energia con l'esterno mantiene costante l'energia interna.\\
Sappiamo che:
$$\vec{F}=m\vec{a}$$
$$\downarrow$$
$$\frac{d^2\vec{x}(t)}{dt^2}=\frac{\vec{F}(\vec{x},t)}{m}$$
Definiamo ora il lavoro di una forza:
\begin{definizione}
Il lavoro di una forza è la rappresentazione dello spostamento di un corpo causato da una certa forza. Quindi se applico una forza costante ad un corpo e si ha uno spostamento si ha che il lavoro è:
$$L=\vec{F}_x\vec{\Delta x}[Nm]$$

Forza e spostamento sono vettori, si ha quindi la direzione degli stessi. Si ha invece che il lavoro è uno scalare, quindi in realtà si ha, con $\theta$ angolo tra i due vettori:
$$L=|\vec{F}|\,|\vec{\Delta x}|cos\theta=\vec{F}\vec{s}$$
ovvero si ha il \textit{prodotto scalare} tra i due vettori. Se la forza è ortogonale allo spostamento si ha che il lavoro è nullo. $\vec{s}$ è lo spostamento e non si ha più necessità di sapere l'orientamento (potrebbe non giacere più sull'asse  delle ascisse). L'unità di misura del lavoro è il \textbf{Joule} ($J=Nm$). \\
Il lavoro totale può essere ottenuto dal lavoro della risultante delle forze o sommando i singoli lavori di ogni forza.
\end{definizione}
Non si hanno però ne forze costanti ne spostamenti lineari. Quindi prendo piccoli intervalli di spazio ne calcolo i vari lavori:
$$\sum	L_i= \sum \vec{F_{x_i}}\vec{\Delta x}$$
quella somma sarà più vera più si riducono gli intervalli di spazio. Calcolo quindi l'integrale:
$$L=\int_{x_0}^x \vec{F_x}\,dx$$

e nello spazio si generalizza così:
$$
\begin{cases}
L_x=\vec{F}_x x\\
L_y=\vec{F}_y y\\
L_z=\vec{F}_z z
\end{cases}$$
$$\downarrow$$
$$L=L_x+L_y+L_z=\vec{F}\vec{s}$$
se le tre componenti non sono costanti si ha, con $a$ indicante una traiettoria:
$$L=\int_a \vec{F}\, ds$$
\subsubsection{Energia}
Il lavoro rappresenta anche il trasferimento di energia che la forza fa su un punto materiale. Partiamo da una forza costante (quindi anche l'accelerazione sarà costante):
$$\vec{F}=\sum\vec{F_i}=m\vec{a} \mbox{ e che } \vec{a}=\frac{\vec{F}}{m}$$
ricordiamo che:
$$v_f^2=v_0^2+2a\Delta x$$
Unendo le due formule sopra si ha che:
$$v_f^2=v_0^2+2\frac{\vec{F}}{m}\Delta x$$
e sappiamo che:
$$L=\vec{F}_x\vec{\Delta x}$$
quindi:
$$v_f^2=v_0^2+2\frac{L}{m}$$
$$\downarrow$$
$$L=\frac{v_f^2-v_0^2}{2}m=\frac{1}{2}mv_f^2-\frac{1}{2}mv_0^2$$
e definiamo quest'ultima relazione come \textbf{variazione dell'energia cinetica}:
$$E_k=\frac{1}{2}mv_f^2-\frac{1}{2}mv_0^2$$
\begin{teorema}[teorema dell'energia cinetica]
Se un corpo possiede un'energia cinetica iniziale e una forza agisce sul corpo effettuando un lavoro si ha che l'energia cinetica finale del corpo è uguale alla somma dell'energia cinetica iniziale e del lavoro compiuto dalla somma delle forze risultati lungo la traiettoria del moto.\\
Questo teorema vale anche per forze variabili con il tempo o con la posizione, per sistemi a massa costante. Ovvero si ha:
$$dW=F\,ds=ma\,ds=m\frac{dv}{dt}ds=m\frac{ds}{dt}dv=mv\, dv$$
e avendo un percorso finito tra $A$ e $B$ si ha:
$$W_{ab}\int_{v_a}^{v_b}mv\,dv=\frac{1}{2}mv_b^2-\frac{1}{2}mv_a^2=E_{k,b}-E_{k,a}=\Delta E_k$$
\end{teorema}
Definiamo ora i lavori per varie forze:
\begin{itemize}
\item \textbf{lavoro della forza peso:}
$$W_{AB}=\int_A^B \vec{F}d\vec{s}=\int_A^B mg\, ds=mg\int_A^B ds=mg\,r_{AB}=mg(z_B-z_A)$$
$$\downarrow$$
$$W_{AB}=-(mgz_b-mgz_A))-(E_{PB}-E_{PA})=-\Delta E_P$$
quindi il lavoro della forza peso è pari alla differenza di energia potenziale tra i due punti.\\
in generale :
$$E_P=mgh$$
\item \textbf{lavoro della forza elastica}:
$$w=\int_A^B -kx\,dx=-k\int_A^Bx\,dx=-\left(\frac{1}{2}kx_b^2-\frac{1}{2}kx_a^2\right)=-(E_{PB}-E_{PA})-\Delta E_P$$
con 
$$E_P=\frac{1}{2}kx^2$$
che è l'\textbf{energia potenziale elastica}
\end{itemize}
\subsubsection{Forze Conservative e non Conservative}
Al moto si possono opporre delle forze, come la \textbf{forza di attrito dinamico:}
$$\vec{f}_{AD}=-\mu_DN$$
che si differenzia dalla forza di attrito statico
$$\vec{f}_{AS}=-\mu_SN$$
per il fatto che si applica su un corpo in movimento.\\
Posso calcolare il lavoro della forza di attrito dinamico:
$$W_{AD}=\int_A^B\vec{f}_{AD}\,ds=-\mu_DN\int_A^Bds=-\mu_DNl_{AB}$$
che è sempre negativo in quanto resistente ed è proporzionale alla lunghezza del tratto da $A$ e $B$. Non si può esprimere come differenza di coordinate tra $A$ e $B$.\\
Si hanno:
\begin{itemize}
\item \textbf{forze conservative}, dove il lavoro non dipende dal percorso, come per la forza peso o quella elastica e si possono esprimere come Energia Potenziale
\item \textbf{forze non conservative}, dove il lavoro dipende dal percorso, come nel caso della forza elastica. NON si possono esprimere mediante l'energia potenziale
\end{itemize}
Nel caso di forze conservative si definisce la \textbf{conservazione dell'energia meccanica:}
$$E_{KB}+E_{PB}=E_{KA}+E_{PA}$$
e si ha che l'energia si conserva durante il moto.\\
In presenza di forze non conservative l'energia meccanica non si conserva:
$$E_{KB}+E_{PB}-E_{KA}+E_{PA}=E_{MB}-E_{MA}=\Delta E_M$$
$$W=W_{cons}+W_{non-cons}$$
$$W_{non-cons}=\Delta E_M$$
quindi, per esempio, nel caso delle forze d'attrito si ha:
$$\Delta E_M=-\mu_DNl_{AB}$$
\subsection{Esercizi}
\begin{esercizio}
\textit{Un corpo di M=33kg è attaccato, mediante un filo inestensibile e privo di massa passante per una carrucola anch'essa priva di massa, ad un'altra massa di m=0,1kg in sospensione. Trovo la tensione sul filo e l'accelerazione}
$$\begin{cases}
T=Ma\\
T=mg-ma
\end{cases}$$
$$\downarrow$$
$$T=Ma=\frac{Mm}{m+M}g$$
$$\downarrow$$
$$\begin{cases}
T=0,98N\\
a=32,28m/s^2
\end{cases}$$
\end{esercizio}
\begin{esercizio}
\textit{ho una massa di m=15kg su un piano inclinato di 27\\ gradi($\theta=\frac{3}{20}\pi$) è attaccata mediante un filo inestensibile e privo di massa all'estremità superiore del piano. Trovo la tensione e la forza normale}
$$T=mg\,sin\left(\frac{3}{20}\pi\right)=15\cdot 9,8\cdot 0,54=66,8N$$
$$F_N=mg\,cos\left(\frac{3}{20}\pi\right)=15\cdot 9,8 \cdot 0,9=132,4N$$
\end{esercizio}
\begin{esercizio}
Ho un disco di 1kg legato a una fune, di 3,2m al palo. Gira senza attrito a 4m/s. Trovo la tensione.\\
$$T=F_c=m\frac{v^2}{R}=5N$$
\end{esercizio}
\begin{esercizio}
ho un oggetto su un piano inclinato scende se $\mu>0$, nel doppio del tempo necessario in assenza di attrito($t_0$): Il piano è inclinato di 35 gradi. Calcolo $\mu$:\\
$$l=\frac{1}{2}at^2$$
senza attrito:
$$t_0=\sqrt{\frac{2l}{a_0}}$$
con attrito:
$$t_1=\sqrt{\frac{2l}{a_1}}$$
trovo il rapporto tra i tempi:
$$\frac{t_0}{t_1}=\frac{1}{2}$$
quindi:
$$\frac{a_1}{a_0}=\frac{1}{4}$$
quindi nel caso di assenza di attrito:
$$F=mg\,sin\theta\to F=ma \to a_0= g\,sin\theta$$
con attrito:
$$F=mg\,sin\theta -\mu_D mg\,cos\theta\to F=ma\to a_1=g\,sin\theta -\mu_D g\, cos\theta$$
quindi:
$$\frac{a_1}{a_0}=\frac{g\,sin\theta -\mu_D g\, cos\theta}{g\,sin\theta}=\frac{1}{4}$$
$$\downarrow$$
$$\frac{\,sin\theta -\mu_D \, cos\theta}{sin\theta}=\frac{1}{4}\to \mu_D=0,52$$
\end{esercizio}
\begin{esercizio}
AGGIUNGERE ESERCIZIO CON TRE CAVI
\end{esercizio}
\begin{esercizio}

\end{esercizio}
\newpage
\section{Gravitazione}
La gravità è una delle 4 forze principali dell'universo (insieme all'i\textit{nterazione forte}, all'\textit{interazione elettromagnetica} e all'\textit{interazione debole}). \\
La gravità è una forza centrale. La forza in un qualsiasi punto \textit{P} è nella direzione $\overline{OP}$, con \textit{O} dentro della forza. Il primo a porre le basi per lo studio della gravitazione è stato Tycho Brahe e dai suoi studi Keplero formulò le tre leggi:
\begin{enumerate}
\item \textbf{prima legge:} le orbite dei pianeti sono ellittiche e il fuoco occupa uno dei tre fuochi
\item \textbf{seconda legge:} la velocità areale del raggio che unisce il sole al pianeta è costante:
$$velocita\,\,areale=\frac{\Delta A}{\Delta t}=const$$
con:
$$dA=\frac{|\vec{r}||d\vec{r}|}{2}$$
e 
$$\frac{dA}{dt}=\frac{r}{2}\frac{dr}{dt}$$
quindi l'area che si crea tra l'orbita e i raggi vettori (vettore che collega il punto al centro della forza) in due istanti di tempo è uguale in ogni punto dell'orbita a parità di $\Delta t$
\item \textbf{terza legge:} il quadrato del periodo di ricoluzione è proporzionale al cubo del semiasse maggiore (ricordiamo che $r_1+r_2=2a$, con $r_1$ e $r_2$ che sono le distanze tra i due fuochi e il punto in questione e $a$ è il semiasse maggiore):
$$T^2=k_Sa^3$$
con $k_S$ costante per tutte le orbite intorno al sole
\end{enumerate}
si definisce anche l'eccentricità:
$$\varepsilon=\sqrt{1-\frac{b^2}{a^2}}\leq 1$$
con $\varepsilon=0$ si ha un cerchio.\\
Si definisce anche l'area dell'ellisse:
$$A=\pi ab=\pi a^2 \sqrt{1-\varepsilon^2}$$
Assumiamo per comodità un'orbita circolare:
$$\frac{\Delta A}{\Delta t}=const=\frac{1}{2}r^2\frac{d\theta}{dt}\,\,\, con\,\,\, \frac{d\theta}{dt}=const$$
e si ha la forza centripeta:
$$F=m\omega^2r=m\frac{4\pi^2}{T^2}r=\frac{4\pi^2}{k_S}\frac{m}{r^2}$$
quindi la forza che il sole esercita sulla terra è:
$$F_{S,T}=\frac{4\pi^2}{k_S}\frac{m_T}{r^2}$$
inoltre per il principio di azione reazione si ha la forza esercitata dalla terra sul sole:
$$|\vec{F_{S,T}}|=|\vec{F_{T,S}}|\to \frac{4\pi^2}{k_S}\frac{m_T}{r^2}=\frac{4\pi^2}{k_T}\frac{m_S}{r^2}$$
$$\downarrow$$
$$\frac{4\pi^2}{k_Sm_S}\frac{1}{r^2}=\frac{4\pi^2}{k_Tm_T}\frac{1}{r^2}$$
definiamo
$$\frac{4\pi^2}{k_Sm_S}=\frac{4\pi^2}{k_Tm_T}=G_{TS }$$
$$\downarrow$$
$$F_{S,T}=G_{TS}\frac{m_Sm_T}{r^2}	=F_{T,S}$$
Si scopre che $G_{TS}$ vale per ogni coppia di masse e quindi la si indica semplicemente con $G$:
$$F=G\frac{m_1m_2}{r^2}$$
che è la legge di gravitazione universale, meglio scritta come:
$$F=-G\frac{m_1m_2}{r^2}\vec{u}_r$$
Possiamo ora scrivere come ottenere l'accelerazione di gravità $g$ sulla superficei terrestre:
$$G\frac{m_Tm}{r_T^2}=mg\to g=\frac{Fm_T}{r_T^2}$$
con G detta costante di \textit{Cavendish}:
$$G=6,67\times10^{-11}\frac{Nm^2}{kg^2}6,67\times10^{-11}\frac{Nm^3}{s^2kg}$$
Infatti fu Cavendish nel 1798 a definire il modulo di G. Utilizzò una bilancia di torsione, un cavo di acciaio, sottoposto a torsione, sviluppa reazioni elastiche che producono un moto armonico. Ad un filo di acciaio e' appesa una sbarra alle cui estremità sono collegate due sfere di massa \textit{m}. Al filo e' saldato uno specchio illuminato da un proiettore. La luce riflessa dallo specchio giunge su una scala graduata. Due ulteriori sfere di massa \textit{M}sono all'inizio tenute lontane dalla sfere più piccole. In queste condizioni si lasciano oscillare le due sfere liberamente e si ottiene il punto di equilibrio delle oscillazioni, ovvero la posizione di equilibrio della sbarra con le due sferette. Spostiamo ora le sfere più grandi vicino alle sferette. La forza di gravità farà in modo da influenzare l'oscillazione della sbarra spostando il centro delle oscillazioni. Per controllare, si puòinvertire la posizione delle sfere grandi per ottenere uno spostamento dal lato opposto. Questo spostamento viene causato quindi dalla forza di gravità che può essere calcolata con considerazioni fisiche e geometriche che vedremo più avanti. Conoscendo la forza $F$, le masse delle due sfere e la loro distanza $r$ si può calcolare $G$ come:
$$G=\frac{Fr^2}{Mm}$$
%cercare bene esperimento
Prendiamo ora una massa inerziale $m$, ovvero la massa resistente ad una certa forza, e una gravitazionale $M$, agente dell'iterazione gravitazionale. Si ha che $F=ma$, poniamoci nel caso della superficie terrestre con $a=g$, $M=M_T$ e $e=r_T$. Si ha:
$$G\frac{M_Tm}{r_T^2}=mg\to g=G\frac{M_t}{r_T^2}$$
possiamo quindi calcolare il campo gravitazionale:
$$F=-G\frac{Mm}{R^2}\vec{u}_{Mm}=\left(-G\frac{M}{r^2}\right)m=\eta m$$
abbiamo definito $\eta$ come il campo generato da $M$. Si definisce anche il \textbf{vettore "campo gravitazionale"}:
$$\vec{\eta}(\vec{r})=\left(-G\frac{M}{r^2}\vec{u}_r\right)$$
se su un punto $P$ si ha la presenza di più campi si può dire che:
$$\vec{\eta}(P)=\sum \vec{\eta}_i=-g\sum \frac{M_i}{r_i^2}\vec{u}_i$$
\subsubsection{Energia Potenziale Gravitazionale}
Si hanno solo forze conservative quindi $E_K+E_P=costante$. Calcolo il lavoro infinitesimo:
$$dW=\vec{F}d\vec{s}=-G\frac{Mm}{r^2}\vec{u}_rd\vec{s}=-G\frac{Mm}{r^2}dr$$
$$W=-GMm\int_a^B\frac{1}{r^2}dr=-GMm\left(-\frac{1}{r_B}-\frac{1}{r_A}\right)=\Delta E_P$$
ne segue che:
$$E_P=-G\frac{Mm}{r}$$
possiamo ora calcolare mediante la conservazione dell'energia la \textbf{velocità di fuga} necesaria a sfuggire ad un campo gravitazionale:
$$\frac{1}{2}mv_f^2-G\frac{Mm}{r}=0$$
$$\downarrow$$
$$v_f=\sqrt{\frac{2GM}{r}}$$
nel caso terrestre, posto $g=\frac{G M_T}{r_T}$
si ha:
$$v_f=\sqrt{2gr_T}=11,2\frac{km}{s}$$
(nel caso di un buco nero $v_f\to c$ velocità della luce, si ottiene $r_s=\frac{2GM}{c^2}$ detto \textbf{raggio di Schwarzschild}, nei pressi di un buco nero nessuna particella può sfuggire al campo gravitazionale).
Si può anche calcolare la velocità orbitale:
$$F=ma$$
$$\downarrow$$
$$F=m\alpha$$
$$F=m\cdot \frac{v^2}{r}$$
$$\downarrow$$
$$\frac{Mm}{r^2}=m\cdot \frac{v^2}{r}$$
$$v=\sqrt{\frac{GM}{r}}$$
\begin{teorema}[del guscio sferico]
Formulato da Newton asserisce che un corpo di massa $M$ di densità uniforme esercita su una particella esterna una fornza gravitazionale pari a quella di una particella di massa $M$ posta al centro del corpo (possiamo quindi approssimare con un punto materiale). Inoltre la fornza gravitazionale di un guscio sferico esercitata su un punto i9nterno al guscio è nulla
\end{teorema}
Analizziamo ora la situazione della forza di gravità all'interno della superficie terrestre.\\
Non possiamo ovviamente usare la legge di gravitazione universale in quanto un corpo posto esattamnete al centro della terra comporterebbe $r=0$, provocando forza e quindi accelerazione infinita. immaginiamo di scavare un tunnel che passi per il centro  della Terra e facciamo muovere un corpo dentro questo tunnel. Man mano  che il corpo si muove all'interno della Terra, dobbiamo immaginare di  dividere questa in due parti. Una prima parte e' una sfera di raggio $r$ dove $r$ è la distanza del corpo dal centro della Terra. Una seconda  parte  e' un guscio sferico nel quale il corpo si trova immerso. Osserviamo   però che un corpo all'interno di un  guscio sferico non risente di  nessun effetto gravitazionale a causa di questo. Quindi il guscio  sferico non esercita  alcuna forza sul corpo. La parte sferica seguirà la legge di gravitazione universale ma con una massa terrestre più piccola. Suppongo densità uniforme, la densità della terra è pari a $$\rho=\frac{M_T}{V_T}=\frac{M_T}{\frac{4}{3}\pi r^3}=5,5\times 10^3 \frac{kg}{m^3}$$. Posso ora ottenere la  massa della terra rimpicciolita:
$$m_T=\rho V_2=\rho\frac{4}{3}\pi r^3$$
quindi:
$$F=G\frac{m_Tm}{r_T^2}=G\frac{\rho\frac{4}{3}\pi r^3 m}{r^2}=G\frac{4}{3}\rho m r$$
ponendo $k=G\frac{4}{3}\rho m $ ottengo:
$$F=kr$$
Questa equazione mostra che la forza  cresce linearmente con la distanza ed è nulla al centro della Terra. Però è una forza attrrattiva quindi:
$$F=-kr$$
ovvero una forza elastica. Quindi, un corpo lasciato cadere all'interno di un tunnel che passa  per il centro della Terra sarebbe sottoposto all'azione di una forza  elastica: la Terra si comporta come una molla. Sappiamo ora che una  forza elastica produce un moto armonico.\\
Facciamo le ultime considerazioni sull'energia della gravitazione:
ricordiamo che:
$$E_P=G\frac{Mm}{r}$$
$$E_K=\frac{1}{2}mv^2=\frac{1}{2}m\omega^2r^2$$
e ricordando che:
$$G\frac{Mm}{r^2}=ma=m\frac{v^2}{r}$$
si ottiene quindi:
$$\omega^2r^2=G\frac{m}{r}$$
ed essendo $v=\omega r$ si ha:
$$E_K=G\frac{Mm}{2r}$$
si ha quindi che:
$$E_M=E_K+E_P=G\frac{Mm}{2r}-G\frac{Mm}{r}=-G\frac{Mm}{2r}$$
si tratta di un sistema chiudo col corpo $m$ che subisce eternamente l'attrazione di $M$. Ecco un grafico:
\begin{center}
\includegraphics[scale=0.5]{img/grav.png}
\end{center}
\chapter{Fluidodinamica}
La materia si può presentare in diverse forme, solida, fluida etc.... Nel caso della materia fluida si ha che i legami tra i vari atomi sono così deboli da non consentire la forma solida. I fluidi si dividono in liquidi e gas. In questo corso ci si occupa solo di liquidi. I fludi non hanno una forma propria (i gas non hanno neanche un volume proprio ma possono essere compressi, a differenza dei liquidi). I fluidi non sopportano sforzi di taglio, ovvero si deformano in presenza di una forza senza opporre alcuna resistenza ( e questo fattore che non permette ad un fluido di avere forma). \\
Passiamo dal concetto di punto materiale a un sistema continuo di punti. Si hanno nuove proprietà:
\begin{itemize}
\item le \textbf{proprietà estensive} del fluido, che sono addittive. Lo sono massa, volume ed energia (se ho due liquindi ognuno di massa $x$ e li unisco ne ottengo uno di massa $2x$, al più di processi chimici...)
\item le \textbf{quantità intensive} che non dipendono dalla quantità del materiale, lo sono per esempio la temperatura((se ho due liquindi ognuno di tempetratura $x$ e li unisco NON ottengo uno di temperatura $2x$...), densità ($\rho_m=\frac{m}{V}$ misurata in $\frac{kg}{m^3}$ ma può variare in un corpo, quindi $\rho(x)=\rho=\frac{dm}{dV}=\lim_{dV\to 0}\frac{dm}{dV}$)
\item la \textbf{pressione} $P=\frac{\vec{F}}{s}$, con $s$ superficie dove si applica la forza. $P$ resta uno scalare. Si dovrebbe considerare solo la componente normale della forza, $F_{\perp}$
\item si ha la cosiddetta \textbf{viscosità}, ovvero l'attrito interno allo scorrimento. Un \textbf{Fluido Ideale} si assume con viscosità nulla
\end{itemize}
Possiamo quindi, per un elemento infinitesimo di un fluido in quiete nei pressi della superficie terrestre:
\begin{itemize}
\item \textbf{Forza di volume (peso):} $dF=g\,dm=g\rho dV$
\item \textbf{Forza di pressione:} $dF= p\, dS$
\end{itemize} 
La pressione non è direzionale ma ogni elemento infinitesimo di volume di un fluido in quiete subisce uguale pressione in ogni parte della sua superficie. Si dimostra infatti che:
\begin{center}
\includegraphics[scale=0.5]{img/flu.png}
\end{center}
infatti:
$$F_2=F_1\cos \theta\to p_2bh=p_1ah\cos\theta$$
essendo $b=a\cos\theta$ si ha:
$$p_2=p_1$$
inoltre:
$$F_3=F_1\sin \theta\to p_3bh=p_1ah\sin\theta$$
essendo $b=a\sin\theta$ si ha:
$$p_3=p_1$$
e quindi:
$$p_1=p_2=p_3$$
Per quanto riguarda il lavoro delle forze di pressione si ha che:
$$dW_p=dF\,dh=p\,dS\,dh=p\,dV$$
con $dh$ indicante lo spostamento:
\begin{center}
\includegraphics[scale=0.5]{img/flu2.png}
\end{center}
ovviamente integrando si ottiene il lavoro complessivo:
$$W_p=\int p\,dV$$
\newpage
consideriamo ora un fluido in quiete:
$$dF_{peso}=g\,dm=g\rho dV=g\rho \,dS\,dh$$
$$dF_{pressione}=-dp\,dS=[p(h)-p(h+dh)]dS$$
con $p(h)-p(h+dh)$ indicante la differenza di pressione tra i piani dello stato $dh$:
\begin{center}
\includegraphics[scale=0.5]{img/flu3.png}
\end{center}
per avere una condizione di equilibrio serve:
$$dF_{peso}+dF_{pressione}=0$$
$$\downarrow$$
$$g\rho \,dS\,dh-dp\,dS=0$$
$$\downarrow$$
$$g\rho \,dh-dp=0$$
si ottiene quindi l'equazione dell'equilibrio idrostatico:
$$\frac{dp}{dh}=g\rho$$
inoltre:
$$dp=g\rho\,dh$$
$$\downarrow$$
$$\int_{p_0}^{p(h)}dp=g\rho\int_0^hdh$$
$$\downarrow$$
$$p(h)=p_0+g\rho h$$
che è la \textbf{legge di Stevino}
\newpage
consideriamo anche il seguente caso:
\begin{center}
\includegraphics[scale=0.5]{img/flu4.png}
\end{center}
$$dF_{peso}=-g\,dm=-g\rho\, dV=-g\rho\, dS\,dz$$
$$dF_{pressione}=dp\,dS$$
si ha la seguente condizione di equilibrio:
$$dF_{peso}=dF_{pressione}$$
$$\downarrow$$
$$-g\rho\, dS\,dz+dp\,dS=0$$
$$\downarrow$$
$$-g\rho\,dz+dp=0$$
$$\downarrow$$
$$dp=-g\rho\,dz$$
quindi si ha la seguente equazione dell'equilibrio idrostatico:
$$\frac{dp}{dz}=-g\rho$$
integriamo:
$$\int_{p(z)}^{p_0}dp=g\rho\int_z^{z_0}dz$$
$$\downarrow$$
$$p(z)-p_0=g\rho(z_0-z)$$
con $z_0-z$ rappresentante la profondità $h$. Si ottiene quindi la legge di Stevino:
$$p(h)=p_0+g\rho h$$
\subsubsection{Principio di Archimede}
isoliamo una porzione generica di fluido di densità $\rho$, in quel punto si ha $F_{peso}=F_{pressione}$. Inserisco al posto di quella porzione un materiale di densità $\rho_1$. Non cambia l'azione delle forze di pressione:
$$F_{peso}=g\rho V\to F_{{peso}_1}=g\rho_1 V$$
e si perde l'equilibrio:
$$F_{{peso}_1}-F_{pressione}\neq 0$$
infatti:
$$F_{{peso}_1}-F_{pressione}=g\rho_1V-g\rho V=g)\rho_1-\rho)V$$
se $\rho_1<\rho$ la spinta di Archimede prevale sulla forza peso e l'oggetto sale verso l'alto. \\
$F_A=g\rho V$ è detta \textbf{Spinta di Archimede}
\subsubsection{Moto di un Fluido}
Si assume un moto in regime stazionario, la velocità varia da un punto all'altro ma non dipende dal tempo. Si definiscono le \textbf{linee di corrente} come traiettorie di elementi di flusso tangenti al vettore velocità che sono fisse in regime stazionario. L'insieme di tutte le linee di corrente in una sezione S è detta \textbf{tubo di flusso}.
\begin{center}
\includegraphics[scale=0.5]{img/flu5.png}
\end{center}
Si definisce la portata come il volume di fluido che attraversa in un secondo una sezione infinitesima del tubo di flusso:
$$dq=v\,dS\,\,\,\left[\frac{m^3}{s}\right]$$
In regime stazionario la portata deve mantenersi costante:
$$q=\int_S dq=costante$$
$$q=\int_S v\,dS=<v>S=costante$$
con $<v>$ media delle velocità nei vari punti della sezione.\\
Questa è la \textbf{Legge di proporzionalità inversa tra velocità e sezione}
\subsubsection{Teorema di Bernoulli}
Studia la relazione tra velocità e pressione del fluido in un condotto costante.\\
Si consideri la seguente immagine:
\begin{center}
\includegraphics[scale=0.7]{img/flu6.png}
\end{center}
considero un fluido ideale, con densità costate e a regime stazionario.\\
considero il volume compreso tra $S_1$ e $S_2$ si sposta e va a riempire il tratto $S_1^{'}$ e $S_2^{'}$.\\
Si ha un fluido incomprimibili e quindi i volumi si conservano:
$$dV_1=S_1dh_1=dV_2=S_2dh_2$$
lo scorrimento quindi equivale a spostare il fluido dal volume $dV_1$ al volume $dV_2$.\\
Si ha che:
$$dW=dW_{peso}+dW_{pressione}=dE_k$$
quindi:
$$dW_{peso}=-dE_p=-g\,dm(z_2-z_1)=-g\rho\,dV(z_2-z_1)$$
$$dW_{pressione}=p_1S_1dh_1-p_2S_2dh_2=(p_1-p_2)dV$$
$$dE_k=\frac{1}{2}dmv_2^2-\frac{1}{2}dmv_1^2$$
$$\downarrow$$
$$-g\rho\,dV(z_2-z_1)+(p_1-p_2)dV=\frac{1}{2}\rho\,dVv_2^2-\frac{1}{2}\rho dVv_1^2$$
$$\downarrow$$
$$-g\rho(z_2-z_1)+(p_1-p_2)=\frac{1}{2}\rho v_2^2-\frac{1}{2}\rho v_1^2$$
$$\downarrow$$
$$g\rho z_1+p_1+\frac{1}{2}\rho v_1^2=g\rho z_2+p_2+\frac{1}{2}\rho v_2^2$$
e dato che gli stati 1 e 2 sono generali vale che:
$$p+\frac{1}{2}\rho v^2+g\rho z=costante$$
che è il \textbf{Teorema di Bernoulli} ovvero in un fluido ideale in regime stazionario 	la somma di pressione, densità di energia cinetica e densità energia potenziale si conserva.\\
Si hanno dei casi particolari:
\begin{itemize}
\item \textbf{fluido statico:} si ha $v=0$ quindi:
$$p+g\rho z=costante$$
$$\downarrow$$
$$p-g\rho h=p_0$$
e si ritrova la legge di Stevino:
$$p(h)=p_0+g\rho h$$
\item \textbf{se il condotto è orizzontale} $z_1=z_2$ si ha:
$$p+\frac{1}{2}\rho v^2=costante$$
\end{itemize}
Quindi pressione e velocità cambiano solo se cambia la sezione (con sezioni più strette aumenta la velocità e diminuisce la pressione e viceversa).
%vedere se come applicazione solo torricelli
\subsubsection{Teorema di Torricelli}
Si ha un recipiente con un piccolo foro di superficie $a$ molto minore a quella posta a contatto con l'atmosfera $A$, $a<<A$:
\begin{center}
\includegraphics[scale=0.5]{img/flu7.png}
\end{center}
si cerca la velocità in uscita dal foro. Sulla superficie $A$ si assume regime stazionario. Poiché $a<<A$ il deflusso è molto lento. Considero un tubo di flusso tra $A$ e $a$ e applico Bernoulli:
$$\left(p+\frac{1}{2}\rho v^2+g\rho z\right)_A=\left(p+\frac{1}{2}\rho v^2+g\rho z\right)_a$$
con $p_A=p_0$, $v_A=0$, $z_A=h$, $p_a=p_0$, $v_a=v$ e $z_a=0$\\
e si ottiene:
$$p_0+g\rho h=p_0+\frac{1}{2}\rho v^2$$
$$\downarrow$$
$$2gh=v^2\to v=\sqrt{2gh}$$
quindi la velocità di deflusso non dipende de dalla densità del fluido ne dalla pressione esterna. Si nota come sia uguale al moto in caduta libera
\newpage
\subsection{esercizi}
\begin{esercizio}
ho un palloncino d'aria sott'acqua di raggio R e m=10g, che forza devo applicare per non farlo salire? \\$\rho_{aria}=1,22 \frac{kg}{m^2}$ $R=0,15m$ $\rho_{acqua}=1000 \frac{kg}{m^2}$\\
$$P_P+P_A+F=F_{archimede}$$
$$\downarrow$$
$$mg+\rho_{aria}\frac{4}{3}\pi R^3g+F=g\rho_{acqua}\frac{4}{3}\pi R^3$$
$$\downarrow$$
$$F=g\rho_{acqua}\frac{4}{3}\pi R^3-mg-\rho_{aria}\frac{4}{3}\pi R^3g=138,2N$$
\end{esercizio}
\begin{esercizio}
Si ha un pallone aereostatico pieno di elio con appesa un amassa m. La massa dell'involucro è 15kg, quella appesa è di 40kg con volume trascurabile. La densità dell'aria in quella quota è $0,035 \frac{kg}{m^2}$. La densità dell'elio è $0,0051 \frac{kg}{m^2}$. Caloclo il volume del pallone.\\
$$F_{archimede}=P_P+P_{elio}+P_M$$
$$\downarrow$$
$$P_P+V_P\rho_{elio}g+P_M-V_P\rho_{aria}g=0$$
$$\downarrow$$
$$V_P=1.8\times 10^3 m^3$$
\end{esercizio}
\begin{esercizio}
ho una diga alta 15m, a $H_1=$6m è posto un tubo tappato di diametro 4cm, calcolo la forza nel tubo\\
l'acqua esercita forza su tutta la perete della diga. Chiamo la forza del tappo $F_T$ e $P_T$ la pressione sul tappo di raggio $R_T$:
$$F_T=P_TR_T^2\pi=\rho g h_12\pi R_T^2=74N$$
\end{esercizio}

\newpage
\chapter{Moto Armonico}
Si usa il moto armonico per descrivere il moto della molla e del pendolo.
Comincio con una molla a riposo su un oiano liscio a cui è attaccata una massa. Allungo la mossa si un certo $x=A$:
\begin{center}
\begin{pspicture}(0,-0.81)(4.82,0.81)
\psline[linecolor=black, linewidth=0.04](0.02,0.81)(0.02,-0.79)(4.82,-0.79)
\psline[linecolor=black, linewidth=0.04](2.42,0.01)(3.22,0.01)(3.22,-0.79)(2.42,-0.79)(2.42,0.01)(2.42,0.01)
\pscircle[linecolor=black, linewidth=0.04, dimen=outer](0.42,-0.39){0.4}
\pscircle[linecolor=black, linewidth=0.04, dimen=outer](0.82,-0.39){0.4}
\pscircle[linecolor=black, linewidth=0.04, dimen=outer](1.22,-0.39){0.4}
\pscircle[linecolor=black, linewidth=0.04, dimen=outer](1.62,-0.39){0.4}
\pscircle[linecolor=black, linewidth=0.04, dimen=outer](2.02,-0.39){0.4}
\rput[bl](2.62,-0.39){$m$}
\end{pspicture}
\end{center}
e si ha:
\begin{center}
\begin{pspicture}(0,-0.8)(3.2,0.8)
\psframe[linecolor=black, linewidth=0.04, dimen=outer](3.2,0.8)(1.6,-0.8)
\psline[linecolor=black, linewidth=0.04, arrowsize=0.05291667cm 2.0,arrowlength=1.4,arrowinset=0.0]{->}(1.6,0.0)(0.4,0.0)
\rput[bl](-0.9,0.4){$F_S=-kx$}
\end{pspicture}
\end{center}
quindi $F=ma\to -kx=ma\to a=-\frac{k}{m}z$ con $a=\frac{dv}{dt}=\frac{d^2x}{dt^2}$ e $v=\frac{dx}{dt}$ quindi:
$$\frac{d^2x}{dt^2}=-\frac{k}{m}x\to \frac{k}{m}=\omega^2$$
$$\downarrow$$
$$\frac{d^2x}{dt^2}=-\omega^2 \to \frac{d^2x(t)}{dt^2}-\omega^2x(t)$$
$$\downarrow$$
$$x(t)=A\cos(\omega t+\phi)$$
con $\phi$ detta fase.
\\faccio derivata prima e ottengo la velocità:
$$\frac{dx(t)}{st}=v(t)=-A\sin(\omega t+\phi)$$
che è sfasata di $\frac{\pi}{2}$ rispetto all'equazione del moto
\\derivo ancora e ottengo l'accelerazione:
$$\frac{dv(t)}{st}=a(t)=-\omega^2A\cos(\omega t+\phi)$$
l'accelerazione è massima quindi con lo spostamento massimo della molla
\begin{center}
\includegraphics[scale=0.7]{img/arm.png}
\end{center}
sappiamo che $\omega=\sqrt{\frac{k}{m}}$, misurata in $\frac{rad}{s}$.
\\Posso calcolare il periodo $T$, ovvero il tempo impiegato dal corpo per tornare nella stessa posizione:
$$x(t)=x(t+T)$$
$$\downarrow$$
$$x(t)=A\cos(\omega t+\phi)=A\cos(\omega (t+T)+\phi)=x(t+T)$$
quindi voglio:
$$(\omega(t+T)+\phi)-(\omega t+\phi)=2\pi$$
$$\downarrow$$
$$\omega t+\omega T-\omega t= 2\pi$$
$$\downarrow$$
$$T=\frac{2\pi}{\omega}$$
quindi si ha anche che:
$$F=2\pi\sqrt{\frac{m}{k}}$$
passiamo quindi all'energia. All'inizio, con la massa ferma si ha l'energia potenziale elastica $\frac{1}{2}kx^2$ e in movimento si avrà anche la cinetica $\frac{1}{2}mv^2$:
$$E_k=\frac{1}{2}mv^2=\frac{1}{2}m(-A\sin(\omega t+\phi))^2=\frac{1}{2}m\omega^2A^2\sin^2(\omega t)$$
$$E_P=\frac{1}{2}kx^2=\frac{1}{2}k(A\cos(\omega t+\phi))^2=\frac{1}{2}kA^2\cos^2(\omega t)$$
$$\downarrow$$
$$E_m=\frac{1}{2}m\omega^2A^2\sin^2(\omega t)+\frac{1}{2}kA^2\cos^2(\omega t)$$
$$\downarrow$$
$$E_m=\frac{1}{2}m\left(\frac{k}{m}\right)^2A^2\sin^2(\omega t)+\frac{1}{2}kA^2\cos^2(\omega t)$$
$$\downarrow$$
$$E_m=\frac{1}{2}kA^2(\sin^2(\omega t)+\cos^2(\omega t))=\frac{1}{2}kA^2\cdot 1=\frac{1}{2}kA^2$$
passiamo al pendolo
\begin{center}
\includegraphics[scale=0.6]{img/pen.png}
\end{center}
$$F_p=-mg\sin\theta$$
$$F=ma=-mg\sin\theta $$
$$\downarrow$$
$$\frac{d^2x}{dt^2}=-g\sin\theta$$
$$\downarrow$$
$$x=L\theta\to \frac{d^2\theta(t)}{dt^2}\theta=-\frac{g}{L}\sin\theta$$
$$\downarrow$$
$$\theta(t)=\theta_{max}\cos(\omega t+\phi)\to \omega=\sqrt{\frac{g}{L}}$$
$$t=\frac{2\pi}{\omega}=2\pi\sqrt{\frac{L}{g}}$$
\chapter{Termodinamica}
%aggiungi parte su meccanca statisica
In meccanica si è visto come in presenza di forze non conservative si ha dispersione di energia. Uno degli argomenti della Termodinamica è appunto lo studio del bilancio energetico complessivo di un sistema fisico estendendo questo studio a scambi energetici non solo macroscopici. \\
Un sistema termodinamico è assimilabile, a livello meccanico, ad un sistema continuo, considerato che a livello microscopico è costituito da un numero di elementi dell'ordine del \textit{numero di Avogadro} $N_a=6,022\times 10^{23}$.\\
I sistemi termodinamici vengono solitamente studiati in uno stato di quiete e si studiano le trasformazioni subite dal sistema e gli scambi energetici.\\
Un \textbf{sistema termodinamico} è una porzione del mondo che può essere composta da una o più parti mentre l'\textbf{ambiente} è costituito anch'esso da una o più parti ed è ciò che interagisce col sistema termodinamico. \\
Sistema termodinamico e ambiente uniti sono detti \textbf{universo termodinamico}.\\
A differenza della Termodinamica la \textbf{meccanica statistica} è quella parte della fisica che studia, mediante metodi statistici, il comportamento di insiemi di un grande numero di particelle (atomi, molecole ecc.), allo scopo di prevederne le proprietà macroscopiche (per esempio, volume, densità, pressione, temperatura ecc.).\\
Si hanno diverse classificazioni dei sistemi termodinamici:
\begin{itemize}
\item \textbf{sistema aperto:} se tra sistema e ambiente avvengono scambi di energia e materia
\item \textbf{sistema chiuso:} se tra sistema e ambiente avvengono scambi di energia ma sono esclusi scambi di materia
\item \textbf{sistema isolato:} se tra sistema e ambiente non avvengono scambi di energia e materia
\end{itemize}
Oggetto di studio sono principalmente i sistemi chiusi.\\
Un sistema termodinamico viene descritto da un numero ridotto di grandezze fisiche direttamente misurabili, che vengono dette \textbf{Variabili Termodinamiche} (per esempio volume, pressione, temperatura, massa, densità, etc...). Alcune variabili esprimono una proprietà globale del sistema, dipendente da dimensione ed estensione; queste variabili sono dette \textbf{estensive} e sono additive. Altre variabili esprimono una proprietà locale, che può variare all'interno del sistema; queste sono dette \textbf{intensive} e non sono additive.\\
Massa e volume sono estensive mentre pressione, temperatura e densità sono intensive.\\
Il numero di coordinate necessarie a descrivere uno stato termodinamico non è fissato a priori ma dipende dalle caratteristiche fisico-chimiche del sistema. Inoltre ad uno stato termodinamico possono corrispondere diversi stati meccanici. \\
Si ha che lo \textbf{stato termodinamico} di un sistema p detto di \textbf{equilibrio} quando le variabili (dette \textbf{variabili di stato}) che lo descrivono sono costanti nel tempo.\\
l'equilibrio termodinamico è il risultato di tre diversi tipi di equilibrio:
\begin{enumerate}
\item \textbf{equilibrio meccanico:} equilibrio di forze e momenti
\item \textbf{equilibrio chimico:} assenza di reazioni chimiche o di trasferimenti interni al sistema
\item \textbf{equilibrio termico:} temperatura costante in tutto il sistema
\end{enumerate}
Quando c'è equilibrio si ha equilibrio tra le forze macroscopiche, si ha equilibrio in ogni parte del sistema e si ha equilibrio con l'ambiente. Inoltre in caso di equilibrio la temperatura del sistema è uguale a quella dell'ambiente.\\
In condizione di equilibrio si ha la cosiddetta \textbf{equazione di stato} che lega le coordinate termodinamiche. Per esempio le coordinate sono pressione ($p$), volume ($V$) e temperatura ($T$) si ha la seguente forma implicita:
$$f(p,V,T)=0$$
o una delle tre forme esplicite:
$$p=p(V,T)$$ 
$$V=V(p,T)$$ 
$$T=T(p,V)$$
il passaggio tra due diversi stati di equilibrio è detto \textbf{trasformazione termodinamica del sistema}. Verranno considerati solo stato iniziale e finale e se gli stati sono molto prossimi si avranno trasformazioni infinitesime, rappresentate da $dp, dV \mbox{ e } dT$.\\
Due sistemi ($A$ e $B$, in equilibrio termodinamico) si dicono in equilibrio termico tra loro quando hanno la stessa temperatura, $T_A=T_B$; la temperatura è quindi indice dell'equilibrio termico tra stati. Si verifica anche il seguente \textbf{principio dell'equilibrio termico, detto anche principio zero della termodinamica:}\\
\textit{due sistemi entrambi in equilibrio termico con un terzo sistema sono in equilibrio termico tra loro}.\\
Per portare due sistemi in equilibrio termico si può usare il sistema del contatto e se questo porta effettivamente all'equilibrio termico si ha a che fare con una \textbf{parete diatermica} altrimenti con una \textbf{parete adiabatica} (che però è un caso limite su tempi brevi). Due sistemi separati da una parete diatermica sono detti in \textbf{contatto termico} e inevitabilmente raggiungono l'equilibrio termico. Il contatto termico può avvenire anche direttamente senza la presenza di una parete, che si rende però necessaria, eventualmente, per contenere un gas. Un sistema è detto \textbf{adiabatico} se circondato da pareti adiabatiche e non può essere messo in contatto termico.
\section{Temperatura} 
Per dare una definizione operativa di temperatura servono due condizioni:
\begin{enumerate}
\item deve esistere una grandezza $X$ che caratterizza un fenomeno fisico e varia con la temperatura. $X$ è detta \textbf{caratteristica termometrica}, registrata dal \textbf{Termometro} e la temperatura è una funzione di $X$, $\theta(X)$, detta \textbf{funzione termometrica}  
\item deve esistere un sistema in stato di equilibrio ben definito e riproducibile facilmente a cui viene attribuito un valore arbitrario di temperatura, detto \textbf{punto fisso} 
\end{enumerate} 
Questo punto fisso fu scelto nel 1954 e fu scelto il \textbf{punto triplo dell'acqua}, ovvero quando ghiaccio, acqua e vapore acqueo saturo sono in equilibrio, e fu scelta la temperatura di 273,15 K (il Kelvin). Tarando un termometro sul punto triplo dell'acqua lo si può mettere a contatto con un qualsiasi sistema e la temperatura sarà così calcolata, con $X_{pt}$ che è la $X$ al punto triplo dell'acqua:
$$T=273,16\frac{X}{X_{pt}}[K]$$
\newpage
Si hanno diverse \textbf{scale termiche}:
\begin{itemize}
\item \textbf{scala Celsius:} si ha la temperatura del punto triplo dell'acqua a 0,01 gradi Celsius ($^{\circ}C$), pertanto lo zero della scala Celsius è a 273,15 K e corrisponde alla fusione del ghiaccio a pressione atmosferica Il punto di ebollizione dell'acqua a pressione atmosferica è di $100^{\circ}C$ e la temperatura ambiente si ha a $20^{\circ}C$. Ovviamente:
$$t(^{\circ}C)=T(K)-273,15$$
\item \textbf{scala Rankine}:
$$t(^{\circ}R)=\frac{9}{5} T(K)$$
\item \textbf{scala Fahrenheit}: si ha il punto di fusione del ghiaccio a $32^{\circ}F$ e il punto di ebollizione dell'acqua a $212^{\circ}F$ con la temperatura ambiente a $68^{\circ}F$. si ha:
$$t(^{\circ}F)=\frac{9}{5} T(K)-459,67$$
$$t(^{\circ}F)=\frac{9}{5} T(^{\circ}C)+32$$
$$t(^{\circ}C)=\frac{5}{9} [T(^{\circ}F)-32]$$
\end{itemize}
la scala in Kelvin è detta \textbf{scala Assoluta} e viene adottata dal SI. Lo zero della scala assoluta è detto \textbf{zero assoluto} ed è la temperatura più bassa raggiungibile. Viene usata perché ottenuta dalle relazioni termodinamiche senza proprietà additive.
\subsection{Termometro a Gas}
La proprietà termometrica si dimostra più soddisfacente con le misure in kelvin e con la pressione di un gas mantenuto a volume fisso. Si è quindi ideato il \textbf{termometro a gas a volume costante}:
\begin{center}
\includegraphics[scale=0.5]{img/term.png}
\end{center}
dove un bulbo riempito di ha ha la forma idonea a contenere la sostanza di cui va misurata la temperatura. Il volume del gas è mantenuto costante abbassando o sollevando il livello di mercurio in modo che il livello di mercurio nel ramo a sinistra coincida con un valore fisso. Procediamo alla misurazione:
\begin{itemize}
\item si immerge il bulbo in una cella a punto triplo e si misura la pressione mediante il manometro a mercurio. Immergo poi il bulbo nel liquido di cui bisogna sapere la temperatura. Pongo $X=p$ e $X_{pt}=p_{pt}$ e applico la formula:
$$T=\frac{X}{X_{pt}}273,16K$$
\item ripetiamo il primo passaggio con una pressione inferiore
\item ripetere n volte
\end{itemize}
ottengo alla fine, a volume costante, una scala di temperatura del gas perfetto:
$$T=\lim_{p_{pt}\to 0}\frac{p}{p{pt}}273,16K$$
un \textbf{gas perfetto} fornisce il valore T del punto triplo a qualsiasi pressione esso sia. Il termometro a gas misura a partire da 1K (usando l'elio a bassa pressione)
\subsubsection{Gas Perfetti}
Innanzitutto si ricorda che un gas è un fluido che non ha forma e volume proprio, ma assume quelli del contenitore, inoltre è facilmente comprimibile. \\
Per descrivere un gas si usano le variabili termodinamiche della pressione, della temperatura e del volume. Quando il volume del contenitore cambia si ha uno scambio di lavoro con l'ambiente esterno (lo scambio di calore dipende dal tipo di pareti). Un gas può quindi scambiare solo lavoro o anche calore.\\
\begin{definizione}
si abbia un gas in equilibrio termodinamico con delle variabili termodinamiche ben descritte. Se si fanno variare pressione e volume mantenendo costante la temperatura si scopre che in tutti gli stati possibili di equilibrio isotermi si ha:
$$pV=costante$$
questa è la \textbf{Legge di Boyle}
\end{definizione}
Una trasformazione isoterma si ha, per esempio, con pareti diatermiche, di cui una mobile, in contatto con una sorgente di calore. Si avrà quindi uno spostamento della parete mobile a seguito della differenza di pressione interna ed esterna.\\
In un passaggio di stato si ha sempre:
$$p_1V_1=p_2V_2$$
\begin{center}
\includegraphics[scale=0.5]{img/term2.png}
\end{center}
\subsubsection{Legge di Avogadro}
Passiamo ora all'aspetto più microscopico.\\
Si ha la \textbf{Legge di Avogadro}: \textit{volumi eguali di gas diversi, alla stessa temperatura e pressione, contengono lo stesso numero di molecole}. Si presuppongono gas ideali.\\
Detta $M$ la massa totale del gas e $m$ la massa della singola molecola si ha che:
$$N_{molecole}=\frac{M}{m}$$
inoltre:
$$m=Am_u=A1,6604\times 10^{-27}kg$$
con $A$ massa molecolare e $m_u$ massa atomica. Si ottiene quindi:
$$N_{molecole}=6,0221\times 10^{26}\frac{M}{A}$$
Considerando una massa $M$ numericamente uguale a $A$, ovvero $A$ chilogrammi di gas, quantità detta chilomole (kmol), si ha:
$$N=N_A=\frac{1}{m_u}=6,0221\times 10^{26}\left[\frac{molecole}{kmol}\right]$$
ma se si passa a $A$ grammi si ottiene una quantità detta \textbf{mole}, e si ottiene il numero di Avogadro:
$$N_A=6,0221\times 10^{-23}\left[\frac{molecole}{mol}\right]$$
si ha quindi che \textit{una mole è una quantità di materia che contiene tante entità elementari quanti sono gli tomi contenuti in 0,0012kg dell'isotopo del carbonio}$^{12}C$. Inoltre ad una quantità di materia data da un certo numero di moli corrispondono masse diverse a seconda della sostanza ma queste masse contengono lo stesso numero di molecole.\\
Si ha un'altra conseguenza: a pressione atmosferica e a 273,15K il volume è:
$$V_m=0,022414m^3=22,414\,\,litri$$
e $V_m$ è detto volume molare. La massa molare di un composto rappresenta la massa in grammi di una mole (espressa in g/mol) e coincide numericamente con la massa molecolare (espressa in grammi ma solitamente indicata con un multiplo della massa atomica, per esempio l'ossigeno è 16u)
\end{document}